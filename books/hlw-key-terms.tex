\documentclass[12pt]{article}

% --- reasonable margins ---
\usepackage[margin=1in]{geometry}

\usepackage[T1]{fontenc}
\usepackage{lmodern}
\usepackage{microtype}
\usepackage{enumitem}

\setlist[description]{style=nextline,leftmargin=0pt,labelsep=0.75em}

\title{Key Terms from \emph{How Logic Works}}
\author{}
\date{}

\begin{document}
\maketitle

\section*{Glossary terms}

\begin{description}
  \item[consequent] The consequent of a conditional is the sentence that occurs after ``then''.
  \item[contingency] A sentence that is true in some situations and false in other situations.
  \item[counterexample] A situation (formalized) in which the premises are true and the conclusion is false.
  \item[dependency number] A number in the leftmost column of a proof indicating which assumptions are in force at that step.
  \item[disjunction] A sentence whose main connective is ``or'' ($\vee$).
  \item[existential quantifier] The symbol $\exists$ (``some'' / ``there is'').
  \item[expressively complete] A collection of connectives that can express all truth functions.
  \item[inconsistency] A sentence that is false in all situations.
  \item[interpretation] An assignment of symbols to set-theoretic structures.
  \item[main column] The column in a truth table corresponding to the main connective of the sentence.
  \item[main connective] For a propositional sentence $\varphi$, the last connective in the construction of $\varphi$.
  \item[model] A model of a theory is an interpretation in which all the theory's sentences are true.
  \item[necessary condition] In ``if $\varphi$, then $\psi$,'' the consequent $\psi$ is a necessary condition for $\varphi$.
  \item[reconstrual] A reconstrual assigns nonlogical symbols to corresponding syntactic structures.
  \item[sequent] A list of premises, a turnstile, and a conclusion; a symbolic representation of a valid argument form.
  \item[signature] A set of nonlogical symbols: propositional constants, relation, function, and constant symbols.
  \item[sound] A sound proof system never proves things it shouldn't.
  \item[subformula] Any formula that occurs in the construction of $\varphi$.
  \item[substitution] Replacing nonlogical symbols with other suitable syntactic structures.
  \item[substitution instance] A sentence resulting from another by a uniform replacement of nonlogical terms.
  \item[sufficient condition] In ``if $\varphi$, then $\psi$,'' the antecedent $\varphi$ is a sufficient condition for $\psi$.
  \item[translation] A map from formulas to formulas, generated by a reconstrual of nonlogical vocabulary.
  \item[truth-functional] A connective is truth-functional iff its truth value is a function of the truth values of its component sentences.
  \item[universal quantifier] The symbol $\forall$ (``all'' / ``every'').
  \item[valid] An argument is valid iff its premises provide decisive support for its conclusion (equivalently: if the truth of its premises guarantees the truth of its conclusion).
  \item[variable] A symbol such as $x$ that plays the role of an open term.
\end{description}

\section*{Common acronyms (proof justifications)}

\begin{description}
  \item[CNF] conjunctive normal form
  \item[DM] DeMorgan's rule
  \item[DNF] disjunctive normal form
  \item[EFQ] \emph{ex falso quodlibet}
  \item[EM] law of excluded middle
  \item[MP] modus ponens
  \item[MT] modus tollens
  \item[QN] quantifier negation equivalences
  \item[RAA] \emph{reductio ad absurdum}
\end{description}

\end{document}

%%% Local Variables:
%%% mode: latex
%%% TeX-master: t
%%% End:
