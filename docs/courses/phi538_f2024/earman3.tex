\documentclass[12pt]{article}
\usepackage{fullpage}
\usepackage{enumitem}
\setlength{\parskip}{1em}
\setlength{\parindent}{0em}
\usepackage{amsfonts,amsmath,amsthm}
\theoremstyle{definition}
\newtheorem*{defn}{Definition}
\newtheorem*{ass}{Assumption}
\usepackage{outlines}
\newcommand{\mf}{\mathfrak}
\begin{document}

Chapter 3 of \emph{World Enough and Spacetime} is challenging because
it has heavy technical \emph{and} philosophical components. I assume
that the majority of you would prefer to focus on the philosophical
component, so let's start with that. The technical component leaves
open many questions --- on which it is possible to make some definite
progress. But it's doubtful that we'll make that progress in the
session today.

\section*{What does it mean to ``choose a spacetime''? And what
  kind of reasons can one have for doing so?}

\begin{outline}[enumerate]

\1 One might think of ``choosing a spacetime'' as a simple exercise
of first-order metaphysical theorizing. Consider e.g.\ the reasons
that David Builes has for preferring monism to other views in
metaphysics. One might adduce similar reasons for thinking that
spacetime is fully Galilean and not just Maxwellian (or vice versa)

\2 For example, is there reason to believe that spacetime has a
preferred foliation?

\2 Granted: we are all convinced now that spacetime is Einsteinian
(e.g.\ a Lorentzian manifold), and none of these. But this is an
exercise in historical reconstruction. What reasons were there for the
participants in this debate. And we will come later to the questions
that face us today. e.g.\ does GTR posit too much spacetime structure?

\1 Material versus formal mode (Carnap)

The metaphysician's question is posed in the material mode. There is a
parallel question in the formal mode: what reasons can we have for
adopting a spacetime theory. The difference here is that the
metaphysical implications of adopting a theory are not fully
transparent. What \emph{does} one believe if one adopts Galilean
relativity? I myself am not convinced that adopting Galilean
relativity is tantamount to asserting that spacetime has the structure
of a quadruple $\langle M,h^{ab},t_c,\nabla \rangle$ with certain
specified properties.

\2 Carnap thought that the philosopher's job is to ask questions in
the formal mode. Earman was reacting against the logical postivists
and poses the question in the material mode.

\2 Earman's approach is tied together with the semantic view of
theories, i.e.\ a theory is a collection of models and believing the
theory is tantamount to believing that the world is represented by one
of them


\1 Empiricist criteria, e.g.\ eliminate unobservable structure

Earman doesn't like these --- neither the stronger (positivist) view
that ``I don't know what that means'' nor the weaker view that ``I
won't believe it if I can't see it''.

Nonetheless, he is perfectly ok with \emph{Ockhamism} as a
methodological principle. His view here is influential, or at least
widely shared.



\1 Symmetry principles 




\end{outline}

\section*{Earman's technical definitions}

\begin{defn} A classical theory of motion $T$ is associated with a
  family $\mf M _T$ of models. Each model has the form
  $\langle M,A_1,A_2,\dots ,P_1,P_2,\dots \rangle$ where the $A_i$
  represent ``fixed spacetime structure'' and the $P_i$ represent
  ``the physical contents of spacetime''. \end{defn}

No reason to write out the full list, or even to specify that the
$A_i$ and $P_i$ are ``geometric-object fields''. We can just say:
$\langle M,A,P\rangle$, where $A$ represents (fixed) spacetime
structure, and $P$ represents material contents. There are various
ways we could do the specification. e.g.\ $A$ might be a metric in the
sense of a function $d:M\times M\to M$, and $P$ might be a curve
representing the trajectory of a particle.

But: there is an ambiguity in Earman about whether $P$ represents
fixed material contents, or is a specification of possible material
contents.

\begin{ass} For any two models $\langle M,A,P\rangle$,
  $\langle M',A',P'\rangle$ in $\mf M_T$, there is an isomorphism in
  the relevant category from $\langle M,A\rangle$ to $\langle
  M',A'\rangle$. \end{ass}

There is also an ambiguity about the role of ``diffeomorphisms'' in
the specification of $\mf M_T$. This ambiguity appears in the
following:

\begin{defn} Let $\Phi$ be a diffeomorphism of $M$ onto $M$. We say
  that $\Phi$ is a \emph{dynamical symmetry} if for any model $\langle
  M,A,P\rangle$ in $\mf M_T$, the structure $\langle M,A,\Phi
  ^*P\rangle$ is also in $\mf M_T$. \end{defn}

For example: a map $\Phi$ that takes straight lines to bent curves is
\emph{not} a dynamical symmetry of Galilean relativity.


\begin{defn}[General covariance] The laws of $T$ are \emph{generally
    covariant} just in case whenever $\langle M,A,P\rangle \in\mf M_T$
  then $\langle M,\varphi ^*A,\varphi ^*P\rangle \in\mf
  M_T$. \end{defn}


\section*{A mini hole argument}

\begin{outline}[enumerate]

  \1 Earman claims that the spacetimes with less structure than
  Galilean correspond to indeterministic theories. In particular,
  there are models $M$ and $N$ and an isomorphism $\varphi :M\to N$
  such that $\varphi |_{t<0}=\mathrm{id}$ but
  $\varphi \neq \mathrm{id}$.

  \1 But doesn't the fact that $\phi :M\to N$ is (according to $T$) an
  isomorphism mean that $M$ and $N$ are (according to $T$) really the
  same model?

  \1 There is, in fact, something syntactically wrong with the
  statement above: $\varphi |_{t<0}=\mathrm{id}$ makes sense only if
  $M=N$. (The identification of the underlying sets $M$ and $N$ is
  permitted by Earman's setup, but we have to wonder if this setup
  isn't problematic.)
  
  \1 Wouldn't a real violation of determinism be two models that are
  isomorphic on an initial segment but not isomorphic subsequently?

 \end{outline}

\end{document}
%%% Local Variables:
%%% mode: latex
%%% TeX-master: t
%%% End:
