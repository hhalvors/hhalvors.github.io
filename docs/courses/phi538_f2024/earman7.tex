\documentclass[12pt,fleqn]{article}
% \usepackage{fullpage}
\usepackage{enumitem}
% \setlength{\parskip}{1em}
% \setlength{\parindent}{0em}
\usepackage{amsfonts,amsmath,amsthm}
\theoremstyle{definition}
\newtheorem*{defn}{Definition}
\newtheorem*{ass}{Assumption}
\usepackage{outlines}
\newcommand{\mf}{\mathfrak}
\newcommand{\7}{\mathbb}
% \newcommand{\2}{\mathcal}
\usepackage{hyperref}





\begin{document}


\section*{Chapter 7: Kant, Incongruent Counterparts, and Absolute
  Space}

\begin{itemize}

  \item This debate flared up in the early 1970s, with work by Sklar,
    Nerlich, and Earman. 

  \item There is a forgetful functor from metric affine spaces to
    metric spaces.

  \item Even topological spaces have something like
    orientation. Consider the fundamental group of $S^1$: it's the
    integers, with both positive and negative numbers. So we have two
    orientations, going clockwise around the circle or going
    counterclockwise.

  \item Question: is orientation in the sense of ordered bases already
    present in the one-dimensional case? 

  \item In a topological space, triangles do not have orientation ---
    this in contrast to vector spaces. One wonders, is there some
    notion of an unoriented vector space? Could we quotient the two
    orientations?

\item Perhaps simplicial sets are unoriented?
 

\end{itemize}

\begin{outline}[enumerate]


  \1 Test for substantivalism: Imagine a possible world with a single
  glove. Would there be a fact of the matter about whether that glove
  is left or right handed?

  Relationalism: ``\dots contrary to Kant, for a hand standing alone
  there aren't two different actions of creative cause for God to
  choose between.'' (p 138)

  \1 Recall that substantivalists supposedly think that shifts create
  new possibilities.

  Would the substantivalist count the reflected triangle as a
  different possible world?

  \1 ``Broad imagines that the set of space points occupied by $B$ and
  the set of points occupied by $B'$ might differ in some geometrical
  properties that are not manifested in the relations among the
  occupying particles of matter \dots '' (p 140)

  HH: But space and its structures provide a common standard --- in
  the same way that space and its structures provide a standard for
  travelling on a straight line.

  \1 ``Nerlich is incorrect in holding that Kant is right if we
  interpret him as saying that the enantiomorphism of a hand depends
  upon the relation between it and the absolute container space
  considered as a unity.'' (p 142)

  \1 Fact: The group of symmetries of Euclidean space has three
  subgroups: translations, rotations, and reflections. The
  translations and rotations are continuously connected to the
  identity $I$, and their determinant is $1$. The reflections are
  \emph{not} continuously connected to the identity $I$, and their
  determinant is $-1$.

  \1 Fact: Symmetries preserve properties of, and relations between,
  shapes.

  \1 Fact: The relation of being incongruent counterparts is definable
  in terms of inner products of vectors. i.e.\ Euclidean geometry
  defines a relation ``same handed'', although it doesn't define
  predicates for ``left'' and ``right''.

  The bases \( \{u_1, \dots, u_n\} \) and \( \{v_1, \dots, v_n\} \)
  are co-oriented if and only if:

  \[ \sum_{i_1, i_2, \dots, i_n = 1}^n \varepsilon_{i_1 i_2 \dots i_n}
    \langle u_1, v_{i_1} \rangle \langle u_2, v_{i_2} \rangle \cdots
    \langle u_n, v_{i_n} \rangle > 0 ,\]

  where $\varepsilon_{i_1 i_2 \dots i_n}$ is the Levi-Civita
  symbol. (The equation on the left is just the determinant of the
  linear operator that maps $u_1,\dots ,u_n$ to $v_1,\dots ,v_n$.)

  So a triangle alone in space ``has an orientation'', it's just
  neither left nor right!

  This fact doesn't contradict what we said above about reflections
  preserving all properties and relations. If $X$ and $Y$ are same
  handed, then $M(X)$ and $M(Y)$ are same handed.

\item Fact: The binary relation of same-handedness cannot be analyzed
  into a conjunction of unary predicates.

  Recall that Leibniz believes that all relational properties are
  ultimately grounded in monadic properties.


\end{outline}

\end{document}



The argument from handedness to substantivalism was put forward by
Kant in 1768, and was revived in the 1970s by Graham Nerlich. Nerlich
is somewhat unique among contemporary philosophers in thinking that
the phenomenon of ``incongruent counterparts'' supports
substantivalism. What's more, Nerlich's argument is set in the context
of an arbitrary Riemannian manifold, rather than in the context of
Euclidean space. So I'll begin with the simpler case and then return
later to Nerlich.

Kant's argument is based on the following factual claim:

\begin{quote} (Hands) In Euclidean space (of dimension greater than 1)
  there can be two objects $X$ and $Y$ such that (i) $X$ and $Y$ have
  identical internal spatial relations, but (ii) no combination of
  translations and rotations can move $X$ into the place of
  $Y$. \end{quote}

The argument then proceeds by reductio ad absurdum: suppose that
substantivalism is false. In that case, all spatial facts are
determined by the distance relations between the parts of physical
bodies. By Hands, $X$ cannot be moved into the place of $Y$. But by
Hands again, all spatial facts about $X$ are true of $Y$ and vice
versa, so there is no fact about $Y$ that could explain why $X$ cannot
be moved into its place. Therefore substantivalism is false.

This is roughly the form of Kant's first argument from handedness to
substantivalism. But this this argument has several notable
weaknesses. First, we no longer take it for granted (as Kant did in
1768) that every fact can be explained by the intrinsic properties of
objects; and so we might not be bothered by their being no fact about
$Y$ that can explain why $X$ cannot take its place.

It seems, however, that the argument has some force even without the
talk of explaining why some transformation cannot be performed. The
point is that relationalism only has $X$ and $Y$; and by the
relationalist's lights, there is no difference between $X$ and
$Y$. But there does seem to be an interesting difference between $X$
and $Y$, and the substantivalist has an account for what that
difference is. The substantivalist can assume that $Z$ is a generic
(unoriented) right triangle with legs of lengths 1 and 2, and he also
has the background space $E^2$. But to put the generic triangle in the
background space, one has to choose between two possible orientations
for the legs of the triangle. The substantivalist can explain the
existence of incongruent counteparts by assuming that space (with its
two possible orientations) is given in advance.

Second, we were a bit imprecise with what we meant by ``all spatial
facts about $X$''. Let's try to tighten that part of the argument.

In the language of metaphysics, we would say that Kant is talking
about \emph{intrinsic} properties of $X$ and $Y$. But thank goodness,
we need not get into details about what ``intrinsic'' means. We can
just talk about formulas in the language of Euclidean geometry. For
simplicity, we'll take the background space to be the Cartesian plane
$E^2$. We can then assume that $X$ has base point at $(0,0)$ and is
generated by a pair of vectors $v_1,v_2$:
\[ X = \{ av_1+bv_2 : a\geq 0,b\geq 0,a+b\leq 1 \} .\] We then take
$Y$ to be the triangle with $v_2$ flipped to $-v_2$, that is,
\[ X = \{ av_1-bv_2 : a\geq 0,b\geq 0,a+b\leq 1 \} .\]
Let's take $v_1$
to be the vector $(0,1)$ and $v_2$ to be the vector $(2,0)$. (We need
$v_1$ and $v_2$ to have different lengths for this example to work.)

Now it's obvious that the reflection $f$ about $v_1$ maps $v_2$ to
$-v_2$, and hence $f(X)=Y$. It then follows that $X$ and $Y$ satisfy
all the same predicates in the language of Euclidean geometry. For
example, consider the claim that $X$ has three vertices
$p_0,p_1,p_2$. Since $f$ maps vertices to vertices, it follows that
$Y$ has three vertices. Similarly, the maximum distance between two
points in $X$ is $\sqrt{5}$, and since $f$ preserves distances, the
maximum distance between two points in $Y$ is also $\sqrt{5}$. For any
points $p_0,\dots ,p_n\in X$, the mirror image points
$f(p_0),\dots ,f(p_0)\in Y$ stand in exactly the same relations.

But this shows that there are certain intuitive properties of $X$ and
$Y$ that \emph{cannot} be expressed in the language of Euclidean
geometry. For example, let $p_0=(0,0)$ and let $p_2=(1,0)$ be the
bottom vertices of $X$. Then $p_2$ is to the right of $p_0$, which we
might want to formalize as $R(p_2,p_0)$. Since $f(p_0)=0$ and
$f(p_2)=-p_2$, it is not the case that $R(f(p_2),f(p_0))$. What this
means is that the relation $R$ simply cannot be defined in the
language of Euclidean geometry.

If you think about it, you wouldn't even expect to be able to define
the relation of one vertex being to the right of another. After all,
that relation wouldn't even be preserved by a rotation through 180
degrees. But the story is different with the vectors $v_1$ and $v_2$
that define those vertices. In the triangle $X$, the longer vector
$v_1$ is to the left of the shorter vector $v_2$, and this would
remain true if $X$ were translated or rotated. A translation would
change the vertex $p_0$, but not the vectors $v_1,v_2$ that generate
the vertices. A rotation would change the vectors $v_1,v_2$, but it
would not reverse their orientation. There is no combination of
translations and rotations that could map the triangle $X$ to the
triangle $Y$. 



Our first definition assumes that space is represented by an affine
space $(A,V)$, with a positive definite inner product on
$V$. (Essentially we are talking about Euclidean space, minus the
preferred origin, basis, or orientation.) This assumption seems to be
the minimum needed to make sense of the notion of enantiomorphs, where
we need both the notion of a \emph{translation} symmetry and the
notion of a \emph{reflection} symmetry. We will return later to the
question of whether it makes sense to talk about enantiomorphs in more
general spaces, e.g.\ those represented by Riemannian manifolds.

We assume that physical objects are represented by subsets of
space. For simplicity I will ignore wild subsets and typically assume
that an object is represented by a polytope, or even more simply, by
an $n$-simplex. For example, when $\dim A=2$, a $2$ simplex in $A$ is
the convex hull of three affinely independent points, viz.\ a
triangle.

\newcommand{\vc}[1]{\overrightarrow{#1}}

There is clearly no mirror symmetry when $\dim A=1$, hence there are
no enantiomorphs. But as soon as $\dim A =2$, there are
enantiomorphs. For example, let $(p_0,p_1,p_2)$ be a triangle
described by the following data:
\[ \langle \vc{p_0p_1},\vc{p_0p_2}\rangle = 0, \: \| \vc{p_0p_1} \| =
  2 ,\: \| \vc{p_0p_2}\| = 1 .\] Let $H$ be the one-dimensional
subspace of $A$ generated by $\vc{p_0p_1}$. Each point $p\in A$ is of
the form $p_0+v$ for some unique $v\in V$. The reflection about $H$
can then be described by changing $v$ to its mirror reflection on the
other side of $H$:
\[ v\mapsto v-2\langle v,n\rangle n ,\] where $n$ is a unit vector
orthogonal to $\vc{p_0p_1}$.


  




\end{document}
%%% Local Variables:
%%% mode: latex
%%% TeX-master: t
%%% End:
