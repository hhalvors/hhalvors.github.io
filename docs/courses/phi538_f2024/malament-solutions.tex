\documentclass[12pt,fleqn]{article}
\setlength{\parskip}{1em}
\setlength{\parindent}{0em}
\usepackage{tensor}
\begin{document}

Here's an alternate solution to 1.6.2.

Using the Leibniz rule and the fact that $L_\xi$ commutes with index
substitution, we have:

\[
  \begin{array}{lll} L_{\xi}(\tensor{\eta}{^a}\tensor{\eta}{^b})
    & = \eta ^aL_{\xi}(\eta ^b)+\eta ^bL_{\xi}(\eta ^a) \\
    & = \eta ^a\tensor{\delta}{^b_a}L_{\xi}(\eta ^a)+\eta ^bL_{\xi}(\eta ^a) \\
    & = \eta ^bL_{\xi}(\eta ^a)+\eta ^bL_{\xi}(\eta ^a) .\end{array}
\]

Thus, if $L_\xi (\eta ^a\eta ^b)=0$ then $\eta ^bL_\xi (\eta
^a)=0$. If $\eta$ is nonvanishing, then it can be cancelled, leaving
$L_{\xi}(\eta ^a)=0$.

One thing I would prefer to confirm: how do we know that there is a
smooth field $\lambda _a$ that is inverse to $\eta ^a$?

Question: Is there an example of a smooth field $\eta$ such that both
$L_{\xi}(\eta ^a\eta ^b)=0$ and $L_{\xi}(\eta ^a)\neq 0$? Obviously it
would have to vanish at some point.





\end{document} 
%%% Local Variables:
%%% mode: latex
%%% TeX-master: t
%%% End:
