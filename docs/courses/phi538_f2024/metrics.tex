\documentclass[12pt,fleqn]{article}
\usepackage{amsthm,amsmath,amsfonts}
\newtheorem*{prop}{Proposition}
\begin{document}

\section*{Metrics on Smooth Manifolds}

When talking about classical spacetimes, we built from the bottom up
--- gradually adding more structure. We began with a smooth manifold
$M$ which we assumed to be foliated into spacelike hypersurfaces. Then
we added a metric on those spacelike hypersurfaces. Then we added a
temporal metric. Then we added an equivalence class of derivative
operators (connections). Then we chose a preferred derivative
operator. Etc.

A \textbf{relativistic spacetime} is a smooth manifold $M$ with a metric
$g$. In fact, $g$ is a Lorentzian metric (as we'll describe below),
but most of the following claims follow for metrics more generally.

\begin{itemize}
\item A metric $g$ on $M$ is actually a tensor \emph{field}, i.e.\ a
  smooth assignment of tensors to points of $M$.
\item A metric $g$ is a $(0,2)$ tensor. So in abstract index notation,
  we would write $g_{ab}$. Speaking functionally, $g$ is a map that
  takes a pair of vectors and returns a real number.
\item We assume that a metric is symmetric, i.e.\ $g(u,v)=g(v,u)$ for
  any vectors $u,v$. So a metric is just like an \emph{inner product}
  on the tangent spaces --- except that we have not assumed that $g$
  is positive definite. In particular, $g(v,v)$ could be negative or
  zero, even when $v\neq 0$.
\item The invertibility condition. Malament writes
  $g_{ab}h^{bc}=\delta _a^c$. This is the same as saying that $g$ is
  non-degenerate. i.e.\ if $g(v,w)=0$ for all $w$, then $v=0$. [Check
  it.]
\end{itemize}

\begin{prop} If a metric $g$ on $V$ is non-degenerate, then $g$
  induces an isomorphism $v\mapsto g(v,-)$ between $V$ and its dual
  space $V^*$. \end{prop}

\begin{proof} Suppose that $g$ is non-degenerate. For $v\in V$, let
  $f(v)$ be a function from $V$ into $\mathbb{R}$ defined by:
  \[ f(v)(w) \: = \: g(v,w) .\] Since $g$ is linear in the second
  argument, $f(v)$ is a linear map. Because $g$ is linear in the first
  argument, $f$ is a linear map of $V$ into $V^*$.

  To see that $f$ is one-to-one, suppose that $f(v)=0$. In this case,
  $g(v,w)=0$ for all $w\in V$. Since $g$ is non-degenerate,
  $v=0$. Hence, $f$ is one-to-one. Since $V$ and $V^*$ have the same
  dimension, it automatically follows that $f$ is onto, hence a linear
  isomorphism.
\end{proof}

\subsection*{From metric to connection}

For every metric $g$, there is a unique connection $\nabla$ that
satisfies a certain compatibility connection. The connection is
sometimes called the \textbf{Levi-Civita connection} for $g$.

\begin{prop} The following are equivalent. 
\begin{enumerate}
\item $\nabla _Xg = 0$ for all vector fields $X$ on $M$.
\item For any smooth curve $\gamma$ and for any vector field $Y$
  defined along $\gamma$, if $\nabla _{\dot{\gamma}}Y =0$ then
  $\dot{\gamma}(g(Y,Y))=0$.
\item For any vector fields $X,Y,Z$ on $M$,
  \[ X(g(Y,Z)) = g(\nabla _XY,Z)+g(Y,\nabla _X Z) .\]
\end{enumerate}
\end{prop}

\begin{proof} The equivalence of (1) and (2) is Malament's Lemma
  1.9.1. Using the definition of $\nabla$ on tensors, and the fact
  that $Y$ is constant along $\dot{\gamma}$, we get: 
  \[ \begin{array}{ll} \nabla _{\dot{\gamma}}(g(Y,Y)) \: =\: g(Y,\nabla
      _{\dot{\gamma}}Y)+g(\nabla _{\dot{\gamma}}Y,Y)+(\nabla
      _{\dot{\gamma}}g)(Y,Y) \: = \: (\nabla _{\dot{\gamma}}g)(Y,Y).
       \end{array} \]

\end{proof}

\end{document}
%%% Local Variables:
%%% mode: latex
%%% TeX-master: t
%%% End:
