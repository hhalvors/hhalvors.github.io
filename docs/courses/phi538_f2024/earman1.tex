\documentclass[12pt]{article}
\usepackage{outlines}
\usepackage{fullpage}
\begin{document}

\subsection*{Notes on \emph{World Enough and Spacetime}, Chap 1}

\begin{outline}[enumerate]

\1 In a 1924 paper, Hans Reichenbach (early logical positivist and
interpreter of relativity theory) claims that:

\2 Newton was a great physicist but a terrible philosopher.

\2 The idea of absolute spacetime is a throwback to unscientific
metaphysics.

\2 Einstein's theory of relativity is a confirmation of the
relationalist views of Leibniz and Huyghens.


\1 Senses of absoluteness

\2 Spacetime has intrinsic structure.

\2 Among these structures are absolute simultaneity and absolute
duration (time intervals).

\2 There is an absolute reference frame.

\2 The structure of space is immutable.

\2 Space is a substratum that underlies physical events and processes.


\1 Relationalism

\begin{description}
\item[R1] All motion is the relative motion of bodies.
\item[R2] Spatiotemporal relations among bodies are direct (perhaps
  ``fundamental'').
\item[R3] No irreducible, monadic spatiotemporal proeprties appear in
  a correct analysis of the spatiotemporal idiom. \end{description}



\1 Earman repeatedly mentions that he thinks there needs to be a third
way besides traditional substantivalism and relationalism.

\2 Even Newton says that space is \emph{sui generis}.

\2 Kant proposed a sort of third way.

 


\1 Newton's Scholium

\2 How can we detect absolute motion?

\3 Forces

An object can change relative motion without any direct application of
force.

An object can change absolute motion only if force is applied directly
to it.


\3 Relative motion as a sufficient condition for (some or other)
absolute motion

If two things are moving relative to each other, then at least one of
the two is moving absolutely. (If two objects are at absolute rest,
then they are at rest relative to each other.)



\2 Bucket experiment


\2 Rotating globes experiment

\3 Tension in the connecting cord indicates that the globes are
rotating.

\3 We can determine the axis of rotation by applying forces to the
globe faces and seeing which reduce the tension in the cord.


\2 Anti-relationalism about quantities (p 25, bottom)

\3 Don't confuse a \emph{sensible measure} with the underlying
\emph{measured quantity}.



\1 Further questions

\2 For Newton, the word ``absolute'' does a lot of work. But it's a
word that one doesn't hear much in analytic philosophy. Can Newton's
``absolute'' be paraphrased into terms that analytic metaphysicians
use?


\end{outline}


\end{document}

%%% Local Variables:
%%% mode: latex
%%% TeX-master: t
%%% End:
