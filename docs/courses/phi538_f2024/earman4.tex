\documentclass[12pt]{article}
\usepackage{fullpage}
\usepackage{enumitem}
\setlength{\parskip}{1em}
\setlength{\parindent}{0em}
\usepackage{amsfonts,amsmath,amsthm}
\theoremstyle{definition}
\newtheorem*{defn}{Definition}
\newtheorem*{ass}{Assumption}
\usepackage{outlines}
\newcommand{\mf}{\mathfrak}
\begin{document}

\section*{Chapter 5: Relational Theories of Motion}

Let's start with a short continuation of last week's discussion (about
whether an individual particle has a velocity vector). This will also
let us start to unpack what John Earman means by an \textbf{absolute
  quantity}.

\begin{quote} Does a lonely particle have a velocity
  vector? \end{quote}

If we make the same assumptions that Earman does, then we must answer
yes. In particular, we assume that spacetime is a smooth manifold $M$,
and that the lonely particle is represented by a smooth curve $\gamma$
in $M$. But every smooth curve has a velocity vector at each
point. Therefore, this particle always has a velocity vector. If we
want to build a theory where the particle does \emph{not} have a
velocity vector, then we need to drop one of Earman's assumptions.

But now natural language ambiguity threatens. When we say ``doesn't
have a velocity,'' we might either mean ``velocity is simply not a
real quantity'' or ``velocity is zero''. The answer above is directed
toward the first disambiguation: velocity is a real quantity, in the
sense that there is a mathematical thing that represents it. However,
here is the crucial fact about Galilean (i.e.\ neo-Newtonian)
spacetime:

\begin{quote} A particle's velocity vector can always be transformed
  to zero. \end{quote}

In particular, in its own frame of reference, the particle's velocity
vector is the zero vector.



\begin{outline}[enumerate]

  

\1 Rotation arguments for absolute spacetime played a central role for
Newton.

\2 Bucket

\2 Two spheres

\1 Intuition pumps

\2 Does the existence of movement require explanation?

\2 Could one thing alone in empty space be moving?

\2 Does the existence of rotation require explanation?

\2 Could two things alone in empty space be rotating? 


\1 Newton's central intuition: No true motion where no force is
applied

So if ``motion'' of $X$ can be created by application of force to
(distant) $Y$, then it's not true motion.

\1 What are the absolutist (substantivalist) and relationalist about
space arguing about?

That can be taken as a historical question (Newton and Leibniz); but
we are more concerned with whether there is some live question for us.

\2 Old idea: This is about whether spacetime exists, or perhaps
counterfactually, whether spacetime would still be there if we took
all the matter away.

\2 Jill North: It's an argument about what's \emph{metaphysically
  fundamental}.

\2 How much structure to put into a spacetime theory?

\2 Where in the theory to put the structure --- i.e.\ as background
geometry or contingent matter distribution?

\2 Or does ``space is absolute'' mean something other than ``the best
theory has such and such in its ideology''?

\2 Or does ``space exists'' or ``space is fundamental'' mean something
other than ``the best theory has such and such in its ideology''?

\1 What is a relational account of motion (or rotation)?

Proposal: The only ``fundamental'' quantities are relative distances,
and anything ``real'' has to be defined from those. E.g. motion (or
rotation) must be defined in terms of change of relative distince.

As we'll see later, this is why Kant concluded (in his pre-critical
stage) that space is absolute: he couldn't define the distinction
between left and right hands in terms of relative distances.

\1 (p 64) Earman: Absolute acceleration does not demand the assumption
of absolute space


\1 Relationist responses

\2 Berkeley and Leibniz thought that there are ``true motions'', but
that they aren't motion relative to absolute space. The criterion of
true motion is inside individual bodies.

\2 Leibniz claims that the bucket thought experiment (and other such)
falsely presuppose that there are genuinely rigid bodies.

\2 Mach claims that the intuition that the water will bend is due to
tacit assumption of motion relative to distant stars.



  


\end{outline}

\end{document}
%%% Local Variables:
%%% mode: latex
%%% TeX-master: t
%%% End:
