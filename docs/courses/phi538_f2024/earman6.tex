\documentclass[11pt,fleqn]{article}
\usepackage{fullpage}
\usepackage{enumitem}
\setlength{\parskip}{1em}
\setlength{\parindent}{0em}
\usepackage{amsfonts,amsmath,amsthm}
\theoremstyle{definition}
\newtheorem*{defn}{Definition}
\newtheorem*{ass}{Assumption}
\usepackage{outlines}
\newcommand{\mf}{\mathfrak}
\newcommand{\7}{\mathbb}
% \newcommand{\2}{\mathcal}
\usepackage{hyperref}





\begin{document}


\section*{Chapter 6: Substantivalism -- Newton versus Leibniz}

\begin{description}
\item[(R1)] All motion is the relative motion of bodies, and
  consequently, spacetime does not have structures that support
  absolute quantities of motion.
\item[(R2)] Spatiotemporal relations among bodies and events are
  direct; i.e.\ they are not derived from relations among a substratum
  of space-time points that underlie events.
\end{description}


\subsection*{Stage setting}

``\dots the best judgment emerging from the evidence marshalled in
chapters 4 and 5 is that (R1) is in fact false in both the classical
and relativistic settings.'' (p 111)

Chapter 3 supposedly shows (R1) and ``determinism is possible''
implies (R2)

``\dots if we want to allow for the possibility that particle motions
are deterministic and if we want to make a substantivalist
interpretation of the space-time manifold (not R2), it follows that
the structure of space-time must be at least as rich as that of
neo-Newtonian spacetime (not R1).'' (p 55)

\textbf{Definition:} A theory is \emph{deterministic} (in the sense of
David Lewis) if for any nomologically possible worlds $W$ and $W'$, if
$W$ and $W'$ agree on an initial segment, then $W=W'$.

Earman will argue that not-R2 and R1 imply not-determinism.

\textbf{Fact:} Let $M$ be a classical spacetime with less structure
than Galilean spacetime. There is a symmetry $\varphi :M\to M$ that is
the identity for $t\leq 0$, but not the identity for $t>0$. (HH
agrees.) Thus, R1 implies the existence of such a symmetry.

``\dots we can thus produce two dynamically possible models where the
world lines of the particles coincide for all $t\leq 0$ but diverge
for $t>0$, a violation of determinism.''

HH: It is not clear how not-R2 is supposed to function in this
argument.

Earman produces a second argument. R1 was taken to mean ``less
structure than Galilean spacetime''. But even if we allow Galilean
spacetime, we still might get failure of determinism.

Let R1' = not full Newtonian spacetime. Hence R1 implies R1' but not
vice versa.

Now Earman argues that not-R2 and R1' implies not-determinism. The key
fact now is that there is a Galilean transformation that is the
identity at $t=0$ but not the identity elsewhere.

\newpage 

\subsection*{Leibniz's argument(s)}

Leibniz argues for (R2), in particular that absolute space is an idle
wheel.

\textbf{Principle of Sufficient Reason:} (1) God would have a good
reason to create one world instead of another. (2) Nothing happens
without a cause.

\textbf{Principle of the Identity of Indiscernibles:} If $a\neq b$,
then there is some property that $a$ has and that $b$ lacks, or vice
versa.

Leibniz's shift argument: If space is a substance, all matter can be
moved $n$ meters to the east of its current location, and the result
is a new possible world. (see p 118)

The existence of these possible worlds $\{ W_r:r\in\mathbb{R}\}$ is a
problem both for PSR (which should God create?), and for PIdIn (the
substantivalist declares them non-equal when they agree on all
properties).

Earman: For Leibniz's invocation of PIdIn to work, it would need to be
supplemented by an implausible verificationist assumption. For
example: if two things agree on observable properties, then they
agree. (``verifiability version of PIdIn'' p 120)



\end{document}




\subsection*{Does a lonely particle have a velocity?}

Let's start where we left off last week --- discussing whether an
individual particle has a velocity. This will let us start to unpack
what John Earman means by an absolute quantity.

Consider a possible world with just a single particle all alone. Does
that particle have a velocity? I think different people will have
different intuitions about this. But if we make the same background
assumptions that Earman does, then we must answer yes. In particular,
if spacetime is represented by a smooth manifold $M$, and a particle
by a smooth curve $\gamma$ in $M$, then this particle has a definite
velocity vector at each point along its trajectory. This is true in
any smooth manifold, without the need to add any additional
structure. Granted, if the manifold $M$ cannot be sliced up, at least
locally, into space+time, then we wouldn't know whether to call a
smooth curve a ``trajectory''. But we don't need a connection, much
less a metric, to have such a local slicing. The conclusion, then, is
that having the structure of a smooth manifold is in itself sufficient
for each physical object (represented by a smooth curve in that
manifold) having an \emph{absolute} velocity at each moment of its
life.\footnote{What I'm saying here can be thought of as the opposite
  of the lesson that some people draw from Galilean relativity, i.e.\
  that nothing is \emph{really} moving. I'm suggesting that everything
  is moving, insofar as ``moving'' is cashed out in terms of
  assignment of a non-zero velocity vector.}

It should be clear, however, that there are multiple senses of
``absolute'', and the sense in which velocity is absolute cannot be
what people like John Earman mean with his use of the term. Earman
agrees with the folklore view that velocity is \emph{not} absolute in
Galilean spacetime. So what is this other sense of ``absolute''?

The sense of absolute that I was (perhaps incautiously) using was
being definable without reference to other bodies. ... I mean simply
that the particle's trajectory $\gamma$ defines, without reference to
any other physical bodies, a vector $v(s)\in T_{\gamma (s)}$. (I will
subsequently set $\gamma (s)=p$ and $v(s)=v$ for notational
simplicity.) Or in other words, having the velocity vector $v\in T_p$
is a monadic property of the particle with this trajectory.

What then is the standard sense of ``absolute''? It's not so easy to
think immediately of a fully rigorous mathematical definition. One
wants to to equate ``absolute'' with ``invariant'', and then just turn
the crank: a mathematical thing $X$ represents something absolute just
in case $f(X)=X$ for every symmetry $f$. But there are some subtleties
here. First, symmetries don't act directly on the things we are
interested in (such as an object's acceleration vector); they act on
the points of spacetime, and then we have to decide how to induce
actions on other kinds of mathematical objects. Another serious issue
for any theory that uses the apparatus of smooth manifolds is that a
symmetry $\varphi :M\to M$ maps elements of the fiber over $p\in M$ to
elements of the fiber over $\varphi (p)$ (or the fiber over
$\varphi ^{-1}(p)$, depending on how we set up the definitions). But
elements in the one fiber are not directly comparable to elements in
the other fiber, i.e.\ there is no cross-fiber identity relation. So
how in the world are we supposed to make sense of $f(X)=X$?\footnote{I
  will leave aside for now the common --- and surely not totally
  incorrect --- idea that symmetries act on equations, and that terms
  of equations can be invariant under such transformations. For
  example, it's typically said that Galilean transformations leave the
  form of the law $F=ma$ intact. People of a mathematical bent would
  like to know the rules of this transformation game. What kind of
  things are the objects being acted upon? Are they strings of symbols
  with some kind of formation rules? And if so, under what conditions
  do we say that two such strings are identical?}

The situation is not quite as bad as that makes it sound: we do have
good intuitions about which quantities are absolute in specific
cases. For example:
\begin{quote}
  (*) In Galilean spacetime $M$, the acceleration of a body is
  absolute, while its velocity is not. \end{quote} What exactly does
(*) mean?  Let's resist cashing (*) out in terms of reference frames,
because the notion of a reference frame is no more clear than the
notion of an absolute quantity. Let's try, instead, to find a
mathematical fact that corresponds to (*). (As Carnap would say, we're
looking for an \emph{explication} of (*).) Here we are helped by the
fact that Galilean spacetime can be represented by an affine space
$M\times V\to M$ with a temporal metric $t:V\to\7R$. Here $V$ is the
four-dimensional tangent space, now treated as the same space for all
different points in $M$. In this case, a fixed base point $a\in M$
defines a bijection $u\mapsto a+u$ from $V$ to $M$. A Galilean boost
$\varphi$ based at $a$ and by velocity $v$ is defined by 
\[ \varphi (a+u) \:=\: (a+u)+t(u)v \:=\: a+(u+t(u)v). \] In other
words, $\varphi (a+u)=a+\Phi (u)$, where $\Phi :V\to V$ is the linear
map $u\mapsto u+t(u)v$.

The map $\Phi$ will obviously change the velocity vector

In summary, when the manifold $M$ has a flat connection, then there is
a preferred standard of comparison between vectors in a tangent space
$T_p$ and vectors in a tangent space $T_{\varphi (p)}$. This standard
then allows for a notion of an invariant vector field, either on the
entire space $M$, or on some smooth curve $\gamma$ in $M$.

\begin{defn} Suppose that $M$ is a smooth manifold, and let $\nabla$
  be a flat connection on $M$. For a diffeomorphism $f:M\to M$ and
  vector field $X$ on $M$, we that $X$ is \textbf{invariant} under $f$
  just in case
  \[ \nabla X \: = \: \nabla (f_* X) .\] \end{defn}

Unfortunately this definition is not immediately useful for the case
of interest: the acceleration field $X$ along a smooth curve $\gamma$
in $M$. The problem is that $X$ is not defined on all of $M$, but just
on the image of $\gamma$ in $M$. 


[[TO DO: Galilean spacetime as affine space. Galilean boost as affine
space map.]]

for example, the classic case: in Galilean spacetime $M$, acceleration
is absolute, but velocity is not. Here's one easy way to think of
that: let $\gamma$ be a smooth timelike curve in $M$, and let
$\mathsf{In}$ be the predicate of curves that means ``is inertial'',
which is tantamount to saying that the curve is a
\kw[geodesic]{geodesic} according to the derivative operator $\nabla$
that is part of the definition of $M$. Of course $\nabla$ is invariant
under Galilean transformations in the precise sense that
\[ \varphi_* (\nabla_X Y) = \nabla_{\varphi_* X} (\varphi_* Y) ,
\]
for any Galilean transformation $\varphi :M\to M$, and for any vector
fields $X$ and $Y$.  Thus, Galilean transformations map geodesics to
geodesics, and $\mathsf{In}$ holds of a curve $\gamma$ iff
$\mathsf{In}$ holds of the curve $\varphi \circ \gamma$, where
$\varphi$ is an arbitrary Galilean transformation.

Let $\gamma$ be a timelike geodesic with a timelike tangent vector
$T^a$ satisfying $t_c T^c = 1$, where $t_c$ is the temporal
1-form. Let $X^a$ be a spacelike vector at some point on $\gamma$,
such that $t_c X^c = 0$, and let \(X^a(\tau)\) be the parallel
transport of \(X^a\) along \(\gamma(\tau)\).

Then, the Galilean boost $\varphi: M \to M$ is defined by:
\[ \varphi(p) \:= \: p + t(p)\vec{v}, \] where: $t(p)$ is the time
coordinate of the point $p$.




I am fully aware that it seems meaningless to assert that the velocity
of a particle is $v$, while not saying anything else about the vector
$v$. The problem is that vectors are abstract mathematical objects in
a particularly egregious sense. Let me explain

Assuming that the manifold $M$ is four-dimensional (as we expect
spacetime to be), a tangent space $T_p$ is a four-dimensional vector
space over $\7R$. It's fairly easy to prove that for any two
\emph{non-zero} vectors $v,w$ in a vector space $V$, there is an
isomorphism $L:V\to V$ (i.e.\ an invertible linear mapping) such that
$Lv=w$. (Construct a basis $\{ v,v_1,v_2,\dots \}$ for $V$, and extend
$L$ by setting $L(v_i)=v_i$ for all $i$.) Using the standard
mathematical account of definable properties of mathematical
objects,\footnote{The standard result in model theory is that if
  $h:M\to N$ is an elementary embedding, then $M\models \phi (a)$ iff
  $N\models \phi (h(a))$. Now for any $a\in M$, let $\mathrm{tp}_M(a)$
  be the family of all predicates $\phi$ such that
  $M\models \phi (a)$. Then $\mathrm{tp}_M(a)=\mathrm{tp}_M(h(a))$,
  i.e.\ $a$ and $h(a)$ have all the same definable properties.} there
is then no vector-space-definable property that $v$ has and that $w$
lacks. In other words, any two non-zero vectors in $V$ are
indistinguishable by the lights of vector space theory. (The zero
vector $0$ is the only distinguishable element of $V$ in the sense
that $v\in V$ is preserved under every automorphism of $V$ iff
$v=0$. However, the tangent vector of a smooth curve can never be
$0$.)

It's tempting, then, to think that the assignment of a velocity vector
$v\in T_p$ is meaningless, i.e.\ it's just a mathematical tag that
doesn't mean anything, because $v$ is no different than any other
non-zero vector. There is \emph{some} truth in that claim, viz.\ we
could always change representational conventions and use a different
vector to represent velocity. But this freedom of representational
conventions does not itself mean that the assignment of a specific
vector is meaningless. First, the tangent space $T_p$ at a point
$p\in M$ is not just an abstract vector space: it is the concrete
vector space of \kw[derivations]{derivation} of scalar fields on
neighborhoods around $p$. (As usual, let $\2S (p)$ denote the family
of neighborhoods of $p$.) Thus, the tangent vector of a curve $\gamma$
is uniquely defined by its action on to these scalar fields: it
specifies the rate of change of each scalar field along the trajectory
$\gamma$. A different vector $w\in T_p$ would give different answers
to questions of the form ``how fast is $f$ changing along the
trajectory $\gamma$''.\footnote{So the velocity of an object is not
  defined relative to some fixed standard, but relative to any
  standard of scalar values along its trajectory. But another way of
  putting this is that the velocity of a particle is the standard by
  which change of scalar values can be judged.}

It might be tempting to think that we smuggled in spacetime
substantivalism by assuming that there is a manifold in which the
particle's motion is represented. The temptation here is to think that
the particle has a velocity vector precisely because its position is
changing relative to the background spacetime (represented by the
manifold). But that doesn't make much sense. A tangent vector to a
curve is not a measure, in any sort of way, of how that curve is
changing relative to some fixed standard of reference inside the
manifold, or some structure on the manifold. There are not ``grooves''
in the manifold such that the rate of change of $\gamma$ can be
defined in terms of how many grooves it crosses per unit of time. In
fact, a smooth manifold (without further structure) is like a
``Heraclitus world'' in the sense that nothing remains constant from
point to point. This lack of a fixed standard of reference is
precisely what makes a smooth manifold different from the familiar
space $\7R ^4$ of quadruples of real numbers (or what is the same: by
fixing a preferred global coordinate chart on a manifold).


TODO: There is another sense in (connection doesn't compare different
observers' velocities) ...

There may be some \emph{other} sense of ``absolute'' such that the
velocity of a particle is not absolute; but let's take our time to
explore the options. One immediate thought here is that an abstract
velocity vector $v\in T_p$ is a very different kind of thing than a
numerical value. After all, it's the variability of numerical values
that drives the intuition that velocity is \emph{not} absolute in,
say, Galilean spacetime. For example, the proverbial man on the ship
has zero velocity relative to the ship, while he has non-zero velocity
relative to the shore.



But in order to explain this other sense, we will need to look at
quantities that are normally thought to be absolute, e.g.\
acceleration in Galilean spacetime.




\newpage
\printglossaries


  




\end{document}
%%% Local Variables:
%%% mode: latex
%%% TeX-master: t
%%% End:
