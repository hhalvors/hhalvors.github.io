\documentclass[12pt,fleqn]{article}
\usepackage{amsmath}
\usepackage{amsthm}
\usepackage{amsfonts}
\usepackage{amssymb}
\usepackage{pdflscape}

\usepackage{graphicx}

\usepackage{tensor}

% Theorem and Definition environments
\newtheorem{theorem}{Theorem}

\theoremstyle{definition}
\newtheorem*{definition}{Definition}
\newtheorem*{note}{Note}


\title{Concepts in Differential Geometry}
\author{Hans Halvorson}
\date{\today}

\begin{document}
\maketitle

\section*{Preliminaries}

In this document, I present definitions of fundamental concepts in
differential geometry: connections, parallel transport, and Riemann
curvature. I then explore the relationships between these concepts,
which are fundamental to understanding the geometry of smooth
manifolds.

\begin{definition} If $M$ is a smooth manifold, we let $TM$ denote its
  tangent bundle, and we let $\Gamma (TM)$ denote the set of sections
  of $TM$, i.e.\ smooth vector fields on $M$. \end{definition}

\section*{Connection on a Smooth Manifold}

\begin{definition}[Connection] Let $M$ be a smooth manifold. A
  \emph{connection}, or \emph{covariant derivative operator}, on $M$
  is a bilinear map
  \[ \nabla: \Gamma(TM) \times \Gamma(TM) \to \Gamma(TM)
\]
such that for any vector fields \( X, Y, Z \in \Gamma(TM) \) and any
smooth function $f\in C^\infty(M)$, the following conditions hold:
\begin{itemize}
    \item \textbf{Linearity in the first argument:}
    \[
    \nabla_{X+Y} Z = \nabla_X Z + \nabla_Y Z,
    \]
    \item \textbf{Leibniz rule in the first argument:}
    \[
    \nabla_{fX} Y = f \nabla_X Y,
    \]
  \item \textbf{Leibniz rule in the second argument:}
    \[ \nabla_X (fY) = (Xf)Y + f \nabla_X Y.  \]
\end{itemize}
\end{definition}

\section*{Parallel Transport}

\begin{definition}[Parallel Transport]
  Let \( M \) be a smooth manifold with a connection \( \nabla \), and
  let \( \gamma: [0,1] \to M \) be a smooth path from
  \( p = \gamma(0) \) to \( q = \gamma(1) \). The \emph{parallel
    transport map} along \( \gamma \), denoted by
  \[ \tau_{\gamma}: T_pM \to T_qM, \] is a linear isomorphism defined
  as follows: for any vector \( v \in T_pM \),
  \( \tau_{\gamma}(v) \in T_qM \) is the unique vector such that the
  vector field \( V(t) \) along \( \gamma(t) \), satisfying the
  initial condition \( V(0) = v \), is parallel with respect to the
  connection \( \nabla \), i.e.,
  \[ \nabla_{\dot{\gamma}(t)} V(t) = 0 \quad \text{for all } t \in
    [0,1]. \]
\end{definition}

\section*{Riemann Curvature Tensor}

In abstract index notation, the Riemann curvature tensor $R$ is
typically defined by setting its value on an arbitrary vector $\xi$:
\[ \tensor{R}{^{a}_{b c d}} \tensor{\xi}{^b} \: = \: -2\nabla
  _{[c}\nabla _{d]}\xi ^a . \] If we look at the index structure,
there is another way to think of $R$, i.e.\ as a map that takes two
vectors $\lambda ^c,\rho ^d$ and returns a tensor
$\tensor{R}{^a_{b c d}}\lambda ^c\rho ^d$. The resulting tensor has
form $\tensor{\theta}{^a_b}$, which can be thought of as a map from a
vector $\xi ^b$ to a vector $\tensor{\theta}{^a_b}\xi ^b$. Thinking of
$R$ this way leads to an intuitive picture: given two directions
$\lambda ,\rho$ out of $p$, $\tensor{R}{^a_{b c d}}\lambda ^c,\rho ^d$
measures the deviation from identity for transport around an
infinitesimal parallelogram.

Let's rewrite the definition without abstract indices: $R$ is a map
that takes two input vectors $X,Y\in T_p$ and returns a linear map
$R(X,Y):T_p\to T_p$. To be more precise, $R$ maps vector fields to a
tensor field.

\begin{definition}[Riemann Curvature Tensor]
  Given a smooth manifold $M$ with a connection $\nabla$, the
  \emph{Riemann curvature tensor} is the map
  \[ R: \Gamma(TM) \times \Gamma(TM) \times \Gamma(TM) \to
    \Gamma(TM), \] defined by
  \[ R(X,Y) \: = \: \nabla_X \nabla_Y - \nabla_Y \nabla_X -
    \nabla_{[X,Y]} , \] for vector fields $X, Y\in \Gamma(TM)$.
\end{definition}

It is straightforward, if a bit tedious, to check that $R$ is a
tensor field. 

\newpage

  \rotatebox{90}{

\begin{tabular}{l|l|l}
  structure & definable & not definable  \\ \hline \hline
  manifold + Lorentzian metric & \\  \hline
  manifold + connection & parallel transport, geodesics, curvature & \\ \hline
  manifold & smooth curves, tangent vectors & 
\end{tabular}

}


\newpage


\section*{Relationships Between Connection, Parallel Transport, and
  Curvature}


\subsection*{From Connection to Parallel Transport}

A covariant derivative operator \( \nabla \) naturally defines
parallel transport along any smooth path \( \gamma \). Given a vector
field \( V(t) \) along \( \gamma(t) \), the condition
\( \nabla_{\dot{\gamma}(t)} V(t) = 0 \) (i.e., that \( V(t) \) is
parallel along \( \gamma \)) uniquely determines how to transport any
vector \( v \in T_pM \) to \( T_qM \). This process is called
\emph{parallel transport}.

In this sense, the connection defines how vectors (and more generally,
tensors) are propagated along curves in a manner consistent with the
geometry defined by \( \nabla \).

\begin{definition}
  Let $M$ be a smooth manifold, and let $\nabla$ be a connection on
  $M$. The \emph{holonomy group} \( \text{Hol}(p, \nabla) \) is the
  set of all linear maps \( \tau_{\gamma}: T_p M \to T_p M \), called
  \emph{holonomies}, obtained by parallel transporting vectors along
  smooth closed loops \( \gamma \) based at \( p \), with respect to
  the connection \( \nabla \). That is,
\[
\text{Hol}(p, \nabla) = \left\{ \tau_{\gamma} \in \text{GL}(T_p M) \mid \gamma: [0,1] \to M \text{ is a smooth loop with } \gamma(0) = \gamma(1) = p \right\}.
\]
\end{definition}

\subsection*{From Parallel Transport to Connection}

Conversely, the notion of parallel transport can also define a
derivative operator. If we know how vectors are transported along all
possible curves in a smooth manifold, we can recover the
connection. Specifically, the connection $\nabla_X Y$ at a point can
be defined by considering the infinitesimal limit of parallel
transport along the flow generated by \( X \). Thus, parallel
transport and covariant derivatives are two perspectives on the same
underlying geometric structure.

\subsection*{From Connection to Curvature}

The Riemann curvature tensor \( R \) is derived directly from the
covariant derivative operator \( \nabla \). It measures the
noncommutativity of the covariant derivative, that is, how much the
result of taking two successive derivatives depends on the order in
which they are taken. If the connection \( \nabla \) were to commute
in all directions, the curvature would vanish, meaning the manifold is
\emph{flat}.

\subsection*{Key Relations Between Curvature and Parallel Transport}

One of the most important relations in differential geometry is the
equivalence between curvature and the path-dependence of parallel
transport.

\begin{theorem}
  Parallel transport along a curve \( \gamma \) is independent of the
  path taken between two points \( p \) and \( q \) if and only if the
  curvature of the connection is zero along \( \gamma \).
\end{theorem}

\begin{proof}[Sketch of Proof]
The independence of parallel transport from the path is equivalent to the holonomy group of the connection being trivial, which occurs precisely when the curvature tensor \( R \) vanishes. If \( R(X,Y)Z = 0 \) for all vector fields \( X, Y, Z \), then the connection is flat, meaning that parallel transport around any closed loop returns the vector unchanged, and transport between any two points is independent of the path.
\end{proof}

\subsection*{Flatness and Commutativity of the Covariant Derivative}

The curvature tensor \( R \) vanishes if and only if the covariant derivative operator \( \nabla \) commutes, i.e.,
\[
\nabla_X \nabla_Y Z = \nabla_Y \nabla_X Z
\]
for all vector fields \( X, Y, Z \in \Gamma(TM) \). This commutativity implies that the manifold is locally flat, meaning that in small neighborhoods, the geometry resembles that of Euclidean space.

\end{document}

%%% Local Variables:
%%% mode: latex
%%% TeX-master: t
%%% End:
