\documentclass[12pt]{article}
\usepackage{fullpage}
\usepackage{enumitem}
\setlength{\parskip}{1em}
\setlength{\parindent}{0em}
\usepackage{amsfonts,amsmath}
\begin{document}

Chapter 2 of \emph{World Enough and Spacetime} is the technically most
demanding chapter of the book. Earman makes lots of (mostly correct)
claims that take several lines of math to prove --- rarely mentioning
how one could reconstruct the proof. He also uses a notation that is
slightly ambiguous about whether the indices are abstract (as in
Malament's $a,b,c,\dots $) or whether they are components of the
tensor relative to a chart.

Nonetheless, the upshot of the chapter is clear: there is a stack of
increasing spacetime structures.

\bigskip \noindent \begin{tabular}{l|ll} spacetime & new structure &
  new concept \\ \hline
                     Aristotelean spacetime & worldline & absolute position \\
                     Newtonian spacetime & reference frame & absolute velocity \\
                     Galilean spacetime & affine structure & absolute acceleration \\
                     Maxwellian spacetime & affine structures & rotation \\
                     Leibnizian spacetime & temporal metric & relative acceleration \\
                     Machian spacetime & simultaneity, Euclidean
                                         metric on space \end{tabular}

\subsection*{Machian spacetime}

\begin{itemize}
\item Assumption: $M$ is a smooth manifold that is diffeomorphic with
  $\mathbb{R}^4$. i.e., there is a smooth bijection
  $\varphi :M\to \mathbb{R}^4$. (This assumption saves one from some
  nuances about global structure.)
\item Motivating idea: All there is are metric relations at each
  moment. But we assume (contra Hume!) that we can re-identify
  physical objects at different times. Mathematically, this means that
  we permit ourselves to write $\gamma :\mathbb{R}\to M$ to represent
  the trajectory of a material object.
\item Assumption: There is a metric $g^{ab}$ of signature $(+ + + 0)$
  on $M$. Hence, locally, in each tangent space $T_p$, one can
  distinguish three spacelike dimensions and one timelike
  dimension. But since there is no \textbf{connection} on $M$, there
  is no sense to the question of whether the timelike direction in
  $T_p$ is the same as the timelike direction in $T_q$. (Since
  $g^{ab}$ is assumed to be smooth, there is a sense in which the
  timelike direction changes continuously. But that does \emph{not}
  mean that these timelike directions can be stitched together
  consistently in such a way that we can define a global notion of
  time. The \textbf{kernel distribution} does not necessarily
  correspond to a smooth vector field.)
\item Assumption: There is a family $\mathbb{T}$ of time functions
  $t:M\to\mathbb{R}$, but no particular member of $\mathbb{T}$ is
  privileged with regard to duration between two events. For any
  $t_1,t_2\in \mathbb{T}$, there is a smooth bijection
  $f:\mathbb{R}\to\mathbb{R}$ such that $t_2=f\circ t_1$ and
  $f'=df/dt>0$. (The latter condition is equivalent to $f$ being
  continuous and order-preserving.)
  \begin{itemize}
  \item ``$x$ and $y$ are moving relative to each other'' is
    definable, i.e.\ their distance is changing.
  \item ``The velocity of $x$ relative to $y$'' is not definable.  
  \item ``$x$ and $y$ are accelerating relative to each other'' is not
    definable.
  \end{itemize}  
\item Fact: If two smooth functions $t_1,t_2:M\to\mathbb{R}$ provide
  the same foliation, then they are related as $t_2=f\circ t_1$, for a
  continuous, order-preserving function $f$.
\item Assumption: $g^{ab}t_a=0$, meaning that $g^{ab}t_av_b=0$ for any
  covector $v_b$. (Question: Is there a sense in which $g^{ab}t_a$ is
  the \emph{projection} of $t_a$ onto space?)
\end{itemize}

\subsection*{Leibnizian spacetime}

\begin{itemize}
\item Earman does something strange here. Instead of selecting a
  particular time function $t\in\mathbb{T}$, he defines a temporal
  metric $h_{ij}$. He then requires that $g^{ij}h_{ij}=0$. Of course,
  $h_{ij}$ does defines temporal distances.
\item Question: Couldn't we just pick a particular time function
  $t\in\mathbb{T}$ and then take $h_{ab}=\partial _at\partial _bt$?
  Conjecture: That wouldn't reduce generality, because a metric that
  is non-degenerate on only one dimension is actually a product of
  one-forms.
\item Fact: for two curves $\gamma _1,\gamma _2:\mathbb{R}\to M$,
  instantaneous relative velocity is definable. Reparameterize $\gamma
  _1$ and $\gamma _2$ so that they are functions of the global time
  coordinate. Each each $t$, $v(t)=\gamma _1(t)-\gamma _2(t)$ is a
  vector in $\mathbb{R}^3$.
\item Imagine three curves $\gamma _1,\gamma _2,\gamma _3$ that form a
  triangle with vertices of constant distance. There is no meaning to
  the claim that this triangle is rotating or not. 
\end{itemize}

\subsection*{Maxwellian spacetime}

\begin{itemize}
\item The construction here is complicated because the goal is to add
  a ``standard of rotation'' without adding a standard of absolute
  acceleration. Earman's idea is to fix a particular flat affine
  connection $\nabla$ that is compatible with the spatial and temporal
  metrics. The specification of such a connection is tantamount to the
  specification of a notion of ``straight line'' in $M$. At this
  stage, absolute acceleration would be definable, and we don't want
  that. We just want to know if an alternative definition $\nabla '$
  of ``straight line'' would agree with $\nabla$ about which extended
  things are rotating.
\item Take a spacelike vector field $f^a$ that is constant, at each
  time, relative to the connection $\nabla$. I believe this means that
  $f^a$ corresponds to shifting each spatial hypersurface --- i.e.\ we
  can bevel the deck in a smooth (but not necessarily linear) way. 
\end{itemize}  
  

\subsection*{Galilean spacetime}

A covariant derivative operator $\nabla$ takes a tensor $T$ and
produces a new tensor $\nabla T$ that has one more lower index. The
intuitive interpretation is that if $\xi ^a$ is a directional vector,
then $\xi ^a\nabla _aT$ is the tensor representing the rate of change
of $T$ in the direction $\xi$.

How is it possible that (absolute) acceleration is definable while
(absolute) velocity is not? Starting with the simple, binary case of
``moving versus not moving'' and ``accelerating versus not
accelerating'': an affine connection (covariant derivative operator)
$\nabla$ defines a predicate on timlike curves --- ``$\gamma$ is a
geodesic''. The more precise definition is that if $\vec{\gamma} |_p$
is parallel transported from $p$ to $q$, then the result is
$\vec{\gamma } |_q$. In short, an object (represented by a timeline
curve $\gamma$) is accelerating at a point $p\in M$ just in case
$\xi ^a \nabla _a\xi ^b\neq 0$.

% Or what is equivalent ...

\subsection*{Additional resources}

If you want to learn more about these different spacetimes, it can be
a challenging because typical physics books are focused on the current
theory, viz. general relativity, which is rather different than these
``classical'' spacetimes.

\begin{itemize}
\item Weatherall, Jim (2021). Classical spacetime structure. In
  (Eds.), The Routledge Companion to Philosophy of Physics
  (pp. 33–45). Routledge.
\item Lee, \emph{Introduction to Smooth Manifolds}, chapter 19 is
  about foliations.
\item Claude Godbillon, \emph{Feuilletages: Études Géométriques}
  (1971) is the classic about foliations (although not available in
  English).
\item ``Foliations and Geometric Structures'' by Barbot, Béguin, and
  Labourie (in Géométrie différentielle)
\end{itemize}
 

\end{document}
%%% Local Variables:
%%% mode: latex
%%% TeX-master: t
%%% End:
