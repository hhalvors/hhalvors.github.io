\documentclass[11pt,fleqn]{article}
\usepackage{fullpage}
\usepackage{enumitem}
% \setlength{\parskip}{1em}
% \setlength{\parindent}{0em}
\usepackage{amsfonts,amsmath,amsthm}
\theoremstyle{definition}
\newtheorem*{defn}{Definition}
\newtheorem*{ass}{Assumption}
\usepackage{outlines}
\newcommand{\mf}{\mathfrak}
\newcommand{\7}{\mathbb}
% \newcommand{\2}{\mathcal}
\usepackage{hyperref}
\usepackage[acronym]{glossaries}

\newcommand{\kw}[2][]{\textbf{\glslink{#2}{#1}}}

\makeglossaries


\begin{document}

% Glossary entry defined in the body of the document
\newglossaryentry{derivation}{ name=Derivation, description={For
    $p\in M$, let $\2S (p)$ be the algebra of all smooth functions on
    neighborhoods of $p$. A derivation at the point $p$ is a linear
    map
    \[
    D : C^\infty(U) \to \mathbb{R}
    \]
    where \( U \) is a neighborhood of \( p \) in \( M \), satisfying the Leibniz rule 
    \[
    D(fg) = D(f) g(p) + f(p) D(g)
    \]
    for functions $f,g\in C^\infty (U)$} }

\newglossaryentry{geodesic}{ name=Geodesic, description={Let $\gamma$
    be a smooth curve in $M$. We say that $\gamma$ is a geodesic just
    in case ... } }


\section*{Chapter 5: Relational Theories of Motion}

\begin{outline}[enumerate]

  \1 ``Can there be interesting theories of motion based on classical
  space-times that do not involve \emph{absolute quantities of
    motion}, whether absolute velocity, acceleration, or rotation?''
  (p 91)

  What do you think Earman means by an \textbf{absolute quantity}?

  \2 Proposal: A necessary condition for a quantity to be absolute is
  that it corresponds to a monadic predicate of trajectories that is
  (a) non-trivial, and (b) preserved by all symmetries of the relevant
  type of spacetime.

  \2 Example: ``$x$ is an inertial trajectory'' is a monadic predicate
  for any theory whose models consist of manifold, covariant
  derivative operator (and possibly other structure).

  \2 Note: Some such global properties might be specifiable locally,
  i.e.\ some local property holds at all points along the
  trajectory. In this case: $\nabla _{\dot{\gamma}}\dot{\gamma}=0$.


  \2 Example: ``$x$ is a stationary trajectory'' is \emph{not} a
  monadic predicate in Galilean spacetime. If we treated one inertial
  trajectory as stationary, we'd have to treat them all as such.

  The key fact is that the inertial trajectories form an invariant
  subset of all trajectories under the induced action of the Galilei
  group on trajectories: $\gamma\mapsto \varphi\circ \gamma$.


  \1 HH: It's misleading to say that these theories lack absolute
  velocity. It would be more correct to say that they lack an absolute
  notion of rest, or perhaps even better, that they do not distinguish
  bewteen rest and inertial motion. I might even say: the standard of
  inertia is local to an object itself, and does not depend on its
  relation to some fixed resting point. In particular, $x$ is inertial
  just in case $x$ stays on the trajectory determined by its own
  velocity vector (as opposed to being pushed off it by some external
  force).
  
  Every theory with a covariant derivative operator $\nabla$ has a
  predicate $\mathsf{Iner}(\gamma ,s)$ that says a curve
  $\gamma :\mathbb{R}\to M$ is inertial at a point
  $s\in\mathbb{R}$. \emph{Does this mean that a theory with a
    covariant derivative operator is automatically substantivalist?}


\1 A relativistic spacetime is a smooth manifold $M$ with a Lorentzian
metric $g$.

\2 The metric defines lightcones.

\2 The metric defines a covariant derivative operator $\nabla$.

\2 Einstein's field equations establish a relation between spacetime
metric $g$ and matter distribution $T$.
  

\1 Earman: Relativity is less friendly to relationalism than
classical spacetime theories.


\2 A relativistic spacetime has a metric, which defines a standard of
acceleration (via the covariant derivative) and of rotation. Contrast
with Machian and Leibnizian spacetime

\2 What about the idea that relativity theory entails the following
order of explanation:

\begin{quote} matter $\Rightarrow$ metric $\Rightarrow$
  motion \end{quote}

\2 Fact: There is more than one solution to EFE with $T=0$.


\1 Acceleration in STR and GTR

Given a timelike curve $\gamma$, the acceleration vector field
$\nabla _{\dot{\gamma}}\dot{\gamma}$ along it is a measure of how much
$\gamma$ differs from a geodesic. The metric $g$ determines length of
vectors, and hence, gives an absolute magnitude of acceleration.

Fact: If $X$ is a vector field along a curve such that $g(X,X)< 0$ is
constant, then $\nabla _XX$ is spacelike.

Since $\nabla$ is the Levi-Civita connection for $g$, and $g$ is
symmetric, we have
\[ X(g(X,X)) = 2g(\nabla _X X,X) .\] Since $g(X,X)$ is constant, it
follows that $g(\nabla _X X,X)=0$. The conclusion follows by the
reverse Cauchy-Schwartz inequality.


\1 Einstein's challenge

``What is the reason for this difference in the two bodies? No answer
can be admitted as epistemologically satisfactory, unless the reason
given is an \emph{observable fact of experience}. \dots But the
privileged space $R_1$ of Galileo \dots is a merely \emph{facticious}
cause, and not a thing that can be observed.''




\1 Does spacetime structure \emph{cause} acceleration effects,
e.g. the concavity of the water in the bucket, the tension in the
string between the spheres, the Lorentz contraction?

Does the distinction between accelerated and non-accelerated motion
need to be grounded in the existence of some thing?



% \1 The anti-bucket experiment

% Stage 1: There's a bucket containing water that is concave.

% Stage 2: There's a bucket containing water that is flat.

% Note: We tend to think that water is flat by default. But that is
% earth-centrism! Concave water is just as ``natural'' as flat water.

\1 The Mach-Einstein critique of absolute space

\2 ``It is clear from the foregoing that if Newtonian mechanics and
STR are unsatisfactory because they employ the `factitious cause' of
inertial frames, then GTR is equally unsatisfactory.'' (p 102)

\2 HH: Consider the Newtonian explanation of why one bucket of water
is concave and the other is flat, or why one string between spheres is
taut and another is not

\3 The causal differentia are not observable

The effect cannot be predicted from observing some other fact. There
is no ``constant conjunction''.


\3 The causal differentia are not manipulable

We can give a shove, but we don't know whether we are starting
absolute motion or stopping it!





\end{outline}



\end{document}




\subsection*{Does a lonely particle have a velocity?}

Let's start where we left off last week --- discussing whether an
individual particle has a velocity. This will let us start to unpack
what John Earman means by an absolute quantity.

Consider a possible world with just a single particle all alone. Does
that particle have a velocity? I think different people will have
different intuitions about this. But if we make the same background
assumptions that Earman does, then we must answer yes. In particular,
if spacetime is represented by a smooth manifold $M$, and a particle
by a smooth curve $\gamma$ in $M$, then this particle has a definite
velocity vector at each point along its trajectory. This is true in
any smooth manifold, without the need to add any additional
structure. Granted, if the manifold $M$ cannot be sliced up, at least
locally, into space+time, then we wouldn't know whether to call a
smooth curve a ``trajectory''. But we don't need a connection, much
less a metric, to have such a local slicing. The conclusion, then, is
that having the structure of a smooth manifold is in itself sufficient
for each physical object (represented by a smooth curve in that
manifold) having an \emph{absolute} velocity at each moment of its
life.\footnote{What I'm saying here can be thought of as the opposite
  of the lesson that some people draw from Galilean relativity, i.e.\
  that nothing is \emph{really} moving. I'm suggesting that everything
  is moving, insofar as ``moving'' is cashed out in terms of
  assignment of a non-zero velocity vector.}

It should be clear, however, that there are multiple senses of
``absolute'', and the sense in which velocity is absolute cannot be
what people like John Earman mean with his use of the term. Earman
agrees with the folklore view that velocity is \emph{not} absolute in
Galilean spacetime. So what is this other sense of ``absolute''?

The sense of absolute that I was (perhaps incautiously) using was
being definable without reference to other bodies. ... I mean simply
that the particle's trajectory $\gamma$ defines, without reference to
any other physical bodies, a vector $v(s)\in T_{\gamma (s)}$. (I will
subsequently set $\gamma (s)=p$ and $v(s)=v$ for notational
simplicity.) Or in other words, having the velocity vector $v\in T_p$
is a monadic property of the particle with this trajectory.

What then is the standard sense of ``absolute''? It's not so easy to
think immediately of a fully rigorous mathematical definition. One
wants to to equate ``absolute'' with ``invariant'', and then just turn
the crank: a mathematical thing $X$ represents something absolute just
in case $f(X)=X$ for every symmetry $f$. But there are some subtleties
here. First, symmetries don't act directly on the things we are
interested in (such as an object's acceleration vector); they act on
the points of spacetime, and then we have to decide how to induce
actions on other kinds of mathematical objects. Another serious issue
for any theory that uses the apparatus of smooth manifolds is that a
symmetry $\varphi :M\to M$ maps elements of the fiber over $p\in M$ to
elements of the fiber over $\varphi (p)$ (or the fiber over
$\varphi ^{-1}(p)$, depending on how we set up the definitions). But
elements in the one fiber are not directly comparable to elements in
the other fiber, i.e.\ there is no cross-fiber identity relation. So
how in the world are we supposed to make sense of $f(X)=X$?\footnote{I
  will leave aside for now the common --- and surely not totally
  incorrect --- idea that symmetries act on equations, and that terms
  of equations can be invariant under such transformations. For
  example, it's typically said that Galilean transformations leave the
  form of the law $F=ma$ intact. People of a mathematical bent would
  like to know the rules of this transformation game. What kind of
  things are the objects being acted upon? Are they strings of symbols
  with some kind of formation rules? And if so, under what conditions
  do we say that two such strings are identical?}

The situation is not quite as bad as that makes it sound: we do have
good intuitions about which quantities are absolute in specific
cases. For example:
\begin{quote}
  (*) In Galilean spacetime $M$, the acceleration of a body is
  absolute, while its velocity is not. \end{quote} What exactly does
(*) mean?  Let's resist cashing (*) out in terms of reference frames,
because the notion of a reference frame is no more clear than the
notion of an absolute quantity. Let's try, instead, to find a
mathematical fact that corresponds to (*). (As Carnap would say, we're
looking for an \emph{explication} of (*).) Here we are helped by the
fact that Galilean spacetime can be represented by an affine space
$M\times V\to M$ with a temporal metric $t:V\to\7R$. Here $V$ is the
four-dimensional tangent space, now treated as the same space for all
different points in $M$. In this case, a fixed base point $a\in M$
defines a bijection $u\mapsto a+u$ from $V$ to $M$. A Galilean boost
$\varphi$ based at $a$ and by velocity $v$ is defined by 
\[ \varphi (a+u) \:=\: (a+u)+t(u)v \:=\: a+(u+t(u)v). \] In other
words, $\varphi (a+u)=a+\Phi (u)$, where $\Phi :V\to V$ is the linear
map $u\mapsto u+t(u)v$.

The map $\Phi$ will obviously change the velocity vector

In summary, when the manifold $M$ has a flat connection, then there is
a preferred standard of comparison between vectors in a tangent space
$T_p$ and vectors in a tangent space $T_{\varphi (p)}$. This standard
then allows for a notion of an invariant vector field, either on the
entire space $M$, or on some smooth curve $\gamma$ in $M$.

\begin{defn} Suppose that $M$ is a smooth manifold, and let $\nabla$
  be a flat connection on $M$. For a diffeomorphism $f:M\to M$ and
  vector field $X$ on $M$, we that $X$ is \textbf{invariant} under $f$
  just in case
  \[ \nabla X \: = \: \nabla (f_* X) .\] \end{defn}

Unfortunately this definition is not immediately useful for the case
of interest: the acceleration field $X$ along a smooth curve $\gamma$
in $M$. The problem is that $X$ is not defined on all of $M$, but just
on the image of $\gamma$ in $M$. 


[[TO DO: Galilean spacetime as affine space. Galilean boost as affine
space map.]]

for example, the classic case: in Galilean spacetime $M$, acceleration
is absolute, but velocity is not. Here's one easy way to think of
that: let $\gamma$ be a smooth timelike curve in $M$, and let
$\mathsf{In}$ be the predicate of curves that means ``is inertial'',
which is tantamount to saying that the curve is a
\kw[geodesic]{geodesic} according to the derivative operator $\nabla$
that is part of the definition of $M$. Of course $\nabla$ is invariant
under Galilean transformations in the precise sense that
\[ \varphi_* (\nabla_X Y) = \nabla_{\varphi_* X} (\varphi_* Y) ,
\]
for any Galilean transformation $\varphi :M\to M$, and for any vector
fields $X$ and $Y$.  Thus, Galilean transformations map geodesics to
geodesics, and $\mathsf{In}$ holds of a curve $\gamma$ iff
$\mathsf{In}$ holds of the curve $\varphi \circ \gamma$, where
$\varphi$ is an arbitrary Galilean transformation.

Let $\gamma$ be a timelike geodesic with a timelike tangent vector
$T^a$ satisfying $t_c T^c = 1$, where $t_c$ is the temporal
1-form. Let $X^a$ be a spacelike vector at some point on $\gamma$,
such that $t_c X^c = 0$, and let \(X^a(\tau)\) be the parallel
transport of \(X^a\) along \(\gamma(\tau)\).

Then, the Galilean boost $\varphi: M \to M$ is defined by:
\[ \varphi(p) \:= \: p + t(p)\vec{v}, \] where: $t(p)$ is the time
coordinate of the point $p$.




I am fully aware that it seems meaningless to assert that the velocity
of a particle is $v$, while not saying anything else about the vector
$v$. The problem is that vectors are abstract mathematical objects in
a particularly egregious sense. Let me explain

Assuming that the manifold $M$ is four-dimensional (as we expect
spacetime to be), a tangent space $T_p$ is a four-dimensional vector
space over $\7R$. It's fairly easy to prove that for any two
\emph{non-zero} vectors $v,w$ in a vector space $V$, there is an
isomorphism $L:V\to V$ (i.e.\ an invertible linear mapping) such that
$Lv=w$. (Construct a basis $\{ v,v_1,v_2,\dots \}$ for $V$, and extend
$L$ by setting $L(v_i)=v_i$ for all $i$.) Using the standard
mathematical account of definable properties of mathematical
objects,\footnote{The standard result in model theory is that if
  $h:M\to N$ is an elementary embedding, then $M\models \phi (a)$ iff
  $N\models \phi (h(a))$. Now for any $a\in M$, let $\mathrm{tp}_M(a)$
  be the family of all predicates $\phi$ such that
  $M\models \phi (a)$. Then $\mathrm{tp}_M(a)=\mathrm{tp}_M(h(a))$,
  i.e.\ $a$ and $h(a)$ have all the same definable properties.} there
is then no vector-space-definable property that $v$ has and that $w$
lacks. In other words, any two non-zero vectors in $V$ are
indistinguishable by the lights of vector space theory. (The zero
vector $0$ is the only distinguishable element of $V$ in the sense
that $v\in V$ is preserved under every automorphism of $V$ iff
$v=0$. However, the tangent vector of a smooth curve can never be
$0$.)

It's tempting, then, to think that the assignment of a velocity vector
$v\in T_p$ is meaningless, i.e.\ it's just a mathematical tag that
doesn't mean anything, because $v$ is no different than any other
non-zero vector. There is \emph{some} truth in that claim, viz.\ we
could always change representational conventions and use a different
vector to represent velocity. But this freedom of representational
conventions does not itself mean that the assignment of a specific
vector is meaningless. First, the tangent space $T_p$ at a point
$p\in M$ is not just an abstract vector space: it is the concrete
vector space of \kw[derivations]{derivation} of scalar fields on
neighborhoods around $p$. (As usual, let $\2S (p)$ denote the family
of neighborhoods of $p$.) Thus, the tangent vector of a curve $\gamma$
is uniquely defined by its action on to these scalar fields: it
specifies the rate of change of each scalar field along the trajectory
$\gamma$. A different vector $w\in T_p$ would give different answers
to questions of the form ``how fast is $f$ changing along the
trajectory $\gamma$''.\footnote{So the velocity of an object is not
  defined relative to some fixed standard, but relative to any
  standard of scalar values along its trajectory. But another way of
  putting this is that the velocity of a particle is the standard by
  which change of scalar values can be judged.}

It might be tempting to think that we smuggled in spacetime
substantivalism by assuming that there is a manifold in which the
particle's motion is represented. The temptation here is to think that
the particle has a velocity vector precisely because its position is
changing relative to the background spacetime (represented by the
manifold). But that doesn't make much sense. A tangent vector to a
curve is not a measure, in any sort of way, of how that curve is
changing relative to some fixed standard of reference inside the
manifold, or some structure on the manifold. There are not ``grooves''
in the manifold such that the rate of change of $\gamma$ can be
defined in terms of how many grooves it crosses per unit of time. In
fact, a smooth manifold (without further structure) is like a
``Heraclitus world'' in the sense that nothing remains constant from
point to point. This lack of a fixed standard of reference is
precisely what makes a smooth manifold different from the familiar
space $\7R ^4$ of quadruples of real numbers (or what is the same: by
fixing a preferred global coordinate chart on a manifold).


TODO: There is another sense in (connection doesn't compare different
observers' velocities) ...

There may be some \emph{other} sense of ``absolute'' such that the
velocity of a particle is not absolute; but let's take our time to
explore the options. One immediate thought here is that an abstract
velocity vector $v\in T_p$ is a very different kind of thing than a
numerical value. After all, it's the variability of numerical values
that drives the intuition that velocity is \emph{not} absolute in,
say, Galilean spacetime. For example, the proverbial man on the ship
has zero velocity relative to the ship, while he has non-zero velocity
relative to the shore.



But in order to explain this other sense, we will need to look at
quantities that are normally thought to be absolute, e.g.\
acceleration in Galilean spacetime.




\newpage
\printglossaries


  




\end{document}
%%% Local Variables:
%%% mode: latex
%%% TeX-master: t
%%% End:
