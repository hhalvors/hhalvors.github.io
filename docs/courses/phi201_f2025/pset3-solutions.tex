% Options for packages loaded elsewhere
\documentclass[12pt]{article}
\setlength{\parindent}{0em}
\setlength{\parskip}{1em}
\usepackage{fullpage,amssymb,array}

\title{logic pset3}
\author{}
\date{}

\begin{document}
\maketitle

\thispagestyle{empty}

\vspace*{-3em}

Resources: Lecture 3 and Chapters 3 and 5 of \emph{How Logic
  Works}. (Note that we are skipping over Chapter 4 for now.)

\section*{A. Proofs}

Use any of the rules of inference, including reductio ad absurdum, to
prove the following sequents.

\begin{enumerate}
\item $\neg (P\to Q)\:\vdash\: Q\to R$

  \medskip \begin{tabular}{>{\raggedleft\arraybackslash}p{1.5cm}
      >{\centering\arraybackslash}p{1.0cm} p{6cm}
      >{\raggedright\arraybackslash}p{3.5cm}}
             1 & (1) & $\lnot (\mathit{P} \to \mathit{Q})$ & A \\
             2 & (2) & $\mathit{Q}$ & A \\
             3 & (3) & $\lnot \mathit{R}$ & A \\
             4 & (4) & $\mathit{P}$ & A \\
             2 & (5) & $\mathit{P} \to \mathit{Q}$ & 4,2 CP \\
             1,2 & (6) & $(\mathit{P} \to \mathit{Q}) \land \lnot (\mathit{P} \to \mathit{Q})$ & 5,1 $\wedge$I \\
             1,2 & (7) & $\lnot \lnot \mathit{R}$ & 3,6 RA \\
             1,2 & (8) & $\mathit{R}$ & 7 DN \\
             1 & (9) & $\mathit{Q} \to \mathit{R}$ & 2,8 CP \\
\end{tabular}


  
\item $P\to Q\:\vdash\:\neg P\vee Q$

 \medskip \begin{tabular}{>{\raggedleft\arraybackslash}p{1.5cm} >{\centering\arraybackslash}p{1.0cm} p{6cm} >{\raggedright\arraybackslash}p{3.5cm}}
1 & (1) & $\mathit{P} \to \mathit{Q}$ & A \\
2 & (2) & $\neg (\neg \mathit{P} \vee \mathit{Q})$ & A \\
3 & (3) & $\mathit{Q}$ & A \\
3 & (4) & $\neg \mathit{P} \vee \mathit{Q}$ & 3 $\vee$I \\
2,3 & (5) & $(\neg \mathit{P} \vee \mathit{Q}) \wedge \neg (\neg \mathit{P} \vee \mathit{Q})$ & 4,2 $\wedge$I \\
2 & (6) & $\neg \mathit{Q}$ & 3,5 RA \\
1,2 & (7) & $\neg \mathit{P}$ & 1,6 MT \\
1,2 & (8) & $\neg \mathit{P} \vee \mathit{Q}$ & 7 $\vee$I \\
1,2 & (9) & $(\neg \mathit{P} \vee \mathit{Q}) \wedge \neg (\neg \mathit{P} \vee \mathit{Q})$ & 8,2 $\wedge$I \\
1 & (10) & $\neg \neg (\neg \mathit{P} \vee \mathit{Q})$ & 2,9 RA \\
1 & (11) & $\neg \mathit{P} \vee \mathit{Q}$ & 10 DN \\
\end{tabular}
 

  
\item $P\to (Q\vee R)\:\vdash\: (P\to Q)\vee R$

  My strategy here is to assume the negation of the conclusion for
  reductio ad absurdum. Following the same pattern as DeMorgan's, we
  get $\neg (P\to Q)$ and $\neg R$. The former implies
  $P\wedge \neg Q$. So we have $P,P\to (Q\vee R),\neg Q$ and $\neg
  R$. These form an inconsistent set.

  There are other strategies that might be more intelligible. For
  example, $P\to (Q\vee R)$ implies $\neg P\vee (Q\vee R)$, which
  implies $(\neg P\vee Q)\vee R$, which implies $(P\to Q)\vee
  R$. Similarly, we could first prove $P\vee \neg P$. The former plus
  the premise gives $Q\vee R$ which gives $(P\to Q)\vee R$. The latter
  gives $P\to Q$, which gives $(P\to Q)\vee R$.

\begin{tabular}{>{\raggedleft\arraybackslash}p{1.5cm} >{\centering\arraybackslash}p{1.0cm} p{6cm} >{\raggedright\arraybackslash}p{3.5cm}}
  1 & (1) & $\mathit{P} \to (\mathit{Q} \lor \mathit{R})$ & A \\
  2 & (2) & $\lnot ((\mathit{P} \to \mathit{Q}) \lor \mathit{R})$ & A \\
  3 & (3) & $\lnot \mathit{P}$ & A \\
  4 & (4) & $\mathit{P}$ & A \\
  3,4 & (5) & $\mathit{P} \land \lnot \mathit{P}$ & 3,4 $\wedge$I \\
  6 & (6) & $\lnot \mathit{Q}$ & A \\
  3,4 & (7) & $\lnot \lnot \mathit{Q}$ & 6,5 RA \\
  3,4 & (8) & $\mathit{Q}$ & 7 DN \\
  3 & (9) & $\mathit{P} \to \mathit{Q}$ & 4,8 CP \\
  3 & (10) & $(\mathit{P} \to \mathit{Q}) \lor \mathit{R}$ & 9 $\vee$I \\
  2,3 & (11) & $((\mathit{P} \to \mathit{Q}) \lor \mathit{R}) \land \lnot ((\mathit{P} \to \mathit{Q}) \lor \mathit{R})$ & 10,2 $\wedge$I \\
  2 & (12) & $\lnot \lnot \mathit{P}$ & 3,11 RA \\
  2 & (13) & $\mathit{P}$ & 12 DN \\
  1,2 & (14) & $\mathit{Q} \lor \mathit{R}$ & 1,13 MP \\
  15 & (15) & $\mathit{Q}$ & A \\
  15 & (16) & $\mathit{P} \to \mathit{Q}$ & 4,15 CP \\
  15 & (17) & $(\mathit{P} \to \mathit{Q}) \lor \mathit{R}$ & 16 $\vee$I \\
  18 & (18) & $\mathit{R}$ & A \\
  18 & (19) & $(\mathit{P} \to \mathit{Q}) \lor \mathit{R}$ & 18 $\vee$I \\
  1,2 & (20) & $(\mathit{P} \to \mathit{Q}) \lor \mathit{R}$ & 14,15,17,18,19 $\vee$E \\
  1,2 & (21) & $((\mathit{P} \to \mathit{Q}) \lor \mathit{R}) \land \lnot ((\mathit{P} \to \mathit{Q}) \lor \mathit{R})$ & 20,2 $\wedge$I \\
  1 & (22) & $\lnot \lnot ((\mathit{P} \to \mathit{Q}) \lor \mathit{R})$ & 2,21 RA \\
  1 & (23) & $(\mathit{P} \to \mathit{Q}) \lor \mathit{R}$ & 22 DN \\
\end{tabular}


  
\end{enumerate}

\section*{B. Truth tables}

\begin{enumerate}
\item Use truth table reasoning to show that
  $P\vee (Q\wedge R)\:\vDash\: P\vee Q$. You don't have to display a
  full truth table, but if you do, explain how the table demonstrates
  the result.

  \medskip Consider a line $L$ of the truth table on which
  $P\vee (Q\wedge R)$ is true. In this case either $P$ is true on $L$,
  or $Q\wedge R$ is true on $L$. In the former case, $P\vee Q$ is true
  on $L$. In the latter case, $Q$ is also true on $L$ and hence
  $P\vee Q$ is true on $L$. In either case, $P\vee Q$ is true on
  $L$. Since $L$ was an arbitrary line of a truth table, whenever
  $P\vee (Q\wedge R)$ is true, $P\vee Q$ is also true.
  
\item Use truth table reasoning to show that
  $P\to (Q\vee R)\:\not\vDash \:P\to Q$.

  \medskip Consider the line $L$ where $P$ and $R$ are true, but $Q$
  is false. In that case $Q\vee R$ is true, and hence $P\to (Q\vee R)$
  is true. But since $P$ is true while $Q$ is false, $P\to Q$ is
  false. Thus, there is a scenario in which $P\to (Q\vee R)$ is true
  while $P\to Q$ is false.
  
\item Use truth table reasoning to show that the following ``proof''
  must have a mistake.

  \medskip \begin{tabular}{>{\raggedleft\arraybackslash}p{1.5cm}
      >{\centering\arraybackslash}p{1cm} p{4cm}
      >{\raggedright\arraybackslash}p{3.5cm}}
                1   & (1) & $P\vee Q$  &    A \\
           2   & (2) & $P$    &   A \\
           3   & (3) & $Q$    &   A \\
           2,3 & (4) & $P\wedge Q$  & 2,3 $\wedge$I \\
           2,3 & (5) & $P$    &    4 $\wedge$E \\
           1 & (6) & $P$ & 1,2,2,3,5 $\vee$E
           \end{tabular}

           \medskip Consider line (6), which asserts that
           $P\vee Q\vdash P$. It is clear that $P\vee Q\not\vDash P$
           since $Q$ could be true while $P$ is false. By the
           \emph{soundness} of our proof system,
           $P\vee Q\not\vdash P$. Therefore line (6) cannot be part of
           any correctly written proof. (In fact, line (6) does not
           calculate dependency numbers correctly. An application of
           $\vee$E to lines $1,2,2,3,5$ should result in dependencies
           $1,2$.)
           
\end{enumerate}
  

\end{document}

%%% Local Variables:
%%% mode: latex
%%% TeX-master: t
%%% End:
