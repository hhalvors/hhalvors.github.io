\documentclass[aspectratio=169,17pt,fleqn]{beamer}

\usetheme{Madrid}
\usecolortheme{dolphin}
\setbeamertemplate{navigation symbols}{}
\setbeamertemplate{footline}[frame number]

\usepackage{amsmath,amssymb,array}
% \setmathfont{Latin Modern Math}

\usepackage{fontspec}      % text fonts (beamer uses sans by default)
\usepackage{unicode-math}  % modern math

\AtBeginSection[]{
  \begin{frame}
    \centering
    \vfill
    \Large\insertsectionhead
    \vfill
  \end{frame}
}


\title{Theories}
\author{PHI~201 — Introductory Logic}
\date{Week 8}

\begin{document}

\frame{\titlepage}

\begin{frame}{The utopian vision of symbolic logic}

  \begin{itemize}
  \item Two original hopes for symbolic logic.
    \begin{enumerate}
    \item It provides a universal language for science.
    \item It dissolves philosophical pseudo-problems.
    \end{enumerate}
  \item While this doesn't work out so easily in practice, there is a
    sense in which all ``theories'' in mathematics can be formalized
    in predicate/relational logic.
  \end{itemize}
  
\end{frame}

\section{Equality}

\begin{frame}{Equality is a special relation}

  \begin{itemize}
  \item Equality is a binary relation which we write as an infix
    rather than as a prefix
    \[ c=d ,\: \exists x(x=d) ,\: \forall y\exists x(x=y) \]
  \item Using ``$=$'' allows us to express many new things.
  \end{itemize}

\end{frame}

\begin{frame}{At least $n$}

  \[ \exists x\exists y(x\neq y) \]

  \[ \exists x\exists y\exists z((x\neq y\wedge x\neq z)\wedge y\neq
    z) \]



\end{frame}

\begin{frame}{At most $n$}

  \[ \forall x\forall y (x=y) \]

  \[ \forall x\forall y\forall z((x=y\vee x=z)\vee y=z) \]


\end{frame}

\begin{frame}{Exactly $n$}

  \[ \exists x\exists y(x\neq y\wedge \forall z(z=x\vee z=y)) \]

  \[
\begin{aligned}
\exists x\,\exists y\,\exists z \big(&((x \neq y \wedge x \neq z)\wedge y \neq z) \\
 &\wedge \forall w\,((w=x \vee w=y) \vee w=z)\big)
\end{aligned}
\]

  


\end{frame}


\begin{frame}{There is a unique $P$}

  \[ \exists x(Px\wedge \forall y(Py\to x=y)) \]


\end{frame}

\begin{frame}{Definite descriptions}

  


\end{frame}

\begin{frame}{Superlatives}

  ``There is a tallest student.''
  \[ \exists x\forall y(x\neq y\to Txy) \]

  This sentence entails uniqueness only because we implicitly assume
  that ``taller than'' is asymmetric.
  \[ \forall x\forall y(Txy\to \neg Tyx) \]

\end{frame}

\begin{frame}

\begin{tabular}{>{\raggedleft\arraybackslash}p{1.5cm} >{\centering\arraybackslash}p{1.0cm} p{6cm} >{\raggedright\arraybackslash}p{3.5cm}}
1 & (1) & $\exists x \forall y (x\neq y\to Txy)$ & A \\
2 & (2) & $\forall y (a\neq y\to Tay)$ & A \\
3 & (3) & $\forall y (b\neq y\to Tby)$ & A \\
4 & (4) & $a\neq b$ & A \\
2 & (5) & $a\neq b \to Tab$ & 2 UE \\
3 & (6) & $b\neq a \to Tba$ & 3 UE \\
\end{tabular}


\end{frame}



\begin{frame}{Inference rules for equality}

  \begin{tabular}{>{\raggedleft\arraybackslash}p{1.5cm} >{\centering\arraybackslash}p{1.0cm} p{6cm} >{\raggedright\arraybackslash}p{3.5cm}}
    $\Gamma$ & (m) & $\varphi (a)$ \\[0.5em]
    $\Delta$ & (n) & $a=b$ \\[0.5em]
    $\Gamma ,\Delta$ & (o) & $\varphi (b)$ & m,n $=$E \end{tabular}
  

\end{frame}

   \begin{frame}
     To show: $a=b,b=c\:\vdash\: a=c$

     \bigskip \begin{tabular}{>{\raggedleft\arraybackslash}p{1.5cm}
      >{\centering\arraybackslash}p{1.0cm} p{6cm}
        >{\raggedright\arraybackslash}p{3.5cm}}
        1 & (1) & $a=b$ & A \\
        2 & (2) & $b=c$ & A \\
        1,2 & (3) & $a=c$ & 2,1 $=$E \end{tabular}

 \end{frame}     

\begin{frame}{Inference rules for equality}

    \begin{tabular}{>{\raggedleft\arraybackslash}p{1.5cm}
      >{\centering\arraybackslash}p{1.0cm} p{6cm}
      >{\raggedright\arraybackslash}p{3.5cm}}
       & (m) & $a=a$ & $=$I \end{tabular}

   \end{frame}

   \begin{frame}

     To show: $a=b\:\vdash\: b=a$

     \bigskip
        \begin{tabular}{>{\raggedleft\arraybackslash}p{1.5cm}
      >{\centering\arraybackslash}p{1.0cm} p{6cm}
          >{\raggedright\arraybackslash}p{3.5cm}}
          1 & (1) & $a=b$ & A \\
            & (2) & $a=a$ & $=$I \\
          1 & (3) & $b=a$ & 2,1 $=$E \end{tabular}

      \end{frame}


                   

  %% The only \dots

\begin{frame}{Nobody but}

  Alice respects nobody but Bob.

  \[ Rab\wedge \forall x(Rax\to x=b) \]
  \[ \forall x(Rax\leftrightarrow x=b ) \]


\end{frame}

\begin{frame}{Everybody loves my baby}

  \begin{tabular}{>{\raggedleft\arraybackslash}p{1.5cm} >{\centering\arraybackslash}p{1.0cm} p{5cm} >{\raggedright\arraybackslash}p{3.5cm}}
1 & (1) & $\forall x Lxb$ & A \\
2 & (2) & $\forall y (Lby \to y = a)$ & A \\
1 & (3) & $Lbb$ & 1 UE \\
2 & (4) & $Lbb \to b = a$ & 2 UE \\
\end{tabular}

\end{frame}

\begin{frame}

  \begin{itemize}
  \item The theory of equality is peculiar, because we build its
    axioms in as new inference rules.
  \item Now we look at theories whose axioms are
    sentences. 
  \end{itemize}

\end{frame}

\section{Theory of partial order}

\begin{frame}
transitive:
\[ \forall x\forall y\forall z((x\leq y\wedge y\leq z)\to x\leq z) \]

reflexive:
\[ \forall x(x\leq x) \]

antisymmetric:
\[ \forall x\forall y((x\leq y\wedge y\leq x)\to x=y) \]


\end{frame}

\begin{frame}{A cornucopia of partially ordered sets}


\vspace{6em}
  linear:
\[ \forall x\forall y((x\leq y)\vee (y\leq x)) \]




\end{frame}

\begin{frame}

  \begin{itemize}
  \item What's a sentence that is true of the natural numbers
    $1,2,3,\dots$ but false of the integers
    $\dots ,-2,-1,0,1,2,\dots $?
  \item What's a sentence that is true of the integers but false of
    the rational numbers?
  \end{itemize}



\end{frame}

\section{Set theory}

\begin{frame}

  extensionality
  \[ \forall x\forall y(x=y\leftrightarrow \forall z(z\in
    x\leftrightarrow z\in y)) \]

  existence of an emptyset
  \[ \exists z\forall x(x\not\in z) \]


\end{frame}

\begin{frame}{Uniqueness of the emptyset}

  \bigskip
    \begin{tabular}{>{\raggedleft\arraybackslash}p{1.5cm}
      >{\centering\arraybackslash}p{1.0cm} p{6cm}
      >{\raggedright\arraybackslash}p{3.5cm}}
      1 & (1) & $\forall x(x\not\in a)$ & A \\
      2 & (2) & $\forall x(x\not\in b)$ & A \\
      1 & (3) & $c\not\in a$ & 1 UE \\
      1 & (4) & $c\in a\to c\in b$ & 3 neg par \\
    \end{tabular}

  \end{frame}

  \begin{frame}{Naive set theory}

    comprehension
    \[ \exists x\forall y(y\in x\leftrightarrow \varphi (y)) \]


  \end{frame}

  \begin{frame}{Consistent theories}

    We say that a theory $T$ is \textbf{consistent} if there is no
    sentence $\varphi$ such that both $T\vdash \varphi$ and
    $T\vdash \neg \varphi$.



  \end{frame}

  \begin{frame}{Naive set theory is inconsistent}

    Use comprehension with the predicate ``$y\not\in y$''

    \bigskip \begin{tabular}{>{\raggedleft\arraybackslash}p{1.5cm}
        >{\centering\arraybackslash}p{1.0cm} p{7cm}
        >{\raggedright\arraybackslash}p{3.5cm}}
    1 & (1) & $\exists x \forall y (y\in x\leftrightarrow y\not\in y)$ & A \\
    2 & (2) & $\forall y (y\in a \leftrightarrow y\not\in y)$ & A \\
    2 & (3) & $a\in a \leftrightarrow a\not\in a$ & 2 UE \\
\end{tabular}
  



  \end{frame}  


 \begin{frame}{Sophisticating set theory}

   pairing
   \[ \forall x\forall y\exists z\forall w(w\in z\leftrightarrow
     (w=x\vee w=y)) \]

   separation: For every formula $\varphi(x, b_1, \dots, b_n)$,
   \[ \forall y \exists z\forall x \Bigl( x \in z
     \;\leftrightarrow\; (x \in y \wedge\ \varphi(x, b_1, \dots, b_n))
     \Bigr).
\]

\end{frame}

\begin{frame}{Existence and uniqueness of intersections}

  \bigskip \small
  \noindent \begin{tabular}{>{\raggedleft\arraybackslash}p{0.5cm}
      >{\centering\arraybackslash}p{0.75cm} p{9.7cm}
      >{\raggedright\arraybackslash}p{3cm}}
              & (1) & $\exists z\forall x\bigl(x\in z \leftrightarrow (x\in a \wedge x\in b)\bigr)$ & sep \\
              2 & (2) & $\forall x(x\in c \leftrightarrow (x\in a \wedge x\in b))$ & A \\
              & (3) & $\forall y\forall y'(\forall x(x\in y\leftrightarrow x\in y')\to y=y')$ & ext \\
              4 & (4) & $\forall x(x\in d \leftrightarrow (x\in a \wedge x\in b))$ & A \\
              2,4  & (5) & $\forall x(x\in c \leftrightarrow x\in d)$
                                                                                                    & 2,4   \\
              & (6) & $\forall x(\in c\leftrightarrow x\in d)\to c=d$ &
                                                                       3
                                                                        UE \\
              2,4 & (8) & $c=d$ & 6,5 MP 
\end{tabular}

\end{frame}

\end{document}  

%%% Local Variables:
%%% mode: latex
%%% TeX-master: t
%%% End:
