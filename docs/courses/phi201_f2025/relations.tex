\documentclass[12pt,fleqn]{article}
\usepackage[margin=1in]{geometry}
\usepackage{amsmath,amssymb,amsthm}
\newtheorem*{fact}{Fact}
\usepackage{tikz}
\usetikzlibrary{arrows.meta}

\title{Binary relations on a three-element set} \date{}

\begin{document}
\maketitle

Let $A=\{a,b,c\}$.  A \emph{binary relation} $R$ on $A$ is any subset
of $A\times A$.  Equivalently, $R$ is determined by its $3\times3$
adjacency matrix $(r_{xy})_{x,y\in A}$, where $r_{xy}=1$ iff
$Rxy$. Since $|A\times A|=9$, each ordered pair may or may not belong
to $R$.  Hence the total number of binary relations on $A$ is
$2^{9}=\mathbf{512}$.

We say that two relations $R,R'$ are the same \emph{type} if one can
be obtained from the other by permuting the elements of $A$; i.e.,
there is some $\sigma\in S_3$ with
\[ \langle x,y\rangle \in R \iff \langle \sigma x,\sigma y\rangle \in
  R'. \] The group $S_3$ acts on the set of all relations in this way.
Burnside's lemma gives
\[
\frac{1}{6}\bigl(2^{9}+3\!\cdot\!2^{5}+2\!\cdot\!2^{3}\bigr)=104,
\]
so there are \textbf{104} distinct relation types (isomorphism
classes) under relabeling.\footnote{%
  \textbf{Burnside’s Lemma.}  Let a finite group $G$ act on a finite
  set $X$.  For each $g\in G$, let
  $\mathrm{Fix}(g)=\{x\in X : g\cdot x=x\}$ be the set of elements
  fixed by $g$.  Then the number of distinct orbits of $X$ under the
  action of $G$ is
  \[ \frac{1}{|G|}\sum_{g\in G} |\mathrm{Fix}(g)|. \] In particular,
  if $G$ acts by permuting coordinates of combinatorial structures
  (such as the rows and columns of a matrix), the lemma gives the
  number of unlabeled structures up to relabeling.%
}

\begin{fact} Let $\varphi$ be a sentence with the binary relation
  symbol $R$ and no names. Let $M$ and $N$ be interpretations of $R$
  such that $R^M$ is of the same type as $R^N$. Then $M\vDash\varphi$
  iff $N\vDash\varphi$. \end{fact}

We now describe some of these 104 types in detail, focusing on three
important families: the \emph{equivalence relations}, the
\emph{symmetric relations}, and the \emph{asymmetric relations}.


\section{Equivalence relations}

Equivalence relations correspond to partitions of $A$.  Up to
isomorphism there are precisely three partitions (hence three types).

\medskip
\begin{description}
\item[\textbf{E1:}] $\{\{a\},\{b\},\{c\}\}$ (identity relation)

\begin{center}
\begin{tikzpicture}[scale=0.9,baseline=(current bounding box.center)]
  \node (a) at (0,1.0) {$a$};
  \node (b) at (-1,-0.6) {$b$};
  \node (c) at (1,-0.6) {$c$};
  \draw[loop above,looseness=14] (a) to (a);
  \draw[loop left,looseness=14] (b) to (b);
  \draw[loop right,looseness=14] (c) to (c);
\end{tikzpicture}
\end{center}

\item[\textbf{E2:}] $\{\{a,b\},\{c\}\}$ (one 2--class + one singleton)

\begin{center}
\begin{tikzpicture}[scale=0.9,baseline=(current bounding box.center)]
  \node (a) at (0,1.0) {$a$};
  \node (b) at (-1,-0.6) {$b$};
  \node (c) at (1,-0.6) {$c$};
  \draw[loop above,looseness=14] (a) to (a);
  \draw[loop left,looseness=14] (b) to (b);
  \draw (a) -- (b);
  \draw[loop right,looseness=14] (c) to (c);
\end{tikzpicture}
\end{center}

\item[\textbf{E3:}] $\{\{a,b,c\}\}$ (universal relation)

\begin{center}
\begin{tikzpicture}[scale=0.9,baseline=(current bounding box.center)]
  \node (a) at (0,1.0) {$a$};
  \node (b) at (-1,-0.6) {$b$};
  \node (c) at (1,-0.6) {$c$};
  \draw (a) -- (b) -- (c) -- (a);
  \draw[loop above,looseness=14] (a) to (a);
  \draw[loop left,looseness=14] (b) to (b);
  \draw[loop right,looseness=14] (c) to (c);
\end{tikzpicture}
\end{center}
\end{description}

\section{Symmetric relations (20 types)}

A symmetric relation is an \emph{undirected} graph with optional loops.
Up to isomorphism, start from the underlying simple graph (no loops), 
then add loops modulo its automorphisms.

\paragraph{Underlying graph shapes (no loops):}
\begin{center}
\begin{tabular}{cccc}
Empty & One edge & Path on 3 & Triangle \\
\begin{tikzpicture}[scale=0.8,baseline=(current bounding box.center)]
  \node (a) at (0,0.8) {}; \node (b) at (-0.8,-0.6) {}; \node (c) at (0.8,-0.6) {};
\end{tikzpicture}
&
\begin{tikzpicture}[scale=0.8,baseline=(current bounding box.center)]
  \node (a) at (0,0.8) {}; \node (b) at (-0.8,-0.6) {}; \node (c) at (0.8,-0.6) {};
  \draw (a)--(b);
\end{tikzpicture}
&
\begin{tikzpicture}[scale=0.8,baseline=(current bounding box.center)]
  \node (a) at (0,0.8) {}; \node (b) at (-0.8,-0.6) {}; \node (c) at (0.8,-0.6) {};
  \draw (a)--(b)--(c);
\end{tikzpicture}
&
\begin{tikzpicture}[scale=0.8,baseline=(current bounding box.center)]
  \node (a) at (0,0.8) {}; \node (b) at (-0.8,-0.6) {}; \node (c) at (0.8,-0.6) {};
  \draw (a)--(b)--(c)--(a);
\end{tikzpicture}
\end{tabular}
\end{center}

\vspace{1em}
\noindent Each base shape can receive loops in various ways, counted up to automorphisms.

\paragraph{(a) Empty graph.}
All vertices are indistinguishable; loops can be placed on any subset.
Distinct cases (by number of loops): 0, 1, 2, or 3.
\begin{center}
\begin{tikzpicture}[scale=0.8,baseline=(current bounding box.center)]
  \node (a) at (0,0.8) {}; \node (b) at (-0.8,-0.6) {}; \node (c) at (0.8,-0.6) {};
\end{tikzpicture}
\quad
\begin{tikzpicture}[scale=0.8,baseline=(current bounding box.center)]
  \node (a) at (0,0.8) {}; \node (b) at (-0.8,-0.6) {}; \node (c) at (0.8,-0.6) {};
  \draw[loop above,looseness=14] (a) to (a);
\end{tikzpicture}
\quad
\begin{tikzpicture}[scale=0.8,baseline=(current bounding box.center)]
  \node (a) at (0,0.8) {}; \node (b) at (-0.8,-0.6) {}; \node (c) at (0.8,-0.6) {};
  \draw[loop above,looseness=14] (a) to (a);
  \draw[loop left,looseness=14] (b) to (b);
\end{tikzpicture}
\quad
\begin{tikzpicture}[scale=0.8,baseline=(current bounding box.center)]
  \node (a) at (0,0.8) {}; \node (b) at (-0.8,-0.6) {}; \node (c) at (0.8,-0.6) {};
  \draw[loop above,looseness=14] (a) to (a);
  \draw[loop left,looseness=14] (b) to (b);
  \draw[loop right,looseness=14] (c) to (c);
\end{tikzpicture}
\end{center}

\paragraph{(b) One edge.}
Automorphism swaps the two adjacent vertices; the isolated vertex is distinct.
Loops on the pair: 0, 1, or 2; loop on the isolated: 0 or 1.
This gives $3\times2=6$ types.

\begin{center}
\begin{tikzpicture}[scale=0.8,baseline=(current bounding box.center)]
  \node (a) at (0,0.8) {}; \node (b) at (-0.8,-0.6) {}; \node (c) at (0.8,-0.6) {};
  \draw (a)--(b);
\end{tikzpicture}
\quad
\begin{tikzpicture}[scale=0.8,baseline=(current bounding box.center)]
  \node (a) at (0,0.8) {}; \node (b) at (-0.8,-0.6) {}; \node (c) at (0.8,-0.6) {};
  \draw (a)--(b); \draw[loop right,looseness=14] (c) to (c);
\end{tikzpicture}
\quad
\begin{tikzpicture}[scale=0.8,baseline=(current bounding box.center)]
  \node (a) at (0,0.8) {}; \node (b) at (-0.8,-0.6) {}; \node (c) at (0.8,-0.6) {};
  \draw (a)--(b); \draw[loop above,looseness=14] (a) to (a);
\end{tikzpicture}
\end{center}

\paragraph{(c) Path on three vertices.}
Automorphism swaps the two endpoints; the middle vertex is distinguished.
Loops on endpoints: 0, 1, or 2; loop on middle: 0 or 1.
Again $3\times2=6$ types.

\begin{center}
\begin{tikzpicture}[scale=0.8,baseline=(current bounding box.center)]
  \node (a) at (0,0.8) {}; \node (b) at (-0.8,-0.6) {}; \node (c) at (0.8,-0.6) {};
  \draw (a)--(b)--(c);
\end{tikzpicture}
\quad
\begin{tikzpicture}[scale=0.8,baseline=(current bounding box.center)]
  \node (a) at (0,0.8) {}; \node (b) at (-0.8,-0.6) {}; \node (c) at (0.8,-0.6) {};
  \draw (a)--(b)--(c);
  \draw[loop left,looseness=14] (b) to (b);
\end{tikzpicture}
\quad
\begin{tikzpicture}[scale=0.8,baseline=(current bounding box.center)]
  \node (a) at (0,0.8) {}; \node (b) at (-0.8,-0.6) {}; \node (c) at (0.8,-0.6) {};
  \draw (a)--(b)--(c);
  \draw[loop above,looseness=14] (a) to (a);
\end{tikzpicture}
\end{center}

\paragraph{(d) Triangle.}
All vertices indistinguishable; loops can be placed on any subset.
Distinct cases (by number of loops): 0, 1, 2, or 3.
\begin{center}
\begin{tikzpicture}[scale=0.8,baseline=(current bounding box.center)]
  \node (a) at (0,0.8) {}; \node (b) at (-0.8,-0.6) {}; \node (c) at (0.8,-0.6) {};
  \draw (a)--(b)--(c)--(a);
\end{tikzpicture}
\quad
\begin{tikzpicture}[scale=0.8,baseline=(current bounding box.center)]
  \node (a) at (0,0.8) {}; \node (b) at (-0.8,-0.6) {}; \node (c) at (0.8,-0.6) {};
  \draw (a)--(b)--(c)--(a);
  \draw[loop above,looseness=14] (a) to (a);
\end{tikzpicture}
\quad
\begin{tikzpicture}[scale=0.8,baseline=(current bounding box.center)]
  \node (a) at (0,0.8) {}; \node (b) at (-0.8,-0.6) {}; \node (c) at (0.8,-0.6) {};
  \draw (a)--(b)--(c)--(a);
  \draw[loop above,looseness=14] (a) to (a);
  \draw[loop left,looseness=14] (b) to (b);
\end{tikzpicture}
\quad
\begin{tikzpicture}[scale=0.8,baseline=(current bounding box.center)]
  \node (a) at (0,0.8) {}; \node (b) at (-0.8,-0.6) {}; \node (c) at (0.8,-0.6) {};
  \draw (a)--(b)--(c)--(a);
  \draw[loop above,looseness=14] (a) to (a);
  \draw[loop left,looseness=14] (b) to (b);
  \draw[loop right,looseness=14] (c) to (c);
\end{tikzpicture}
\end{center}

\paragraph{Count check.}
Distinct loop patterns modulo automorphisms:
\[
  4\ (\text{empty})\;+\;6\ (\text{one edge})\;+\;6\
  (\text{path})\;+\;4\ (\text{triangle})\;=\;\boxed{20}.
\]
Thus there are \textbf{20 symmetric relation types} on a 3--element set.

Every symmetric relation on $A=\{a,b,c\}$ can be represented by a
$3\times3$ Boolean matrix $(r_{ij})$ such that $r_{ij}=r_{ji}$ for all
$i,j$.  Each $r_{ij}$ records whether the ordered pair
$\langle a_i,a_j\rangle$ belongs to the relation $R$.

\medskip
\noindent
To understand how many distinct symmetric relations there are up to isomorphism,
it helps to build these matrices column by column.  
The first column determines all entries in the first row by symmetry,  
so we can visualize the process as gradually filling in the upper--left
triangle of the matrix.

\[
\begin{bmatrix}
r_{11} & r_{12} & r_{13} \\
r_{12} & r_{22} & r_{23} \\
r_{13} & r_{23} & r_{33}
\end{bmatrix}
\]

\medskip
\noindent
The first column corresponds to the loops and connections involving the first element~$a$.
We can freely choose:
\begin{itemize}
  \item whether $a$ has a loop ($r_{11}=0$ or $1$), and
  \item whether $a$ is connected to $b$ ($r_{12}$) and to $c$ ($r_{13}$).
\end{itemize}
Hence there are $2^3=8$ possibilities for the first column (and its symmetric first row).

\medskip
\noindent
Now fix one of those eight possibilities.  
Once the first column is chosen, some entries in the second column are already
constrained by symmetry: the entry $r_{21}$ is equal to $r_{12}$.
Thus the second column still has two unconstrained positions:
$r_{22}$ (loop on~$b$) and $r_{23}$ (connection between $b$ and~$c$).
That gives $2^2=4$ degrees of freedom for the second column.

\medskip
\noindent
Finally, once we have filled in the first and second columns,
the third column is almost completely determined by symmetry:
$r_{31}=r_{13}$ and $r_{32}=r_{23}$ are fixed,
and only $r_{33}$ (the loop on~$c$) remains free.

\medskip
\noindent
Altogether we have
\[
2^3 \text{ (for the first column)} \;\times\;
2^2 \text{ (for the second column)} \;\times\;
2^1 \text{ (for the third column)} \;=\; 2^6 = 64
\]
labeled symmetric relations.  

\medskip
\noindent
When we ignore the names of $a$, $b$, and $c$—that is, when we identify
relations that differ only by renaming the elements of~$A$—
many of these $64$ labeled matrices coincide.
For example, the two matrices
\[
\begin{bmatrix}
0 & 1 & 0 \\
1 & 0 & 0 \\
0 & 0 & 0
\end{bmatrix}
\qquad\text{and}\qquad
\begin{bmatrix}
0 & 0 & 1 \\
0 & 0 & 0 \\
1 & 0 & 0
\end{bmatrix}
\]
represent the same relation \emph{type}:
each depicts a single undirected edge connecting two of the three elements,
with the remaining element isolated.  
They differ only in which pair of vertices happens to be connected.
When we collapse all such relabelings, the 64 labeled matrices fall into
exactly \textbf{20 distinct unlabeled patterns}.
These correspond precisely to the $20$ symmetric relation types
classified earlier by their underlying graph shapes and possible loop configurations.

\section{Asymmetric relations}

A relation $R$ on $A=\{a,b,c\}$ is \emph{asymmetric} if
\[
Rxy \;\Rightarrow\; \lnot Ryx.
\]
Equivalently, $R$ has no loops and no pair of opposite arrows; in graph terms,
it is a loopless digraph without $2$--cycles.  (Asymmetric $\Rightarrow$ irreflexive.)

\medskip
\noindent
For each unordered pair $\{x,y\}$ there are exactly three options:
$(x\!\to\!y)$, $(y\!\to\!x)$, or neither.  With $\binom{3}{2}=3$ pairs, the number
of \emph{labeled} asymmetric relations on $A$ is
\[
3^3 = 27.
\]

\medskip
\noindent
Representative shapes up to relabeling (no loops anywhere):

\begin{itemize}
  \item \textbf{Empty relation.}\;
  \begin{tikzpicture}[scale=0.8,baseline=(current bounding box.center)]
    \node (a) at (0,1.0) {};
    \node (b) at (-1,-0.6) {};
    \node (c) at (1,-0.6) {};
  \end{tikzpicture}

  \item \textbf{Single arrow} ($a\!\to b$).\;
  \begin{tikzpicture}[scale=0.8,baseline=(current bounding box.center)]
    \node (a) at (0,1.0) {};
    \node (b) at (-1,-0.6) {};
    \node (c) at (1,-0.6) {};
    \draw[->] (a) -- (b);
  \end{tikzpicture}

  \item \textbf{Fork (two arrows with a common source)} ($a\!\to b$, $a\!\to c$).\;
  \begin{tikzpicture}[scale=0.8,baseline=(current bounding box.center)]
    \node (a) at (0,1.0) {};
    \node (b) at (-1,-0.6) {};
    \node (c) at (1,-0.6) {};
    \draw[->] (a) -- (b);
    \draw[->] (a) -- (c);
  \end{tikzpicture}

  \item \textbf{Chain} ($a\!\to b\!\to c$).\;
  \begin{tikzpicture}[scale=0.8,baseline=(current bounding box.center)]
    \node (a) at (0,1.0) {};
    \node (b) at (-1,-0.6) {};
    \node (c) at (1,-0.6) {};
    \draw[->] (a) -- (b);
    \draw[->] (b) -- (c);
  \end{tikzpicture}

  \item \textbf{3--cycle (tournament on three)} ($a\!\to b\!\to c\!\to a$).\;
  \begin{tikzpicture}[scale=0.8,baseline=(current bounding box.center)]
    \node (a) at (0,1.0) {};
    \node (b) at (-1,-0.6) {};
    \node (c) at (1,-0.6) {};
    \draw[->] (a) -- (b);
    \draw[->] (b) -- (c);
    \draw[->] (c) -- (a);
  \end{tikzpicture}
\end{itemize}

\section{Summary}
\begin{itemize}
  \item $2^{9}=512$ labeled binary relations on $A=\{a,b,c\}$.
  \item $104$ relation types up to relabeling of the domain.
  \item Equivalence relations split into $3$ types (partitions of $A$).
  \item Symmetric relations split into $20$ types (simple-graph shape $\times$ loop--pattern modulo automorphisms).
\end{itemize}

\end{document}