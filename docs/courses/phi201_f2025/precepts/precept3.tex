\documentclass[12pt]{article}
\usepackage{fullpage}
\title{Logic precept 3}
\date{}
\setlength{\parindent}{0em}
\setlength{\parskip}{1em}
\begin{document}

\maketitle

\vspace*{-4em}

\section*{Proofs}

\subsection*{Review of $\vee$-elimination}

\begin{enumerate}
\item $(P\to Q)\vee (P\to R)\:\vdash\: P\to (Q\vee R)$
\item $\neg P\vee\neg Q\:\vdash\: \neg (P\wedge Q)$
\item $\neg P\vee Q\:\vdash\: P\to Q$
\end{enumerate}

\subsection*{Reductio ad Absurdum}

\begin{enumerate}
\item $P\to Q\:\vdash\: \neg (P\wedge \neg Q)$
\item $\neg (P\to Q)\:\vdash\: \neg Q$
\item pset1 $\neg (P\to Q)\:\vdash\: Q\to R$
\item $\neg (P\vee Q)\:\vdash \: \neg P$  
\item pset2 $P\to Q\:\vdash\:\neg P\vee Q$
\item pset3 $P\to (Q\vee R)\:\vdash\: (P\to Q)\vee R$
\end{enumerate}



\subsubsection*{Challenge problem: Pierce's law}

$\vdash\: ((P\to Q)\to P)\to P$

\section*{Truth tables}

\subsection*{Key Concepts}

\begin{itemize}
\item
  arguments: valid, invalid
\item
  counterexample
\item
  truth-value
\item
  main connective
\item
  sentences (syntactic): atomic, conjunction, negation, disjunction,
  conditional, biconditional
\item
  sentences (semantic): tautology, inconsistency, contingency
\item
  two sentences: equivalent, inconsistent, independent
\end{itemize}

\subsection*{For arguments}

Determine whether the following arguments are valid or not. Explain
your answer by showing the existence of a row of a truth table, or by
pointing to a full truth table, or something of the sort.  Your answer
should be articulated in English prose so that it can convince anyone
else who is familiar with truth tables.

\begin{enumerate}
\item $P\to (Q\vee R)\:\vdash \: (P\to Q)\vee R$
\item
  $\vdash \: (P\leftrightarrow Q)\vee (P\leftrightarrow R)\vee
  (Q\leftrightarrow R)$
\item $P\to (Q\to R)\:\vdash\: (P\wedge Q)\to R$
\item $P\to R\:\vdash\: (P\vee Q)\to R$
\item $(P\leftrightarrow Q)\leftrightarrow R \:\vdash\: P\vee R$
\item $\:\vdash\: (P\to Q)\vee (Q\to R)$
\end{enumerate}

\end{document}


\subsection*{Sentence classification
(syntactic)}\label{sentence-classification-syntactic}

\textbf{Exercise.} What is the \textbf{main connective} of each of the
following formulas?

\begin{enumerate}
\def\labelenumi{\arabic{enumi}.}
\item
  \(\neg (P\to Q)\)
\item
  \(\neg P\to Q\)
\item
  \(\neg (P\to \neg Q)\)
\item
  \((P\wedge Q)\vee \neg (P\to Q)\)
\item
  \(((P\to Q)\to P)\to P\)
\end{enumerate}

\section{Sentence classification
(semantic)}\label{sentence-classification-semantic}

\textbf{Exercise.} Classify each of the following sentences as
tautology, inconsistency, or contingency.

\begin{enumerate}
\def\labelenumi{\arabic{enumi}.}
\item
  \((P\to \neg P)\to \neg P\)
\item
  \((P\wedge Q)\vee (\neg P\wedge \neg Q)\)
\item
  \((P\wedge (Q\wedge \neg R))\vee (\neg P\wedge (\neg Q\wedge R))\)
\item
  \((P\leftrightarrow Q)\leftrightarrow R\)
\item
  \((P\wedge Q)\vee \neg (P\to Q)\)
\item
  \(((P\to Q)\to R)\to Q\)
\end{enumerate}

\textbf{Exercise.} Show that if \(B\) is a tautology, then \(A\wedge B\)
is logically equivalent to \(A\).

\textbf{Exercise.} Show that if \(B\) is an inconsistency, then
\(A\vee B\) is logically equivalent to \(A\).

\subsection{For multiple sentences}

\textbf{Exercise:} What is the semantic relationship between
\((P\wedge Q)\) and \(\neg (P\to Q)\)?

\textbf{Exercise:} If \(\phi\wedge\psi\) is a contingency, then what are
the possibilities for \(\phi\) and \(\psi\)?

\textbf{Exercise:} If \(\phi\) is a tautology, then what are the
possibilities for \(\phi\wedge\psi\)? What are the possibilities for
\(\phi\vee\psi\)?




\end{document}
%%% Local Variables:
%%% mode: latex
%%% TeX-master: t
%%% End:
