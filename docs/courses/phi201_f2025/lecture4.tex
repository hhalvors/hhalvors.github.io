% !TeX program = xelatex
\documentclass[aspectratio=169,14pt,fleqn]{beamer}

% THEME & FONTS
\usetheme{Madrid}
\usecolortheme{dolphin}
\setbeamertemplate{navigation symbols}{}
\setbeamertemplate{footline}[frame number]

\usepackage{fontspec}
% \usepackage{unicode-math}
% \usefonttheme{professionalfonts} % let fontspec take control

\setmonofont{Fira Code}



% PACKAGES
\usepackage[utf8]{inputenc}
\usepackage[T1]{fontenc}
% \usepackage{lmodern}
\usepackage{amsmath,amssymb,mathtools}
\usepackage{booktabs}
\usepackage{array}

\usepackage{tikz}
\usetikzlibrary{trees}

% MACROS
\newcommand{\seq}{\vdash}
\newcommand{\A}{\mathrm{A}}

\AtBeginSection[]{
  \begin{frame}
    \centering
    \vfill
    \Huge\insertsectionhead
    \vfill
  \end{frame}
}

\title{Lecture 4}
\author{Hans Halvorson}
\institute{Princeton University}
\date{September 29, 2025}

\begin{document}

% TITLE
\begin{frame}
  \titlepage
\end{frame}

\begin{frame}{Midterm Exam}

  \begin{itemize}
  \item Monday, October 6 at 1:20pm
  \item 80 minutes to complete exam
  \item Cheat sheet: You may bring one sheet of paper with whatever
    information you can fit on it (front and back)
  \item No precepts next week (after exam)
  \item No pset this week
  \item To do: Work on practice midterm
  \item To do: Practice problems
  \end{itemize}

\end{frame}

% ROADMAP
\begin{frame}{Plan for today}
  \begin{itemize}
    \setlength\itemsep{4pt}
  \item Not much new content --- mostly stuff that will help you
    become more confident with proofs.
  \item Semantics (truth-tables) again
    \begin{itemize}
    \item New: Biconditional
    \item New: Classification of sentences
    \end{itemize}
  \item Meta-rules for proofs
  \item Inferring the semantic type of compound sentences
  \end{itemize}
\end{frame}

\section{Semantics}

\begin{frame}{Truth table: Biconditional}
\begin{center}
\begin{tabular}{c c | c}
$P$ & $Q$ & $P \leftrightarrow Q$ \\
\hline
1 & 1 & 1 \\
1 & 0 & 0 \\
0 & 1 & 0 \\
0 & 0 & 1 \\
\end{tabular}
\end{center}

\medskip
The biconditional $P \leftrightarrow Q$ is true (1) exactly when $P$ and $Q$ have the same truth value.
\end{frame}

\begin{frame}{Semantic classification of sentences}

\begin{tabular}{>{\bfseries\color{blue}}r p{0.7\linewidth}}
Tautology:    & The column under the main connective is always True (1) \\
Inconsistency:& The column under the main connective is always False
                (0) \\
Contingency:  & The column under the main connective is a mix of True
                (1) and False (0)
\end{tabular}  

\end{frame}

\begin{frame}{Semantic classification of sentences}

\[ (P\leftrightarrow Q)\vee ((Q\leftrightarrow R)\vee
  (P\leftrightarrow R)) \]

\vspace{2em} This sentence is a tautology: for any three sentences
$P,Q,R$, at least two must have the same truth-value. 


\end{frame}


\begin{frame}{Equivalent sentences}

  Two sentences are said to be \textbf{logically equivalent} just in
  case they have the same truth-value in all rows of their joint truth
  table.

\[
  \begin{array}{c c || c@{\;}c@{\;}c | c@{\;}c@{\;}c@{\;}c}
    P & Q & P & \to & Q & \neg & P & \vee & Q \\
\hline
1 & 1 & 1 & 1 & 1 & 0 & 1 & 1 & 1 \\
1 & 0 & 1 & 0 & 0 & 0 & 1 & 0 & 0 \\
0 & 1 & 0 & 1 & 1 &  1 & 0 & 1 & 1 \\
0 & 0 & 0 & 1 & 0 & 1 & 0 & 1 & 0 \end{array}
\]

\end{frame}



\begin{frame}{Equivalent sentences}

\[
\begin{array}{c c || c@{\;}c@{\;}c@{\;}c@{\;}c | c@{\;}c@{\;}c@{\;}c }
P & Q & \neg & ( P & \to & Q ) & & P & \wedge & \neg & Q \\
\hline
1 & 1 & 0 & 1  & 1 & 1  &  & 1 & 0 & 0 & 1 \\
1 & 0 & 1 & 1  & 0 & 0  &  & 1 & 1 & 1 & 0 \\
0 & 1 & 0 & 0  & 1 & 1  &  & 0 & 0 & 0 & 1 \\
0 & 0 & 0 & 0  & 1 & 0  &  & 0 & 0 & 1 & 0 
\end{array}
\]  



\end{frame}

 \begin{frame}{Equivalent sentences}

    \begin{align*}
  P \to Q       &\;\equiv\; \neg P \vee Q \\
  \neg(P \to Q) &\;\equiv\; P \wedge \neg Q \\
    \neg(P \vee Q)&\;\equiv\; \neg P \wedge \neg Q \\
    \neg(P\wedge Q) &\;\equiv\; \neg P\vee\neg Q
\end{align*}

\end{frame}

 \begin{frame}{Equivalent sentences}

    \begin{align*}
  P\wedge Q       &\;\equiv\; Q\wedge P \\
  P\wedge P &\;\equiv\; P \\
    P\vee P &\;\equiv\; P \\
    P\to \neg P &\;\equiv\; \neg P
\end{align*}

\end{frame}


\section{Meta-theorems}

\begin{frame}{Summary}

  \begin{itemize}
  \item Soundness: If an argument form has a counterexample, then it
    cannot be proven.
  \item Completeness: If an argument form has no counterexample, then
    it can be proven.
  \item Cut: Proven sequents can act as \textbf{derived rules}.
  \item Replacement: Replacing a subformula of $\varphi$ with an
    equivalent subformula results in an equivalent formula
    $\varphi '$.
  \end{itemize}


\end{frame}


\begin{frame}{} 
  \begin{block}{Soundness}
    If the argument from $A_1,\dots ,A_j$ to $B$ is \textbf{not}
    truth-functionally valid (if it has a counterexample), then
    $A_1,\dots ,A_j\vdash B$ can \textbf{not} be proven.
  \end{block}
  
  \begin{block}{Completeness}
    If the argument from $A_1,\dots ,A_j$ to $B$ is truth-functionally
    valid, then there is a proof of $A_1,\dots ,A_j \seq B$.
  \end{block}
  
  \begin{itemize}
  \item If $A_1,\dots ,A_j\not\vDash B$, then no correct proof can end
    with $A_1,\dots ,A_j\; (n)\; B$.
  \item If $A_1,\dots ,A_j\vDash B$, then there is a correct proof
    that ends with that line.
  \end{itemize}
\end{frame}

\begin{frame}{Consequences of soundness and completeness}

  Two sentences are \textbf{logically equivalent} if and only if they
  are \textbf{inter-derivable}.

  \begin{align*}
    P \to Q       &\;\equiv\; \neg P \vee Q \\
    \neg(P \to Q) &\;\equiv\; P \wedge \neg Q \\
    \neg(P \vee Q)&\;\equiv\; \neg P \wedge \neg Q \\
    \neg(P\wedge Q) &\;\equiv\; \neg P\vee\neg Q
\end{align*}



\end{frame}

% EXAMPLE 1
\begin{frame}{Fragment check I}
  Can there be a correct proof with these line fragments? 

\bigskip \begin{tabular}{>{\raggedleft\arraybackslash}p{1.5cm} >{\centering\arraybackslash}p{1.0cm} p{5cm} >{\raggedright\arraybackslash}p{3.5cm}}
1 & (1) & $P\lor Q$ & A \\
  2 & (2) & $P\lor \neg Q$ & A \\
   & \,\vdots \\ 
1,2 & (n) & $P$ & \\
\end{tabular}

 \bigskip Yes, $P\lor Q,\ P\lor \neg Q\:\vDash \:P$ (easy
truth-table reasoning). By completeness, some proof exists.
\end{frame}


% EXAMPLE 3
\begin{frame}{Fragment check II: Explosion from inconsistency}

\bigskip \begin{tabular}{>{\raggedleft\arraybackslash}p{1.5cm} >{\centering\arraybackslash}p{1.0cm} p{8cm} >{\raggedright\arraybackslash}p{3.5cm}}
           1 & (1) & $\neg(P\leftrightarrow Q) \wedge (\neg(Q\leftrightarrow R) \wedge \neg(P\leftrightarrow R))$ & A \\
           & \,\vdots \\ 
1 & (n) & $P\wedge\neg P$ & \\
\end{tabular}

\bigskip Line 1 is inconsistent. From an inconsistency one can derive
any formula. By completeness, there is a correct proof to
$P\wedge\neg P$ depending only on 1.
\end{frame}

% EXAMPLE 4
\begin{frame}{Fragment check III: Tautology does not entail
    contingency}

  \begin{tabular}{>{\raggedleft\arraybackslash}p{1.5cm} >{\centering\arraybackslash}p{1.0cm} p{5cm} >{\raggedright\arraybackslash}p{3.5cm}}
  1 & (1) & $P\lor \neg P$ & A \\
  & \, \vdots \\ 
1 & (n) & $Q$ & \\
\end{tabular}

 \bigskip $P\lor\neg P$ is a tautology; $Q$ is a
contingency. Since $P\lor\neg P\nvDash Q$, soundness forbids such a
proof.
\end{frame}


\section{Derived rules}

\begin{frame}{Derived rules}

  \begin{itemize}
  \item The relationship between the basic rules and derived rules is
    like the relationship between machine language and a high-level
    programming language (such as Python).
  \item Your thinking can operate at two levels: you can use derived
    rules to find a path to a proof, and then fill out the details
    with basic rules.
  \item Two kinds of derived rules:
    \begin{itemize}
      \item \textbf{Cut:} Inference rules that operate on entire lines
      \item \textbf{Replacement:} Inference rules that operate on
        subformulas
     \end{itemize}
  \end{itemize}


\end{frame}

\begin{frame}{\textbf{Ex Falso Quodlibet} is a derived inference rule}

  \begin{tabular}{>{\raggedleft\arraybackslash}p{1.5cm} >{\centering\arraybackslash}p{1.0cm} p{5cm} >{\raggedright\arraybackslash}p{3.5cm}}
1 & (1) & $\neg P$ & A \\
2 & (2) & $P$ & A \\
3 & (3) & $\neg Q$ & A \\
1,2 & (4) & $P \wedge \neg P$ & 2,1 $\wedge$I \\
1,2 & (5) & $\neg \neg Q$ & 3,4 RA \\
1,2 & (6) & $Q$ & 5 DN \\
\end{tabular}



\end{frame}


\begin{frame}{\textbf{Negative paradox} is a derived inference rule}

   \begin{tabular}{>{\raggedleft\arraybackslash}p{1.5cm}
                >{\centering\arraybackslash}p{1.0cm}
                p{5cm}
               >{\raggedright\arraybackslash}p{5cm}}
               1 & (1) & $\neg P$ & A \\
               2 & (2) & $P$ & A \\
               1,2 & (3) & $Q$ & 1,2 EFQ \\
               1 & (4) & $P\to Q$ & 2,3 CP \end{tabular}



\end{frame}

\begin{frame}{\textbf{Chain order} from derived rules}

  $\vdash\: (P\to Q)\vee (Q\to P)$

  \bigskip \begin{tabular}{>{\raggedleft\arraybackslash}p{1.5cm}
                >{\centering\arraybackslash}p{1.0cm}
                p{5cm}
             >{\raggedright\arraybackslash}p{5cm}}
             $\varnothing$ & (1) & $Q\vee \neg Q$ & Excluded middle \\
             2 & (2) & $Q$ & A \\
             2 & (3) & $P\to Q$ & Positive paradox \\
             2 & (4) & $(P\to Q)\vee (Q\to P)$ & 3 $\vee$I \\
             5 & (5) & $\neg Q$ & A \\
             5 & (6) & $Q\to P$ & Negative paradox \\
             5 & (7) & $(P\to Q)\vee (Q\to P)$ & 6 $\vee$I \\
             $\varnothing$ & (8) & $(P\to Q)\vee (Q\to P)$ & 1,2,4,5,7
                                                           $\vee$E              
           \end{tabular}
         \end{frame}

\begin{frame}{Using derived rules}

  $P\to (Q\vee R)\: \vdash \: (P\to Q)\vee R$

  \bigskip \begin{tabular}{>{\raggedleft\arraybackslash}p{1.5cm}
                >{\centering\arraybackslash}p{1.0cm}
                p{5.5cm}
             >{\raggedright\arraybackslash}p{5.5cm}}
             1 & (1) & $P\to (Q\vee R)$ & A \\
             2 & (2) & $\neg (P\to Q)$ & A \\
             2 & (3) & $P$ & 2 Material conditional \\
             1,2 & (4) & $Q\vee R$ & 1,3 MP \\
             2 & (5) & $\neg Q$ & 2 Material conditional \\
             1,2 & (6) & $R$ & 4,5 Disjunctive syllogism \\
             1 & (7) & $\neg (P\to Q)\to R$ & 2,6 CP \\
             1 & (8) & $(P\to Q)\vee R$ & 7 Material conditional \end{tabular}

\end{frame}


\begin{frame}{Using derived rules}
$(P\wedge Q)\to R\:\vdash \: (P\to R)\vee (Q\to R)$

  \bigskip \begin{tabular}{>{\raggedleft\arraybackslash}p{1.5cm}
                >{\centering\arraybackslash}p{1.0cm}
                p{5.5cm}
             >{\raggedright\arraybackslash}p{6cm}}
             1 & (1) & $(P\wedge Q)\to R$ & A \\
             2 & (2) & $\neg (P\to R)$ & A \\
             2 & (3) & $\neg R$ & 2 Material conditional \\
             1,2 & (4) & $\neg (P\wedge Q)$ & 1,3 MT \\
             1,2 & (5) & $\neg P\vee \neg Q$ & 4 DeMorgans \\
             2 & (6) & $P$ & 2 Material conditional \\
             1,2 & (7) & $\neg Q$ & 5,6 Disjunctive syllogism \\
             1,2 & (8) & $Q\to R$ & 7 Negative paradox \\
             1 & (9) & $\neg (P\to R)\to (Q\to R)$ & 2,8 CP \\
             1 & (10) &  $(P\to R)\vee (Q\to R)$ & 9 Material
                                                   conditional \end{tabular}
                                                   
             
                                               \end{frame}

\section{Substitution instances}


\begin{frame}{Substitution instances}

  We implicitly assumed that proof rules should be read
  \textbf{schematically}: while written as $P\to Q,P\vdash P$ with
  specific propositional constants $P$ and $Q$, it applies to any
  sentences of these forms.

  \bigskip 

\begin{tabular}{>{\raggedleft\arraybackslash}p{1.5cm}
                >{\centering\arraybackslash}p{1.0cm}
                p{5cm}
  >{\raggedright\arraybackslash}p{5cm}}
  1 & (1) & $(P\wedge Q)\to (Q\to R)$ & A \\
  2 & (2) & $P\wedge Q$ & A \\
  1,2 & (3) & $Q\to R$ & 1,2 MP \end{tabular}

\bigskip More precisely: the rule applies to \textbf{substitution
  instance} of $P\to Q$ and $P$. 
\end{frame}

\begin{frame}{Substitution Instances}

\begin{block}{Definition}
A \textbf{substitution instance} of a formula schema is obtained by 
uniformly replacing its propositional variables with arbitrary 
sentences of propositional logic.
\end{block}

\bigskip

\textbf{Schema:} \quad $P \to Q$

\begin{itemize}
  \item Substitution $P := R \wedge S$, $Q := T$  
    \[
      (R \wedge S) \to T
    \]
  \item Substitution $P := \neg R$, $Q := (S \vee T)$  
    \[
      \neg R \to (S \vee T)
    \]
\end{itemize}

Each of these is a substitution instance of the schema $P \to Q$.
\end{frame}

\begin{frame}{What is \emph{not} a substitution instance?}

\begin{block}{Reminder}
  A substitution instance of a formula results from \emph{uniformly
    replacing} its propositional variables with formulas.  It does
  \emph{not} allow adding, deleting, or re-arranging structure.
\end{block}

\bigskip

\textbf{Not substitution instances:}

\begin{itemize}
\item $Q$ is not a substitution instance of $\neg P$. (We cannot
  ``drop'' the negation sign by substitution.)
\item $S \to T$ is not a substitution instance of $P \to (Q \to P)$.
  (No substitution for $P,Q$ will collapse the schema into $S \to T$.)
\end{itemize}

\bigskip

\textit{Moral:} Substitution preserves the \emph{tail form} of the
formula.
\end{frame}

\begin{frame}{Parse trees}

  A substitution instance of a formula results from extending the
  leaves in that formula's parse tree.

\bigskip \centering
\begin{tikzpicture}[level distance=1.2cm,
  every node/.style={font=\large},
  level 1/.style={sibling distance=5cm},
  level 2/.style={sibling distance=2.5cm}]

\node {$\to$}
  child {node {$\wedge$}
    child {node {$P$}}
    child {node {$Q$}}}
  child {node {$R$}};

\end{tikzpicture}

\end{frame}



\begin{frame}[fragile]{How to generate a substitution instance}

\begin{block}{Idea}
A substitution maps each propositional variable to a formula.
To generate a substitution instance, recursively replace variables.
\end{block}

\footnotesize
\textbf{Pseudo-Python:}
\begin{verbatim}
def substitute(formula, mapping):
    if is_var(formula):
        return mapping[formula]
    elif is_neg(formula):          # ¬φ
        return Neg(substitute(formula.arg, mapping))
    elif is_and(formula):          # φ ∧ ψ
        return And(substitute(formula.left, mapping),
                   substitute(formula.right, mapping))
    elif is_or(formula):           # φ ∨ ψ
        return Or(substitute(formula.left, mapping),
                  substitute(formula.right, mapping))
\end{verbatim}

\end{frame}

                                               


\begin{frame}{A substitution consequence}
  Substitution of $R\mapsto P\wedge Q$ in the provable sequent
  \[ (P\wedge Q)\to R\ \seq\ (P\to R)\lor (Q\to R) ,\] yields
  \[ (P\wedge Q)\to(P\wedge Q)\ \seq\ (P\to(P\wedge
    Q))\lor(Q\to(P\wedge Q)). \] Since the premise of the latter
  sequent is a tautology, its conclusion is a tautology.

\end{frame}
        

  
 \begin{frame}{Using already proven results}

   $\vdash\: (P\to (P\wedge Q))\vee (Q\to (P\wedge Q))$

    \bigskip \begin{tabular}{>{\raggedleft\arraybackslash}p{1.5cm}
                >{\centering\arraybackslash}p{1.0cm}
                p{7cm}
               >{\raggedright\arraybackslash}p{5cm}}
               $\varnothing$ & (1) & $Q\vee\neg Q$ & Excluded middle \\
               2 & (2) & $Q$ & A \\
               3 & (3) & $P$ & A \\
               2,3 & (4) & $P\wedge Q$ & 3,2 $\wedge$I \\
               2 & (5) & $P\to (P\wedge Q)$ & 2,4 CP \\
               6 & (6) & $\neg Q$ & A \\
               6 & (7) & $Q\to (P\wedge Q)$ & 6 Negative paradox \\
               $\varnothing$ & (8) & $(P\to (P\wedge Q))\vee (Q\to
                                     (P\wedge Q))$ & 1,2,5,6,7
                                                     $\vee$E$^*$ \end{tabular}
               
\end{frame}

\section{Replacement rules}

\begin{frame}{An unsound rule}

  $\wedge$E$^+$: Any subformula $P\wedge Q$ may be replaced by $P$.

  \bigskip \begin{tabular}{>{\raggedleft\arraybackslash}p{1.5cm}
                >{\centering\arraybackslash}p{1.0cm}
                p{5.5cm}
  >{\raggedright\arraybackslash}p{5.5cm}}
  1 & (1) & $(P\wedge Q)\to R$ & A \\
  1 & (2) & $P\to R$ & 1 $\wedge$E$^+$ \end{tabular}

\bigskip Line (2) is not semantically valid: if $P$ is true and $Q$
and $R$ are false, then the dependency is true but $P\to R$ is false.


\end{frame}

\begin{frame}{A sound rule}

  \textbf{Material conditional:} Any occurence of $P\to Q$ as a
  subformula may be replaced by $\neg P\vee Q$.

  \bigskip Why is this sound?

  \medskip  \begin{tabular}{>{\raggedleft\arraybackslash}p{2.5cm}
                >{\centering\arraybackslash}p{1.0cm}
                p{4.5cm}
              >{\raggedright\arraybackslash}p{5.5cm}}
              $m_1,\dots ,m_j$ & (m) & $\varphi$ \\
                               & \,\vdots & \\
              $m_1,\dots ,m_j$ & (n) & $\varphi [\neg P\vee Q/P\to Q]$
                               & Material conditional \end{tabular}


\end{frame}  

\begin{frame}{Replacement meta-rule}

\begin{block}{Statement}
  $\Gamma \vdash \varphi$ is provable if and only if
  $\Gamma \vdash \varphi'$ is provable, where $\varphi'$ is the result
  of replacing some \textbf{subformula} of $\varphi$ with a logically
  equivalent subformula.
\end{block}

\bigskip

\textbf{Example:}
\[
  \neg (P\to Q) \;\;\equiv\;\; P\wedge\neg Q 
\]

So $\Gamma \vdash \neg (P\to Q) \to R$ if and only if
$\Gamma \vdash (P\wedge \neg Q)\to R$.
\end{frame}


\begin{frame}{Useful equivalences}

  \[ \begin{aligned}
       P\to Q &\equiv\; \neg P\vee Q \\
       \neg (P\to Q) & \equiv\; P\wedge\neg Q \\
       P\to Q & \equiv\; \neg Q\to\neg P \\
       \neg (P\vee Q) & \equiv\; \neg P\wedge \neg Q \\
       \neg (P\wedge Q) & \equiv\; \neg P\vee\neg Q \\
       P\leftrightarrow Q & \equiv\: (P\wedge Q)\vee (\neg
                            P\wedge\neg Q) \end{aligned} \]



\end{frame}

\begin{frame}{Useful equivalences}

  \[ \begin{aligned}
       P\vee Q &\equiv\; Q\vee P \\
       P\vee (Q\vee R) & \equiv\: (P\vee Q)\vee R \\
       P\vee P & \equiv\: P \end{aligned} \]


\end{frame}

\begin{frame}{Useful equivalences}

  \[ \begin{aligned}
       P\to (Q\to R) &\equiv\; (P\wedge Q)\to R \\
       P\wedge (Q\vee R) & \equiv\; (P\wedge Q)\vee (P\wedge R) \\
       P\vee (Q\wedge R) & \equiv\; (P\vee Q)\wedge (P\vee
                           R) \end{aligned} \]

\end{frame}


\begin{frame}{Chain of equivalences}

\[
\begin{aligned}
 (P \wedge Q) \to R 
   &\;\equiv\; P \to (Q \to R) \\
   &\;\equiv\; \neg P \vee (\neg Q \vee R) \\
   &\;\equiv\; \neg P \vee (\neg Q \vee (R \vee R)) \\
   &\;\equiv\; (\neg P \vee R) \vee (\neg Q \vee R) \\
   &\;\equiv\; (P \to R) \vee (Q \to R)
\end{aligned}
\]

\end{frame}

\begin{frame}{Proofs with replacement rules}

\begin{tabular}{>{\raggedleft\arraybackslash}p{2.5cm}
                >{\centering\arraybackslash}p{1.0cm}
                p{5.5cm}
  >{\raggedright\arraybackslash}p{5.5cm}}
  $\varnothing$ & (1) & $P\vee \neg P$ & Excluded middle \\
  $\varnothing$ & (2) & $(\neg P\vee Q)\vee (\neg Q\vee P)$ & 1 $\vee$I \\
  $\varnothing$ & (3) & $(P\to Q)\vee (Q\to P)$ & 2 Material conditional \end{tabular}
                                                  
          
                             

\end{frame}

\section{Translation aided by semantics}

\begin{frame}

  I will leave Princeton unless they give me a substantial raise.

  \bigskip Option 1: $R\vee \neg P$ \\
  Option 2: $\neg R\to \neg P$ \\
  Option 3: $R\to P$ \\
  Option 4: $\neg R\leftrightarrow \neg P$ \\
  Option 5: $R\leftrightarrow P$ 


\end{frame}

\begin{frame}

  I will stay at Princeton only if they give me a substantial raise.

  \bigskip Option 1: $P\to R$ \\
  Option 2: $R\to P$ \\
  Option 3: $P\leftrightarrow R$

\end{frame}

\begin{frame}

  Desmond is either in Princeton or in Queens.

  \bigskip Option 1: $P\vee Q$ \\
  Option 2: $P\leftrightarrow \neg Q$ \\
  Option 3: $(P\vee Q)\wedge \neg (P\wedge Q)$

\end{frame}

\section{Inferring types of sentences}

\begin{frame}{Type of $\Phi\lor\Psi$ when both contingencies}
\begin{itemize}
  \item Cannot be an inconsistency (since $\Phi$ is true on some row, making $\Phi\lor\Psi$ true there).
  \item Could be a contingency (e.g. $P\lor Q$).
  \item Could be a tautology (e.g. $P\lor\neg P$).
\end{itemize}
\end{frame}

\begin{frame}{Type of $\Phi\to\Psi$ when $\Phi$ is a tautology}
If $\Phi$ is a tautology, then $\Phi\to\Psi\equiv\Psi$. Therefore $\Phi\to\Psi$ has the same type as $\Psi$ (contingency if $\Psi$ is).
\medskip

\textbf{Exercise.} Build a $3\times3$ table for $\Phi\to\Psi$ over the cases where each of $\Phi,\Psi$ is a tautology, inconsistency, or contingency.
\end{frame}



\begin{frame}{Wrap-up}
\begin{itemize}
\item Soundness/Completeness connect proofs to truth-tables, giving
  another way to discern logical relations.
\item Using standard moves (e.g.\ material conditional) plus
  cut/replacement can transform difficult proofs into routine
  exercises.
\item When translating, consider whether the target sentence has the
  intended logical relations.
\end{itemize}
\end{frame}

\end{document}
%%% Local Variables:
%%% mode: latex
%%% TeX-master: t
%%% End:
