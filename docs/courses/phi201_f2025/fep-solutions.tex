\documentclass[12pt,fleqn]{article}
\usepackage{fullpage}
\usepackage{amsmath,amssymb}
\setlength{\parindent}{0em}
\begin{document}

\thispagestyle{empty}

%% Hey, remember next time to PRINT OUT grade sheets

\section*{PHI 201, Practice Final Exam (Solutions)} 

\textbf{Instructions:} Write your name, preceptor's name, and pledge
on the exam booklet. Write your answers \emph{legibly} in the exam
booklet. While you may take up to three hours to complete the exam, it
was designed to take no more than two.

% When finished, place your exam booklet in the box at the front of the
% room, and leave quietly.

\begin{enumerate}

\item Translate the following into predicate logic.  You can assume
  that the domain is people, and so you don't need an additional
  predicate symbol for ``$x$ is a person''.

\begin{enumerate}
\item There is a person who loves all people who love her.  (Use $Lxy$
  for ``$x$ loves $y$''.)

\item Every lover loves herself.

\item There are exactly two people.

\end{enumerate}

\item Could the following sentence be true?  Explain your answer.
  \[ (\neg P\vee Q)\wedge ((Q\to (\neg R\wedge \neg P))\wedge (P\vee R)) \]

% \item Grade the following attempted proof:

%   \begin{tabular}{lllll}
% 1 & (1) & $p\vee q$ & & A \\
% 2 & (2) & $p$ & & A \\
% 3 & (3) & $q$ & & A \\
% 2,3 & (4) & $p\wedge q$ & & 2,3 $\andd$I \\
% 2,3 & (5) & $p$ & & 4 $\andd$E \\
% 1 & (6) & $p$ & & 1,2,2,3,5 $\orr$E \end{tabular}

\item Explain what's wrong with the following attempted proof:

  \bigskip \begin{tabular}{lllll}
  1 & (1) $Fa$ & & A \\
  $\varnothing$  & (2) $Fa \to Fa$ & & 1,1 CP \\
  $\varnothing$  & (3) $\forall y(Fy \to Fa)$ & & 2 UI \\
  $\varnothing$  & (4) $\exists x\forall y(Fy\to Fx)$ & & 3 EI \end{tabular}

\item Prove the following sequent.  You can use ``cut'' or
  ``replacement'', but only if you prove the relevant sequents in your
  exam booklet. 
  \[ \vdash\: \exists x\forall y(Fy\to Fx) \]

\item Prove the following fact of set theory:
  \[ C\setminus (A\cap B) \: \subseteq \: (C\setminus A)\cup (C\setminus B) ,
  \]
  where $C\setminus X$ is defined by
  \[ \forall x((x\in (C\setminus X))\leftrightarrow (x\in C\wedge
    x\not\in X)) .\] Your proof should be rigorous, but it can
  (preferably) be written in English prose.

  We need to show that every $a$ in $C\setminus (A\cap B)$ is either
  in $C\setminus A$ or in $C\setminus B$. So suppose that
  $a\in C\setminus (A\cap B)$, but $a\not\in C\setminus A$. The latter
  implies that either $a\not\in C$ or $a\in A$. Since
  $a\in C\setminus (A\cap B)$, it follows that $a\in C$ but
  $a\not\in A\cap B$. If $a\in B$, then $a\in A$ and $a\in B$, in
  contradiction with the fact that $a\not\in A\cap B$. Therefore
  $a\not\in B$, which means that $a\in C\setminus B$. We have shown
  that if $a\in C\setminus (A\cap B)$, then if $a\not\in C\setminus A$
  then $a\in C\setminus B$. The latter conditional is equivalent to
  $a\in C\setminus A$ or $a\in C\setminus B$, which is equivalent to
  $a\in (C\setminus A)\cup (C\setminus B)$. So we have
  \[ a\in C\setminus (A\cap B) \,\to \, a\in (C\setminus A)\cup
    (C\setminus B) ,\] for arbitrary $a$. By UI and the definition of
  $\subseteq$ we have
  \[ C\setminus (A\cap B)\subseteq (C\setminus A)\cup (C\setminus B) .\]

\item Let $\Gamma$ be the set of sentences defined inductively by:
  \begin{itemize}
  \item $P\in \Gamma$
  \item If $\varphi\in\Gamma$ then $\neg\varphi\in\Gamma$.
  \item If $\varphi\in \Gamma$ and $\psi\in \Gamma$ then
    $\varphi\to\psi\in \Gamma$.
  \end{itemize} Show that for every $\varphi\in\Gamma$, either
  $P\vdash \varphi$ or $P\vdash\neg\varphi$.

  We argue by induction that $\forall \varphi D(\varphi )$, where $D$
  is the property that either $P\vdash\varphi$ or $P\vdash\neg\varphi$.

  Base case: $P$ has property $D$ since $P\vdash P$ (Rule of
  Assumptions).

  Inductive step: Suppose that both $\varphi$ and $\psi$ have property
  $D$. We show that $\varphi\to\psi$ also has property $D$. We break
  it down into two cases. In the first case, either
  $P\vdash\neg\varphi$ or $P\vdash\psi$. In both cases
  $P\vdash\varphi\to\psi$, either by negative paradox or by positive
  paradox. In the second case, $P\vdash\varphi$ and
  $P\vdash\neg\psi$. In this case, $P\vdash\varphi\wedge\neg\psi$, and
  so $P\vdash\neg (\varphi\to\psi )$. The latter step can be seen by
  doing RA on $\varphi\wedge\neg\psi$ and $\neg (\varphi\to\psi )$. We
  have shown that in all cases, either $P\vdash \varphi\to\psi$ or
  $P\vdash\neg (\varphi\to\psi )$; that is, $\varphi\to\psi$ has
  property $D$. This completes the inductive step.

  
\item Provide a countermodel to show that the following sequent cannot
  be proven. 
\[ \exists x(Fx\to \exists yGy)\:\vdash\: \exists xFx\to \exists yGy \]

\end{enumerate}

\end{document}


 

\textbf{Proofs and Counterexamples} [20 points total]

\begin{enumerate}
\item Prove the following tautology.  (You may use SI only for things
  provable in propositional logic, and for the quantifier-negation
  equivalences.)  [8 points]

$$\vdash \; \exists x\forall y (Fy\to Fx)$$

% \item{2.} Prove the following using the basic rules of inference.
%   (If you use Sequent Introduction, then also prove the cited
%   sequent(s).)
% $$ (x)(y)[(\exists z)Ryz \to Rxy] \; \vdash \; (\exists x)(\exists
% y)Rxy \to (x)(y)Rxy$$

\item The two sentences ``$\forall x\forall y(Rxy \to Ryx)$'' and
  ``$\forall x(Fx\leftrightarrow \exists yRxy )$'' together imply one
  of (a) and (b) below, but not the other.  Find which is implied and
  show the implication by giving a proof (using SI only for
  propositional logic proofs or the quantifier-negation equivalences).
  Show that lack of implication in the other case by giving a suitable
  interpretation.
\begin{enumerate}
\item $\forall xFx\to \exists x\forall yRxy$
\item $\exists x\forall yRxy \to \forall xFx$
\end{enumerate}

\end{enumerate}



\textbf{Metatheory} [6 points each; 12 points total] Please complete
\underbar{two} of the following three problems.  If you give solutions
to all three, please clearly designate which two problems you want to
be graded.
\begin{enumerate}
\item State and prove the soundness of Conditional Proof (CP) relative
  to propositional logic interpretations (i.e.\ truth tables).  That
  is, show that CP takes ``good lines'' to ``good lines.''

% \item We showed in lecture (and in the handout) that if $A$ is an
%   inconsistent sentence of propositional logic, then there is a proof
%   $A\vdash P\wedge \neg P$.  Use this fact to show that our proof
%   rules for propositional logic are \emph{complete} relative to truth
%   tables, that is if $A_1,\dots A_n\models B$ then $A_1,\dots
%   ,A_n\vdash B$.

\item Let $\Sigma$ be the set of sentences of propositional logic
  whose only atomic sentence is $P$.  Show that for any consistent
  sentence $A$ in $\Sigma$, either $P\vdash A$ or $\neg P\vdash A$.
  (You may cite any result that we established during the semester.)

\item Prove that the set $\{ \wedge ,\rightarrow \}$ of connectives is
  \emph{not} truth-functionally complete.

% \item True or False (explain your answer): If a propositional logic
%   argument is semantically valid (i.e. $A_1,\dots ,A_n\models B$), and
%   if all occurrences of an atomic sentence $P$ in the argument are
%   replaced with some other (possibly non-atomic) sentence $X$, then
%   the resulting argument is also semantically valid.

\end{enumerate}






\begin{center} -- THE END -- \end{center}

\end{document}

%%% Local Variables:
%%% mode: latex
%%% TeX-master: t
%%% End:
