\documentclass[fleqn,12pt]{article}
\usepackage{amsmath}
\usepackage[colorlinks]{hyperref}
\setlength{\parindent}{0em}
\setlength{\parskip}{1em}
\begin{document}
\thispagestyle{empty}

\section*{Logic pset 6}

Resources: HLW \href{https://doi.org/10.2307/j.ctvxrpz0q.10}{Ch 7} pp
116-127

\begin{enumerate}
\item Represent the form of the following sentences in predicate logic
  using the $=$ symbol where necessary.
  \begin{enumerate}
  \item There is one and only one Princeton University. (Use $Px$ for
    ``$x$ is a Princeton University'')
  \item There is at most one Ivy League university in New Jersey. (Use
    $Ix$ for ``$x$ is an Ivy League university'', and use $Nx$ for
    ``$x$ is in New Jersey.'')
  \item There is a smallest prime number. ($Px,Sxy$, variables are
    restricted to numbers.)
 \end{enumerate}

\item Prove the following sequents using any of the rules, including
  $=$E and $=$I.
  \begin{enumerate}
  \item $\exists x(Px\wedge \forall y(Py\to x=y))\: \vdash \: \forall
    x\forall y((Px\wedge Py)\to x=y)$
  \item $\vdash\: \forall x\forall y((x=y)\to (y=x))$
  \end{enumerate}

\item Let $Rxy$ be a binary relation symbol that satisfies the
  transitivity axiom (page 126). Suppose that $Rxy$ satisfies two
  other axioms: serial $\forall x\exists yRxy$ and irreflexive
  $\forall x\neg Rxx$. Show that there are at least three distinct
  things, i.e.,
  \[ \exists x\exists y\exists z((x\neq y\wedge x\neq z)\wedge y\neq
    z) .\] It would also suffice to show that the claim ``there are at
  most two things'' contradicts the assumptions. You may write your
  proof in English prose (not our formal system), but you need to
  convince the reader that you would be able to write a full formal
  proof.
\end{enumerate}



\end{document}


%%% Local Variables:
%%% mode: latex
%%% TeX-master: t
%%% End:
