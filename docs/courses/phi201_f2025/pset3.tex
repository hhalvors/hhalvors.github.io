% Options for packages loaded elsewhere
\documentclass[12pt]{article}
\setlength{\parindent}{0em}
\setlength{\parskip}{1em}
\usepackage{fullpage,amssymb,array}

\title{logic pset3}
\author{}
\date{}

\begin{document}
\maketitle

\thispagestyle{empty}

\vspace*{-3em}

Resources: Lecture 3 and Chapters 3 and 5 of \emph{How Logic
  Works}. (Note that we are skipping over Chapter 4 for now.)

\section*{A. Proofs}

Use any of the rules of inference, including reductio ad absurdum, to
prove the following sequents.

\begin{enumerate}
\item $\neg (P\to Q)\:\vdash\: Q\to R$
\item $P\to Q\:\vdash\:\neg P\vee Q$
\item $P\to (Q\vee R)\:\vdash\: (P\to Q)\vee R$
\end{enumerate}

\section*{B. Truth tables}

\begin{enumerate}
\item Use truth table reasoning to show that
  $P\vee (Q\wedge R)\:\vDash\: P\vee Q$. You don't have to display a
  full truth table, but if you do, explain how the table demonstrates
  the result.
\item Use truth table reasoning to show that
  $P\to (Q\vee R)\:\not\vDash \:P\to Q$.
\item Use truth table reasoning to show that the following ``proof''
  must have a mistake.

  \medskip \begin{tabular}{>{\raggedleft\arraybackslash}p{1.5cm}
      >{\centering\arraybackslash}p{1cm} p{4cm}
      >{\raggedright\arraybackslash}p{3.5cm}}
                1   & (1) & $P\vee Q$  &    A \\
           2   & (2) & $P$    &   A \\
           3   & (3) & $Q$    &   A \\
           2,3 & (4) & $P\wedge Q$  & 2,3 $\wedge$I \\
           2,3 & (5) & $P$    &    4 $\wedge$E \\
           1 & (6) & $P$ & 1,2,2,3,5 $\vee$E
\end{tabular}
\end{enumerate}
  

\end{document}

%%% Local Variables:
%%% mode: latex
%%% TeX-master: t
%%% End:
