% Options for packages loaded elsewhere
\PassOptionsToPackage{unicode}{hyperref}
\PassOptionsToPackage{hyphens}{url}
\documentclass[
  12pt,
]{article}
\usepackage{xcolor}
\usepackage{amsmath,amssymb}
\setcounter{secnumdepth}{-\maxdimen} % remove section numbering
\usepackage{iftex}
\ifPDFTeX
  \usepackage[T1]{fontenc}
  \usepackage[utf8]{inputenc}
  \usepackage{textcomp} % provide euro and other symbols
\else % if luatex or xetex
  \usepackage{unicode-math} % this also loads fontspec
  \defaultfontfeatures{Scale=MatchLowercase}
  \defaultfontfeatures[\rmfamily]{Ligatures=TeX,Scale=1}
\fi
\usepackage{lmodern}
\ifPDFTeX\else
  % xetex/luatex font selection
  \setmainfont[]{TeX Gyre Termes}
  \setmonofont[]{Menlo}
\fi
% Use upquote if available, for straight quotes in verbatim environments
\IfFileExists{upquote.sty}{\usepackage{upquote}}{}
\IfFileExists{microtype.sty}{% use microtype if available
  \usepackage[]{microtype}
  \UseMicrotypeSet[protrusion]{basicmath} % disable protrusion for tt fonts
}{}
\makeatletter
\@ifundefined{KOMAClassName}{% if non-KOMA class
  \IfFileExists{parskip.sty}{%
    \usepackage{parskip}
  }{% else
    \setlength{\parindent}{0pt}
    \setlength{\parskip}{6pt plus 2pt minus 1pt}}
}{% if KOMA class
  \KOMAoptions{parskip=half}}
\makeatother
\setlength{\emergencystretch}{3em} % prevent overfull lines
\providecommand{\tightlist}{%
  \setlength{\itemsep}{0pt}\setlength{\parskip}{0pt}}
\usepackage{fullpage}
\usepackage{bookmark}
\IfFileExists{xurl.sty}{\usepackage{xurl}}{} % add URL line breaks if available
\urlstyle{same}
\hypersetup{
  pdftitle={logic precept 2},
  pdfauthor={}
  hidelinks,
  pdfcreator={LaTeX via pandoc}}

\title{logic precept 2}
\author{ }
\date{}

\usepackage{titling}

\setlength{\droptitle}{-4em} % move title up (adjust as needed)

\usepackage{array}

\begin{document}
\maketitle

\vspace*{-4em}
\section{Warmup: Deducing}

\textbf{Exercise 2.5} ($\wedge$E, $\wedge$I, $\vee$I, MP, MT, DN)

$P\to \neg Q,Q\: \vdash \: \neg P$

$\neg \neg P\: \vdash \: \neg \neg P\wedge (P\vee Q)$

$\neg (P\wedge Q)\to R,\neg R\:\vdash \: P$

$\neg P\to\neg Q,Q\:\vdash \: P$

$P\:\vdash \: \neg \neg (P\vee Q)$

\section{New proof rules}

We will work in blocks. The first block leads to problem A1.

$P\to Q\:\vdash\: P\to (Q\vee R)$

$(P\vee Q)\to R\:\vdash \: P\to R$

$P\to Q\:\vdash\: (R\to P)\to (R\to Q)$

$P\to Q\:\vdash\: (Q\to R)\to (P\to R)$

\textbf{(A1)} $P\to (Q\to R)\:\vdash\: Q\to (P\to R)$

The second block leads to A2. The focus is on the ``contrapositive
maneuver''.

$P\to Q\:\vdash\: \neg Q\to \neg P$

$\:\vdash\: P\to (P\vee Q)$

$\neg (P\vee Q)\:\vdash\: \neg P$

$\:\vdash\: (P\wedge Q)\to P$

\textbf{(A2)} $\neg P\:\vdash\: \neg (P\wedge Q)$

The third block leads to A3 and A4.

$P\:\vdash\: (P\to Q)\to Q$

\textbf{(A3)} $P\:\vdash\: (P\to \neg P)\to \neg P$

\textbf{(A4)} $Q\:\vdash\: \neg (Q\to\neg Q)$

The fourth block leads to problem B1.

$P\vee Q\:\vdash\: Q\vee P$

$P\vee (Q\wedge R)\:\vdash\: (P\wedge Q)\vee (P\wedge R)$

\textbf{(B1)} $P\wedge (Q\vee R)\:\vdash\: (P\wedge Q)\vee (P\wedge R)$

The fifth block leads to problem B2.

$P,\neg P\:\vdash\: Q$

\textbf{(B2)} $P\vee Q,\neg P\:\vdash\: Q$



\section{Evaluating proofs}\label{evaluating-proofs}

\textbf{Exercise} Which of the following proofs with CP is correct? If
a proof is not correct, explain what is wrong with it, and say whether
there is a simple fix, or whether it is fatally flawed. 

  \begin{tabular}{>{\raggedleft\arraybackslash}p{1.5cm} >{\centering\arraybackslash}p{1cm} p{4cm} >{\raggedright\arraybackslash}p{3.5cm}}
\textbf{Deps} & \textbf{Line} & \textbf{Formula} & \textbf{Justification} \\ \hline
1   & (1) & $P\wedge Q$ & A \\
1   & (2) & $P$ & 1 $\wedge$E \\
1   & (3) & $Q$ & 2 $\wedge$E \\ 
              & (4) & $P\to Q$ & 2,3 CP
  \end{tabular}

\medskip \begin{tabular}{>{\raggedleft\arraybackslash}p{1.5cm} >{\centering\arraybackslash}p{1cm} p{4cm} >{\raggedright\arraybackslash}p{3.5cm}}
\textbf{Deps} & \textbf{Line} & \textbf{Formula} & \textbf{Justification} \\ \hline  
1  & (1) & $Q$ & A \\
2  & (2) & $P$ & A \\
1  & (3) & $P\to Q$ & 2,1 CP \end{tabular}

\bigskip \textbf{Exercise} Explain what is wrong with the following
``proof''.

\medskip \begin{tabular}{>{\raggedleft\arraybackslash}p{1.5cm}
    >{\centering\arraybackslash}p{1cm} p{4cm}
           >{\raggedright\arraybackslash}p{3.5cm}}
     \textbf{Deps} & \textbf{Line} & \textbf{Formula} & \textbf{Justification} \\ \hline        
           1   & (1) & $P\vee Q$  &    A \\
           2   & (2) & $P$    &   A \\
           3   & (3) & $Q$    &   A \\
           2,3 & (4) & $P\wedge Q$  & 2,3 $\wedge$I \\
           2,3 & (5) & $P$    &    4 $\wedge$E \\
           1 & (6) & $P$ & 1,2,2,3,5 $\vee$E
\end{tabular}

\section{Additional practice
  problems}

$\neg P\vee \neg Q\:\dashv\vdash\: \neg (P\wedge Q)$

$P\to (P\to Q)\:\vdash\: P\to Q$

$(P\vee Q)\to R\:\vdash\: P\to R$

$P\to (Q\to R),P\to Q\:\vdash \: P\to R$

$P\to (Q\to R)\:\vdash\: (P\to Q)\to (P\to R)$

$(P\to Q)\to P\:\vdash\: (P\to Q)\to Q$

$(P\to Q)\to P\: \vdash \: \neg P\to P$

$(P\to R)\wedge (Q\to R)\:\vdash\: (P\vee Q)\to R$

$P\vee (Q\vee R)\:\dashv\vdash\: (P\vee Q)\vee R$

$P\wedge (Q\vee R)\:\dashv\vdash\: (P\wedge Q)\vee (P\wedge R)$

$P\vee (Q\wedge R)\:\dashv\vdash\: (P\vee Q)\wedge (P\vee R)$

$\neg P\vee Q\:\dashv\vdash\: P\to Q$

$\neg (P\to Q)\:\dashv\vdash\: P\wedge \neg Q$

$\vdash (P\to Q)\vee (Q\to P)$

$P\to (Q\vee R)\:\vdash\: (P\to Q)\vee (P\to R)$

$\vdash\: ((P\to Q)\to P)\to P$ (Hint: One possibility is to first
prove $\vdash P\vee \neg P$, and then argue by cases. The first case
is easy if you remember ``positive paradox''. For the second case,
remember ``negative paradox'', i.e.~that $\neg P$ implies $P\to
  Q$.)

$P\to (Q\vee R)\:\vdash \: \neg R\to (\neg Q\to \neg P)$

$P\to\neg P\:\dashv\vdash\:\neg P$

$(P\to Q)\to Q\:\vdash\: (Q\to P)\to P$

$(P\to Q)\to R\:\vdash\: (P\to R)\to R$

$(P\to R)\to R\:\dashv\vdash\: P\vee R$ (Hint: derive $\neg P\to R$
from the sentence on the left.)

$(P\to Q)\to P\:\dashv\vdash \: P$ (Hint: assume $\neg P$ and derive
$P\to Q$.)

\end{document}

%%% Local Variables:
%%% mode: latex
%%% TeX-master: t
%%% End:
