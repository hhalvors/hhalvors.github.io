\documentclass[fleqn,12pt]{article}
\usepackage{amsmath,amssymb}
\usepackage{fullpage}
\usepackage[colorlinks]{hyperref}
\setlength{\parindent}{0em}
\setlength{\parskip}{1em}
\usepackage{enumitem}
\usepackage{array}
\begin{document}
\thispagestyle{empty}

\section*{Logic pset 6}

Resources: HLW \href{https://doi.org/10.2307/j.ctvxrpz0q.10}{Ch 7} pp
116-127

\begin{enumerate}[leftmargin=*, labelindent=0pt]  
\item Represent the form of the following sentences in predicate logic
  using the $=$ symbol where necessary.
  \begin{enumerate}
  \item There is one and only one Princeton University. (Use $Px$ for
    ``$x$ is a Princeton University'')
    
  \[ \exists x(Px\wedge \forall y(Py\to y=x)) \]
    
  \item There is at most one Ivy League university in New Jersey. (Use
    $Ix$ for ``$x$ is an Ivy League university'', and use $Nx$ for
    ``$x$ is in New Jersey.'')

  \[ \forall x\forall y(((Ix\wedge Nx)\wedge (Iy\wedge Ny))\to x=y) \]
    
  \item There is a smallest prime number. ($Px,Sxy$, variables are
    restricted to numbers.)

    \[ \exists x(Px\wedge \forall y((Py\wedge y\neq x)\to Sxy)) \]

    This translation does not capture the uniqueness that may be
    implicit in ``smallest''. But if it's written in a context where
    $Sxy$ is a linear order, then there can only be one $x$ with the
    feature that $\forall y((Py\wedge y\neq x)\to Sxy)$.
    
 \end{enumerate}

\item Prove the following sequents using any of the rules, including
  $=$E and $=$I.
  \begin{enumerate}
  \item $\exists x(Px\wedge \forall y(Py\to x=y))\: \vdash \: \forall
    x\forall y((Px\wedge Py)\to x=y)$

    \bigskip  \noindent\hspace*{-1.5cm}%
\begin{tabular}{@{}>{\raggedleft\arraybackslash}p{1.5cm}
                >{\centering\arraybackslash}p{1.0cm}
                 p{7cm}
                >{\raggedright\arraybackslash}p{3.5cm}@{}}
1 & (1) & $\exists x (Px \wedge \forall y (Py \to x = y))$ & A \\
2 & (2) & $Pc \wedge \forall y (Py \to c = y)$ & A \\
3 & (3) & $Pa \wedge Pb$ & A \\
3 & (4) & $Pa$ & 3 $\wedge$E \\
3 & (5) & $Pb$ & 3 $\wedge$E \\
2 & (6) & $\forall y (Py \to c = y)$ & 2 $\wedge$E \\
2 & (7) & $Pa \to c = a$ & 6 UE \\
2 & (8) & $Pb \to c = b$ & 6 UE \\
2,3 & (9) & $c = a$ & 7,4 MP \\
2,3 & (10) & $c = b$ & 8,5 MP \\
2,3 & (11) & $a = b$ & 10,9 =E \\
2 & (12) & $(Pa \wedge Pb) \to a = b$ & 3,11 CP \\
2 & (13) & $\forall y ((Pa \wedge Py) \to a = y)$ & 12 UI \\
2 & (14) & $\forall x \forall y ((Px \wedge Py) \to x = y)$ & 13 UI \\
1 & (15) & $\forall x \forall y ((Px \wedge Py) \to x = y)$ & 1,2,14 EE \\
\end{tabular}



    
\item $\vdash\: \forall x\forall y((x=y)\to (y=x))$

\bigskip \begin{tabular}{>{\raggedleft\arraybackslash}p{1.5cm} >{\centering\arraybackslash}p{1.0cm} p{5cm} >{\raggedright\arraybackslash}p{3.5cm}}
1 & (1) & $a = b$ & A \\
$\varnothing$ & (2) & $a = a$ & =I \\
1 & (3) & $b = a$ & 2,1 =E \\
$\varnothing$ & (4) & $a = b \to b = a$ & 1,3 CP \\
$\varnothing$ & (5) & $\forall y (a = y \to y = a)$ & 4 UI \\
$\varnothing$ & (6) & $\forall x \forall y (x = y \to y = x)$ & 5 UI \\
\end{tabular}

  
  \end{enumerate}

\item Let $Rxy$ be a binary relation symbol that satisfies the
  transitivity axiom (page 126). Suppose that $Rxy$ satisfies two
  other axioms: serial $\forall x\exists yRxy$ and irreflexive
  $\forall x\neg Rxx$. Show that there are at least three distinct
  things, i.e.,
  \[ \exists x\exists y\exists z((x\neq y\wedge x\neq z)\wedge y\neq
    z) .\] It would also suffice to show that the claim ``there are at
  most two things'' contradicts the assumptions. You may write your
  proof in English prose (not our formal system), but you need to
  convince the reader that you would be able to write a full formal
  proof.

  \bigskip Here's a sub-proof that establishes that there are two
  things $a,b$. We use $a$ as an arbitrary name for a thing that bears
  the relation $R$ to some $y$. We then use $b$ as an arbitrary name
  for some $y$ to which $a$ bears the relation $R$. The irreflexivity
  axiom (4) entails that $a\neq b$.

\medskip \begin{tabular}{>{\raggedleft\arraybackslash}p{2.5cm} >{\centering\arraybackslash}p{1.0cm} p{8cm} >{\raggedright\arraybackslash}p{3.5cm}}
1 & (1) & $\forall x \exists y Rxy$ & A \\
1 & (2) & $\exists y Ray$ & 1 UE \\
3 & (3) & $Rab$ & A \\
4 & (4) & $\forall x \neg Rxx$ & A \\
4 & (5) & $\neg Rbb$ & 4 UE \\
6 & (6) & $a = b$ & A \\
3,6 & (7) & $Rbb$ & 3,6 =E \\
3,4,6 & (8) & $Rbb \wedge \neg Rbb$ & 7,5 $\wedge$I \\
3,4 & (9) & $a\neq b$ & 6,8 RA \\
         \end{tabular}

We now repeat the exact same process to establish that there is a $c$
such that $c\neq b$. 

 \medskip \begin{tabular}{>{\raggedleft\arraybackslash}p{2.5cm}
            >{\centering\arraybackslash}p{1.0cm} p{8cm}
            >{\raggedright\arraybackslash}p{3.5cm}}
1 & (10) & $\exists y Rby$ & 1 UE \\
11 & (11) & $Rbc$ & A \\
12 & (12) & $c = b$ & A \\
11,12 & (13) & $Rbb$ & 11,12 =E \\
4,11,12 & (14) & $Rbb \wedge \neg Rbb$ & 13,5 $\wedge$I \\
            4,11 & (15) & $c \neq b$ & 12,14 RA \end{tabular}

          \medskip Now we need to show that $c\neq a$. If $c=a$, then
          $Rba$, and the transitivity axiom implies that $Raa$.

 \medskip \begin{tabular}{>{\raggedleft\arraybackslash}p{2.5cm}
            >{\centering\arraybackslash}p{1.0cm} p{8cm}
            >{\raggedright\arraybackslash}p{3.5cm}}
16 & (16) & $c = a$ & A \\
11,16 & (17) & $Rba$ & 11,16 =E \\
18 & (18) & $\forall x \forall y \forall z ((Rxy \wedge Ryz) \to Rxz)$ & A \\
18 & (19) & $\forall y \forall z ((Ray \wedge Ryz) \to Raz)$ & 18 UE \\
18 & (20) & $\forall z ((Rab \wedge Rbz) \to Raz)$ & 19 UE \\
18 & (21) & $(Rab \wedge Rba) \to Raa$ & 20 UE \\
3,11,16 & (22) & $Rab \wedge Rba$ & 3,17 $\wedge$I \\
3,11,16,18 & (23) & $Raa$ & 21,22 MP \\
4 & (24) & $\neg Raa$ & 4 UE \\
3,4,11,16,18 & (25) & $Raa \wedge \neg Raa$ & 23,24 $\wedge$I \\
            3,4,11,18 & (26) & $c \neq a$ & 16,25 RA \end{tabular}

          \medskip To tidy up the proof, we need to collect everything
          together and do some steps of EI and EE. We end with
          dependency on the axioms $1,4,18$.

 \medskip \begin{tabular}{>{\raggedleft\arraybackslash}p{2.5cm}
            >{\centering\arraybackslash}p{1.0cm} p{8cm}
            >{\raggedright\arraybackslash}p{3.5cm}}
3,4,11 & (27) & $a\neq b \wedge c \neq b$ & 9,15 $\wedge$I \\
3,4,11,18 & (28) & $(a \neq b \wedge c \neq b) \wedge c \neq a$ & 27,26 $\wedge$I \\
3,4,11,18 & (29) & $\exists z ((a \neq b \wedge z \neq b) \wedge z
                   \neq a)$ & 28 EI \\
3,4,11,18 & (30) & $\exists y \exists z ((a\neq y \wedge z \neq y)
                   \wedge z \neq a)$ & 29 EI \\
3,4,11,18 & (31) & $\exists x \exists y \exists z ((x \neq y \wedge z
                   \neq y) \wedge z \neq x)$ & 30 EI \\
1,3,4,18 & (32) & $\exists x \exists y \exists z ((x \neq y \wedge z
                  \neq y) \wedge  z \neq x)$ & 10,11,31 EE \\
1,4,18 & (33) & $\exists x \exists y \exists z ((x \neq y \wedge z
                \neq y) \wedge z \neq x)$ & 2,3,32 EE \\
\end{tabular}
           
            

  

  
\end{enumerate}



\end{document}


%%% Local Variables:
%%% mode: latex
%%% TeX-master: t
%%% End:
