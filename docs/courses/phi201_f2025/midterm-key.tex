\documentclass[12pt,fleqn]{article}
\usepackage{fullpage}
\usepackage{amsmath}
\usepackage{array}
\renewcommand{\thesubsection}{\Alph{subsection}.}
\usepackage{enumitem}
\setlength{\parindent}{0em}
\usepackage{amssymb,amsmath}
\begin{document}

\section*{Intro Logic: Midterm Exam 2025 --- Answer Key}

\textbf{Instructions:} (same as on exam)

\subsection{Translation}

Translate the following sentences into propositional logic. In each case, clearly indicate what letters you are assigning to atomic sentences. (2 points each)

\begin{enumerate}[leftmargin=*]
\item If Bob does not exercise regularly or does not eat healthy
  meals, then Bob will not maintain good health.

\textbf{Answer:} Let $E$ = Bob exercises regularly; $H$ = Bob eats healthy meals; $G$ = Bob maintains good health.  
Translation: $(\neg E \vee \neg H) \to \neg G$.

\item Carla will increase her chances of admission only if Carla
submits her application early and asks for strong recommendation letters.

\textbf{Answer:} Let $I$ = Carla increases her chances of admission; $S$ = Carla submits her application early; $R$ = Carla asks for strong recommendation letters.  
Translation: $I \to (S \wedge R)$.

\item Either David gets a front-row seat, or if David does not go with
friends then David will not enjoy the concert.

\textbf{Answer:} Let $F$ = David gets a front-row seat; $K$ = David goes with friends; $J$ = David enjoys the concert.  
Translation: $F \vee (\neg K \to \neg J)$.
\end{enumerate}


\subsection{Semantics (truth tables)}

\begin{enumerate}[leftmargin=*]

\item For each of the following sentences, state whether it is a
  tautology, contingency, or inconsistency, and justify your claim in
  terms of truth tables. (3 points each)

  \begin{enumerate}
  \item $(P \wedge Q) \vee (\neg P \wedge \neg Q)$

    \textbf{Answer:} Contingency.  True when $P$ and $Q$ have the same
    truth value (true/true or false/false). False otherwise.

  \item $P \to (Q \to (R \to (S \to P)))$

    \textbf{Answer:} Tautology.  If $P$ is true, all nested
    implications yield true; if $P$ is false, the outer conditional is
    true. No false rows.
  \end{enumerate}
  
\item For each of the following arguments, state whether it is valid
  or invalid, and justify your claim in terms of truth tables. (3
  points each)
  \begin{enumerate}
  \item $P \to Q \:\vdash\: P \to (Q \wedge R)$  

    \textbf{Answer:} Invalid.  Counterexample: $P=1, Q=1, R=0$ makes
    the premise true and conclusion false.

  \item $Q \to R \:\vdash\: (P \vee Q) \to (P \vee R)$  

    \textbf{Answer:} Valid.  If the conclusion is false, then
    $P\vee Q$ is true and $P\vee R$ is false. The only way for this to
    be the case is if $Q$ is true and $R$ is false, which means that
    the premise $Q\to R$ is false.
  \end{enumerate}
\end{enumerate}


\subsection{Proofs}

Prove the following. Besides the basic rules, you may also use cut and
replacement, but only if you include a proof of the relevant
``lemmas'' in your exam booklet. (4 points each)

\begin{enumerate}[leftmargin=*]

\item $P,\neg P \:\vdash\: Q$

\begin{tabular}{>{\raggedleft\arraybackslash}p{1.5cm} >{\centering\arraybackslash}p{1.0cm} p{5cm} >{\raggedright\arraybackslash}p{3.5cm}}
1 & (1) & $P$ & A \\
2 & (2) & $\neg P$ & A \\
3 & (3) & $\neg Q$ & A \\
1,2 & (4) & $P \wedge \neg P$ & 1,2 $\wedge$I \\
1,2 & (5) & $\neg \neg Q$ & 3,4 RA \\
1,2 & (6) & $Q$ & 5 DN \\
\end{tabular}
  
\item $P \vee Q, \neg P \:\vdash\: Q$

\begin{tabular}{>{\raggedleft\arraybackslash}p{1.5cm} >{\centering\arraybackslash}p{1.0cm} p{5cm} >{\raggedright\arraybackslash}p{3.5cm}}
  1 & (1) & $P\vee Q$ & A \\
  2 & (2) & $\neg P$ & A \\
  3 & (3) & $P$ & A \\
  2,3 & (4) & $Q$ & 3,2 problem 1 \\
  5 & (5) & $Q$ & A \\
  1,2 & (6) & $Q$ & 1,3,4,5,5 $\vee$E 
\end{tabular}

\item $\neg P\to Q \:\vdash\: P\vee Q$

\begin{tabular}{>{\raggedleft\arraybackslash}p{1.5cm} >{\centering\arraybackslash}p{1.0cm} p{5cm} >{\raggedright\arraybackslash}p{3.5cm}}
1 & (1) & $\neg P \to Q$ & A \\
2 & (2) & $\neg (P \vee Q)$ & A \\
3 & (3) & $P$ & A \\
3 & (4) & $P \vee Q$ & 3 $\vee$I \\
2,3 & (5) & $(P \vee Q) \wedge \neg (P \vee Q)$ & 4,2 $\wedge$I \\
2 & (6) & $\neg P$ & 3,5 RA \\
1,2 & (7) & $Q$ & 1,6 MP \\
1,2 & (8) & $P \vee Q$ & 7 $\vee$I \\
1,2 & (9) & $(P \vee Q) \wedge \neg (P \vee Q)$ & 8,2 $\wedge$I \\
1 & (10) & $\neg \neg (P \vee Q)$ & 2,9 RA \\
1 & (11) & $P \vee Q$ & 10 DN \\
\end{tabular}

\item $(\neg P \vee Q) \to (P \vee Q) \:\vdash\: P \vee Q$

\begin{tabular}{>{\raggedleft\arraybackslash}p{1.5cm} >{\centering\arraybackslash}p{1.0cm} p{5cm} >{\raggedright\arraybackslash}p{3.5cm}}
1 & (1) & $(\neg P \vee Q) \to (P \vee Q)$ & A \\
2 & (2) & $\neg P$ & A \\
  2 & (3) & $\neg P\vee Q$ & 2 $\vee$I \\
  1,2 & (4) & $P\vee Q$ & 1,3 MP \\
  1,2 & (5) & $Q$ & 4,2 problem 2 \\
  1 & (6) & $\neg P\to Q$ & 2,5 CP \\
  1 & (7) & $P\vee Q$ & 6 problem 3 
\end{tabular}

\end{enumerate}


\subsection{Conceptual}

\begin{enumerate}[leftmargin=*]
  
\item Is there a correctly written proof with the following line
  fragment? Justify your answer by showing that the relevant argument
  is valid or invalid, and by invoking soundness or completeness. (4
  points)
  \[ \varnothing \vdash ((P\to Q)\to \neg P)\to \neg P \]
  
  \textbf{Answer:} No.  The sentence is not a tautology.
  Counterexample: $P=1, Q=0$ makes the formula false.  By soundness,
  no correct proof from $\varnothing$ exists.

\item Suppose that $\varphi$ and $\psi$ are contingencies. Can
  $\varphi \to \psi$ be a tautology, contingency, or inconsistency?
  Justify your answers. (4 points)

  \textbf{Answer:}
  \begin{itemize}
    \item \textbf{Tautology:} Yes. Example: $\varphi$ and $\psi$ are identical.
    \item \textbf{Contingency:} Yes. Example: $\varphi=P$, $\psi=Q$.
    \item \textbf{Inconsistency:} No. Since $\varphi$ is a
      contingency, it is false on some row of its truth table. On this
      row, $\varphi\to\psi$ is true. Hence $\varphi\to\psi$ cannot be
      an inconsistency.
  \end{itemize}
\end{enumerate}

\end{document}


%%% Local Variables:
%%% mode: latex
%%% TeX-master: t
%%% End:
