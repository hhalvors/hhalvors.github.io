\documentclass[aspectratio=169,17pt,fleqn]{beamer}

\usetheme{Madrid}
\usecolortheme{dolphin}
\setbeamertemplate{navigation symbols}{}
\setbeamertemplate{footline}[frame number]

\usepackage{amsmath,amssymb,array}

\usepackage{booktabs}



\usepackage{fontspec}    
\usepackage{unicode-math}  

\AtBeginSection[]{
  \begin{frame}
    \centering
    \vfill
    \Large\insertsectionhead
    \vfill
  \end{frame}
}

\usepackage{tikz}
\usetikzlibrary{positioning,matrix,backgrounds}
\usepackage{tikz-cd}


\title{Lecture 13}
\author{PHI~201 -- Introductory Logic}
\date{December 1, 2025}

\begin{document}

\frame{\titlepage}

\begin{frame}{Review}

\begin{itemize}
\item We have formulated a theory \emph{about} propositional logic,
  and now we're proving some facts.
\item Our new inference rule is mathematical induction --- usually on
  the construction of sentences, or on the construction of sequents.
\item E.g., in the previous lecture, I showed that every sentence is
  equivalent to one in which just $\vee$ and $\neg$ occur.
\end{itemize}

\end{frame}

\section{Soundness}

\begin{frame}

  \textbf{Theorem.} Every line in a correctly written proof is
  semantically valid. That is, the sentence in the center column is a
  semantic consequence of the dependency sentences.

  \bigskip Method of proof:

  \begin{enumerate}
  \item Show that Rule of Assumptions lines are semantically valid.
  \item Show that the other inference rules transform semantically
    valid lines to semantically valid lines.
  \end{enumerate}


\end{frame}

\begin{frame}{Induction MP}

Suppose that $\Gamma ,\Delta\vdash \psi$ is derived from
$\Gamma\vdash\varphi\to\psi$ and $\Delta\vdash\varphi$.
  \[ \begin{array}{r c l l}
       \Gamma & (a) & \varphi\to\psi & \\
       \Delta & (b) & \varphi & \\[0.3em]
       \Gamma ,\Delta & (c) & \psi \end{array} \]
Suppose that (a) and (b) are semantically valid.

\end{frame}

\begin{frame}{Induction MP}

  Let $v$ be an arbitrary valuation, and suppose that $v$ assigns $1$
  to all elements of $\Gamma ,\Delta$. Since line (a) is valid,
  $v(\varphi\to\psi )=1$. Since line (b) is valid, $v(\varphi )=1$. By
  the truth table for $\to$, it follows that $v(\psi )=1$. Since $v$
  was an arbitrary valuation, any valuation that assigns $1$ to all
  elements of $\Gamma ,\Delta$ also assigns $1$ to $\psi$. Therefore,
  line (c) is semantically valid.

\end{frame}

\begin{frame}{Induction RA}

Suppose that $\Delta '$ is derived from $\varphi\vdash\varphi$ and
$\Delta\vdash\bot$.
\[ \begin{array}{c c l l}
     a  & (a) & \varphi & \text{A} \\
     \Delta  & (b) & \bot \\[0.3em]
     \Delta ' & (c) & \neg\varphi \end{array} \]
 Suppose that (a) and (b) are semantically valid.

\end{frame}

\begin{frame}{Induction RA}

  Let $v$ be an arbitrary valuation, and suppose that $v$ assigns $1$
  to every element of $\Delta '$. Since (b) is valid, $v$ does not
  assign $1$ to every element of $\Delta$. Therefore, $v(\varphi )=0$,
  since $\varphi$ is only thing in $\Delta$ that is not in $\Delta
  '$. Therefore, $v(\neg\varphi )=1$. Since $v$ was an arbitrary
  valuation, every valuation that assigns $1$ to all elements of
  $\Delta '$ also assigns $1$ to $\neg\varphi$, and line (c) is valid.   


\end{frame}

\section{Disjunctive normal form}

%---------------------------------------------------------
\begin{frame}[t]{Goal: Disjunctive Normal Form (DNF)}

 \small 

\textbf{DNF:} A sentence is in disjunctive normal form if it is a disjunction
of conjunctions of literals.

\begin{itemize}
  \item A \textbf{literal} is either an atomic sentence
    (e.g.\ $P,Q,R$) or the negation of an atomic sentence (e.g.\ $\neg P$).
  \item A \textbf{conjunction of literals} has the form
    \[
      L_1 \wedge L_2 \wedge \cdots \wedge L_n,
    \]
    where each $L_i$ is a literal.
  \item A sentence is in \textbf{DNF} if it has the form
    \[
      C_1 \vee C_2 \vee \cdots \vee C_k,
    \]
    where each $C_j$ is a conjunction of literals
    (or a single literal).
\end{itemize}

\end{frame}

\begin{frame}

  \textbf{Fact 1:} Every sentence is provably equivalent to a sentence
  in DNF.

  \bigskip \textbf{Fact 2:} A DNF sentence $C_1\vee\cdots\vee C_n$ is
  a semantic tautology iff for each elementary conjunction $E$, there
  is a $C_i$ such that $E\vdash C_i$.


\end{frame}

\begin{frame}{DNF and truth tables}

  You can ``guess'' a DNF equivalent of a sentence by looking at its
  truth table and taking a disjunction of all the rows in which its
  true. For example:

  \[ \begin{array}{c c | c }
       P & Q & \varphi \\ \hline
       1 & 0 & 1  \end{array} \]

\end{frame}  

\begin{frame}[t]{Truth Table for $P \to Q$}

\[
\begin{array}{c c | c | c }
P & Q & P \to Q & (P\wedge Q)\vee (\neg P\wedge Q)\vee (\neg
                  P\wedge\neg Q) \\ \hline
1 & 1 & 1 \\
1 & 0 & 0 \\
0 & 1 & 1 \\
0 & 0 & 1
\end{array}
\]


\end{frame}



  




% ---------------------------------------------------------
\begin{frame}[t]{DNF algorithm: High-level strategy}

Given any sentence $\varphi$ built from
$\wedge, \vee, \neg, \to$ and atomic $P,Q,R,\dots$:

\bigskip

\begin{enumerate}
  \item Eliminate all occurrences of $\to$.
  \item Push all occurrences of $\neg$ inwards so that
        they apply only to atomic sentences.
  \item Distribute $\wedge$ over $\vee$ to obtain a disjunction
        of conjunctions.
  \item Clean up: remove unnecessary parentheses, reorder
        conjuncts/disjuncts, and combine duplicates if desired.
\end{enumerate}

\end{frame}
%---------------------------------------------------------
\begin{frame}[t]{Step 1: Eliminate conditionals}

Replace every occurrence of $A \to B$ with $\neg A\vee B$.

\begin{itemize}
  \item Do this recursively on all subformulas:
    \[
      (\varphi \to \psi) \wedge (\chi \to \theta)
      \;\rightsquigarrow\;
      (\neg \varphi \vee \psi) \wedge (\neg \chi \vee \theta).
    \]
  \item After this step, your sentence uses only
        $\wedge, \vee, \neg$ and atomic letters.
\end{itemize}

\end{frame}
%---------------------------------------------------------
\begin{frame}[t]{Step 2: Push Negations Inward}

Use these equivalences repeatedly until $\neg$ appears only
directly in front of atomic sentences:

{\small $\neg\neg A \;\equiv\; A \qquad \neg(A \wedge B) \;\equiv\; \neg A
\vee \neg B \qquad \neg(A \vee B) \;\equiv\; \neg A \wedge \neg B $  }

\begin{itemize}
  \item Apply these rules from the outside in, simplifying
        as you go.
  \item After this step, the sentence is built from $\wedge,\vee$
        and literals (atoms or negated atoms).
\end{itemize}

\end{frame}
%---------------------------------------------------------
\begin{frame}[t]{Step 3: Distribute $\wedge$ over $\vee$}

\small 

To get a disjunction of conjunctions, repeatedly use
the distributive laws:
\[ A \wedge (B \vee C) \;\equiv\; (A \wedge B) \vee (A \wedge C)
\] \[ (A \vee B) \wedge C \;\equiv\; (A \wedge C) \vee (B \wedge C)
\]

\begin{itemize}
  \item Whenever you see a conjunction whose parts contain
        disjunctions, distribute.
  \item Also use associativity and commutativity of
        $\wedge, \vee$ to rearrange and “flatten”:
  $A \vee (B \vee C) \;\equiv\; A \vee B \vee C,
          \qquad
          A \wedge (B \wedge C) \;\equiv\; A \wedge B \wedge C.
        $
\end{itemize}

\end{frame}
%---------------------------------------------------------
\begin{frame}[t]{Step 4: Cleanup to Get DNF}

After distribution, your sentence should be a disjunction of
conjunctions of literals. Then:

\begin{itemize}
  \item Remove redundant parentheses using associativity.
  \item Optionally, reorder literals and conjunctions to a
        standard order (e.g.\ alphabetically).
  \item Optionally, simplify obvious redundancies, e.g.
        $(P \wedge P \wedge Q) \;\equiv\; (P \wedge Q)$
\end{itemize}

The resulting sentence is in disjunctive normal form and is
logically equivalent to the original sentence.

\end{frame}
%---------------------------------------------------------
\begin{frame}[t]{Worked Example: From Formula to DNF}

\small

Start with $(P \to Q) \wedge \neg R$ 

\medskip
\textbf{1. Eliminate $\to$:}
\[
  (P \to Q) \wedge \neg R
  \;\equiv\;
  (\neg P \vee Q) \wedge \neg R.
\]

\medskip
\textbf{2. Push negations inward:} nothing to do (already on atoms).

\medskip
\textbf{3. Distribute $\wedge$ over $\vee$:}
\[
  (\neg P \vee Q) \wedge \neg R
  \;\equiv\;
  (\neg P \wedge \neg R) \vee (Q \wedge \neg R).
\]
Now we have a disjunction of conjunctions of literals, which is in
DNF.

\end{frame}
%---------------------------------------------------------
\begin{frame}[t]{Algorithm in Pseudocode}

\small

\textbf{Input:} sentence $\varphi$ built from
$\wedge, \vee, \neg, \to$ and atomic $P,Q,R,\dots$

\bigskip

\begin{enumerate}
  \item \textbf{ElimCond}($\varphi$): recursively replace each
        subformula of the form $(A \to B)$ by $(\neg A \vee B)$.
  \item \textbf{PushNeg}($\varphi$): recursively apply
        \(\neg\neg A \equiv A\),
        \(\neg(A \wedge B) \equiv \neg A \vee \neg B\),
        \(\neg(A \vee B) \equiv \neg A \wedge \neg B\)
        until every $\neg$ is on an atom.
  \item \textbf{Distribute}($\varphi$): recursively apply
        the distributive laws to move all $\wedge$ inside all $\vee$.
  \item \textbf{Output} the resulting disjunction of conjunctions
        of literals as the DNF of the original sentence.
\end{enumerate}

\end{frame}
%---------------------------------------------------------

\section{Completeness}

\begin{frame}{Substitution theorem}

  For a formula $\varphi$, let $\varphi '$ denote the result of
  uniformly substituting formulas for the atomic sentences that occur
  in $\varphi$. We say that $\varphi '$ is an \alert{substitution
    instance} of $\varphi$. 

  \bigskip \textbf{Proposition.} If
  $\varphi _1,\dots ,\varphi _n\vdash\psi$ then
  $\varphi '_1,\dots ,\varphi '_n\vdash\psi '$.

\end{frame}

\begin{frame}

  \textbf{Proposition.} If $\varphi$ is not provable, then it has a
  substitution instance $\varphi '$ such that $\vdash \neg \varphi '$.

  \medskip By the DNF theorem, $\varphi$ is provably equivalent to a
  sentence $C_1\vee\cdots\vee C_n$, where each $C_i$ is a consistent
  conjunction of literals.

  \medskip It's not hard to see that $\vdash E_1\vee\cdots \vee E_m$,
  where the $E$s are an exhaustive set of elementary conjunctions.

\end{frame}

\begin{frame}

  If each $E_j$ entailed some $C_i$, then $C_1\vee\cdots\vee C_n$
  would be provable.

  \bigskip Since $\varphi$ is not provable, there is an elementary
  conjunction $E$ that does not imply any $C_i$.

  \bigskip $E\vdash \neg C_i$


\end{frame}

\begin{frame}

  A substitution that takes $E$ to something provable will take each
  $C_i$ to something whose negation is provable. 
  \[ E\vdash \neg C_i \: \Longrightarrow \: \top \vdash \neg C_i' \]
  Therefore $\vdash \neg (C_1'\vee\cdots \vee C_n')$, and hence
  $\vdash \neg\varphi '$.

\end{frame}

\begin{frame}

  \textbf{Theorem.} If $\varphi$ is not provable, then there is a
  valuation $v$ such that $v(\varphi )=0$.

\bigskip Take the elementary conjunction $E$ from the previous
argument and use it to define $v$.

\medskip $v(C_i)=0$ for $i=1,\dots ,n$. Therefore $v(C_1\vee\cdots\vee C_n)=0$.

\medskip Since $\varphi\vdash C_1\vee\cdots\vee C_n$, soundness implies that
$v(\varphi )=0$.



\end{frame}

\begin{frame}

  \textbf{Corollary.} If $\varphi _1,\dots ,\varphi _n\not\vdash\psi$,
  then there is a valuation $v$ such that $v(\varphi _i)=1$ and
  $v(\psi )=0$.

  \bigskip If
  $\vdash (\varphi _1\wedge\cdots\wedge \varphi _n)\to \psi$, then
  $\wedge$I and MP give $\varphi _1,\dots ,\varphi _n\vdash\psi$. So
  $(\varphi _1\wedge\cdots\wedge\varphi _n )\to \psi$ is not
  provable. By the previous theorem, there is a valuation $v$ such
  that $v((\varphi _1\wedge\cdots\wedge\varphi _n )\to \psi )=0$. By
  the truth-tables for $\wedge$ and $\to$, it follows that
  $v(\varphi _i)=1$ for $i=1,\dots ,n$, while $v(\psi )=0$.


\end{frame}

\end{document}
%%% Local Variables:
%%% mode: latex
%%% TeX-master: t
%%% End:
