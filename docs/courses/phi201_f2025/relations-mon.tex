\documentclass[12pt,fleqn]{article}
\usepackage[margin=1in]{geometry}
\usepackage{amsmath,amssymb,enumitem}
\setlength{\parskip}{0.6em}
\setlength{\parindent}{0pt}

\title{Worksheet: Defining one relation from another}
\date{}
\begin{document}
\maketitle

We work in ordinary first-order logic with identity, with:
\begin{itemize}
  \item monadic (unary) predicate symbols $P,Q$,
  \item a binary predicate symbol $R$,
  \item and, later, a binary predicate symbol $S$.
\end{itemize}
Throughout, $x,y,z$ range over objects in the domain.

This worksheet explores what we can \emph{prove} about a binary relation when it is
defined in terms of simpler predicates.

\bigskip
\hrule
\bigskip

\section*{1. Defining a Product Relation}

Suppose we \emph{define} a binary relation $R$ by:
\[
  Rxy \;\leftrightarrow\; Px\ \wedge\ Qy.
\]

Intuitively, $R$ relates exactly those pairs $(x,y)$ such that $x$ has property $P$
and $y$ has property $Q$.

\subsection*{Tasks}

\begin{enumerate}[label=\textbf{(1.\arabic*)}]
  \item Show that for all $x$ and $y$:
  \[
    Rxy \rightarrow Px \qquad\text{and}\qquad Rxy \rightarrow Qy.
  \]
  (A very easy warm-up.)

  \item Define two unary predicates \emph{from} $R$:
  \[
    P_R(x) \;\equiv\; \exists y\,Rxy
    \qquad\text{and}\qquad
    Q_R(y) \;\equiv\; \exists x\,Rxy.
  \]
  Show that in any structure satisfying the definition $Rxy \leftrightarrow Px\wedge Qy$, we have:
  \[
    \forall x\bigl(P_R(x) \leftrightarrow Px\bigr)
    \qquad\text{and}\qquad
    \forall y\bigl(Q_R(y) \leftrightarrow Qy\bigr).
  \]

  \item \textbf{Rectangle Law.} Show that $R$ satisfies the following ``rectangle'' property:
  \[
    \forall x\forall x'\forall y\forall y'\bigl(
      (Rxy \wedge Rx'y') \rightarrow (Rxy' \wedge Rx'y)
    \bigr).
  \]
  In words: whenever $(x,y)$ and $(x',y')$ are in $R$, then the ``crossed'' pairs
  $(x,y')$ and $(x',y)$ are also in $R$.

  \item Explain why the Rectangle Law expresses the idea that the extension of $R$
  in the domain is a \emph{full rectangular block} between the $P$-objects
  and the $Q$-objects. (It has no ``holes'' inside that block.)
\end{enumerate}

\bigskip
\hrule
\bigskip

\section*{2. Defining an Equivalence from a Monadic Predicate}

Now suppose we define a new binary relation $R$ from a single unary predicate $P$:
\[
  Rxy \;\leftrightarrow\; \bigl(Px\leftrightarrow Py\bigr).
\]
Intuitively: $x$ is $R$-related to $y$ iff $x$ and $y$ either both have property $P$,
or both lack property $P$.

\subsection*{Tasks}

\begin{enumerate}[label=\textbf{(2.\arabic*)}]
  \item Prove that $R$ is \textbf{reflexive}:
  \[
    \forall x\,Rxx.
  \]
  (Hint: what is $Px\leftrightarrow Px$?)

  \item Prove that $R$ is \textbf{symmetric}:
  \[
    \forall x\forall y\,(Rxy \rightarrow Ryx).
  \]
  (Hint: use the symmetry of the biconditional: $Px\leftrightarrow Py$ iff $Py\leftrightarrow Px$.)

  \item Prove that $R$ is \textbf{transitive}:
  \[
    \forall x\forall y\forall z\bigl((Rxy \wedge Ryz) \rightarrow Rxz\bigr).
  \]
  (Hint: if $Px$ and $Py$ have the same truth-value, and $Py$ and
  $P(z)$ have the same truth-value, then $Px$ and $Pz$ have the same
  truth-value.)

  \item Conclude that $R$ is an \textbf{equivalence relation}.
  
  \item Describe informally what the $R$-equivalence classes look like.  
  (How many equivalence classes are there? Which objects are in each class?)
\end{enumerate}

\bigskip
\hrule
\bigskip

\section*{3. Defining a Relation from Another}

Now start with an arbitrary binary relation $R$.  
Define a new binary relation $S$ by:
\[
  Sxy \;\leftrightarrow\; \forall z\bigl(Rxz \rightarrow Ryz\bigr).
\]

Intuitively:
\begin{itemize}
  \item $Sxy$ means: \emph{every} $R$-successor of $x$ is also an $R$-successor of $y$.
  \item So $y$ has at least all the $R$-successors that $x$ has (perhaps more).
\end{itemize}

\subsection*{Tasks}

\begin{enumerate}[label=\textbf{(3.\arabic*)}]
  \item Prove that $S$ is \textbf{reflexive}:
  \[
    \forall x\,Sxx.
  \]
  (Hint: for any $x$ and $z$, $Rxz \rightarrow Rxz$ is always true.)

  \item Prove that $S$ is \textbf{transitive}:
  \[
    \forall x\forall y\forall w\bigl((Sxy \wedge Syw) \rightarrow Sxw\bigr).
  \]
  (Hint: unpack the definition: if every $R$-successor of $x$ is an $R$-successor of $y$, and every $R$-successor of $y$ is an $R$-successor of $w$, then what can you say about the $R$-successors of $x$ and $w$?)

  \item Is $S$ necessarily \textbf{symmetric}?  
  Either:
  \begin{itemize}
    \item give a proof that $\forall x\forall y(Sxy \rightarrow Syx)$ is valid, or
    \item give a countermodel (a structure and an interpretation of $R$) where $Sxy$ holds but $Syx$ fails for some $x,y$.
  \end{itemize}

  \item Based on your answers above, what kind of relational structure is $S$?  
  (For example: is it an equivalence relation, a partial order, a preorder, \dots?)

  \item Explain in ordinary language what $Sxy$ says about the relationship between $x$ and $y$,
  in terms of their $R$-successor sets.
\end{enumerate}

\bigskip
\hrule
\bigskip

\section*{Optional Challenge}

\begin{enumerate}[label=\textbf{(C.\arabic*)}]
  \item In Part~1, we saw that if $Rxy \leftrightarrow Px\wedge Qy$, then $R$ satisfies the Rectangle Law.  
  Prove the \emph{converse}: if a binary relation $R$ satisfies
  \[
    \forall x\forall x'\forall y\forall y'
    \bigl( (Rxy \wedge Rx'y') \rightarrow (Rxy' \wedge Rx'y) \bigr),
  \]
  then there exist monadic predicates $P$ and $Q$ such that
  \[
    Rxy \leftrightarrow Px\wedge Qy
  \]
  holds in the structure.

  (Hint: let $Px$ say ``row $x$ of $R$ is nonempty'', and let $Qy$ say ``column $y$ of $R$ is nonempty''.)
\end{enumerate}

\end{document}
%%% Local Variables:
%%% mode: latex
%%% TeX-master: t
%%% End:
