\documentclass[12pt]{article}
\usepackage[margin=1in]{geometry}
\setlength{\parskip}{1em}
\setlength{\parindent}{0em}
\title{Logic precept: Week 8}
\date{}
\begin{document}

\maketitle

\vspace*{-4em}

\section*{Translation}

\begin{enumerate}
\item Mary  is the only student who didn’t miss any questions on the
  exam.
\item All professors except $a$ are boring.
\item There is no greatest prime number.
\item The smallest prime number is even.
\item For each natural number, there is a unique next-greater natural
  number.
\item There are at least two Ivy League universities in New York
  state.

\end{enumerate}

\section*{Proofs with equality}

\begin{enumerate}
\item $Fa\:\vdash\: \forall x((x=a)\to Fx)$
\item $\forall x((x=a)\to Fx)\:\vdash\: Fa$  
\item $\exists x\forall y(x=y)\:\vdash\: \forall x\forall y (x=y)$
\end{enumerate}


\section*{Partial order}

In real life, rigorous proofs are rarely written with line numbers,
dependencies, or named justifications. But the idea is to give the
reader enough information so that s/he could reconstruct such a proof.

\begin{enumerate}
\item Write down a predicate logic sentence that expresses the claim
  that every two elements have a least upper bound.
\item Give an example of a partially ordered set in which that
  sentence is false.
\item Prove (informally) that if any two elements have a least upper
  bound, then so do any three elements.
\item We say that $\leq$ is a serial relation just in case
  $\forall x\exists y(x\leq y\wedge x\neq y)$. Is there a
  \emph{finite} partially ordered set that satisfies the serial axiom?
\end{enumerate}


 


\section*{Set theory}

For sets $a$ and $b$, we write $a\subseteq b$ for the claim that
$\forall x(x\in a\to x\in b)$.

We let $a\cap b$ be the set defined by
$\forall x((x\in a\cup b)\leftrightarrow (x\in a\wedge
x\in b))$.

\begin{enumerate}
\item Show that if $a\subseteq b$ and $b\subseteq c$ then
  $a\subseteq c$.
\item Show that $a\subseteq b$ if and only if $a\cap b=a$.
\end{enumerate}



\end{document}


%%% Local Variables:
%%% mode: latex
%%% TeX-master: t
%%% End:
