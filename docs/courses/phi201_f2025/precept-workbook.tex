\documentclass[12pt]{article}

\usepackage[margin=1in]{geometry}
\usepackage{amsmath,amssymb}
\usepackage{enumitem}
\usepackage{array}
\usepackage{hyperref}

\usepackage{amsmath,amssymb}
\usepackage{tikz}
\usetikzlibrary{arrows.meta,positioning}
\usepackage{tikz-cd}

% Optional: slightly nicer spacing for a workbook
\setlength{\parindent}{0pt}
\setlength{\parskip}{0.7em}

\title{Logic Precepts Workbook}
\author{PHI 201 – Introductory Logic}
\date{Fall 2025}

\begin{document}

\maketitle
\tableofcontents
\clearpage

% ---------- Precept 2 ----------
\section{Conditional proof and $\vee$ elimination}

\subsection*{Warmup: Deducing}

\textbf{Exercise 2.5} ($\wedge$E, $\wedge$I, $\vee$I, MP, MT, DN)

$P\to \neg Q,Q\: \vdash \: \neg P$

$\neg \neg P\: \vdash \: \neg \neg P\wedge (P\vee Q)$

$\neg (P\wedge Q)\to R,\neg R\:\vdash \: P$

$\neg P\to\neg Q,Q\:\vdash \: P$

$P\:\vdash \: \neg \neg (P\vee Q)$

\subsection*{New proof rules}

We will work in blocks. The first block leads to problem A1.

$P\to Q\:\vdash\: P\to (Q\vee R)$

$(P\vee Q)\to R\:\vdash \: P\to R$

$P\to Q\:\vdash\: (R\to P)\to (R\to Q)$

$P\to Q\:\vdash\: (Q\to R)\to (P\to R)$

\textbf{(A1)} $P\to (Q\to R)\:\vdash\: Q\to (P\to R)$

The second block leads to A2. The focus is on the ``contrapositive
maneuver''.

$P\to Q\:\vdash\: \neg Q\to \neg P$

$\:\vdash\: P\to (P\vee Q)$

$\neg (P\vee Q)\:\vdash\: \neg P$

$\:\vdash\: (P\wedge Q)\to P$

\textbf{(A2)} $\neg P\:\vdash\: \neg (P\wedge Q)$

The third block leads to A3 and A4.

$P\:\vdash\: (P\to Q)\to Q$

\textbf{(A3)} $P\:\vdash\: (P\to \neg P)\to \neg P$

\textbf{(A4)} $Q\:\vdash\: \neg (Q\to\neg Q)$

The fourth block leads to problem B1.

$P\vee Q\:\vdash\: Q\vee P$

$P\vee (Q\wedge R)\:\vdash\: (P\wedge Q)\vee (P\wedge R)$

\textbf{(B1)} $P\wedge (Q\vee R)\:\vdash\: (P\wedge Q)\vee (P\wedge R)$

The fifth block leads to problem B2.

$P,\neg P\:\vdash\: Q$

\textbf{(B2)} $P\vee Q,\neg P\:\vdash\: Q$

\subsection*{Evaluating proofs}\label{evaluating-proofs}

\textbf{Exercise} Which of the following proofs with CP is correct? If
a proof is not correct, explain what is wrong with it, and say whether
there is a simple fix, or whether it is fatally flawed. 

  \begin{tabular}{>{\raggedleft\arraybackslash}p{1.5cm} >{\centering\arraybackslash}p{1cm} p{4cm} >{\raggedright\arraybackslash}p{3.5cm}}
\textbf{Deps} & \textbf{Line} & \textbf{Formula} & \textbf{Justification} \\ \hline
1   & (1) & $P\wedge Q$ & A \\
1   & (2) & $P$ & 1 $\wedge$E \\
1   & (3) & $Q$ & 2 $\wedge$E \\ 
              & (4) & $P\to Q$ & 2,3 CP
  \end{tabular}

\medskip \begin{tabular}{>{\raggedleft\arraybackslash}p{1.5cm} >{\centering\arraybackslash}p{1cm} p{4cm} >{\raggedright\arraybackslash}p{3.5cm}}
\textbf{Deps} & \textbf{Line} & \textbf{Formula} & \textbf{Justification} \\ \hline  
1  & (1) & $Q$ & A \\
2  & (2) & $P$ & A \\
1  & (3) & $P\to Q$ & 2,1 CP \end{tabular}

\bigskip \textbf{Exercise} Explain what is wrong with the following
``proof''.

\medskip \begin{tabular}{>{\raggedleft\arraybackslash}p{1.5cm}
    >{\centering\arraybackslash}p{1cm} p{4cm}
           >{\raggedright\arraybackslash}p{3.5cm}}
     \textbf{Deps} & \textbf{Line} & \textbf{Formula} & \textbf{Justification} \\ \hline        
           1   & (1) & $P\vee Q$  &    A \\
           2   & (2) & $P$    &   A \\
           3   & (3) & $Q$    &   A \\
           2,3 & (4) & $P\wedge Q$  & 2,3 $\wedge$I \\
           2,3 & (5) & $P$    &    4 $\wedge$E \\
           1 & (6) & $P$ & 1,2,2,3,5 $\vee$E
\end{tabular}

\subsection*{Additional practice problems}

$\neg P\vee \neg Q\:\dashv\vdash\: \neg (P\wedge Q)$

$P\to (P\to Q)\:\vdash\: P\to Q$

$(P\vee Q)\to R\:\vdash\: P\to R$

$P\to (Q\to R),P\to Q\:\vdash \: P\to R$

$P\to (Q\to R)\:\vdash\: (P\to Q)\to (P\to R)$

$(P\to Q)\to P\:\vdash\: (P\to Q)\to Q$

$(P\to Q)\to P\: \vdash \: \neg P\to P$

$(P\to R)\wedge (Q\to R)\:\vdash\: (P\vee Q)\to R$

$P\vee (Q\vee R)\:\dashv\vdash\: (P\vee Q)\vee R$

$P\wedge (Q\vee R)\:\dashv\vdash\: (P\wedge Q)\vee (P\wedge R)$

$P\vee (Q\wedge R)\:\dashv\vdash\: (P\vee Q)\wedge (P\vee R)$

$\neg P\vee Q\:\dashv\vdash\: P\to Q$

$\neg (P\to Q)\:\dashv\vdash\: P\wedge \neg Q$

$\vdash (P\to Q)\vee (Q\to P)$

$P\to (Q\vee R)\:\vdash\: (P\to Q)\vee (P\to R)$

$\vdash\: ((P\to Q)\to P)\to P$ (Hint: One possibility is to first
prove $\vdash P\vee \neg P$, and then argue by cases. The first case
is easy if you remember ``positive paradox''. For the second case,
remember ``negative paradox'', i.e.~that $\neg P$ implies $P\to
  Q$.)

$P\to (Q\vee R)\:\vdash \: \neg R\to (\neg Q\to \neg P)$

$P\to\neg P\:\dashv\vdash\:\neg P$

$(P\to Q)\to Q\:\vdash\: (Q\to P)\to P$

$(P\to Q)\to R\:\vdash\: (P\to R)\to R$

$(P\to R)\to R\:\dashv\vdash\: P\vee R$ (Hint: derive $\neg P\to R$
from the sentence on the left.)

$(P\to Q)\to P\:\dashv\vdash \: P$ (Hint: assume $\neg P$ and derive
$P\to Q$.)

\clearpage


\section{Reductio ad absurdum}

\subsection*{Proofs}

\subsubsection*{Review of $\vee$-elimination}

\begin{enumerate}
\item $(P\to Q)\vee (P\to R)\:\vdash\: P\to (Q\vee R)$
\item $\neg P\vee\neg Q\:\vdash\: \neg (P\wedge Q)$
\item $\neg P\vee Q\:\vdash\: P\to Q$
\end{enumerate}

\subsubsection*{Reductio ad Absurdum}

\begin{enumerate}
\item $P\to Q\:\vdash\: \neg (P\wedge \neg Q)$
\item $\neg (P\to Q)\:\vdash\: \neg Q$
\item pset1 $\neg (P\to Q)\:\vdash\: Q\to R$
\item $\neg (P\vee Q)\:\vdash \: \neg P$  
\item pset2 $P\to Q\:\vdash\:\neg P\vee Q$
\item pset3 $P\to (Q\vee R)\:\vdash\: (P\to Q)\vee R$
\end{enumerate}



\subsection*{Challenge problem: Pierce's law}

$\vdash\: ((P\to Q)\to P)\to P$

\subsection*{Truth tables}

\subsubsection*{Key Concepts}

\begin{itemize}
\item
  arguments: valid, invalid
\item
  counterexample
\item
  truth-value
\item
  main connective
\item
  sentences (syntactic): atomic, conjunction, negation, disjunction,
  conditional, biconditional
\item
  sentences (semantic): tautology, inconsistency, contingency
\item
  two sentences: equivalent, inconsistent, independent
\end{itemize}

\subsubsection*{For arguments}

Determine whether the following arguments are valid or not. Explain
your answer by showing the existence of a row of a truth table, or by
pointing to a full truth table, or something of the sort.  Your answer
should be articulated in English prose so that it can convince anyone
else who is familiar with truth tables.

\begin{enumerate}
\item $P\to (Q\vee R)\:\vdash \: (P\to Q)\vee R$
\item
  $\vdash \: (P\leftrightarrow Q)\vee (P\leftrightarrow R)\vee
  (Q\leftrightarrow R)$
\item $P\to (Q\to R)\:\vdash\: (P\wedge Q)\to R$
\item $P\to R\:\vdash\: (P\vee Q)\to R$
\item $(P\leftrightarrow Q)\leftrightarrow R \:\vdash\: P\vee R$
\item $\:\vdash\: (P\to Q)\vee (Q\to R)$
\end{enumerate}

\subsubsection*{Sentence classification
(syntactic)}\label{sentence-classification-syntactic}

\textbf{Exercise.} What is the \textbf{main connective} of each of the
following formulas?

\begin{enumerate}
\def\labelenumi{\arabic{enumi}.}
\item
  \(\neg (P\to Q)\)
\item
  \(\neg P\to Q\)
\item
  \(\neg (P\to \neg Q)\)
\item
  \((P\wedge Q)\vee \neg (P\to Q)\)
\item
  \(((P\to Q)\to P)\to P\)
\end{enumerate}

\subsection*{Sentence classification
(semantic)}

\textbf{Exercise.} Classify each of the following sentences as
tautology, inconsistency, or contingency.

\begin{enumerate}
\def\labelenumi{\arabic{enumi}.}
\item
  \((P\to \neg P)\to \neg P\)
\item
  \((P\wedge Q)\vee (\neg P\wedge \neg Q)\)
\item
  \((P\wedge (Q\wedge \neg R))\vee (\neg P\wedge (\neg Q\wedge R))\)
\item
  \((P\leftrightarrow Q)\leftrightarrow R\)
\item
  \((P\wedge Q)\vee \neg (P\to Q)\)
\item
  \(((P\to Q)\to R)\to Q\)
\end{enumerate}

\textbf{Exercise.} Show that if \(B\) is a tautology, then \(A\wedge B\)
is logically equivalent to \(A\).

\textbf{Exercise.} Show that if \(B\) is an inconsistency, then
\(A\vee B\) is logically equivalent to \(A\).

\subsubsection*{For multiple sentences}

\textbf{Exercise:} What is the semantic relationship between
\((P\wedge Q)\) and \(\neg (P\to Q)\)?

\textbf{Exercise:} If \(\phi\wedge\psi\) is a contingency, then what are
the possibilities for \(\phi\) and \(\psi\)?

\textbf{Exercise:} If \(\phi\) is a tautology, then what are the
possibilities for \(\phi\wedge\psi\)? What are the possibilities for
\(\phi\vee\psi\)?


\clearpage

\section{Universal quantifier rules}

\subsection*{Review}

\begin{enumerate}
  \item What kinds of sentences are there in predicate logic?
  \item What is the difference between a \textbf{formula} and a \textbf{sentence}?
  \item What is an \textbf{instance} of a universal sentence?
  \item What is the restriction on UI?
\end{enumerate}

\subsection*{Proofs}

\subsubsection*{Warmup Problems}

\begin{enumerate}
\item
  $\forall x (F x \rightarrow G x) \ \vdash \ \forall x F x
  \rightarrow \forall x G x$
\item
  $\forall x (P x \rightarrow Q x), \ \forall x (Q x \rightarrow R x)
  \ \vdash \ \forall x (P x \rightarrow R x)$
\item
  $P \rightarrow \forall x F x \ \vdash \ \forall x (P \rightarrow F
  x)$
\item $\forall x \forall y R x y \ \vdash \ \forall y \forall x R x y$
  
\end{enumerate}

\subsubsection*{Pset Problems}

\begin{enumerate}
\item $\forall x(Fx \rightarrow \forall y Gy) \ \vdash \ \forall x
  \forall y (Fx \rightarrow Gy)$
\item
  $\forall x \forall y (F x \rightarrow G y) \ \vdash \ \forall x (F x
  \rightarrow \forall y G y)$
\item $\vdash \ \forall x (\forall y R x y \rightarrow R x x)$
\end{enumerate}

\subsection*{Translation}

\subsubsection*{Exercise}

How do you symbolize the following?

\begin{enumerate}
  \item All $F$ are $G$.
  \item No $F$ are $G$.
  \item Some $F$ are $G$.
  \item Some $F$ are not $G$.
  \end{enumerate}

\subsubsection*{Exercise}

Use $F$ for ``is French'', $G$ for ``is German'', $C$ for ``is
Canadian'', $L x y$ for ``$x$ likes $y$'', $a$ for Alice, and $b$ for
Bob. How would you symbolize:

\begin{enumerate}
\item Alice likes Canadians.
\item Alice likes Bob only if Bob likes Canadians.
\item Alice likes Bob only if he likes her.
\item Alice is a German who likes Canadians.
\item Alice is French only if she doesn't like Canadians.
\item Alice likes only those people who don't like Canadians.
\item Someone likes only those people who like Canadians.
\item French people only like Canadians who don't like Germans.
\item Some French people like only those Germans who don't like
  themselves.
\end{enumerate}


\clearpage

\section{Theories}

\subsection*{Translation}

\begin{enumerate}
\item Mary  is the only student who didn’t miss any questions on the
  exam.
\item All professors except $a$ are boring.
\item There is no greatest prime number.
\item The smallest prime number is even.
\item For each natural number, there is a unique next-greater natural
  number.
\item There are at least two Ivy League universities in New York
  state.

\end{enumerate}

\subsection*{Proofs with equality}

\begin{enumerate}
\item $Fa\:\vdash\: \forall x((x=a)\to Fx)$
\item $\forall x((x=a)\to Fx)\:\vdash\: Fa$  
\item $\exists x\forall y(x=y)\:\vdash\: \forall x\forall y (x=y)$
\end{enumerate}


\subsection*{Partial order}

In real life, rigorous proofs are rarely written with line numbers,
dependencies, or named justifications. But the idea is to give the
reader enough information so that s/he could reconstruct such a proof.

\begin{enumerate}
\item Write down a predicate logic sentence that expresses the claim
  that every two elements have a least upper bound.
\item Give an example of a partially ordered set in which that
  sentence is false.
\item Prove (informally) that if any two elements have a least upper
  bound, then so do any three elements.
\item We say that $\leq$ is a serial relation just in case
  $\forall x\exists y(x\leq y\wedge x\neq y)$. Is there a
  \emph{finite} partially ordered set that satisfies the serial axiom?
\end{enumerate}


\subsection*{Set theory}

For sets $a$ and $b$, we write $a\subseteq b$ for the claim that
$\forall x(x\in a\to x\in b)$.

We let $a\cap b$ be the set defined by
$\forall x((x\in a\cup b)\leftrightarrow (x\in a\wedge
x\in b))$.

\begin{enumerate}
\item Show that if $a\subseteq b$ and $b\subseteq c$ then
  $a\subseteq c$.
\item Show that $a\subseteq b$ if and only if $a\cap b=a$.
\end{enumerate}

\clearpage 

\section{Models}


\begin{enumerate}[leftmargin=*]
\item What does it mean for a predicate logic sentence to be
  \emph{consistent}?

\item Suppose that we had an algorithm that determined whether
  sentences are consistent. Explain how we could use this algorithm to
  determine if arguments are valid.
  
\item Consider the following English sentences, along with the two
  possible translations into predicate logic. Are the two PL sentences
  logically equivalent? Does one imply the other? How does this
  information bear on your judgment about which is the best
  translation?

\begin{enumerate}
\item Only students who do the homework will learn logic. ($Sx,Hx,Lx$)
  \[ \forall x(Lx\to (Sx\wedge Hx)) \]
  \[ \forall x((Sx\wedge Lx)\to Hx) \]
  
\item There is some student who respects only those professors who
  respect all students. ($Sx,Px,Rxy$)
\[ \exists x(Sx\wedge \forall y(Rxy\to (Py\wedge \forall z(Sz\to
  Ryz)))) \]
\[ \exists x(Sx\wedge \forall y((Py\wedge Rxy)\to \forall z(Sz\to
  Ryz))) \]  

\end{enumerate}

\item Explain why the sentence $\exists x(Mx\to Dx)$ is \emph{not} a
  good translation of ``There is a melancholy Dane.''

\item Provide models to show that the following sequents are invalid:
\begin{enumerate}
\item $\forall x(Fx\vee Gx)\:\vdash \:\forall xFx\vee \forall xGx$
\item $\forall xFx\to \forall xGx\:\vdash \: \forall x(Fx\to Gx)$
\item $\exists x(Fx\to P)\:\vdash \: \exists xFx\to P$
\end{enumerate}

\item The EE rule requires that the arbitrary name that is used
in the instance of the existential formula does \emph{not} appear in
(a) the existential formula, (b) the auxiliary assumptions used to
derive the conclusion, and (c) the conclusion itself. Explain why
dropping any one of these three restrictions would lead to an unsound
rule.

\end{enumerate}

\newpage

\begin{enumerate}[resume]

\item Which of the following sentences are true in which of the
diagrams below.

\begin{enumerate}
  \item $\forall x \forall y (Rxy \rightarrow Ryy)$
  \item $\forall x \exists y (Rxy \wedge Ryx)$
  \item $\exists x \forall y (Rxy \rightarrow \exists z Ryz)$
  \item $\forall x \exists y (Rxy \wedge \forall z (Ryz \rightarrow Rxz))$
  \item $\exists x \exists y (Rxy \wedge \neg Ryx)$
\end{enumerate}

\begin{tikzpicture}[>=Stealth,thick]
  % nodes
  \node (a) at (0,1.5) {$a$};
  \node (b) at (-1,-1) {$b$};
  \node (c) at (1,-1) {$c$};

  % arrows
  \draw[->] (b) -- (a);
  \draw[->] (c) -- (a);
  \draw[->, looseness=15, out=60, in=120] (a) to (a); % loop on a

\begin{scope}[xshift=4cm]
    \node (a2) at (0,1.5) {$a$};
    \node (b2) at (-1,-1) {$b$};
    \node (c2) at (1,-1) {$c$};

    \draw[->] (b2) -- (a2);
    \draw[->] (a2) -- (c2);
    \draw[->] (c2) -- (a2);
    \draw[->, looseness=15, out=-120, in=-60] (b2) to (b2);
  \end{scope}

 \begin{scope}[xshift=8cm]
    \node (a) at (0,1.5) {$a$};
    \node (b) at (-1,-1) {$b$};
    \node (c) at (1,-1) {$c$};

    \draw[->] (a) -- (b);
    \draw[->] (b) -- (c);
    \draw[->] (c) -- (a);
  \end{scope}

   \begin{scope}[xshift=12cm]
    \node (a) at (0,1.5) {$a$};
    \node (b) at (-1,-1) {$b$};
    \node (c) at (1,-1) {$c$};

    % Single arrows
    \draw[->] (a) -- (b);
    \draw[->] (c) -- (a);

    % Parallel arrows between b and c
    \draw[->,transform canvas={yshift=3pt}] (b) -- (c);
    \draw[->,transform canvas={yshift=-3pt}] (c) -- (b);
  \end{scope}


  
\end{tikzpicture}

\end{enumerate}




\end{document}

%%% Local Variables:
%%% mode: latex
%%% TeX-master: t
%%% End:


\end{document}



%%% Local Variables:
%%% mode: latex
%%% TeX-master: t
%%% End:



