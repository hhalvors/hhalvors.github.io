\documentclass[12pt,fleqn]{article}
\usepackage{fullpage}
\usepackage{amsmath}
\usepackage{array}
\renewcommand{\thesubsection}{\Alph{subsection}.}
\usepackage{enumitem}
\setlength{\parindent}{0em}
\usepackage{amssymb,amsmath}
\begin{document}

\section*{Intro logic: midterm exam 2025}

Version 2 -- for makeup exam

\bigskip \textbf{INSTRUCTIONS:} (1) Please note that there are exam
questions on the front and the back of this sheet. (2) You have 80
minutes to complete the exam. (3) Please write your \textbf{name} and
\textbf{preceptor's name} and \textbf{honor pledge} on the front of
the exam booklet. (4) The only resource you may consult is your
prepared page of notes. (5) You may use additional exam booklets to
write scratch work.

\subsection{Translation}

Translate the following sentences into propositional logic. In each
case, clearly indicate what letters you are assigning to atomic
sentences.  (2 points each)

\begin{enumerate}[leftmargin=*]
\item Either Bob gets a front-row seat, or if Bob does not go with
  friends then Bob will not enjoy the concert.
\item If David does not exercise regularly or does not eat healthy
  meals, then David will not maintain good health.
\item Carla will increase her chances of admission only if Carla
  submits her application early and asks for strong recommendation
  letters.

\end{enumerate}

\subsection{Semantics (truth tables)}

\begin{enumerate}[leftmargin=*]

\item For each of the following sentences, state whether it is a
  tautology, contingency, or inconsistency, and justify your claim in
  terms of truth tables. (3 points each)
  \begin{enumerate}
  \item $P\to (Q\to (R\to (S\to P)))$
  \item $(P\wedge Q)\vee (\neg P\wedge \neg Q)$
  \end{enumerate}
  
\item For each of the following arguments, state whether it is valid
  or invalid, and justify your claim in terms of truth tables. (3
  points each)
  \begin{enumerate}
  \item $P\to Q\:\vdash \: P\to (Q\wedge R)$
  \item $Q\to R\:\vdash \: (P\vee Q)\to (P\vee R)$
\end{enumerate}


\end{enumerate}

\newpage

\subsection{Proofs}

Prove the following. Besides the basic rules, you may also use cut and
replacement, but only if you include a proof of the relevant ``lemma''
in your exam booklet. (4 points each)

\begin{enumerate}[leftmargin=*]

\item $P,\neg P\: \vdash \: Q$ 

\item $P\vee Q,\neg P\:\vdash \: Q$

\item $\neg P\to Q\:\vdash\: P\vee Q$  

\item $(\neg P\vee Q)\to (P\vee Q)\: \vdash\: P\vee Q$

\end{enumerate}


\subsection{Conceptual}

\begin{enumerate}[leftmargin=*]
  
\item Is there a correctly written proof with the following line
  fragment? Justify your answer by showing that the relevant argument
  is valid or invalid, and by invoking soundness or completeness. The
  symbol ``$\varnothing$'' means no dependency numbers. (4 points)
  \[ \begin{array}{lc>{$}p{5cm}<{$}p{3cm}l}
       \varnothing & (n) & ((P\to Q)\to \neg P)\to \neg P  & \end{array}
   \]
   
\item Suppose that $\varphi$ and $\psi$ are contingencies. Can
  $\varphi \to \psi$ be a tautology, contingency, or inconsistency?
  Justify your answers. (4 points)
\end{enumerate}
  

\bigskip

\noindent
\rule{0.45\textwidth}{0.4pt} {\large\bf END} 
\rule{0.45\textwidth}{0.4pt}
  

  


  


  



\end{document}

%%% Local Variables:
%%% mode: latex
%%% TeX-master: t
%%% End:
