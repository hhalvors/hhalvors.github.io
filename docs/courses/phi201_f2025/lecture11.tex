\documentclass[aspectratio=169,17pt,fleqn]{beamer}

\usetheme{Madrid}
\usecolortheme{dolphin}
\setbeamertemplate{navigation symbols}{}
\setbeamertemplate{footline}[frame number]

\usepackage{amsmath,amssymb,array}


\usepackage{fontspec}    
\usepackage{unicode-math}  

\AtBeginSection[]{
  \begin{frame}
    \centering
    \vfill
    \Large\insertsectionhead
    \vfill
  \end{frame}
}

\usepackage{tikz}
\usetikzlibrary{positioning,matrix}


\title{Review: translations and proofs}
\author{PHI~201 -- Introductory Logic}
\date{November 17, 2025}

\begin{document}

\frame{\titlepage}

\section{Proofs}

\begin{frame}{Quantifiers of the same type}

  \[ \forall x\forall y\,\varphi \:\equiv \: \forall y\forall
    x\,\varphi \]
  \[ \exists x\exists y\,\varphi \:\equiv \: \exists y\exists
    x\,\varphi \]

  \textbf{Fact:} Quantifiers of the same type commute. That is,
  universal quantifiers commute with each other, and existential
  quantifiers commute with each other.



\end{frame}

\begin{frame}{Prenex normal form}

  \textbf{Fact:} Every sentence in predicate logic is equivalent to a
  sentence where all the quantifiers occur at the beginning.

\end{frame}

\begin{frame}{Pushing negation inside}

  \[ \neg \forall x\,\varphi \equiv \exists x\neg \varphi \]
  \[ \neg\exists x\,\varphi\equiv\forall x\neg \varphi \]



\end{frame}

\begin{frame}{Equivalent formulas}

  For purposes of establishing equivalence, you can treat the
  variables inside a formula as names.

  \bigskip If quantifiers are stripped, and variables replaced by
  distinct names, then the process can be reversed to put the
  quantifiers back on.

\end{frame}

\begin{frame}{Equivalent formulas}

\textbf{Fact:} If $\varphi (a_1,\dots ,a_n)$ is equivalent to
$\psi (a_1,\dots ,a_n)$, then
$Q_1x_1\cdots Q_nx_n\,\varphi (x_1,\dots ,x_n)$ is equivalent to
$Q_1x_1\cdots Q_nx_n\,\psi (x_1,\dots ,x_n)$.
  
\bigskip Practical upshot: you can detect equivalence by looking at
the formula inside quantifiers.




\end{frame}

\begin{frame}
\small 
  
  \begin{tabular}{>{\raggedleft\arraybackslash}p{1.5cm} >{\centering\arraybackslash}p{1.0cm} p{8cm} >{\raggedright\arraybackslash}p{3.5cm}}
    1 & (1) & $\forall x \exists y \forall z ((Fx \wedge Gy) \to Hz)$ & A \\
    1 & (2) & $\exists y \forall z ((Fa \wedge Gy) \to Hz)$ & 1 UE \\
    3 & (3) & $\forall z ((Fa \wedge Gb) \to Hz)$ & A \\
    3 & (4) & $(Fa \wedge Gb) \to Hc$ & 3 UE \\
    3 & (5) & $(\neg Fa\vee \neg Gb)\vee Hc$ & 4 TTV \\
    3 & (6) & $\forall z((\neg Fa\vee \neg Gb)\vee Hz)$ & 5 UI \\
    3 & (7) & $\exists y\forall z((\neg Fa\vee Gy)\vee Hz)$ & 6 EI \\
    1 & (8) & $\exists y\forall z((\neg Fa\vee Gy)\vee Hz)$ & 2,3,7 EE
    \\
    1 & (9) & $\forall x\exists y\forall z((\neg Fx\vee Gy)\vee Hz)$ &
                                                                       8
                                                                       UI \end{tabular}

\end{frame}

\begin{frame}

  \[ \begin{array}{rcl}
       Fy\wedge \exists xGx & \equiv & \exists
                                                  x(Fy\wedge Gx) \\[5pt]
       Fy\vee \exists xGx &\equiv & \exists
                                               x(Fy\vee Gx) \\[5pt]
       Fy\to \exists xGx & \equiv & \exists
                                               x(Fy\to Gx) \end{array}
   \]

 \end{frame}

 \begin{frame}

   \[ \begin{array}{rcl}
        \exists x(Gx\wedge Fy) & \equiv & \exists xGx\wedge Fy \\[5pt]
        \exists x(Gx\vee Fy) & \equiv & \exists xGx\wedge Fy \\[5pt]
        \exists x(Gx\to Fy) & \equiv & \forall xGx\to Fy \end{array} \]


\end{frame}

\begin{frame}

  \[ \begin{array}{rcl}
       \forall y(\exists xGx\to Fy) & \equiv & \forall y\forall
                                               x(Gx\to Fy)  \\[5pt]
     & \equiv & \exists xGx\to \forall yGy \end{array} \]


\end{frame}

\begin{frame}

  $\forall xFx\to Gb\:\vdash\:\exists x(Fx\to Gb)$

  \vspace{10em}




\end{frame}
 
\begin{frame}{Test your understanding}

  Use what you just learned to quickly see that the following sentence
  is a tautology:

  \[ \forall y\exists x(Rxy\to \forall zRzy) \]

\end{frame}



\section{Translation}

\begin{frame}{Nested Quantifiers and Scope}

  \textbf{Source note.} The next few slides are based in part on
  Warren Goldfarb's \emph{Deductive Logic}.

\medskip

\textbf{Nested quantifiers introduce a new complication:} \\
we must determine \emph{which quantifier governs which.}

\medskip
\begin{itemize}
  \item Quantifiers have \textbf{scope}, and one may fall inside another.
  \item To paraphrase, it helps to work \textbf{from the outside in}.
  \item First decide whether the statement as a whole is \textbf{universal} $(\forall)$
        or \textbf{existential} $(\exists)$.
  \item Then paraphrase the \textbf{open formula} inside its scope.
\end{itemize}
\end{frame}

\begin{frame}{Example Sentences}
We assume—for now—that the \textbf{universe of discourse} is the class of \textbf{persons}.

\medskip
Consider the following English sentences:

\begin{enumerate}
  \item Every critic admires some painter.
  \item Every critic is admired by some painter.
  \item Every critic admires all painters.
\end{enumerate}

All three sentences are \textbf{universal quantifications}.
\end{frame}

\begin{frame}{Example (1): Every critic admires some painter}

\textbf{Step 1 (outer quantifier):}
\[
  \forall x\, (x \text{ is a critic} \rightarrow \dots)
\]

\textbf{Step 2 (inner scope):}
\[
  x \text{ admires some painter}
  \quad \leadsto \quad
  \exists y\, (y \text{ is a painter} \wedge x \text{ admires } y)
\]

\textbf{Final formalization:}
\[
  \forall x\, (Cx \rightarrow \exists y\, (Py \wedge Axy))
\]

\end{frame}


\begin{frame}{Example (2): Every critic is admired by some painter}

\textbf{Paraphrase structure:}
\[
  \forall x\, (Cx \rightarrow \exists y\, (Py \wedge Ayx))
\]

Note the reversal in the predicate:
\[
  Ayx \equiv y \text{ admires } x.
\]

\textbf{Final symbolic form:}
\[
  \forall x\, (Cx \rightarrow \exists y\, (Py \wedge Ayx))
\]
\end{frame}

\begin{frame}{Example (3): Every critic admires all painters}

\textbf{Step 1 (outer quantifier):}
\[
  \forall x\, (Cx \rightarrow \dots)
\]

\textbf{Step 2 (scope):}
\[
  x \text{ admires all painters}
  \quad \leadsto \quad
  \forall y (Py \rightarrow Axy)
\]

\textbf{Final formalization:}
\[
  \forall x\, (Cx \rightarrow \forall y (Py \rightarrow Axy))
\]
\end{frame}

\begin{frame}{Existential sentences}


  There is a painter who is admired by every critic.

  \bigskip $\exists x(x\;\text{is a painter}\;\wedge x \;\text{is admired by every critic})$

\vspace{2em} Some critics admire all painters.

\vspace{2em} There is a critic who admires no painters.


\end{frame}



\begin{frame}

  $P$'s bear the relation $R$ only to $Q$'s
  \[ \forall x(Px\to \forall y(Rxy\to Qy)) \]

  Conversational implicature of if and only if: ``Danes only trust
  other Danes.''
  \[ \forall x(Dx\to \forall y(Txy\leftrightarrow Dy)) \]

  
\end{frame}

\begin{frame}

  Only the $F$ that/who are $G$ are $H$.

  \[ \forall x(Fx\to (Hx\to Gx)) \]
  \[ \forall x(Hx\to (Fx\wedge Gx)) \]


\end{frame}

\begin{frame}

  $P$'s bear the relation $R$ only to $Q$'s that/who are $S$.

  
  
\end{frame}

\begin{frame}

  $a$ only respects Harvard professors who acknowledge that Princeton
  is superior.

\[ \forall x((Hx\wedge Rax)\to Ax) \]
\[ \forall x(Rax\to (Hx\wedge Ax)) \]

\end{frame}



\end{document}

%%% Local Variables:
%%% mode: latex
%%% TeX-master: t
%%% End:
