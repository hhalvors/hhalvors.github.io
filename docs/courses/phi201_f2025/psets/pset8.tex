\documentclass[fleqn,12pt]{article}
\usepackage{amsmath}
\usepackage[colorlinks]{hyperref}
\usepackage{fullpage}
\setlength{\parindent}{0em}
\setlength{\parskip}{1em}
\usepackage{array}
\usepackage{tikz}
\usetikzlibrary{positioning}
\begin{document}
\thispagestyle{empty}

\section*{Logic pset 8}

Please answer \textbf{ONE} of the following questions. Some of these
problems are rather complex, but your answer shouldn't be longer than
two pages. All of these are exercises in Chapter 9 of HLW.

\begin{enumerate}

\item Prove that the set $\{ \neg ,\leftrightarrow \}$ is not
  expressively complete.

\item Prove the steps of the soundness theorem for the $\vee$I and
  $\vee$E rules.

% \item Show that if $\varphi$ is inconsistent, then any substitution
%  instance of $\varphi$ is inconsistent.  Here we mean ``inconsistent''
%  in the semantic sense of being assigned $0$ by all valuations.

\item Show that if $\varphi$ is a contingent sentence, then
  $\varphi$ has an inconsistent substitution instance.

\item In this exercise, you're asked to show that the RA rule is
  redundant.  We write $\Gamma\succ\varphi$ to indicate that there is
  a proof of $\varphi$ from $\Gamma$ that does \emph{not} use RA. Show
  that if $\Gamma\vdash\varphi$ then $\Gamma\succ\varphi$.

\item Let's say that a sentence $\varphi$ of propositional logic is
  ``conjunctive'' just in case it contains no connectives besides
  $\wedge$ and $\neg$. That is, conjunctive sentences do not contain
  $\vee$ or $\to$. If $\Gamma$ is a set of conjunctive sentences and
  $\varphi$ is a conjunctive sentence, we write $\Gamma\succ\varphi$
  to indicate that there is a proof of $\varphi$ from $\Gamma$ that
  uses only DN, RA, $\wedge$I, or $\wedge$E. Show that if
  $\Gamma\vdash\varphi$ then $\Gamma\succ\varphi$. (You don't need to
  prove the cases for all inference rules that define $\vdash$; a
  representative sample will suffice.)

\item Give introduction and elimination rules for the nand connective
  $\uparrow$. (You may use the symbol $\bot$ for a generic
  contradiction.) Use these rules to derive the sequent:
  \[ P\:\vdash \: (P\uparrow P)\uparrow (Q\uparrow Q) \] Prove that
  your rules are sound relative to the truth-table for $\uparrow$.

\end{enumerate}  

\end{document}


%%% Local Variables:
%%% mode: latex
%%% TeX-master: t
%%% End:
