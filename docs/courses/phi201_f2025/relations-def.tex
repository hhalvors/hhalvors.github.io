\documentclass[12pt,fleqn]{article}
\usepackage[margin=1in]{geometry}
\usepackage{amsmath, amssymb}

\title{Worksheet: Relational Properties and Spans}
\date{}
\begin{document}
\maketitle

\section*{Background}

Let $S$ and $R$ be binary relation symbols.  
We say that \emph{$x$ and $y$ are spanned by $S$} if there exists a $z$ such that
\[
S z x \;\wedge\; S z y.
\]
We consider two different ways of relating $R$ and $S$:

\begin{description}
    \item[(A) One-way span axiom:] 
    \[
    \forall x \forall y\bigl( \exists z(Szx \land Syx) \rightarrow Rxy \bigr).
    \]
    This says: if $x$ and $y$ share a common $S$-predecessor, then $Rxy$ holds.

    \item[(B) Definitional equivalence:]
    \[
    \forall x \forall y\bigl( Rxy \leftrightarrow \exists z(Szx \land Syx) \bigr).
    \]
    Here $R$ is \emph{exactly} the span of $S$.
\end{description}

We investigate which relational properties transfer from $S$ to $R$ under each assumption.

\bigskip

%%%%%%%%%%%%%%%%%%%%%%%%%%%%%%%%%%%%%%%%%%%%%%
\section*{Part I: Countermodels under the One-Way Span Axiom}

\subsection*{1. Nothing interesting follows from (A)}

\textbf{Task:}  
Give a structure $\mathcal{M}$ with domain $D$ and interpretations of $S$ and $R$ such that:

\begin{enumerate}
    \item $\mathcal{M} \models \forall x\forall y( \exists z(Szx \land Syx) \to Rxy)$, but
    \item $R$ fails to be reflexive, symmetric, and transitive.
\end{enumerate}

\textbf{Hint:}  
Let $S$ be empty, and let $R$ be \emph{anything at all}.  
Explain why the implication in (A) is automatically satisfied.

\bigskip

\subsection*{2. A more interesting countermodel}

Now give a structure where $S$ \emph{does} have nontrivial spans (i.e.\ some pairs $x,y$ share an $S$-predecessor), but $R$ still fails to have any nice property you choose (reflexivity, symmetry, or transitivity).

\textbf{Write down explicitly:}

\begin{itemize}
    \item domain $D$,
    \item extension of $S$,
    \item extension of $R$,
    \item verification that (A) holds,
    \item verification that the chosen property of $R$ fails.
\end{itemize}

\bigskip
\hrule
\bigskip

%%%%%%%%%%%%%%%%%%%%%%%%%%%%%%%%%%%%%%%%%%%%%%
\section*{Part II: Property Transfer under Definitional Equivalence}

Now assume the stronger connection (B):
\[
Rxy \quad\text{iff}\quad \exists z(Szx \land Syx).
\]

\subsection*{3. Symmetry of $R$}

Show that under (B),
\[
\forall x\forall y(Rxy \to Ryx)
\]
is valid in all structures.

\textbf{Task:} Prove the sequent
\[
\forall x\forall y\bigl(Rxy \leftrightarrow \exists z(Szx \land Syx)\bigr) 
\;\;\vdash\;\;
\forall x\forall y(Rxy \rightarrow Ryx)
\]
using the HLW/Lemmon natural deduction rules.

\bigskip

\subsection*{4. Failure of Transitivity}

Show that even under (B), $R$ need \emph{not} be transitive.  
Construct a countermodel.

\textbf{Task:} Provide a structure $\mathcal{M}$ such that:

\begin{itemize}
    \item $\mathcal{M} \models$ (B), i.e.\ $R$ is \emph{exactly} the $S$-span;
    \item $\mathcal{M} \not\models \forall x\forall y\forall z\bigl( (Rxy \land Ryz) \to Rxz \bigr)$.
\end{itemize}

\textbf{Hint:} Make three points $a,b,c$ which pairwise share different $S$-predecessors, but no single predecessor is shared by all three.

\bigskip
\hrule
\bigskip

%%%%%%%%%%%%%%%%%%%%%%%%%%%%%%%%%%%%%%%%%%%%%%
\section*{Part III: Reflexivity Transfer}

\subsection*{5. When does $R$ become reflexive?}

We want $Rxx$ to hold for every $x$.  
Under (B), this means:
\[
\forall x\, Rxx 
\quad\text{iff}\quad
\forall x\, \exists z( Szx \land Sxx ).
\]

\textbf{Task A:}  
Give conditions on $S$ that ensure $\forall x Rxx$ holds.  
(Hint: consider \emph{left-seriality} \(\forall x \exists z\, Szx\) and \emph{reflexivity} \(\forall x\, Sxx\).)

\medskip

\textbf{Task B:}  
Prove the following sequent in Lemmon/HLW style:
\[
\begin{aligned}
&\forall x\forall y\bigl(Rxy \leftrightarrow \exists z(Szx \land Syx)\bigr), \\
&\forall x Sxx, \\
&\forall x \exists z Szx
\quad\vdash\quad 
\forall x Rxx.
\end{aligned}
\]

You may assume standard relational equivalences and use EI/EG and UG in the HLW system.

\bigskip
\hrule
\bigskip

%%%%%%%%%%%%%%%%%%%%%%%%%%%%%%%%%%%%%%%%%%%%%%
\section*{Part IV: Extra Exploration (Optional)}

\subsection*{6. Defining spans the other way around}

Instead of taking $S$-predecessors, suppose we define $R$ by common $S$-\emph{successors}:
\[
Rxy \quad\text{iff}\quad \exists z(Sxz \land Syz).
\]

\textbf{Tasks:}
\begin{enumerate}
    \item Show that $R$ is automatically symmetric, regardless of what $S$ is.
    \item Investigate: under what conditions on $S$ will $R$ be reflexive? transitive?
    \item Compare your answers with Parts II--III.
\end{enumerate}

\bigskip

\end{document}

%%% Local Variables:
%%% mode: latex
%%% TeX-master: t
%%% End:
