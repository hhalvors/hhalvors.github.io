% Options for packages loaded elsewhere
\documentclass[12pt]{article}
\setlength{\parindent}{0em}
\setlength{\parskip}{1em}
\usepackage{fullpage,amssymb,array}

\title{logic pset4}
\author{}
\date{}

\begin{document}
\maketitle

\thispagestyle{empty}

\vspace*{-3em}

Resources: Lecture 4 and Chapter 6 (pp 84-99) of \emph{How Logic
  Works}.

\section*{A. Translation}

Represent the form of the following sentences in predicate logic.
We've suggested appropriate symbols. (We assume that quantifiers are
restricted to persons, so you don't need to add an extra predicate for
``$x$ is a person.'')

\begin{enumerate}
\item Only students who do the homework will learn logic. ($Sx,Hx,Lx$)

\item All students and professors get a discount. ($Sx,Px,Dx$)

\item Every student respects every professor who respects some
student. ($Sx,Px,Rxy$)
   
\item There is some student who respects only those professors who
   respect all students. ($Sx,Px,Rxy$)
\end{enumerate}


\section*{B. Proofs}

Prove the following sequents with the propositional logic rules plus
UE and UI. You may also use cut and replacement with any of the
``useful sequents'' from the back of the textbook.

\begin{enumerate}
\item $\forall x(Fx \to\forall yGy)\:\vdash\: \forall x\forall y(Fx\to
Gy)$
\item $\forall x\forall y(Fx\to Gy)\:\vdash\: \forall x(Fx\to \forall
yGy)$
\item $\vdash\:\forall x(\forall yRxy\to Rxx)$
\end{enumerate}


\section*{C. Conceptual}

It can be proven that
$\forall xFx\to \forall xGx\vdash \forall x(Fx\to Gx)$, but the
following attempt at a proof has a mistake. What is the mistake? A
good answer can be as short as one sentence.

\begin{tabular}{>{\raggedleft\arraybackslash}p{1.5cm}
    >{\centering\arraybackslash}p{1cm} p{4cm}
    >{\raggedright\arraybackslash}p{3.5cm}}
1 & (1) & $\forall xFx \to \forall xGx$ & A \\
2 & (2) & $Fa$ & A \\
2 & (3) & $\forall xFx$ & 2 UI \\
1,2 & (4) & $\forall xGx$ & 1,3 MP \\
1,2 & (5) & $Ga$ & 4 UE \\
1 & (6) & $Fa \to Ga$ & 2,5 CP \\
1 & (7) & $\forall x(Fx \to Gx)$ & 6 UI \\
\end{tabular}

\end{document}

%%% Local Variables:
%%% mode: latex
%%% TeX-master: t
%%% End:
