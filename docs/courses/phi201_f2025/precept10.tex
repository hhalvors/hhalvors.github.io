\documentclass[12pt,fleqn]{article}
\usepackage[margin=1in]{geometry}
\setlength{\parskip}{1em}
\setlength{\parindent}{0em}
\title{Precept exercises: Week 9}
\date{}

\usepackage{amsmath,amssymb}

\usepackage{enumitem}

\usepackage{tikz}
\usetikzlibrary{arrows.meta,positioning}
\usepackage{tikz-cd}

\begin{document}

\maketitle

\vspace*{-4em}


\begin{enumerate}[leftmargin=*]
\item What does it mean for a predicate logic sentence to be
  \emph{consistent}?

\item Suppose that we had an algorithm that determined whether
  sentences are consistent. Explain how we could use this algorithm to
  determine if arguments are valid.
  
\item Consider the following English sentences, along with the two
  possible translations into predicate logic. Are the two PL sentences
  logically equivalent? Does one imply the other? How does this
  information bear on your judgment about which is the best
  translation?

\begin{enumerate}
\item Only students who do the homework will learn logic. ($Sx,Hx,Lx$)
  \[ \forall x(Lx\to (Sx\wedge Hx)) \]
  \[ \forall x((Sx\wedge Lx)\to Hx) \]
  
\item There is some student who respects only those professors who
  respect all students. ($Sx,Px,Rxy$)
\[ \exists x(Sx\wedge \forall y(Rxy\to (Py\wedge \forall z(Sz\to
  Ryz)))) \]
\[ \exists x(Sx\wedge \forall y((Py\wedge Rxy)\to \forall z(Sz\to
  Ryz))) \]  

\end{enumerate}

\item Explain why the sentence $\exists x(Mx\to Dx)$ is \emph{not} a
  good translation of ``There is a melancholy Dane.''

\item Provide models to show that the following sequents are invalid:
\begin{enumerate}
\item $\forall x(Fx\vee Gx)\:\vdash \:\forall xFx\vee \forall xGx$
\item $\forall xFx\to \forall xGx\:\vdash \: \forall x(Fx\to Gx)$
\item $\exists x(Fx\to P)\:\vdash \: \exists xFx\to P$
\end{enumerate}

\item The EE rule requires that the arbitrary name that is used
in the instance of the existential formula does \emph{not} appear in
(a) the existential formula, (b) the auxiliary assumptions used to
derive the conclusion, and (c) the conclusion itself. Explain why
dropping any one of these three restrictions would lead to an unsound
rule.

\end{enumerate}

\newpage

\begin{enumerate}[resume]

\item Which of the following sentences are true in which of the
diagrams below.

\begin{enumerate}
  \item $\forall x \forall y (Rxy \rightarrow Ryy)$
  \item $\forall x \exists y (Rxy \wedge Ryx)$
  \item $\exists x \forall y (Rxy \rightarrow \exists z Ryz)$
  \item $\forall x \exists y (Rxy \wedge \forall z (Ryz \rightarrow Rxz))$
  \item $\exists x \exists y (Rxy \wedge \neg Ryx)$
\end{enumerate}

\begin{tikzpicture}[>=Stealth,thick]
  % nodes
  \node (a) at (0,1.5) {$a$};
  \node (b) at (-1,-1) {$b$};
  \node (c) at (1,-1) {$c$};

  % arrows
  \draw[->] (b) -- (a);
  \draw[->] (c) -- (a);
  \draw[->, looseness=15, out=60, in=120] (a) to (a); % loop on a

\begin{scope}[xshift=4cm]
    \node (a2) at (0,1.5) {$a$};
    \node (b2) at (-1,-1) {$b$};
    \node (c2) at (1,-1) {$c$};

    \draw[->] (b2) -- (a2);
    \draw[->] (a2) -- (c2);
    \draw[->] (c2) -- (a2);
    \draw[->, looseness=15, out=-120, in=-60] (b2) to (b2);
  \end{scope}

 \begin{scope}[xshift=8cm]
    \node (a) at (0,1.5) {$a$};
    \node (b) at (-1,-1) {$b$};
    \node (c) at (1,-1) {$c$};

    \draw[->] (a) -- (b);
    \draw[->] (b) -- (c);
    \draw[->] (c) -- (a);
  \end{scope}

   \begin{scope}[xshift=12cm]
    \node (a) at (0,1.5) {$a$};
    \node (b) at (-1,-1) {$b$};
    \node (c) at (1,-1) {$c$};

    % Single arrows
    \draw[->] (a) -- (b);
    \draw[->] (c) -- (a);

    % Parallel arrows between b and c
    \draw[->,transform canvas={yshift=3pt}] (b) -- (c);
    \draw[->,transform canvas={yshift=-3pt}] (c) -- (b);
  \end{scope}


  
\end{tikzpicture}

\end{enumerate}




\end{document}

%%% Local Variables:
%%% mode: latex
%%% TeX-master: t
%%% End:
