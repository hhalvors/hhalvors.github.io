\documentclass[11pt,fleqn]{article}
\sloppy
\usepackage{fullpage}
\usepackage{outlines}
\usepackage{soul}
\usepackage{enumitem} 
\setlength{\parskip}{1em}
\setlength{\parindent}{0em}
% \usepackage{setspace}
% \spacing{2}
\usepackage{tikz}
\usepackage{tikz-cd}
\usepackage{tikz-3dplot}

\begin{document}

\section*{Whence the quantum conundrum?}

\begin{outline}[enumerate]

\1 The philosophical lesson of quantum theory

\2 Randomness?

\2 Nonlocality?

\2 Many worlds?
  

\1 Brief history of classical physics 1600--1910

\2 Galileo: eliminate all but primary qualities  

\2 Newton: particles and gravity

\2 Laplace: the clockwork universe 

\2 Maxwell: electromagnetic waves 

\2 Einstein: relativity

Locality and the speed of light


\1 Elements of a classical worldview

\2 Ontology (what is the world made of?)

\2 Determinacy

State description

\2 Determinism

\3 Principle of sufficient reason

\3 Laplace's demon

\2 Locality (no ``spooky action at a distance'')

\2 Separability

The properties of a composite are determined by the properties of its
parts

\2 The aim of physics: what is a ``good'' theory? 






\1 Catastrophic fail

\2 Blackbody radiation

\2 Photoelectric effect 

\2 The Rutherford atom



\1 Niels Bohr (1885--1962)

\2 Energy levels and quantum leaps

\2 Prediction of spectral lines

\2 The periodic table


\1 Cobbling together a new ``theory'' (1915--1930)

\2 Matrix mechanics (Heisenberg 1925)

\3 \emph{beobachtbare Gr\"osse}

\3 uncertainty principle and disturbance interpretation

\2 Wave mechanics (Schr{\"o}dinger 1925)

\3 \emph{verdammte Quantenspringerei}

\3 Schr\"odinger equation

\2 Complementarity (Bohr 1927)

\2 Hilbert space formalism (von Neumann 1926--32)

No hidden variables theorem


\1 Einstein-Bohr debates (1927--1935)

\2 Photon box (1927)

\2 Einstein-Podolsky-Rosen argument (1935)

\2 Realism versus antirealism?  Rational versus irrational?
Determinism versus ``god rolls dice''?  

\1 Bohr for dummies, a.k.a.\ Copenhagen/Orthodox interpretation (1935--)

Collapse of the wavefunction

\1 Heretics

\2 David Bohm's ``hidden variable'' theory (1952)

\2 Many worlds (Hugh Everett 1957)

\2 Nonlocality (John Bell 1965)

\2 ``Quantum theory without observers'' (Bell 1986)

\2 Bell's disciples (1990-)


\end{outline}

\section*{Sean Carroll: physicists don't understand QM}

\begin{enumerate}

\item What criteria for \emph{understanding} does Carroll use to judge
  the current state of quantum physics?  How does quantum physics ---
  or how do quantum physicists --- fail to meet these criteria?  Do
  you agree with Carroll's criteria?  What might motivate them?

\item What, for Carroll, would a good physical theory be like?  What
  do you think motivates his vision?  Do you agree with it?

\item What are some other vices that Carroll sees in how physicists
  use, and think of, QM?  (Hint: look for where he uses words with
  negative connotations.)  Do you think that these are necessarily
  vices?  Could they be avoided?

\end{enumerate}


\section*{Stern-Gerlach experiments}

\subsection*{First experiment}

Two magnets, perpendicular orientation.  i.e.\ send ``up'' from
spin-$z$ to a spin-$x$ magnet. \newline
\begin{tikzcd}   \\
Z  \arrow[d] \arrow[r] & X \arrow{d} \arrow{r} & D \\
E  &  D & 
\end{tikzcd}

\bigskip \noindent Phenomena: the detectors click individually (not at
the same time), and in general there is a $50\%$ chance of each
detector clicking.  

% \begin{tikzcd}
% & & &  \\ 
% & A  \arrow[ru] \arrow[rd] & \\ 
% B \arrow[ru] \arrow[rd] &  & \\ \end{tikzcd}

\subsection*{Second experiment}

Two magnets, same orientation.  i.e.\ send ``up'' from spin-$z$ to a
spin-$z$ magnet. \newline
\begin{tikzcd}   \\
Z \arrow[d] \arrow[r] & Z  \arrow{d} \arrow{r} & D \\
E  &  D & 
\end{tikzcd}

\bigskip \noindent Phenomena: the top detector always clicks, and the
bottom detector never clicks.

\section*{Maudlin: intro to book}

\begin{enumerate}
\item Warm up (easy question): who are the bad guys in Maudlin's
  story?  Who are the good guys?
\item When you're going through the text, note words with strong
  positive or negative connotations.  (e.g.\ in application to a
  statement or theory, ``clear'' has a strong positive connotation.)
  For each such word, write down --- in a sentence or two --- what you
  think Maudlin means by it.
\item What does Maudlin mean by saying that quantum mechanics is not a
  theory?  What does he think it takes to be a theory?
\end{enumerate}




\end{document}



%%% Local Variables:
%%% mode: latex
%%% TeX-master: t
%%% End:
