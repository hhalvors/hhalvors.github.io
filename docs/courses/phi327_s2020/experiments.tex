\documentclass[12pt]{article}
\title{The Phenomena}
\author{Hans Halvorson}
\date{\today}
\usepackage{tikz}
\usepackage{tikz-cd}
\usepackage{tikz-3dplot}

\begin{document}

\maketitle

\section{Historical introduction}

In retrospect, there was evidence for quantum theory all the way back
in the time of Newton, when physicists started to understand that
light displays both wave-like and particle-like features.  But the
story really gets started in the late 19th century with four puzzling
phenomena.

\subsection{Spectral lines}

\subsection{Blackbody radiation}

\subsection{Stability of the atom}

\subsection{Quantum two-valuedness}

\section{Two slit experiments}

\subsection{Single slit}




\subsection{Two slit}



\subsection{Two slit with monitoring}

\section{Stern-Gerlach}

(Note: The experiments described in this section do \emph{not}
explicitly involve tensor products.)

In the following diagrams, the letters $X$ and $Z$ indicate
Stern-Gerlach magnets oriented, respectively along the $x$ and $z$
axes.  The arrows between nodes indicate the paths that the neutrons
can follow.  A path going to the right is ``up'' and a path going down
is ``down''.  The letter $E$ indicates a eraser that wipes out
neutrons.  The letter $D$ indicates a detector that counts the number
of incoming neutrons.  The letter $R$ indicates a reflector that
redirects that path of neutrons.

\subsection*{First experiment}

Two magnets, perpendicular orientation.  i.e.\ send ``up'' from
spin-$z$ to a spin-$x$ magnet

\begin{tikzcd}   \\
Z  \arrow[d] \arrow[r] & X \arrow{d} \arrow{r} & D \\
E  &  D & 
\end{tikzcd}

\bigskip \noindent Phenomena: the detectors click individually (not at
the same time), and in general there is a $50\%$ chance of each
detector clicking.  

% \begin{tikzcd}
% & & &  \\ 
% & A  \arrow[ru] \arrow[rd] & \\ 
% B \arrow[ru] \arrow[rd] &  & \\ \end{tikzcd}

\subsection*{Second experiment}

Two magnets, same orientation.  i.e.\ send ``up'' from spin-$z$ to a
spin-$z$ magnet

\begin{tikzcd}   \\
Z \arrow[d] \arrow[r] & Z  \arrow{d} \arrow{r} & D \\
E  &  D & 
\end{tikzcd}

\bigskip \noindent Phenomena: the top detector always clicks, and the
bottom detector never clicks.


\subsection*{Third experiment}

Two magnets, $45^\circ$ orientation

\subsection*{Fourth experiment}

Three magnets: spin-$z$, spin-$x$, spin-$z$

\begin{tikzcd}   \\
Z \arrow[d] \arrow[r] & X \arrow{d} \arrow{r} &  Z \arrow{r}\arrow{d} & D  \\
E  &  E  & D & 
\end{tikzcd}

\bigskip \noindent Phenomena: equal chance of detection for the top
and bottom detectors.

\subsection*{Fifth experiment (recombination)}

\begin{tikzcd}   \\
Z \arrow[d] \arrow[r] & X \arrow{d} \arrow{r} &  R  \arrow{d} &   \\
E  &  R \arrow[r]   & Z \arrow{r} \arrow{d} & D \\
   &      & D 
 \end{tikzcd}

 \bigskip \noindent Phenomena: top detector always clicks, bottom
 detector never clicks.

\begin{tikzcd}   \\
Z \arrow[d] \arrow[r] & X \arrow{d} \arrow{r} &  R  \arrow{d} &   \\
E  &  E     & Z \arrow{r} \arrow{d} & D \\
   &      & D 
 \end{tikzcd}

\bigskip \noindent Phenomena: equal chance of detection, both
detectors.


\bigskip \noindent Things to add?
\begin{enumerate}
\item Changing path lengths
\item Polarization
\end{enumerate}


\section{Nonlocality}

\begin{tikzcd}   
  S \arrow[rr] \arrow[dd] & & Z \arrow[r] \arrow[d] & A^+ \\
  & & A^- \\
  Z \arrow[r] \arrow[d] & B^+ \\
  B^- 
\end{tikzcd}

\bigskip \noindent Phenomenon: the detectors on the two sides are
perfectly anti-correlated.  $A^+$ clicks iff $B^-$ clicks, and $A^-$
clicks iff $B^+$ clicks. 

\bigskip \noindent 
\begin{tikzcd}   
  S \arrow[rr] \arrow[dd] & & X \arrow[r] \arrow[d] & A^+ \\
  & & A^- \\
  X \arrow[r] \arrow[d] & B^+ \\
  B^- 
\end{tikzcd}

\bigskip \noindent Phenomenon: the detectors on the two sides are
perfectly anti-correlated.  $A^+$ clicks iff $B^-$ clicks, and $A^-$
clicks iff $B^+$ clicks.

\bigskip\noindent \begin{tikzcd}   
  S \arrow[rr] \arrow[dd] & & Z \arrow[r] \arrow[d] & A^+ \\
  & & A^- \\
  X \arrow[r] \arrow[d] & B^+ \\
  B^- 
\end{tikzcd}

\bigskip \noindent Phenomenon: the detectors on the two sides are
uncorrelated.  For any pair, the probability of a joint detection is
$0.25$.

\subsection{EPR argument}

\noindent \textbf{Reality criterion:} If, without disturbing an
electron, we can predict with certainty what it would do, then it has
a feature (i.e.\ there is an \textbf{element of reality}) that
determines that it would do that.

\bigskip\noindent \textbf{Concluson:} The electron has a feature that
determines whether it would go up or down for $X$, and it has a
feature that determines whether it would go up or down for $Z$.








\end{document}


 \tdplotsetmaincoords{75}{140}

 \begin{figure}[htp]
 \centering
 \begin{tikzpicture}[tdplot_main_coords,scale=0.7]
  \newcommand{\nodin}[1]{coordinate(#1)node{#1}}%draw nodes on coordinates, I alter           
                                                   %this later to hide the nodes


  \draw[vlak,rand](0,-2,0)--(0,2,0)\nodin{A}--(0,2,2)\nodin{B}--(0,1,2)\nodin{C}--(0,1,1)\nodin{Q}--(0,-1,1)\nodin{D}--(0,-1,2)\nodin{E}--(0,-2,2)\nodin{F}--cycle;

  \draw[vlak,rand](F)++(-6,0,0)--(F)--(E)--++(-6,0,0)--cycle;
  \draw[vlak,rand](D)++(-6,0,0)--(D)--(E)--++(-6,0,0)--cycle;

  \draw[vlak,rand](0,-1,3)--(0,0,2)\nodin{G}--(0,1,3)\nodin{H}--(0,1,4)\nodin{I}--(0,-1,4)\nodin{J}--cycle;
  \draw[vlak,rand](G)++(-6,0,0)--(G)--(H)--++(-6,0,0)--cycle;
  \draw[vlak,rand](H)++(-6,0,0)--(H)--(I)--++(-6,0,0)--cycle;
  \draw[vlak,rand](J)++(-6,0,0)--(J)--(I)--++(-6,0,0)--cycle;
  \draw[vlak,rand](Q)++(-6,0,0)--(Q)--(D)--++(-6,0,0)--cycle;

  \foreach \x in{0,2,4,6}{
   \draw[vector,->] (-\x,0,2)--(-\x,1,1);
   \draw[vector,->] (-\x,0,2)--(-\x,0.5,1);
   \draw[vector,->] (-\x,0,2)--(-\x,0,1);
   \draw[vector,->] (-\x,0,2)--(-\x,-0.5,1);
   \draw[vector,->] (-\x,0,2)--(-\x,-1,1);
   }


  \draw[vlak,rand](0,-2,0)--(0,2,0)\nodin{A}--(0,2,2)\nodin{B}--(0,1,2)\nodin{C}--(0,1,1)\nodin{Q}--(0,-1,1)\nodin{D}--(0,-1,2)\nodin{E}--(0,-2,2)\nodin{F}--cycle;

  \draw(-9,0,1.8)\nodin{R};


  \draw[vlak,rand](R)++(0,-0.5,-0.5)--++(0,0,1)\nodin{R3}--++(0,1,0)\nodin{R2}--++(0,0,-1)\nodin{R1}--cycle;
  \draw[vlak,rand](R1)++(-1,0,0)--(R1)--(R2)--++(-1,0,0)--cycle;
  \draw[vlak,rand](R3)++(-1,0,0)--(R3)--(R2)--++(-1,0,0)--cycle;

  \tdplotsetrotatedcoords{0}{90}{0}
  \tdplotdrawarc[tdplot_rotated_coords,fill=black,rand]{(R)}{0.1}{0}{360}{}{}
  \draw[thick](R)--(-5,0,1.8)\nodin{M};

  \draw[thick](M)--(5,0,2)\nodin{M1};
  \draw[thick](M)--(5,0,1.6)\nodin{M2};

  \draw[vlak,rand](C)++(-6,0,0)--(C)--(B)--++(-6,0,0)--cycle;
  \draw[vlak,rand](A)++(-6,0,0)--(A)--(B)--++(-6,0,0)--cycle;


  \draw(5,0,0)\nodin{S};
  \draw[opp,rand](S)--++(0,2,0)--++(0,0,3.6)--++(0,-4,0)--++(0,0,-3.6)--cycle;

  \draw(5,0,1.8)\nodin{K};

  \foreach \x in{0.025,0.05,...,1}{
   \tdplotdrawarc[tdplot_rotated_coords,fill=black!80,opacity=0.15*\x,draw=none]{(K)}{1-     \x}{0}{360}{}{}
   }

  \tdplotdrawarc[tdplot_rotated_coords,fill=black]{(M1)}{0.03}{0}{360}{}{}
  \tdplotdrawarc[tdplot_rotated_coords,fill=black]{(M2)}{0.03}{0}{360}{}{}
  \draw[->](S)++(3,0,4)node[left]{classical}to[out=0,in=130](5,-0.6,2.4);

  \draw[->,shorten >=0.8pt](S)++(3,0,2)node[left]{quantum}to[out=0,in=150](M1);
  \draw[->,shorten >=0.8pt](S)++(3,0,2)to[out=0,in=220](M2);

  \draw(R1)++(-1,0,2.5)\nodin{R4};
  \draw(S)++(2,0,0)\nodin{S1};

  \pgfresetboundingbox 

  \draw[use as bounding box](S1)rectangle(R4);

 \end{tikzpicture}\caption{The Stern-Gerlach Experiment}\label{}\end{figure}



%%% Local Variables:
%%% mode: latex
%%% TeX-master: t
%%% End:
