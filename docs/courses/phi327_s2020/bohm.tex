
\documentclass[12pt,fleqn]{article}
\usepackage{amsfonts,amssymb,amsthm,amsmath}
\title{Bohmian Mechanics}
\author{Hans Halvorson}
\date{\today}
\usepackage{tikz}
\usepackage{url}
\usepackage{natbib}

\newcommand{\bh}{\mathbf{B}(\mathcal{H})}
\newcommand{\bk}{\mathbf{B}(\mathcal{K})}

\newcommand{\2}{\mathcal}      

\renewcommand{\emph}{\textbf}

\newcommand{\vp}{\varphi}

% \newcommand{\b}[1]{\mathbf{#1}}


\newcommand{\Ex}{\mathcal{E}}

\newcommand{\7}[1]{\mathbb{#1}}

\swapnumbers

\newtheorem{prop}{Proposition}
\newtheorem{thm}[prop]{Theorem}
\newtheorem{lemma}[prop]{Lemma}
\newtheorem*{bellthm}{Bell's Theorem}
\theoremstyle{definition}
\newtheorem*{fact}{Fact}
\newtheorem*{defn}{Definition}
\newtheorem*{example}{Example}
\newtheorem*{question}{Question}
\newtheorem{exercise}{Exercise}
\newtheorem*{disc}{Discussion}

% \usepackage{dsfont}


%% TO DO explain: it's possible to have a non-local theory with commutative
%% algebras

%% separable and non-separable states

%% TO DO: explain how the Bell inequality could be violated in
%% classical physics

%% TO DO: Werner states

\newcommand{\ke}[1]{|#1\rangle}
 \newcommand{\ket}[1]{|#1\rangle}

\newcommand{\singa}{\ensuremath{\varphi _{+z}\otimes \varphi _{-z}}}
\newcommand{\singb}{\ensuremath{\varphi _{-z}\otimes \varphi _{+z}}}
\newcommand{\s}[1]{\varphi _{#1}}

\begin{document}

%% TO DO: primitive ontology

%% contextuality

%% momentum

%% status of the wavefunction

\maketitle

In 1952, a young Princeton physicist named David Bohm discovered a new
theory whose predictions match, for all practical purposes, those of
standard quantum mechanics.  (I add ``for all practical purposes'' for
those who doubt that standard QM makes clear predictions.)  What's
more, defenders of Bohmian mechanics claim that this new theory has
none of the drawbacks of standard QM: it doesn't treat ``measurement''
as a primitive, it doesn't require violations of Schr{\"o}dinger's
equation, it has a clear ontology (i.e.\ local beables), it has a
deterministic equation of motion, etc.

In the past, opponents of Bohmian mechanics often dismissed it on the
basis that it is a ``metaphysical addition'' to QM, i.e.\ it adds new
entities without adding predictive power, therby violating Ockham's
razor.  However, that style of criticism has fallen out of fashion
along with other radical forms of empiricism that were popular in the
twentieth century.  Defenders of Bohmian mechanics will say that it's
not an addition to an already well defined theory (QM), but that Bohm
finally gave us a theory does the explanatory work that QM was
supposed to do.  What's more, they'll say, Bohm's theory doesn't call
for a radical reconceptualization of the human epistemic predicament
(as Bohr suggests), nor does it call for a radical rethinking of the
nature of personal identity (as Everett requires), nor does it call
for rejection of the Schr{\"o}dinger equation (as GRW requires).

The purpose of this chapter is to assess the extent to which Bohmian
mechanics can live up to these promises.  Now that radical empiricist
critiques of Bohm are out of fashion, the next main complaint about
Bohm is that it's nonlocal, and hence conflicts with relativity
theory.  Bohmians have an answer.  They claim that Bell's theorem
shows that any empirically adequate theory will be nonlocal.  They
have also argued (persuasively, I think) that Bohmian mechanics is
consistent with the empirical predictions of relativity theory.  So,
there's no easy argument from locality to not-Bohm.

In this chapter, I'll focus primarily on other issues with Bohmian
mechanics.  In particular, I will focus on the motivation behind
Bohmian mechanics, and whether what it gives us in the end is really
better than the alternatives.

\section{The preferred quantity}

Consider again the example from earlier in the course.  There are two
boxes, one labelled $L$ and one labelled $R$, and there is a marble
that we place into one of the boxes.  Let $\ket{L}$ be the state where
the marble is in the left box and let $\ket{R}$ be the state where the
marble is in the right box.  In addition to these two states, we have
the state $\ket{0}$, in which the marble is stationary, and the state
$\ket{1}$ in which the marble is moving.  We stipulate the following
relations (which are the structural relations between position and
momentum in any quantum system):
\begin{eqnarray*} \ket{0} & = & \tfrac{1}{\sqrt{2}}\left(
    \ket{L}+\ket{R} \right)
                                ,\\
  \ket{1} & = & \tfrac{1}{\sqrt{2}}\left( \ket{L}-\ket{R} \right)
                .\end{eqnarray*} The basic idea behind Bohmian
mechanics, in a quick snapshot, is that we described the states
$\ket{0}$ and $\ket{1}$ in the wrong way.  We said that $\ket{0}$ is
the state in which ``the marble is stationary.''  But why say that?
After all, the marble is stationary just in case it stays in the same
box from one moment to the next.  So we don't need any additional
states besides $\ket{L}$ and $\ket{R}$.  Sure, the vectors $\ket{0}$
and $\ket{1}$ are in the Hilbert space, but the question is what they
mean {\it qua} states.  The Bohmian proposal is that $\ket{0}$ and
$\ket{1}$ should be interpreted as probability distributions over the
space $\{ \ket{L},\ket{R} \}$.  In particular, both $\ket{0}$ and
$\ket{1}$ correspond to the flat distribution that assigns $0.5$ to
both $\ket{L}$ and $\ket{R}$.  (That might make $\ket{0}$ and
$\ket{1}$ seem like the same state.  But we will see later that $\ket{0}$ and
$\ket{1}$ have different dynamical properties.)

The move we just made can be generalized.  In any quantum system where
there is a well-behaved position operator $Q$, the Hilbert space $\2H$
is isomorphic to a space of functions $L(X)$, where $X$ is the set of
values that $Q$ can take.\footnote{I am oversimplifying here.
  Typically there will be several position operators, three for each
  particle.}  Thus, any state $\psi\in\2H$ can be {\it interpreted} as
a probability distribution over $X$, and that's exactly what the
Bohmian does.  The Bohmian treats the quantity $Q$ as privileged in
the sense that (1) $Q$ always has a definite value, and (2) every
state should be interpreted as a probability distribution over $Q$
values.

Bohmian mechanics is often described as a {\it hidden variable
  theory}, but that is misleading in a couple of ways.  First, it's
misleading from a mathematical point of view, because Bohmian
mechanics does not add new states to the formalism of QM.  Notice how
we described the situation above: the states of the marble are just
$\ket{L}$ and $\ket{R}$.  There was no need to supplement with any
further states.  Second, it's misleading from an epistemic point of
view to describe Bohmian mechanics as a hidden variable theory,
because the variables aren't hidden.  In fact, the states $\ket{L}$
and $\ket{R}$ are the opposite of hidden: they are what we see.

If interpreting states as distributions over positions were all there
were to Bohmian mechanics, then it could have been discovered by
anyone who understood Hilbert space.  But there is more to Bohmian
mechanics.  The genuinely new thing that Bohm discovered is a
``sub-dynamics'' on the position eigenstates.  Here's what I mean by a
sub-dynamics:

Suppose that the quantum-mechanical time evolution is represented by a
family of unitary operators $U_t$, where $t$ is a real-number
parameter.  In other words, as time ticks from $t$ to $t'$, the
quantum state changes from $U_t\psi$ to $U_{t'}\psi$.  Then typically
a position eigenstate such as $\ket{L}$ will be transformed by $U_t$
to something that is not a position eigenstate, say
\[ U_t\ket{L} \: = \: \tfrac{1}{\sqrt{2}}\left(\ket{L}+\ket{R} \right)
  .\] But that doesn't make any sense as a genuine change of the way
things are, because the superposition state on the right is not a
``way things are.''  The superposition state on the right represents
our ignorance of the way things are.

Now Bohm responds to this challenge not by adding something new to the
formalism of QM, but essentially by allowing there to be two states.
The first state can be called the \emph{value state}, and it must be
an element of the set $\{ \ket{L} ,\ket{R} \}$.  The second state can
be called the \emph{wavefunction}, and it can be any element of $\2H$.
The quantum dynamical evolution $U_t$ is only applied to the
wavefunction.  Bohm's big discovery was finding a second dynamical law
that applies to the value state and that meshes nicely with the first
dynamical law.

The one tricky thing about Bohm's second dynamical law is that it
depends on {\it both} the present value state and the wavefunction.
In other words, the future value state $\ket{j}$ is a function of the
present value state $\ket{i}$ and the present wavefunction $\psi$.
This is the reason that the wavefunction is sometimes called the
``pilot wave'' and the corresponding dynamical law is called the
``guiding equation.''

\section{Missing quantities}

What then are the challenges for Bohmian mechanics?  The first
challenge is to explain the utility of the quantum-mechanical
formalism, in particular the fact that operators (such as $P$) appear
to represent quantities (such as momentum) that occasionally have
values.  Unfortunately, momentum itself is not the best example to
start with.  It's tempting to think that momentum is nothing but
velocity times mass, and velocity is nothing more than a description
of position over time, and hence, if one has positions at all times,
then one automatically has velocities.  This also might tempt you to
think that there is nothing to explain vis-a-vis momentum, because to
measure momentum one just measures a series of positions.  So let's
start with a different example.

Consider a two-dimensional Hilbert space with spin-$z$ and spin-$x$
operators.  Suppose that we prefer spin-$z$ in the way that Bohmians
prefer position: at each time, the particle has value state either
$\ket{z+}$ or $\ket{z-}$, and its quantum state (i.e.\ wavefunction)
$\psi$ happens to give the best guess (prior to measurement) of what
this value state is.  Now suppose that we ``measure'' spin-$x$ and the
particle comes out up.  How are we supposed to understand what just
happened?  And how should we explain the fact that if we immediately
measure spin-$x$ again, we will get the same value?

Notice, in fact, that Bohm violates the EPR reality criterion at
precisely this point.  We can predict with certainty that spin-$x$
will have a value, but there Bohm says that there is no corresponding
element of reality!  I myself don't take this to be a damning feature
of Bohm, at least not if we can explain our predictive ability in
terms of the behavior of the fundamental elements of reality (in this
case, the values of spin-$z$).










\nocite{bohm-sep}

\nocite{daumer}


\bibliographystyle{chicago}

\bibliography{/Users/hhalvors/teaching/phi327_s2020/qbib}




\end{document}

%%% Local Variables:
%%% mode: latex
%%% TeX-master: t
%%% End:
