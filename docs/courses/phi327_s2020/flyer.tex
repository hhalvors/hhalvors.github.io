\documentclass[12pt]{article}
\setlength{\parskip}{1em}
\setlength{\parindent}{0em}
\begin{document}

\section*{Philosophy of Physics}

\textbf{Professor:} Hans Halvorson (hhalvors)

There is a (non-local) AI for the course (Alex Meehan). He will give
the lectures in the fourth week, but otherwise he act upon us from a
distance of approximately 250 miles.

The course is split into two components of approximately eight and four
weeks respectively. In the first component, we will ask why so many
people find quantum theory to be worth philosophizing about. (Is there a
problem with quantum theory? Does quantum theory demand a change in the
way we think about the world? Does quantum theory demand a change in how
we think about ourselves?) We'll focus on the measurement problem,
entanglement, no hidden variable theorems, and the relationship between
quantum nonlocality and relativistic causality. (We will mostly rely on
handouts, but there is one required book: Richard Healey, \emph{The
Quantum Revolution in Philosophy}.) In the second component, we will
have guest lectures from physicists and philosophers working at the
cutting edge of building this new worldview of physics.

\textbf{Assessment:} There will be approximately five problem sets for
the first component of the course. For the second component of the
course, there will be one longer paper (due on Dean's date), and a few
response exercises in conjunction with the guest lectures.

\subsection*{First component}

\begin{description}

\item[Week 1:] Overview, a little history, and a couple experiments

  For Wednesday's lecture, please read Weinberg, ``The trouble with
  quantum mechanics'' and Maudlin, selection from \emph{Philosophy of
    Physics: Quantum Theory} (found in the Course Materials section
  of the Blackboard site)

\item[Week 2:] Experiments and superposition

\item[Week 3:] The measurement problem

\item[Week 4:] More experiments, tensor products and dynamics

\item[Week 5:] The uncertainty principle, Kochen-Specker theorem

\item[Week 6:] EPR and Bohr

\item[Week 7:] Bell, quantum nonlocality and relativity

\item[Week 8:] Interpretations

\end{description}

\subsection*{Second component}

\begin{description}

\item[April 1:] Juan Maldacena, IAS. String theory

\item[April 6:] Adam Becker. Author of \emph{What is Real? The Unfinished Quest
for the Meaning of Quantum Physics.}

\item[April 8:] Carlo Rovelli, Marseille. Quantum gravity. Author of
\emph{Seven Brief Lessons on Physics}

\item[April 22:] Sean Carroll, CalTech. Physicist, philosopher, and author of
many books such as \emph{Something Deeply Hidden, The Big Picture,} and
\emph{From Eternity to Here}

\item[April 27:] Jill North, Rutgers Philosophy

\item[April 29:] Sheldon Goldstein, Rutgers Math, Physics, and
  Philosophy.  Alternatives to orthodox QM, especially Bohm

\item[TBD:] Herman Verlinde, Princeton Physics. Black holes, string theory

\end{description}

\end{document}


%%% Local Variables:
%%% mode: latex
%%% TeX-master: t
%%% End:
