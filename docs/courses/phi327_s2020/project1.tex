
\documentclass[12pt,fleqn]{article}
\usepackage{amsfonts,amssymb,amsthm,amsmath}
\title{Project ideas: EPR and Bell}
\author{} 
\date{}
\usepackage{tikz}
\usepackage{url}
\usepackage{natbib}

\newcommand{\bh}{\mathbf{B}(\mathcal{H})}
\newcommand{\bk}{\mathbf{B}(\mathcal{K})}

\newcommand{\2}{\mathcal}      

\renewcommand{\emph}{\textbf}

\newcommand{\vp}{\varphi}

% \newcommand{\b}[1]{\mathbf{#1}}


\newcommand{\Ex}{\mathcal{E}}

\newcommand{\7}[1]{\mathbb{#1}}

\swapnumbers

\newtheorem{prop}{Proposition}
\newtheorem{thm}[prop]{Theorem}
\newtheorem{lemma}[prop]{Lemma}
\newtheorem*{bellthm}{Bell's Theorem}
\theoremstyle{definition}
\newtheorem*{fact}{Fact}
\newtheorem*{defn}{Definition}
\newtheorem*{example}{Example}
\newtheorem*{question}{Question}
\newtheorem{exercise}{Exercise}
\newtheorem*{disc}{Discussion}

% \usepackage{dsfont}


%% TO DO explain: it's possible to have a non-local theory with commutative
%% algebras

%% separable and non-separable states

%% TO DO: explain how the Bell inequality could be violated in
%% classical physics

%% TO DO: Werner states

\newcommand{\ke}[1]{|#1\rangle}
 \newcommand{\ket}[1]{|#1\rangle}

\newcommand{\singa}{\ensuremath{\varphi _{+z}\otimes \varphi _{-z}}}
\newcommand{\singb}{\ensuremath{\varphi _{-z}\otimes \varphi _{+z}}}
\newcommand{\s}[1]{\varphi _{#1}}

\begin{document}

% \maketitle

\section*{Project ideas: EPR, Bohr, Bell, Nonlocality}

A ``project'' is meant to be something half way between a homework
assignment and a paper.  For technically oriented projects, the length
might be no more than a couple pages.  For historically oriented
projects, the length might be closer to six pages (double spaced).
Our goal is for you to use this project as an excuse to investigate
something that you are curious about.  Please feel free to use one of
the ideas below as your jumping off point, or to make up your own
project.

\begin{enumerate}

\item In the Bell notes, it's claimed that the Bell inequality can
  only be violated if both Alice and Bob have a pair of
  \emph{non-commuting} operators.  Prove this fact.
\item Graph the expectation values in the singlet state of the Bell
  observable
  \[ R = A_1(B_1+B_2)+A_2(B_1-B_2) ,\] as a function of $\theta$,
  where $B_1=I\otimes S_{\theta}$.  Find the range in which Bell's
  inequality is violated, and find the value where Bell's inequality
  is maximally violated.
\item Write out the EPR argument for the simultaneous reality of $B_1$
  and $B_2$ (Bob's position and momentum, or $I\otimes S_z$ and
  $I\otimes S_x$).  In doing so, provide a clear formulation of the
  EPR reality criterion.  Take a stance on whether or not the argument
  is valid.
\item What is locality and why is it important?  Is there a sense of
  locality that is not undermined by the violation of Bell's
  inequality?
\item Evaluate the debate between Maudlin and Werner over whether the
  violation of Bell's inequality implies nonlocality.  Or simply argue
  for one side of the issue.
\item In his answer to EPR, Bohr repeatedly accuses them of being
  ``ambiguous.''  Based on what he says in this paper, what do you
  think he's getting at?
\end{enumerate}




\end{document}

%%% Local Variables:
%%% mode: latex
%%% TeX-master: t
%%% End:
