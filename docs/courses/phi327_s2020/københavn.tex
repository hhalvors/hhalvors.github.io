\documentclass[12pt]{article}
\begin{document}

Det næste spørgsmål er: hvorvidt burde (eller kan) vores matematiske
konstruktioner afspejler verdenens egen struktur?  Lad mig introducere
et begreb som er dukket op inden videnskabsteori i det sidste par år.
På engelsk hedder begrebet ``surplus structure'', dvs. overflødig
struktur.  Ideen er at mange fysiske teorier bruger matematiske
strukturer, uden at ville sige at der er noget indenfor fysiske
realitet som korresponderer til den struktur.  For eksempel:

\begin{enumerate}
  \item Tænk på teorier om tid og rum, som specielle eller almene
relativitets teorier.  Disse teorier bruger koordinateensystemer, men
ikke nogen mener at rum har sin egen koordinatsystem.  Vi har et frit
valg ift en koordinatesystem.

\item Tænk nu på electromagnetisme.  (I USA og England taler man om
Maxwells teorie, men her i Danmark ved man at teorien blev skabt af
H.C. Ørsted.)  Den elektromagnetiske potentiel
\end{enumerate}

\end{document}


%%% Local Variables:
%%% mode: latex
%%% TeX-master: t
%%% End: