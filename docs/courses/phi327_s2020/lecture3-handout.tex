\documentclass[12pt,fleqn]{article}
\sloppy
% \usepackage{fullpage}
\usepackage{outlines}
\usepackage{soul}
\usepackage{enumitem} 
% \setlength{\parskip}{1em}
% \setlength{\parindent}{0em}
% \usepackage{setspace}
% \spacing{2}
\usepackage{tikz}
\usepackage{tikz-cd}
\usepackage{tikz-3dplot}

\usepackage{amsthm,amsmath,amsfonts}
\theoremstyle{definition}
\newtheorem*{exercise}{Exercise}
\title{Phil Physics: Week 2}
\date{}

\renewcommand{\emph}{\textbf}

\begin{document}
\maketitle

\section*{Maudlin: intro to book}

\begin{enumerate}
\item Warm up (easy question): who are the bad guys in Maudlin's
  story?  Who are the good guys?
\item When you're going through the text, note words with strong
  positive or negative connotations.  (e.g.\ in application to a
  statement or theory, ``clear'' has a strong positive connotation.)
  For each such word, write down --- in a sentence or two --- what you
  think Maudlin means by it.
\item What does Maudlin mean by saying that quantum mechanics is not a
  theory?  What does he think it takes to be a theory?
\end{enumerate}

\section*{Stern-Gerlach experiments}

We will begin now to look at the sort of phenomena that are explained
by quantum mechanics.  We start with a kind of experiment that was
conceived by Otto Stern in 1921, and carrier out by Walther Gerlach in
1922.  This experiment was supposed to be a crucial test of the new
quantum theory, because it predicted something different than
classical physics.  In particular, QM says that electrons have a
quantity called ``spin'' that has only two possible values ``up'' and
``down.''

Here's the setup of the experiment.  First of all, we have a source
$S$ that is emitting some stuff.  The nature of this stuff is not
``plain to sight'': it's so small that we cannot see whether we have a
bunch of particles, or a continuous field.  But we do have control
over the direction that this stuff travels, and we can see how it
interacts with certain measuring devices.

The original Stern-Gerlach experiment involved warming up silver in an
oven, which was thought to result in the emission of individual silver
atoms, and which were then collimated into a single ray.  The
important thing about silver atoms is that they have 47 electrons, so
the magnetic moments of the first 46 of them cancel each other out.
The net magnetic moment of the atom is the same as a single electron.
So from now on, we'll speak as the source is emitting a stream of
individual electrons.  That interpretation is supported by the fact
that if the stream is directed to a detector (e.g.\ a screen), then
that detector lights up at discrete moments, and not continuously.

The second thing we have is a pair of magnets that create an
inhomogeneous magnetic field between them.  When the electron passes
through this field, it's motion is expected to be altered by the
interaction of its magnetic moment with this magnetic field.  

The third thing we have is a screen behind the pair of magnets.  When
an electron hits this screen, it makes a flash.  

The prediction of classical physics was that, because the electrons
coming out of the oven have randomly distributed magnetic moments, the
flashes on the screen should be uniformly distributed throughout the
possible range.  The prediction of quantum mechanics was that ...

\subsection*{First experiment}

\begin{figure}[h]
\begin{tikzcd}   \\
  & & D_u \\
S \arrow[r] & Z \arrow{ru} \arrow{rd}  \\
&  & D_d  
\end{tikzcd} \end{figure}

As stated, the result of this first experiment is that $D_u$ registers
$50\%$ of the hits, and $D_d$ registers $50\%$ of the hits.  While
that result was not expected by classical physics, it's not as if it
cannot be explained by classical physics.

\begin{exercise} Construct a ``deterministic hidden variable model''
  of this experiment.  \end{exercise}

\begin{proof}[Solution] Suppose that of the particles emitted by $S$,
  $50\%$ are in state $Z_+$ and $50\%$ are in state $Z_{-}$.  To
  elaborate, we suppose that the state space of the system is
  $\{ Z_+,Z_{-}\}$, and that the quantity $Z$ takes value $i$ in state
  $Z_i$.  \end{proof}

\begin{exercise} Suppose that we ``look inside'' the particles coming
  out of $S$, and we can see nothing that explains why some go up and
  others go down.  In other words, we can find no labels like
  ``$Z_+$'' or ``$Z_{-}$''.  What kinds of theories could we use to
  explain the phenomena?  \end{exercise}



\subsection*{Second experiment}

For the second experiment, we replace the detector $D_u$ with an
eraser $E$, and we replace the detector $D_d$ with another $Z$ magnet.

\bigskip \begin{tikzcd}
                        &                &  E     &  \\
            S \arrow[r] & Z \arrow{ru} \arrow{rd} & & D_u \\
                        &                & Z \arrow{ru} \arrow{rd}
            \\
                        &                &       & D_d  
\end{tikzcd}


\bigskip \begin{exercise} What do you predict for the relative number of clicks
  in $D_u$ and $D_d$? \end{exercise}

If you predicted that $D_u$ and $D_d$ would each get $50\%$ of the
clicks, then that goes to show that the results of experiments can be
surprising.  For, in fact, $D_d$ now clicks $100\%$ of the time.

\begin{exercise} Consider the various explanations we gave for the
  first experiment.  Do any of them fail to explain this second
  experiment?  Does this second experiment give a more clear
  indication of what electron spin states are like? \end{exercise}

\begin{exercise} Consider the hypothesis:
  \begin{quote} (ND) The state of an electron is not disturbed by its
    passage through $Z$. \end{quote}
Does this experiment give any positive or negative evidence for
ND?  \end{exercise}

\subsection*{Third experiment}

We now consider what happens if we rotate the magnet through $90$
degrees.  If the previous magnet was called $Z$, this new one will be
called $X$.  In real life, we would now have to think of the
experiment as occurring in three-dimensions; but the schematic below
projects onto a two-dimensional plane.

\bigskip \begin{tikzcd}
                        &                &  E     &  \\
            S \arrow[r] & Z \arrow{ru} \arrow{rd} & & D_u \\
                        &                & X \arrow{ru} \arrow{rd}
            \\
                        &                &       & D_d  
\end{tikzcd}

\bigskip The result of this experiment is that $D_u$ and $D_d$ each
register $50\%$ of the hits.

\begin{exercise} Provide a ``deterministic hidden variable model'' for
  this experiment.  \end{exercise}

\subsection*{Fourth experiment}

This is where things start getting weird.  What we know want to know
is what happens if we determine that the electron is $Z_{-}$, then we
determine that it's $X_{-}$, and then we measure $Z$ again.  What do
you expect the result to be?

\bigskip \begin{tikzcd}
                        &                &  E     &  \\
            S \arrow[r] & Z \arrow{ru} \arrow{rd} & & E  \\
                        &                & X \arrow{ru} \arrow{rd} & &
                        D_u            \\
                        &                &       & Z \arrow{ru}
                        \arrow{rd} \\
                        & & & & D_d 
                      \end{tikzcd}


\bigskip\noindent The outcome of this  experiment is $50\%$ of clicks
for both $D_u$ and $D_d$.  

\begin{exercise} What does this experiment say about the
  non-disturbance (ND) hypothesis? \end{exercise}

\begin{exercise} Consider the hypothesis of determinism:
  \begin{quote} (D) Each electron has hidden variables $X_i$ and $Z_j$
    that determine whether it will go up or down through the various
    magnets. \end{quote} Does this experiment rule out D?  What about
  the combination of D and ND? \end{exercise}


\subsection*{Fifth experiment}

\begin{tikzcd}
              &                         & E     \\
S \arrow[r] & Z \arrow{ru} \arrow{rd} &                  &  R
\arrow{rd} & & D_u \\
            &                         & X  \arrow{ru} \arrow{rd} & & Z
            \arrow{ru} \arrow{rd}
            \\
            &  &  & R \arrow{ru} & & D_d  \end{tikzcd}

\bigskip\noindent In this more complicated experiment, we have two
reflectors labelled with $R$.  When turned on, these reflectors don't
do anything besides changing the electron's path.  When turned off,
these reflectors act like erasers.  Here, then, are the results of the experiment: 
\begin{enumerate}
\item Top reflector off: $D_u$ and $D_d$ register $50\%$.
\item Bottom reflector off: $D_u$ and $D_d$ register $50\%$.
\item Both reflectors on: $D_d$ registers $100\%$.
\end{enumerate}
It seems like something strange is going on here.  However, we can
modify the experiment to make sure that our $Z$ and $X$
measuring devices are functioning properly.  For example, if we
replace the reflectors with detectors, then we again get $50\%$ up and
$50\%$ down.  

\begin{exercise} Consider the hypothesis:
  \begin{quote} (EO) Just after the electron goes through the
    $X$ magnet, it is either $100\%$ on the up path, or $100\%$ on the
    down path. \end{quote}
  What evidence is there that EO is true? 
 \end{exercise}

\begin{exercise} Suppose that we begin the experiment without having
  decided whether to turn one of the reflectors off, and suppose that
  the electron has just gone through the $X$ magnet.  What do you
  think is the best hypothesis about the location of the
  electron?  \end{exercise}

\begin{exercise} Consider the following argument:
  \begin{quote}
    After passing through the $X$ magnet, an electron is either on the
    top path, or on the bottom path.  If it's on the bottom path, then
    we could turn off the top reflector (without disturbing the
    electron's state), and then it might end up at $D_u$.  Ditto for
    the top path.  In either case, the electron might end up at
    $D_u$. \end{quote} Do you think this argument is good or bad, and
  why?  \end{exercise}

\begin{exercise} Give a single deterministic hidden variable model
  that explains both the experiment with the top reflector off, and
  the experiment with both reflectors on. \end{exercise}

\begin{proof}[Solution] Suppose that $X$ magnets cause a particle to
  spawn a ``ghost twin'' that goes the opposite direction.  (This
  ghost twin is itself undetectable.)  If a particle doesn't meet its
  twin again, then it behaves no differently than before.  If a
  particle does meet its twin, then it kills him, and changes its own
  state to $Z_{-}$.
\end{proof}

\section*{Quantum models of Stern-Gerlach}

It's time to see how to use the quantum formalism in order to predict
the results of these kinds of experiments.

We assume that each electron has a state $v$, which is represented by
an element of a two-dimensional vector space.  For now, it will
suffice to think of the most familiar such vector space: the plane
$\mathbb{R}^2$ of real numbers.

The following table is an assignment of properties to vectors in
$\mathbb{R}^2$.

\[ \begin{array}{r c l c r c l}
     z_0 & = & \begin{pmatrix} 1 \\ 0 \end{pmatrix} & & z_1 & = & \begin{pmatrix} 0 \\ 1 \end{pmatrix} \\ \\
     x_1 & = & \frac{1}{\sqrt{2}}\begin{pmatrix} 1 \\ 1 \end{pmatrix}
         & & x_0 & = & \frac{1}{\sqrt{2}}\begin{pmatrix} 1 \\
       -1 \end{pmatrix} \end{array} \]

 \noindent The way we've set things up here, the electron cannot have
 any of the properties $z_i$ and $x_j$ simultaneously.  That's even
 stronger than the \emph{uncertainty relations}, which would say that
 you cannot \ul{know} the values of $Z$ and $X$ simultaneously.  So we
 might do better to talk about the \emph{indeterminacy relations},
 since $Z$ and $X$ cannot simultaneously have determinate values.

 But what happens if we measure $X$ when the particle is in state
 $z_1$ or $z_0$?  First a short answer, then a more elaborate answer.
 The short answer is that we should think of $X$ as associated with
 the two states $x_0$ and $x_1$; and if we measure $X$ when the system
 is in state $z$, then we should:
\begin{itemize}
\item Expect $0$ with probability $|\langle x_0,z\rangle |^2$, and in
  this case, change the state to $z_0$.
\item Expect $1$ with probability $|\langle x_1,z\rangle |^2$, and in
  this case, change the state to $z_1$. 
\end{itemize}
(What we've just written is called \emph{Born's rule}.)  Here
$\langle x_i,z\rangle$ is the \emph{inner product} of the two vectors
$x_i$ and $z$.  In the case at hand, it's none other than
$\cos ^2(\theta )$, where $\theta$ is the angle between $x_i$ and $z$.

This formalism already predicts the result of the second experiment:
after the first measurement of $Z$, the electrons that go down are in
state $z_0$, and so in the second measurement of $Z$, they will
definitely go down again.

The formalism also predicts the results of the third experiment: after
the measurement of $Z$, the electrons that go down are in state $z_0$.
Hence, a measurement of $X$ should give probability $0.5$ for going
up, and probability $0.5$ for going down.

The fifth experiment is a bit more tricky, and forces us to think
harder about what happens at the Stern-Gerlach magnet.  Does a $X$
magnet itself cause an electron to ``collapse'' into either state
$x_0$ or $x_1$?  If that were the case, then we would expect a
subsequent measurement of $Z$ to assign equal probabilities to $0$ and
$1$, whereas we know that we will always get $0$.

The ``orthodox'' answer from physics is that a $X$ magnet does not
itself cause the state to collapse to $x_0$ or $x_1$.  What does that
is the detectors placed in the two paths after the magnet.  Instead,
when the electron passes through the $X$ magnet, its spin degree of
freedom becomes ``entangled'' with its spatial location.  For
simplicity, let's write this as follows:
\[ \begin{array}{r c l }
     x_0\otimes m  & \Longrightarrow & x_0\otimes d \\
     x_1\otimes m  & \Longrightarrow & x_1\otimes u \end{array} \]
(Don't worry if you don't understand the symbols yet; they will be
explained.)  Then if both reflectors are on, the $d$ and $u$ states transform back
to $m$, which means that the states $x_0$ and $x_1$ can again be
``superposed'' to yield the state $z_0$.

\subsection*{Superposition}

As you know, vectors can be added.  The math is straightforward.  The
physics is also straightforward, if we're talking about \emph{waves}.
Just think of two waves approaching the shore from slightly different
directions.  When they come together, they superpose --- at some
points, their peaks meet and form a higher wavecrest, and at other
points their troughs meet and form a lower depression.  Of course,
there can also be points where they interfere, or cancel each other
out.

We have already represented the state of an electron by a vector.
Mathematically, these vectors can be superposed, i.e.\ added together.
But what does that mean physically?  Suppose we take the state $z_0$
where the electron has the property of ``down'' for $Z$, and the state
$z_1$ where the electron has the property of ``up'' for $Z$, and then
we add them together.  Does the resulting vector define a physical
state, and what is that state like?  

Since we have 
\[ z_0 = \begin{pmatrix} 1 \\ 0 \end{pmatrix} \qquad
  z_1=\begin{pmatrix} 0 \\ 1 \end{pmatrix} , \]
it follows that
\[ x_0 =\frac{1}{\sqrt{2}}(z_0+z_1) ,\qquad
  x_1=\frac{1}{\sqrt{2}}(z_0-z_1) .\] Hence, $x_0$ is a superposition
of $z_0$ and $z_1$, and $x_1$ is a different superposition of $z_0$
and $z_1$.  That is curious for several different reasons.  First,
what in the world does $Z$ have to do with $X$?  Aren't these supposed
to be independent axes?  How could summing a state with one apple and
a state with two apples yield a state with one orange?  Second, how
can summing together states where $Z$ is sharp give rise to states
where $Z$ is fuzzy?  That's especially puzzling because electrons
can't remain ambivalent about which way they'll go through a $Z$
magnet: they have to go up or down.

To get the feel for superposition, it might help to look at another
kind of experiment: the famous two-slit interference experiment.
Suppose that there's a stream of particles directed toward a screen
with two slits, and behind the screen there is another detector
screen.  Suppose also that there are little doors on the slits that we
can open and close.

In the first experiment, we close the bottom door so that the stream
only goes through the top door, and we see a pattern of detections on
the back screen like this:

\begin{figure}

  [Figure to be supplied in lecture]

\end{figure}

That's not surprising: we expect that the particles emerge from the
slit with fairly random momentum.  What's surprising is what happens
when we open the second door.  If the source were producing discrete
particles, then the prediction of classical physics would be two lumps
on the back screen, like this:


In contrast, if the source were producing waves, then classical
physics would predict that the waves coming out of the two slits would
interfere with each other, producing an interference pattern on the
back screen.


Quantum mechanics also predicts the interference pattern, and the
explanation goes like this: if only the top slit is open, then it
prepares particles in the state $z_0$.  If only the bottom slit is
open, then it prepares particles in the state $z_1$.  However, if both
slits are open, then the state is $\frac{1}{\sqrt{2}}(z_0+z_1)$.  This
latter state is {\it not} a state in which the particle definitely
goes through the top or bottom slit.  Instead, it's more like a wave
that goes through {\it both} the top and bottom slits, and then
interferes with itself on the other side.








\end{document}



%%% Local Variables:
%%% mode: latex
%%% TeX-master: t
%%% End:
