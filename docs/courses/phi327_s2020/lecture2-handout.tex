\documentclass[11pt,fleqn]{article}
\sloppy
\usepackage{fullpage}
\usepackage{outlines}
\usepackage{soul}
\usepackage{enumitem} 
\setlength{\parskip}{1em}
\setlength{\parindent}{0em}
% \usepackage{setspace}
% \spacing{2}
\usepackage{tikz}
\usepackage{tikz-cd}
\usepackage{tikz-3dplot}

\begin{document}

\section*{Whence the quantum conundrum?}

\begin{outline}[enumerate]

\1 Cobbling together a new ``theory'' (1915--1930)

\2 Matrix mechanics (Heisenberg 1925)

\3 \emph{beobachtbare Gr\"osse}

\3 uncertainty principle and disturbance interpretation

\2 Wave mechanics (Schr{\"o}dinger 1925)

\3 \emph{verdammte Quantenspringerei}

\3 Schr\"odinger equation

\2 Complementarity (Bohr 1927)

\2 Hilbert space formalism (von Neumann 1926--32)

No hidden variables theorem


\1 Einstein-Bohr debates (1927--1935)

\2 Photon box (1927)

\2 Einstein-Podolsky-Rosen argument (1935)

\2 Realism versus antirealism?  Rational versus irrational?
Determinism versus ``god rolls dice''?  

\1 Bohr for dummies, a.k.a.\ Copenhagen/Orthodox interpretation (1935--)

Collapse of the wavefunction

\1 Heretics

\2 David Bohm's ``hidden variable'' theory (1952)

\2 Many worlds (Hugh Everett 1957)

\2 Nonlocality (John Bell 1965)

\2 ``Quantum theory without observers'' (Bell 1986)

\2 Bell's disciples (1990-)


\end{outline}

\end{document}

\section*{Maudlin: intro to book}

\begin{enumerate}
\item Warm up (easy question): who are the bad guys in Maudlin's
  story?  Who are the good guys?
\item When you're going through the text, note words with strong
  positive or negative connotations.  (e.g.\ in application to a
  statement or theory, ``clear'' has a strong positive connotation.)
  For each such word, write down --- in a sentence or two --- what you
  think Maudlin means by it.
\item What does Maudlin mean by saying that quantum mechanics is not a
  theory?  What does he think it takes to be a theory?
\end{enumerate}


\section*{Stern-Gerlach experiments}

What we \emph{do} know in advance: (1) We can measure ``spin'' along
any axis in the $xz$ plane. \newline (2) Electrons always come out
``up'' or ``down''.

What we do \emph{not} know in advance: (1) Are electrons indivisible
particles, or should we be thinking of electron-waves?  (2) Does each
electron have a feature that determines whether it will come out
``up'' or ``down''?  (2) Does passing through a magnet change the
features of electrons?

\subsection*{First experiment}

Two magnets, perpendicular orientation.  i.e.\ send ``up'' from
spin-$z$ to a spin-$x$ magnet. Here $E$ is an eraser, and $D$ are
detectors.  \newline
\begin{tikzcd}   \\
Z  \arrow[d] \arrow[r] & X \arrow{d} \arrow{r} & D \\
E  &  D & 
\end{tikzcd}

\bigskip \noindent Phenomena: (1) The detectors click individually
(not at the same time).  (2) In the long run, each detector clicks in
$50\%$ of trials.

% \begin{tikzcd}
% & & &  \\ 
% & A  \arrow[ru] \arrow[rd] & \\ 
% B \arrow[ru] \arrow[rd] &  & \\ \end{tikzcd}

\subsection*{Second experiment}

Two magnets, same orientation.  i.e.\ send ``up'' from spin-$z$ to a
spin-$z$ magnet. \newline
\begin{tikzcd}   \\
Z \arrow[d] \arrow[r] & Z  \arrow{d} \arrow{r} & D \\
E  &  D & 
\end{tikzcd}

\bigskip \noindent Phenomena: (1) The top detector always clicks.  (2)
The bottom detector never clicks.

\subsection*{Third experiment}

Three magnets: spin-$z$, spin-$x$, spin-$z$ \newline 
\begin{tikzcd}   \\
Z \arrow[d] \arrow[r] & X \arrow{d} \arrow{r} &  Z \arrow{r}\arrow{d} & D  \\
E  &  E  & D & 
\end{tikzcd}

\bigskip \noindent Phenomena: equal chance of detection for the top
and bottom detectors.


\subsection*{Fourth experiment}

Here $R_i$ is a reflector. \newline 
\begin{tikzcd}   \\
Z \arrow[d] \arrow[r] & X \arrow{d} \arrow{r} &  R_1  \arrow{d} &   \\
E  &  R_2 \arrow[r]   & Z \arrow{r} \arrow{d} & D_a \\
   &      & D_b 
 \end{tikzcd}

%  \bigskip \noindent Phenomena: top detector always clicks, bottom
%  detector never clicks.

 \subsection*{Problem set 1}

 Consider the following argument in relation to the fourth experiment:
 \begin{description}
 \item[P1.] Half of the electrons bounce off $R_1$ and half of the
   electrons bounce off $R_2$.
 \item[P2.] Of the electrons that bounce off $R_1$, half hit $D_a$ and
   half hit $D_b$.
 \item[P3.] Of the electrons that bounce off $R_2$, half hit $D_a$ and
   half hit $D_b$.
 \item[C.] Therefore, half of the electrons hit $D_a$ and half hit
   $D_b$.
 \end{description}
 Please answer the following on one (neatly written or typed) page.
 Please upload your answer by Monday, Feb 10 at 3pm.  A link will be
 emailed.
 \begin{enumerate}
\item  Give some motivation for premises P1,P2, and P3 from the
  experimental data.
\item Explain the pattern of inference from P1,P2,P3 to C.  Do you
  think that this pattern of inference is valid?  If so, then how can
  it be that C is false?
  \end{enumerate}





\end{document}



%%% Local Variables:
%%% mode: latex
%%% TeX-master: t
%%% End:
