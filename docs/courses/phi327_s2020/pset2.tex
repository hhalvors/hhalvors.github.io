\documentclass[11pt,fleqn]{article}
\sloppy
% \usepackage{fullpage}
\usepackage{outlines}
\usepackage{soul}
\usepackage{enumitem} 
% \setlength{\parskip}{1em}
% \setlength{\parindent}{0em}
% \usepackage{setspace}
% \spacing{2}
\usepackage{tikz}
\usepackage{tikz-cd}
\usepackage{tikz-3dplot}

\newcommand{\zu}{|z+\rangle}
\newcommand{\zd}{|z-\rangle}

\newcommand{\xu}{|x+\rangle}
\newcommand{\xd}{|x-\rangle}




\usepackage{amsthm,amsmath,amsfonts}
\theoremstyle{definition}
\newtheorem*{exercise}{Exercise}
\newtheorem*{defn}{Definition}
\newtheorem*{convention}{Convention}
\title{Phil Physics: Week 3}
\date{}

\renewcommand{\emph}{\textbf}

\begin{document}

\thispagestyle{empty}

\section*{phil physics: pset 2}

\newcommand{\ke}[1]{|#1\rangle}

Let $\psi$ be the singlet state.  That is,
\[ \psi \: = \: \frac{1}{\sqrt{2}}\left (\ke{L}\ke{R}-\ke{R}\ke{L}\right ).\]
Recall that $\ke{L}\ke{R}$ is shorthand for the more cumbersome notation $\ke{L}\otimes\ke{R}$.  You are given the equations
\[ \ke{0} \: =\: \frac{1}{\sqrt{2}}\left( \ke{L}+\ke{R} \right) ,\qquad      \ke{1} \: = \: \frac{1}{\sqrt{2}}\left( \ke{L}-\ke{R} \right) . \]
 Show that
 \[ \psi \: = \: \frac{1}{\sqrt{2}}\left( \ke{1}\ke{0}-\ke{0}\ke{1} \right) .\]

\end{document}

%%% Local Variables:
%%% mode: latex
%%% TeX-master: t
%%% End:
