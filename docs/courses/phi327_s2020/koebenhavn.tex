\documentclass[12pt]{article}
\title{Matematisering af Fysik}
\author{Hans Halvorson}
\begin{document}

\maketitle

\section{Surplus structure}

Det næste spørgsmål er: hvorvidt burde, eller kan, vores matematiske
konstruktioner afspejler verdenens egen struktur?  Her er der to
ekstreme synspunkter:
\begin{enumerate}
\item En teoretiker burde søge efter den perfekt match mellem
  matematiske struktur og verdenen.  Målsætningen er finde ud af
  verdenens egen matematiske struktur.

  Dette billede betragter verdenen som en person, der har et nummer i
  tankerne.  I dette tilfælde er vores opgave at gætte det rigtige
  nummer.  På den samme måde, er fysikens målsætning at skimte
  nummrene som verdenen ``har i tankerne''.
  
\item Fysikkens målsætning er at forudsige resultaterne af
  eksperimeter.  Der er ikke noget forkert valg for matematiske
  beskrivelse, fordi matematiske strukturer hører til idealitetens
  verdenen, og ikke til den fysiske verdenen.

  Dette billede betragter verdenen som noget uden sin egen matematiske
  struktur.  
\end{enumerate}

Lad mig introducere et begreb som er dukket op inden videnskabsteori i
det sidste par årtier.  På engelsk hedder begrebet \textbf{surplus
  structure}, dvs. overflødig struktur.  Ideen er at mange fysiske
teorier bruger matematiske strukturer, uden at ville sige at der er
noget indenfor fysisk realitet som korresponderer til de strukturer.
For eksempel:

\begin{enumerate}
\item (Koordinater) Tænk på teorier om tid og rum, som specielle eller
  almene relativitets teorier.  Disse teorier bruger
  koordinateensystemer, men ikke nogen mener at rum eller tid har sit
  eget koordinatensystem.  Man har frit valg ift en koordinatensystem.
  Jeg kunne sig at det her punkt er nullenpunkt, men jeg kunne også
  sige at det der punkt er nullenpunkt.  På den samme måde er man ikke
  nødt til at måle i meter, man kunne måle i feet.  Metersystemet er
  ...

\item (Enheder)

\item (Potentielfunktioner) Tænk nu på electrostatikken.  (I USA og
  England taler man om ``Maxwell's theory,'' men her i Danmark ved
  man, at elektromagnetisme blev opdaget af H.C. Ørsted.)  I
  elektrostatikken er der en potentialfunktion som tilskriver ethvert
  punkt et fast antal volter.
\end{enumerate}

\section{Mellem Pythagoras og Poincar{\'e}}

\end{document}


%%% Local Variables:
%%% mode: latex
%%% TeX-master: t
%%% End: