%% TO DO: faster than light causality

%% Often we think of non-locality is a dynamic thing.  But in QM, some
%% states are local, and some states are not.  A state is a way that
%% things are at some time.  Ergo, quantum non-locality is not
%% dynamical.
%%
%% I guess that in Bohmian mechanics, an entangled state gives rise to
%% dynamical nonlocality?

%% TO DO: quantum no-signalling and connection to no-cloning

%% TO DO: quantum teleportation ... beating classical limits on
%% transmission of information

%% counterfactuals

%% outcome dependence - parameter dependence

%% act-outcome correlations - outcome-outcome correlations

%% passion at a distance 

%% TO DO: quantum no-signaling

%% TO DO: relationship of non-locality to contextuality

%% what is realism? Maudlin points out that a theory cannot be realist
%% ...

%% TO DO: show that the singlet state is tracial on subsystems

%% TO DO: more about realism and locality




\documentclass[12pt,fleqn]{article}
\usepackage{amsfonts,amssymb,amsthm,amsmath}
\title{Bell's Theorem and Nonlocality}
\author{Hans Halvorson}
\date{\today}
\usepackage{tikz}
\usepackage{url}
\usepackage{natbib}

\newcommand{\bh}{\mathbf{B}(\mathcal{H})}
\newcommand{\bk}{\mathbf{B}(\mathcal{K})}

\newcommand{\2}{\mathcal}      

\renewcommand{\emph}{\textbf}

\newcommand{\vp}{\varphi}

% \newcommand{\b}[1]{\mathbf{#1}}


\newcommand{\Ex}{\mathcal{E}}

\newcommand{\7}[1]{\mathbb{#1}}

\swapnumbers

\newtheorem{prop}{Proposition}
\newtheorem{thm}[prop]{Theorem}
\newtheorem{lemma}[prop]{Lemma}
\newtheorem*{bellthm}{Bell's Theorem}
\theoremstyle{definition}
\newtheorem*{fact}{Fact}
\newtheorem*{defn}{Definition}
\newtheorem*{example}{Example}
\newtheorem*{question}{Question}
\newtheorem{exercise}{Exercise}
\newtheorem*{disc}{Discussion}

% \usepackage{dsfont}


%% TO DO explain: it's possible to have a non-local theory with commutative
%% algebras

%% separable and non-separable states

%% TO DO: explain how the Bell inequality could be violated in
%% classical physics

%% TO DO: Werner states

\newcommand{\ke}[1]{|#1\rangle}
 \newcommand{\ket}[1]{|#1\rangle}

\newcommand{\singa}{\ensuremath{\varphi _{+z}\otimes \varphi _{-z}}}
\newcommand{\singb}{\ensuremath{\varphi _{-z}\otimes \varphi _{+z}}}
\newcommand{\s}[1]{\varphi _{#1}}

\begin{document}

\maketitle

The objective of this document is to introduce you to Bell's theorem,
which has been called ``the most profound discovery of science''
\citep{stapp1975}.  A typical way of discussing these issues is to
begin with the famous 1935 paper by Einstein, Podolsky, and Rosen, and
then to see Bell's 1964 paper as a sort of extension of it.  However,
the issues raised by EPR are, in one sense, more conceptually subtle
than those raised by Bell.  So we'll go straight to Bell, and then
circle back to EPR.

\bigskip \noindent {\it Primary reading:} The original Bell paper,
``On the Einstein-Podolsky-Rosen paradox.''  I suggest reading the
words in this paper, but skim over the equations (at least on a first
reading).  In the current document, I'll present the technical results
in a more systematic and transparent way.  (When you read the Bell
paper, keep in mind the question, ``what assumptions does he make in
order to derive the inequality?''  For that, you'll at least have to
glance at the equations --- to see how Bell translates words into
equations.)

\bigskip \noindent {\it Secondary reading:} We'll read Maudlin's paper
``What Bell did'', and the response by Werner.

\bigskip \noindent {\it Additional resources:} Here are two survey
articles about Bell's theorem:
\begin{itemize}
\item \url{http://www.scholarpedia.org/article/Bell%27s_theorem}
\item \url{https://plato.stanford.edu/entries/bell-theorem/} 
\end{itemize}
These articles are really long and quite technical --- please don't
feel like you need to read them straight through.  I suggest that you
skim them, and then refer back to them when they might add to the
discussion in this document.

\section{Introduction}

It's difficult state Bell's theorem without immediately taking sides
on contentious issues.  To try to speak neutrally, let me say that
\cite{bell} shows that if a certain set $\Gamma$ of assumptions holds,
then an experimentally testable inequality $\phi$ holds.  Bell then
points out that QM violates the inequality $\phi$, i.e.\
QM$\,\Rightarrow\neg\phi$.  Hence, QM$\,\Rightarrow\neg\Gamma$, i.e.\
QM implies that one of the assumptions in $\Gamma$ is false.

What's interesting about this result is that the assumptions in
$\Gamma$ appear to be significant metaphysical theses --- such as
``realism'' or ``locality''.  If that's right, then Bell's theorem
shows that QM violates either realism or locality.  But that
undersells Bell's result.  It's not just that QM implies that Bell's
inequality is violated, we now have strong experimental evidence that
Bell's inequality is violated.  Hence, we have strong evidence that
one of the assumptions in $\Gamma$ must be false.  For this reason,
some philosophers have said that Bell's theorem enables us to do
``experimental metaphysics.''

But there is an intense debate about how to think about Bell's
theorem, in particular, how to think about the assumptions that are
needed to derive the inequality.  From about 1980 to 2000, the common
story was that the assumptions needed for Bell's theorem are realism
and locality --- so that the upshot of Bell's theorem is that one must
reject {\it either} realism {\it or} locality.  It was then thought
that orthodox or standard QM rejects realism, while Bohmian mechanics
(and other realist interpretations) reject locality.  (For further
thought: where does the Everett interpretation sit in this dialectic?)

More recently, some philosophers and physicists have rejected this
analysis.  For example, \cite{goldstein2011} say that,
\begin{quote}
  ($\neg$LOC) One can prove the CHSH–Bell inequality from the
  assumption of locality alone and, therefore, no matter what one
  believes about the role of non-commuting observables, it follows
  that the violation of the CHSH–Bell inequality implies
  non-locality.  \end{quote} I think that Goldstein et al.\ have
overstated the case, because there's really no such thing as a
mathematical proof that has just one assumption.  To prove anything,
you've got to make other assumptions about which inferential moves are
permitted, and you've got to make other ``framework assumptions.''

To play devil's advocate, I'll take the opposing side to Goldstein et
al., i.e.\ the side that says that Bell's theorem isn't really about
locality.\footnote{The view that violation of Bell's inequality isn't
  really about nonlocality was put forward long ago by the philosopher
  Arthur Fine.  He says that ``our investigations suggest that what
  the different hidden variables programs have in common, and the
  common source of their difficulties, is the provision of joint
  distributions in those cases where quantum mechanics denies them''
  \cite[p 1309]{fine}.  The joint distributions that Fine is talking
  about are assignments of joint probabilities to incompatible
  quantities, such as position and momentum, or spin-$z$ and spin-$x$.
  So, Fine's opinion is: Bell's inequality is violated in QM {\it
    because} QM does not provide joint distributions for non-commuting
  quantities.  Fine's analysis was disputed by \cite{shimony}.  Some
  support for Fine's position can be derived from the fact that Bell's
  inequality is violated even for different degrees of freedom of a
  single object.  If the explanation for violations should be the same
  in all cases, then the explanation cannot be nonlocality.}  Here by
``V'' I'll mean the claim that all properties can simultaneously
possess values.
\begin{quote}
  ($\neg$VAL) One can prove the CHSH-Bell inequality from the
  assumption of V \underline{alone} and, therefore, no matter what one
  believes about the role of locality, it follows that the violation
  of the CHSH-Bell inequality implies not-V. \end{quote} Your job will
be to decide which one of these claims is better supported by the
evidence.

Here is an outline of the rest of this document: First we'll talk
about classical (i.e.\ non-quantum) probability theory.  We restrict
the discussion mostly to the (boring) finite case, just because that's
all we'll need to derive the Bell inequality.  On a first reading, you
can skim this section, and proceed straight to the section on the
derivation of the Bell inequality --- referring back to the earlier
section if you have questions about terminology.

% At the end of this unit, you should be able to:
% \begin{itemize}
% \item Sketch in outline the derivation of Bell's inequality.
% \item Defend a position on the questions: ``What, if any, are the
%   metaphysical implications of the violations of Bell's inequality?''
% \item Take a position on the question of whether QM predicts
%   faster-than-light causality or ``spooky action at a distance.''
% \item Explain how various interpretations of QM account for violations
%   of Bell's inequality.
% \end{itemize}


\section{Classical probability}\label{dens}

\begin{defn} Let $X$ be a finite set; the elements of $X$ may be
  thought of as \emph{pure states}, i.e.\ complete, classical
  configurations of some system or world.  A \emph{probability
    measure} is a map $p: X \to [0, 1]$ such that
  $\sum_{x\in X} p(x)=1$.  \end{defn}

\begin{example} When the space $X$ is finite --- as we have assumed
  --- there is one probability measure that seems special or
  preferred, viz.\ the flat distribution:
\[ p_0(x) \: = \: \frac{1}{n} ,\]
where $n$ is the number of elements in $X$.  However, $p_0$ is by no
means the only probability distribution on $X$.  For example, for
each point $x\in X$, there is a probability measure that is
concentrated on $x$:
\[ p_x(y) \: = \: \begin{cases} 0 & \textrm{if}\; x=y, \\
    1 & \textrm{if}\; x\neq y . \end{cases} \] If $p$ and $q$ are
probability measures on $X$, and $\lambda\in [0,1]$, then
$\lambda p+(1-\lambda )q$ is a probability measure on $X$, called a
\emph{convex combination} of $p$ and $q$. \end{example}

\begin{defn} If a state $p$ (i.e.\ a probability measure) can be
  written as a non-trivial convex combination of other states, then we
  say that $p$ is a \emph{mixed state}.  Otherwise we say that $p$ is
  a \emph{pure state}.  In other words, the pure states are the
  \emph{extremal points} of the convex set of states.  \end{defn}

%% TO DO: define dispersion

The following result shows that the pure states of $X$ are precisely
the point masses, and hence stand in one-to-one correspondence with
elements of $X$.

\begin{prop} If $p(x)>0$, then $p=\lambda p_x+(1-\lambda )q$ for some
  probability measure $q$ and some $\lambda >0$.  \end{prop}

\begin{proof} If $p(x)=1$, then $p=p_x$ and we're finished.  If
  $p(x)<1$, then we may define
  \[ q \: = \: (1-\lambda )^{-1}(p-\lambda p_x) ,\] where
  $\lambda =p(x)$. Then
  \[ \begin{array}{rcl} \sum _{y\in X}q(y) & = & (1-p(x))^{-1}\sum
      _{y\in
        X}(p(y)-\lambda p_x(y))  \\
      & = & (1-p(x))^{-1}(1-p(x)) \\
      & = & 1 .\end{array} \] Hence $q$ is a probability measure, and
  by definition $p=\lambda p_x+(1-\lambda )q$. \end{proof}

\begin{prop} A probability measure $p$ on $X$ is pure iff $p=p_x$ for
  some $x\in X$.  \label{gelfand} \end{prop}

\begin{proof} Suppose that $p$ is pure.  Let $x\in X$ such that
  $p(x)>0$.  By the previous result, $p=\lambda p_x+(1-\lambda )q$.
  Since $p$ is pure, $p=p_x$.

  Now we show that $p_x$ is pure.  Suppose that $p_x=\lambda
  p+(1-\lambda )q$.  If $y\neq x$ then 
  \[ 0 \: = \: p_x(y) \: = \: (1-\lambda )p(y)+\lambda q(y) .\] Hence
  $p(y)=0=q(y)$.  Since $y$ was arbitrary, it follows that $p=p_x=q$.
\end{proof}

\begin{defn} A \emph{property} (or \emph{event} or \emph{proposition})
  $E$ is defined to be a subset of $X$. \end{defn}

The events/propositions on $X$ form a \emph{Boolean algebra} with the
operations $\wedge$ (intersection), $\vee$ (union), and $\neg$
(complement).  Any event $E$ also has an associated probability,
denoted $p(E)$ and defined by
\[ 
  p(E) \: = \: \sum_{x\in E} p(x) .\] The map $p$ from subsets of $X$
to probabilities is called a \emph{probability measure}, and it
satisfies some obvious equations such as
\[ p(E\vee F) \: = \: p(E)+p(F) ,\] when $E$ and $F$ are disjoint.

\begin{disc} In classical physics, it's standard to assume that every
  subset of state space corresponds to a property.  That's not the
  general understanding in QM.  For example, consider the subset
  $E=\{ \ket{0},\ket{1} \}\subseteq \2H$, which one might think
  represents the property,
  \begin{quote} \dots being in an eigenstate of $S_z$. \end{quote}
  However, since $E$ is not a subspace of $\2H$, there is no
  projection operator corresponding to it, and it's not usually
  considered to represent a property.  But why not?  Should it be?

  Let's use the name {\it abundant properties view} for the view that
  all subsets of state space represent properties.  An old-fashioned
  argument against the abundant view went like this:
  \begin{quote}
    The only properties that can be measured are those that are
    represented by projection operators.  If something cannot be
    measured, then there's no reason to think it's there.  Therefore,
    there's no reason to think that there are properties besides those
    represented by projection operators. \end{quote} The second
  premise of that argument is beyond questionable.  What's more, there
  are many projection operators that don't really correspond to things
  that we can measure.  (After all, there is an uncountable infinite
  of projection operators, and for most of them, we don't have any
  idea what physical measuring procedure they would correspond to.)

  However, there are better arguments against the abundant view.  For
  example, there's the argument from paradoxes in the foundations in
  mathematics.  Similarly, there are philosophical arguments such as
  Goodman's grue paradox.

  My favorite argument against the abundant view is the simple ``tell
  me more'' argument.  It does like this: if you want to modify the
  standard approach to QM, then please explicitly specify your new
  theory, including its states, its properties, and the rules for
  which properties are possessed in which states.  My suspicion is
  that if somebody tells me more, then their new properties will end
  up doing no work, i.e.\ they can be excised without the theory
  losing any explanatory power. \end{disc}

\begin{defn} Suppose that $p(E)>0$.  Then we define the
  \emph{conditional probability} of $F$ given $E$ as
  \[ p(F|E) \: = \: \frac{p(F\wedge E)}{p(E)} .\] \end{defn} \noindent
In fact, $F\mapsto p_E(F)=p(F|E)$ is the probability measure generated
by the function
\[ p_E(x) \: = \:  \begin{cases} p(E)^{-1}p(x) & x\in E \\
    0 & x\not\in E .\end{cases} \]
itself a probability measure, and
it's the most conservative choice of a new probability measure once
one learns that $E$ holds.  Indeed, define a distance between
probability measures on $X$ as follows:
\[ \| p-q\| \: = \: \sum _{x\in X}|p(x)-q(x)| .\] Now let $M_E(X)$ be
the set of all probability measures on $X$ with the feature that
$q(E)=1$.  Clearly $p_E\in M_E(X)$, and it can be shown that
\[ \| p - p_E \| \: \leq \: \| p-q \| ,\] for all $q\in M_E(X)$.  We
will leave the details of a proof to the reader, but intuitively,
$p_E$ is the only measure on $E$ that results from uniformly
stretching values $p(x)$ for $x\in E$.  If that goal is to minimize
the distance from $p$, then no measure $q$ can do better than a
uniform stretch.  If $q$ were closer to $p$ at some point $x\in E$,
then $q$ would have to be that much further away from $p$ at some
other point $y\in E$.


\begin{defn} A \emph{random variable} is a function
  $f:X\to\mathbb{R}$.  We will sometimes call $f$ a \emph{quantity},
  or for the sake of compassion with quantum theory, an
  \emph{observable}. \end{defn}

\begin{example} If $X$ is the classical configuration space
  $\mathbb{R}^3$, then the function $f(x_1,x_2,x_3)=x_1$ represents
  the quantity ``first coordinate of position.'' \end{example}

Let $\mathbb{R}^X$ be the set of random variables, i.e.\ functions
from $X$ to $\mathbb{R}$.  This set $\mathbb{R}^X$ naturally forms an
algebra where the operations are defined pointwise.  That is, given
$f,g$, we define
\[ \begin{array}{rcl}
    (f+g)(x) & = & f(x)+g(x) ,\\
    (fg)(x) & = & f(x)g(x) ,\\
    (rf)(x) & = & rf(x) .\end{array} \] Clearly this algebra has a
multiplicative identity (the constant $1$ function), and is
commutative, i.e.\ $fg=gf$.

\newcommand{\spec}{\mathrm{sp}}

\begin{defn} The \emph{spectrum} of $f$,
  $\spec (f) \subseteq \mathbb{R}$, is the image of $X$ under $f$,
  i.e.
\begin{equation}
  \spec (f) = \{ f(x) \in \mathbb{R} : x \in X \} .
\end{equation} \end{defn}

\begin{prop} For a quantity $f$, the following are equivalent.
  \begin{enumerate}
  \item $\spec (f)\in \{ 0,1\}$.

  \item $f$ is the characteristic function of some subset $E$ of $X$.
      \item $f^2=f$. \end{enumerate}
  \end{prop}

  \begin{proof} Suppose first that $\spec (f)\in \{ 0,1\}$.  If $E=\{
    x\in X\mid f(x)=1\}$ then $f$ is the characteristic function of
    $E$.  It's also clear that a characteristic function $f$ has the
    property that $f^2=f$.  Hence $(1)\Rightarrow
    (2)\Rightarrow (3)$.

    Now suppose that $f^2=f$.  Then for any $x\in X$, $f(x)^2=f(x)$,
    which implies that $f(x)=0$ or $f(x)=1$.  Therefore $\spec (f)\in
    \{ 0,1\}$.    \end{proof}  

  There is a natural probability density on $\spec (f)$ denoted by
  $p_f$ and defined (for $\lambda \in \spec (f)$) by
\[  p_f (\lambda) = \sum_{x\in f^{-1}(\lambda )} p(x) .\]

More generally, for any $n$ random variables $f_1, \dotsc, f_n$, there
is a discrete probability density on
$\spec ( f_1) \times \cdots \times \spec (f_n)$ given by
\[ 
p_{f_1, \dots, f_n} (\lambda_1, \dots , \lambda_n) = \sum_{x\in S} p (x),
\] 
where $S= \bigcap_{i =1}^n f^{-1}_i(\lambda _i)$.

\begin{defn} Given a random variable $f: X\to \mathbb{R}$, we define
  the \emph{expectation value} of $f$ as
\[ 
p(f) \:=\: \sum_{x\in X} p(x)f(x) .
\]  \end{defn}



\begin{defn} If $E\subseteq X$, then the \emph{characteristic function} of $E$ is
the function $e:X\to \{ 0,1\}$ that assigns $1$ to $x$ iff $x\in
E$. \end{defn}

It follows that $p(E)=p(e)$, where $p(E)=\sum _{x\in E}p(x)$, and
$p(e)=\sum _{x\in X}e(x)p(x)$.  Hence, we can freely interchange
application of $p$ to a subset and that subset's characteristic
function.

\begin{exercise} Show that expectation value is linear, i.e.\
  $p(f+g)=p(f)+p(g)$, and $p(rf)=rp(f)$.  Show that expectation value
  is positive, i.e.\ $p(f)\geq 0$ for any function $f$ such that
  $\spec (f)\subseteq \mathbb{R}^+$.  Show that expectation value is
  normalized, i.e.\ $p(1)=1$, where the first ``$1$'' is the constant
  function on $X$.  Show that expectation value is not necessarily
  multiplicative, i.e.\ $p(fg)\neq p(f)p(g)$.  \end{exercise}

% dispersion

\begin{example} If $p_a$ is the measure concentrated on $a\in X$, then
  \[ p_a(f) \: =\: \sum _{x\in X}p_a(x)f(x) \: = \: f(a) ,\] for all
  $f\in\mathbb{R}^X$.  (Here $\mathbb{R}^X$ is the set of functions
  from $X$ to $\mathbb{R}$.)  Conversely, if $p(f)=f(a)$ for all
  $f\in \mathbb{R}^X$, then $p(e)=1$ where $e$ is the characteristic
  function of $\{ a\}$, and it follows that $p=p_a$. \end{example}

\begin{prop} A state $p$ is pure iff $p(E)\in \{ 0,1\}$ for all events
  $E$. \label{purity} \end{prop}

\begin{proof} Suppose first that $p$ is pure.  By Prop \ref{gelfand},
  $p=p_a$ for some $a\in X$.  Hence
  \[ p(E) \:= \: p_a(E) \: = \: \sum _{x\in E} p_a(x) \: =
    \: \begin{cases} 0 & x\not\in E , \\
      1 & x\in E .\end{cases} \]

Suppose now that $p(E)\in \{ 0,1\}$ for all events $E$.  In
particular, $p(\{ x\})\in \{ 0,1\}$ for each $x\in X$.  Since
\[ 1 \: = \: p(X) \: = \: \sum _{x\in X}p(\{ x\}) ,\] it follows that
$p(\{ a\})=1$ for some $a\in X$, and hence $p=p_a$.  Therefore, $p$ is
pure.
\end{proof}

\begin{defn} Let $f$ be a quantity (aka random variable) and let $p$
  be a state.  The \emph{dispersion} (aka variance) $V_p(f)$ of
  $f$ in $p$ is defined by
  \[ V_p(f) \: = \: p(f^2)-p(f)^2 . \] We say that $p$ is
  \emph{dispersion free} on $f$ just in case $V_p(f)=0$.  We say
  that $p$ is \emph{dispersion free} if it is dispersion free on all
  quantities.  \end{defn}

If $f$ is a projection, then $f^2=f$, and hence $V_p(f)=p(f)^2-p(f)$.
Therefore $p(f)\in \{ 0,1\}$ iff $V_p(f)=0$.

We now look at the \emph{spectral decomposition} of a function.  For
any subset $\lambda \in \spec (f)$, let
\[ E(\lambda ) \: = \: \{ x\in X \mid f(x)=\lambda \} .\] Using the
correspondence between a subset $E(\lambda )$ of $X$ and its
characteristic function $e(\lambda )$, we have 
\[ f \: = \: \lambda _1e(\lambda _1)+\cdots +\lambda _ne(\lambda _n)
  .\] This fact is completely trivial in the case we are dealing with.
But it will be important to remember the analogy when we derive an
analogous result for quantum probability spaces.

\begin{prop} A state $p$ is pure iff $p$ is dispersion-free on all
  quantities. \end{prop}

\begin{proof} Suppose that $p$ is pure.  By Prop \ref{purity},
  $p(g)\in \{ 0,1\}$ for all $g$ such that $g^2=g$.  In particular,
  $p(e(\lambda _i))\in \{ 0,1\}$, for any spectral projection
  $e(\lambda _i)$ of $f$.  Hence,
  \[ p(f) \: = \: \sum _i \lambda _ip(e(\lambda _i)) \: = \: \lambda
    _j ,\]
  for some $\lambda _j\in \spec (f)$.  A similar calculation shows
  that $p(f^2)=\lambda _j^2$.

  Now suppose that $V_p(f)=0$ for all quantities $f$.  In particular,
  $p(f^2)=p(f)^2$ for any idempotent $f$, and hence
  $p(f)\in \{ 0,1\}$.  By Prop \ref{purity}, $p$ is pure.  \end{proof}

%% simplex



%% multiplicative <-> dispersion-free

The following result shows the precise sense in which there are always
\textbf{hidden variables} for classical systems, i.e.\ any state
whatsoever can be interpreted as an ignorance mixture of determinate
(i.e.\ dispersion-free) states.

\begin{prop}[unique decomposition] Every state $p$ on $X$ decomposes
  uniquely as a convex combination of dispersion-free
  states. \end{prop}

\begin{proof} Suppose that $X=\{ x_1,\dots ,x_n\}$.  Since $p(X)=1$,
  it follows that $p(x_i)>0$ for some $i\in [1,n]$.  Let
  $S=\{ i\in [1,n] : p(x_i)> 0\}$, and let $\lambda _i=p(x_i)$.  Then
  $p=\sum _{i\in S}\lambda _ip_{x_i}$.

  To see that this decomposition is unique, suppose that
  $p=\sum _{i\in S'}\lambda _i'p_{x_i}$ where $0<\lambda '_i<1$ and
  $\sum _{i\in S'}\lambda _i'=1$.  If $k\in S$, then
  \[ \sum _{i\in S'}\lambda _i'p_{x_i}(x_k) \: > \: 0 ,\]
  and it follows that $k\in S'$.  By symmetry, if $k\in S'$ then $k\in
  S$.  Finally, for $k\in S=S'$, we have $\lambda _k=p(x_k)=\lambda
  '_k$. 
\end{proof}

\begin{defn} Let $V$ be a linear space over the real numbers.  A
  subset $K\subseteq V$ is said to be \emph{convex} just in case for
  each $x,y\in K$ and $\lambda \in (0,1)$, we have
  $\lambda x+(1-\lambda )y\in K$. \end{defn}

If we use $M(X)$ to denote the set of probability measures on $X$,
then $M(X)$ can be seen as living inside the linear space of functions
from $\7R ^X$ to $\7R$, i.e.\ functionals on the algebra of random
variables.  Clearly $M(X)$ is a convex set, and the previous result
tells us that $M(X)$ is a special kind of convex set, called a
simplex.

\begin{defn} Let $V$ be a finite-dimensional vector space, and let $K$
  be a convex subset of $V$.  Then $K$ is called a \emph{simplex} just
  in case each $x\in K$ has a unique decomposition in terms of
  extremal points in $K$. \end{defn}

In fact, there is a nice geometric representation of $M(X)$ when $X$
has $n$ elements: it is the standard simplex with $n$ extreme points.
In particular, for $n=2$, $M(X)$ is a line segment; for $n=3$, $M(X)$
is a triangle; for $n=3$, $M(X)$ is a tetrahedron, etc.

In contrast, quantum state spaces are not simplices.  (The quantum
state space is not actually the Hilbert space $\2H$, but the convex
set of density operators on $\2H$.)  Take, for example, the density
operators on a two-dimensional Hilbert space.  If $E_0$ and $E_1$ are
the spectral projections of $S_z$, and $F_0$ and $F_1$ the spectral
projections of $S_x$, then
\[ \tfrac{1}{2}E_0+\tfrac{1}{2}E_1 \: = \: \tfrac{1}{2}I \: = \:
  \tfrac{1}{2}F_0+\tfrac{1}{2}F_1 .\] Thus, an equal mixture of the
eigenstates of $S_z$ is the same state as an equal mixture of the
eigenstates of $S_x$, i.e.\ a state can be decomposed into pure states
in more than one way.  This feature of quantum state spaces is known
as the \emph{non-unique decomposability of mixtures}.  In fact, the
convex set of density operators on a two-dimensional Hilbert space is
shaped like a sphere, and it's usually called the \emph{Bloch sphere}.


\begin{prop} Let $q:\mathbb{R}^X\to \mathbb{R}$ be a positive linear
  functional such that $q(1)=1$.  Then there is a unique probability
  measure $p$ on $X$ such that $q(f)=\sum _{x\in X}p(x)f(x)$, for each
  $f\in \mathbb{R}^X$. \end{prop}

\begin{proof} Let $e_1,\dots ,e_n$ be characteristic functions of all
  singleton subsets of $X$.  Since $q$ is linear and normalized, we
  have
  \[ 1 \: = \: q(e_1+\cdots +e_n) \: = \: q(e_1)+\cdots +q(e_n) .\]
  Since $q$ is positive, $q(e_i)\in [0,1]$.  Hence if we define
  $p(x)=q(\{ x\})$, then $p$ is a probability measure on $X$.  Now let
  $f$ be an arbitrary element of $\mathbb{R}^X$, and let
  $e(\lambda _1),\dots ,e(\lambda _m)$ be its spectral decomposition,
  which means that $f(x)=\lambda _i$ iff $e(\lambda _i)(x)=1$.
  Clearly we have
  \[ \sum _{x\in e(\lambda _i )}p(x) \: = \: \sum _{x\in e(\lambda
      _i)}q(\{ x\}) \: = \: q(e(\lambda _i)) ,\] and hence
  \[ \begin{array}{rcl}
      q(f) & = & \lambda _1q(e(\lambda _1))+\cdots \lambda _m q(e(\lambda
      _m)) \\
      & = &  \lambda _1p(e(\lambda _1))+\cdots \lambda _m p(e(\lambda
      _m)) \\
      & = & \sum _{x\in X}p(x)f(x) . \end{array} \]


\end{proof}




\section{Composite systems} \label{derive}

Given two state spaces $X$ and $Y$, the state space of the composite
system is the Cartesian product
\[ X\times Y \: = \: \{ (x,y) \mid x\in X,y\in Y \} .\] In this case,
Prop \ref{purity} implies that every pure state is of the form
$p_{(x,y)}$.

The space $X\times Y$ has the feature that for any functions
$f:X\to \7R$ and $g:X\to\mathbb{R}$, there is a unique function
$f\times g:X\times Y\to\mathbb{R}$ given by
\[ (f\times g)(x,y) \: = \: f(x)g(y) , \] for all $x\in X$ and
$y\in Y$.  However, there are also functions on $X\times Y$ that do
not decompose in this way.  For example, let $X=Y=\{ a,b\}$, and
consider the function $p$ such that
\[ p(x,y) \: = \: \begin{cases} \frac{1}{2} & x=y , \\
    0 & x\neq y .\end{cases} \] In fact, this function $p$ is a
probability measure on $X\times Y$.  Intuitively, it's a state in
which the two systems are \emph{strictly correlated}: either both are
in state $a$, or both are in state $b$.  Nonetheless, each state on
$X\times Y$ is a convex combination of pure states.  In particular,
$p=\sum _i \lambda _ip_i$, where each $p_i$ is a state of the form
$p_x\times p_y$.  This mathematical fact corresponds to the physical
fact that correlated states can be interpreted \textit{epistemically},
e.g.\ as representing our ignorance of the real state of the system,
which is a logical sum of the state of the individual subsystems.


%% TO DO: Bell's inequality

In order to derive Bell's inequality, we first need to prove a trivial
little result about inequalities of real numbers:

\begin{prop} For real numbers $a,b\in [-1,1]$, we
  have \begin{equation} |a+b|+|a-b| \:\leq\:
    2. \label{trivial} \end{equation} \end{prop}

\begin{proof} If $a+b$ is positive, then $|a+b|+|a-b|=2\max \{ a,b\}$,
  and if $a+b$ is negative, then $|a+b|+|a-b|=2\max \{
  -a,-b\}$. \end{proof}

Now we consider two systems with state spaces $X$ and $Y$.  Let
$f_1,f_2\in \mathbb{R}^X$ such that $\spec (f_i)\subseteq [-1,1]$, and
let $g_1,g_2\in \mathbb{R}^Y$ such that $\spec (g_i)\subseteq [-1,1]$.
That is, $f_1$ and $f_2$ are quantities associated with system $X$,
and $g_1$ and $g_2$ are quantities associated with system $Y$.
Consider the quantity represented by the function
\[ r \: = \: f_1\times (g_1+g_2) + f_2\times (g_1-g_2) .\] This $r$ is
called a \emph{Bell observable}, and it could, in principle, be
measured by two observers with systems $X$ and $Y$.  We then have
\[ \begin{array}{rcl} | r(x,y) | & = &
    |f_1(x)g_1(y)+f_1(x)g_2(y)+f_2(x)g_1(y)-f_2(x)g_2(y)| \\
    & \leq & |f_1(x)+f_2(x)|+|f_1(x)-f_2(x)| \\ & \leq & 2
    , \end{array} \] where the final inequality follows from
Eq.~\ref{trivial}, since $f_1(x),f_2(x)\in [-1,1]$.

\begin{bellthm} If $r$ is a Bell observable, then for any classical
  state $\omega$,
  \begin{equation} -2\: \leq \: \omega (r) \: \leq \: 2 . \label{bell} \end{equation}  \end{bellthm}
\noindent Equation \ref{bell} is called \emph{Bell's inequality}, or
to be more accurate the \emph{CHSH} variant of Bell's inequality (in
honor of Clauser, Horne, Shimony, and Holt).

\begin{proof} The discussion above shows that $-2\leq p(r)\leq 2$ for
  any pure state $p$.  An arbitrary state $\omega$ is a convex
  combination of pure states, and so the result holds for $\omega$ as
  well.  \end{proof}

\begin{question} The word ``locality'' does not occur in this section.
  So how is it that the derivation of Bell's inequality requires the
  assumption of locality?  \end{question}


\section{QM predicts violation of Bell's inequality}

%% TO DO: Show that if [B_1,B_2]=0, then there is no violation of Bell

The Bell experiment uses an idea that is due to David Bohm, and which
is based on the Einstein-Podolsky-Rosen thought experiment.  In the
EPR thought experiment, we have two systems whose locations and
momenta are strictly correlated.  However, that thought experiment is
difficult to describe mathematically, because position and momentum
are continuous quantities.  So, Bohm suggests that we look at two
systems whose spins are strictly anticorrelated.

To be more precise, let $\2H$ and $\2K$ be two-dimensional Hilbert
spaces, i.e.\ both are isomorphic to $\mathbb{C}^2$.  Then the state
space of the joint system is the tensor product $\2H\otimes \2K$.  The
\emph{singlet state} of $\2H\otimes \2K$ is given by
\[ \Omega \: = \: 2^{-1/2}\left( \ket{01}+\ket{10} \right) ,\] where
$\ket{0}$ is the $+1$ eigenstate of $S_z$, and $\ket{1}$ is the $-1$
eigenstate of $S_z$, and $\ket{ij}=\ket{i}\otimes\ket{j}$.  We now let
\[ \begin{array}{l c l l}
     A_1 = S_x\otimes I & & B_1 =  I\otimes -\frac{1}{\sqrt{2}}(S_y+S_x)  \\ \\
     A_2= S_y\otimes I  & & 
                                                 B_2= I\otimes
                                                 \frac{1}{\sqrt{2}}(S_y-S_x)
                                                 .\end{array} \] %
Note that each $A_i$ and $B_j$ is self-adjoint with spectrum
$\{-1,+1\}$.  [Here $B_1$ is a measurement of spin along an axis
tilted $\pi /4$ from the $x$-axis in the $xy$ plane, and $B_2$ is a measurement of spin
along an axis tilted $-\pi /4$ from the $x$-axis.]  Since $[A_i,B_j]=0$, it follows that $A_iB_j$ is also
self-adjoint with spectrum $\{-1,+1\}$.

Combining $A_1,A_2,B_1,B_2$, we have four different global measurement
contexts: $A_1B_1,A_1B_2,A_2B_1$, and $A_2B_2$.  (Here a ``global''
measurement context means that measurements have been chosen for both
subsystems.)  

For any quantity $X$, let
$\omega (X)=\langle \Omega ,X\Omega \rangle$, i.e.\ $\omega (X)$ is
the quantum expectation value of $X$ in state $\Omega$.  From the
equations in the appendix, we have $\omega (A_iB_j)=2^{-1/2}$ when $i=1$ or
$j=1$, and $\omega (A_2B_2)=-2^{-1/2}$.  Hence, \begin{equation}
  \omega (A_1B_1)+
  \omega (A_1B_2)+ \omega (A_2B_1) - \omega (A_2B_2) \: = \: \frac{4}{\sqrt{2}} \: = \:
  2\sqrt{2} ,\label{viol} \end{equation} and if we let
$R=A_1B_1+A_1B_2+A_2B_1-A_2B_2$, then $\omega (R)= 2\sqrt{2}$.

What Bell showed is that if $A_1,A_2,B_1,B_2$ are classical random
variables with spectrum in $[-1,1]$, then
\[ -2 \: \leq \: \omega (A_1B_1)+\omega (A_1B_2) + \omega
  (A_2B_1)-\omega (A_2B_2) \: \leq \: 2 .\] Therefore, Eqn \ref{viol}
shows that the predictions of QM cannot be reproduced by a classical
probabilistic model.
                                                                
Let's pause to note a crucial difference between the classical and
quantum cases: in the quantum case, no two of the four measurements
are compatible.  To put it in neutral, mathematical terms: since $A_1$
and $A_2$ do not commute, $A_1B_1$ and $A_2B_1$ do not commute, etc.
What do those neutral, mathematical terms mean?  Well, according to
Kochen-Specker, if operators do not commute, then they cannot
simultaneously possess values.  I don't say that because I'm an
operationalist (i.e.\ ``if it can't be measured then it's not real''),
but because I cannot assign values to both $A_1$ and $A_2$ without
contradicting myself.\footnote{The current example is precisely the
  exceptional case to Kochen-Specker: it is, in fact, possible to
  assign joint values to $A_1$ and $A_2$.  Does that change the
  overall dialectic?  I don't think so, because nobody thinks that the
  world is made of two-dimensional systems.}

The fact we just noted (i.e.\ that the four measurement contexts are
incompatible) adds a wrinkle to the dialectic of the Bell inequality.
Think of it this way: Bell argues that {\it if} there were a classical
probabilistic model of the four measurement contexts, {\it then} that
model would be nonlocal.  However, we already know that there can be
no classical probabilistic model of the four measurement contexts.
Thus, one might wonder whether Bell's result really adds anything new
to the Kochen-Specker result.  We'll come back to that question after
we finish showing that QM violates Bell's inequality.

%% To Do: cite Arthur Fine

\begin{exercise} In this exercise, you will show that the Bell
  inequality is violated only if Anne and Bjarke both make use of
  incompatible measurement settings --- a fortiori, violations of
  Bell's inequality require at least four quantities.  As a warm-up,
  check that if
  $B_1=B_2=-2^{-1/2}I\otimes (S_y+S_x)$, then
  \[ \omega (A_1B_1)+\omega (A_1B_2)+\omega (A_2B_1)-\omega (A_2B_2)
    \: = \: \sqrt{2} \: < \: 2 .\] Then there are a couple different
  ways one can generalize that result.  First, one can note that if
  $[B_1,B_2]=0$, then there is a single self-adjoint operator $C$ such
  that both $B_1$ and $B_2$ are functions of $C$.  (Intuitively: if
  $B_1$ and $B_2$ can be simultaneously diagonalized, then it's easy
  to construct another diagonal matrix $C$ and functions $f_1$ and
  $f_2$ such that the eigenvalues of $B_i$ result from applying the
  function $f_i$ to the eigenvalues of $C$.)  Second, one can show
  that if $R$ is a Bell observable, then
\[ R^2 \: = \: 4I + [A_1,A_2][B_1,B_2] .\]
Hence, if either $[A_1,A_2]=0$ or $[B_1,B_2]=0$, then $R^2=4I$ which
entails that $\spec{(R)}\subseteq [-2,2]$.                                                            \end{exercise}

The previous exercise raises the following puzzle for the defenders of
the nonlocality thesis (i.e.\ the thesis that the violation of Bell's
inequality establishes non-locality): why is Bell's inequality only
violated for measurements of incompatible quantities?  If there is
nonlocality in nature, then why does it not display itself in
measurements of a single quantity?


\section{Locality and contextuality}

In this section we will investigate the relation between Bell's
inequality and contextuality.  Our original derivation of Bell's
inequality (in Section \ref{derive}) tacitly assumes non-contextuality
(in having just one probability space $X$ for all four experiments).
Some people have said, however, that Bell does not assume
non-contextuality; he assumes locality, and derives non-contextuality
from it.  If that's right, the our derivation in Section \ref{derive}
contains an implicit assumption of locality.

Recall that the Bell setup talks about four separate experiments
(measurement of $A_i$ and $B_j$ for $i,j=1,2$), and that according to
QM, no two of these experiments are compatible with each other.  Thus,
one might suggest that there are really four state spaces in play, say
$X_{11},X_{12},X_{21},X_{22}$.  The elements of $X_{ij}$ are states
that a system can be relative to the quantities $A_i$ and $B_j$, say
$(-1,-1),(-1,+1),(+1,-1),(-1,-1)$.

Our first question is: are there classical probability distributions
$p_{ij}$ on $X_{ij}$ that reproduce the statistics of the singlet
state?  Even without checking details, it is obvious that the answer
is yes.  For any quantity $R$, a quantum state $\Omega$ induces a
classical probability measure over the eigenstates of $R$.  For the
specific case at hand, we can let $p_{ij}$ be the probability
distribution induced by the singlet state on the eigenstates of the
operator $A_iB_j$.

%% TO DO: show that this model is not local / ... is contextual /
%% ... not factorizable

%% TO DO: show that locality => non-contextuality

Here's another version of the Bell theorem.  Suppose in this case that
we have two propositions $A$ and $B$ about the left system, and two
propositions $B'$ and $C$ about the right system.  For example, $A$
might be the proposition that $S_a\otimes I$ has value $+1$, and $B'$
might be the proposition that $I\otimes S_b$ has value $+1$.  We
assume that $B$ and $B'$ are strictly anticorrelated, i.e.\ for any
truth-assignment (aka hidden variable) $\omega$, the value
$\omega (B)$ must be the opposite of $\omega (B')$, i.e.\ if the
former is true, then the latter is false, and vice versa.  If a hidden
variable $\omega$ did not have this feature, then it would not
contribute in an significant way to the statistics of the singlet
state.  In Bell's terminology, the measure of these hidden variables
would be zero, so they can be omitted from further consideration.

Now, we claim that if $\omega (A\wedge C)=1$ then either
$\omega (A\wedge B')=1$ or $\omega (B\wedge C)=1$.  Indeed, either
$\omega (B)=0$ or $\omega (B)=1$, and in the former case,
$\omega (B')=1$ and hence $\omega (A\wedge B')=1$.  In the latter
case, $\omega (B\wedge C)=1$.  Thus, in either case,
\[ \omega (A\wedge C) \: \leq \: \omega (A\wedge B')+ \omega (A\wedge
  C) .\] Since integration is linear and preserves inequalities, it
follows that for any probability measure $\mu$ on hidden variables,
\begin{equation} \mu (A\wedge C) \: \leq \: \mu (A\wedge B')+\mu
  (B\wedge C) .\label{pbell} \end{equation} This last equation is a
simple probabilistic version of Bell's inequality.  In particular,
suppose that we've chosen three measurement directions ($a,b,c$) and
suppose that:
 \begin{itemize}
   \item $A$ is the statement that $S_a\otimes I$ has value
     $+1$.
   \item $B$ is the statement that $S_b\otimes I$ has value
     $+1$.
   \item $B'$ is the statement that $I\otimes S_b$ has value
     $+1$.
   \item $C$ is the statement that $I\otimes S_c$ has value
     $+1$.  \end{itemize}
   Then we have the following translation scheme:
   \begin{eqnarray*} 
        \mu (A\wedge B') &  = & P_{ab}(++) \\
        \mu (A\wedge C) &   = & P_{ac}(++) \\
        \mu (B\wedge C) & = & P_{bc}(++) ,\end{eqnarray*} 
                              and Equation \ref{pbell} becomes
\begin{equation} P_{ac}(++) \: \leq \: P_{ab}(++)+P_{bc}(++)
  .\end{equation}
For QM, if $a=\pi /3$, $b=0$, and $c=-\pi /3$, then one can calculate
(as shown in the appendix) that: 
\[ P_{ac}(++)=3/8,\qquad P_{ab}(++)=1/8 ,\qquad P_{bc}(++)=1/8 .\] So,
once again, QM violates Bell's inequality.


\section{What did Bell do?}

At the time when Bell proved his theorem, it was already known that QM
would predict a violation of Bell's inequality.  However, at that time
--- in the 1960s --- no direct experiment had been undertaken to
verify this prediction of QM.  In the intervening years, many
different experiments have confirmed that Bell's inequality is
violated.  (Many people would say that the decisive experiment was the
one undertaken by Alain Aspect in 1982.)  Let's look at the
significance of these two facts in reverse order.

First, what is the significance of the fact that there are
experimental violations of Bell's inequality?  In the first instance,
the only significance of this fact is that the model we made above
does not adequately describe those experimental situations.  In other
words, if we thought that there were two systems $X$ and $Y$, and that
classical probability theory was applicable in the simple way we
described above, then we would derive a false prediction about the
outcomes the experiments.

Some other philosophers and physicists have not been so modest in
their claims about what these experiments show.  For example,
according to Tim Maudlin, the violation of Bell's inequality show
quite simply that the physical universe has a feature called
``non-locality''.
\begin{quote} [John Bell] taught us something about the world we live
  in, a lesson that will survive even the complete abandonment of
  quantum theory. For what cannot be reconciled with locality is an
  observable phenomenon: the violations of Bell’s inequality for
  ‘measurements’ performed at arbitrary distances apart, or at least
  at space-like separation. And this phenomenon has been verified, and
  continues to be verified, in the lab. Neither indeterministic nor
  deterministic theories can recover these predictions in a local way.
  Non-locality is here to stay. \cite[p 22]{maudlin} \end{quote}
Similarly, \citet{norsen} claims that, ``nonlocality really is
required to coherently explain the empirical data.''  This is an
interesting point of view, and there are a couple of different ways to
read it --- either in the material mode, or in the formal mode.  (The
material mode is speaking about the universe, and the formal mode is
speaking about theories.)  In the material mode, the claim seems to be
that some possible universes are local, and others are non-local, but
that any universe that displays violations of Bell's inequalities is
one of the non-local universes.  But that claim doesn't look anything
like what Bell actually proved.  Bell didn't talk about varieties of
universes, and he didn't give us any insight into what a non-local
universe would look like.

To read Maudlin and Norsen's claims in the formal mode would have Bell
showing something like this:
\begin{quote}
  There is no theory $T$ with property $\Phi$ such that $T$ predicts
  violations of Bell's inequalities \end{quote} where, in this
particular case, $\Phi$ is the property of being a local theory.  Once
again, the claim seems too strong.  Bell didn't do any surveying of
the space of all possible theories, so it's not clear how his result
could show anything of this sort.  Instead, what Bell showed is that a
certain familiar kind of modeling strategy --- classical probability
--- makes the wrong predictions for these kinds of experiments.  We
have a long way to go before we can say anything about all possible
future theories.

In fact, in the decades immediately following Bell's theorem, there
was a different consensus about the physical significance of the
result.  In particular, the common view was that Bell's theorem should
be thought of as a derivation of an (experimentally testable)
inequality from the conjunction of two premises:
\begin{description}
\item[realism] The moon is there even when no-one is looking.
\item[locality] Things that happen in one place cannot have an
  instantaneous effect on things in another place.
\end{description}
(The classic ``Jarrett analysis'' of Bell's derivation can be found in
\cite{jarrett}.)  I've purposely stated these premises in both a
vague, and an overspecific, way.  The point of doing so is that as
soon as one starts explicating (i.e.\ formalizing) these premises,
then one has to beg some questions about the framework.  In standard
analyses of Bell's theorem, one begins immediately to translate
\textbf{locality} into a statement about conditional probabilities.
But to apply classical probability theory to a complicated situation
requires making quite a few physical assumptions about what's going
on.

In the case at hand, note that the Bell observable
\[ f_1\times (g_1+g_2)+f_2\times (g_1-g_2) ,\] is built out of four
different observables: $f_1$ and $f_2$ belong to the first
experimenter, and $g_1$ and $g_2$ belong to the second experimenter.
Hence, to successfully carry out a test of Bell's inequality, the
first experimenter must perform at least two different measurements,
and the second experimenter must also perform at least two different
measurements.  So, we're not talking about any single state of
affairs, but a sequence of different experiments.  If we assume that
these four experiments can be jointly modeled in the way that
classical physics suggests, then we get a false prediction (i.e.\ that
Bell's inequality would be satisfied).

To be clear, to prove a claim of the form
\begin{quote} \textbf{locality} $\:\Longrightarrow\quad |p(r)|\leq 2$
  ,\end{quote} one first has to make \textbf{locality} into a
mathematically precise statement.  So let's say that
\textbf{i-locality} is our intuitive concept of locality, and let's
say that \textbf{m-locality} is a mathematical precisification of
\textbf{i-locality}.  Then Bell's theorem is of the form
\begin{quote} \textbf{m-locality}
  $\:\Longrightarrow\quad |p(r)|\leq 2$ ,\end{quote} and the
experimental result $p(r)>1$ shows that \textbf{m-locality} doesn't
hold.  Does it follow that \textbf{i-locality} doesn't hold?  Well,
not unless the intuitive concept of locality demands a particular
mathematical explication.  Perhaps it does; we will have to think
about that.  (For a similar argument, see \cite{werner-maudlin}.)

Update: what I wrote about i-locality and m-locality is not clear.
(Thanks to Alex for raising some questions.)  The point I was trying
to make is that it's possible (I think!) that QM does satisfy some
form(s) of locality, even though it violates Bell's inequality.  To
try to make that point more clearly, I've added the following section.

\section{How QM is local}

In this section, I'll explain some ways in which QM does seem to be
local.  In particular, QM has the feature that what is measured at one
location seems to be irrelevant for the statistics of measurements at
other locations.  If that's right, then one cannot send signals by
choosing what to measure, nor can one figure out what somebody else
measured (at a distance) by looking at the result of one's local
measurements.  This is not to say that QM forbids non-local causality
--- only that if there is such causality, then its effects are well
hidden from us.

Suppose that Anne is in Amsterdam and Bente is in Beijing, and Anne
wants to send an instantaneous (faster than light) message to Bente.
Suppose, in particular, that Bente wants to know whether Ajax
Amsterdam won their game.  Here then is a strategy for Anne to signal
to Bente: 
\begin{quote} (Protocol A) Anne and Bente prepare a pair of electrons
  in the singlet state $\psi$.  If Anne measures $S_z\otimes I$, then
  after the measurement, their state will either be $\ket{01}$, or it
  will be $\ket{10}$.  On the other hand, if Anne measures $S_x$, then
  after the measurement, their state will either be $\ket{\!+\!-}$, or
  it will be $\ket{\!-\!+}$.  (Here we use $\ket{\pm}$ for the $\pm 1$
  eigenstate of $S_x$.)  Before the experiment begins, Anne and Bente
  agree that if the state is one of the $S_z$ eigenstates, then the
  message is ``Yes'', and if the state is one of the $S_x$
  eigenstates, then the message is ``No.'' \end{quote}
\begin{exercise} Recall that according to the collapse postulate, if
  Anne measures $S_z\otimes I$ on state
  $\psi=\frac{1}{\sqrt{2}}(\ket{01}-\ket{10})$, then the subsequent
  quantum state is either $\ket{01}$ or $\ket{10}$.  (1) Explain how
  Protocol A assumes the collapse postulate.  (2) Explain why, even
  with the collapse postulate, Protocol A will not
  work. \end{exercise}

Back in section ?, we assumed that composite quantum systems could be
described using the tensor product operation: ``$\otimes$''.  I also
told you not to worry too much about why the tensor product is the
right way to represent composite systems.  Indeed, I'm not sure that
anyone has a good argument for the claim that the tensor product is
``natural'' or ``demanded'' by the concept of spatial parts.  There
is, however, at least one sense in which the tensor product operation
ensures that the two systems (or two parts of one system) are {\it
  independent} from each other.  In particular, if $H_1\otimes H_2$ is
the state space of a composite system, then the quantities associated
with the first system are of the form $A\otimes I$, and the quantities
associated with the second system are of the form $I\otimes B$.  Then
we have
\[ (A\otimes I)(I\otimes B) \: = \: A\otimes B \: = \: (I\otimes
  B)(A\otimes I) .\]
In other words,
\[ [A\otimes I,I\otimes B] \: = \: 0 ,\] which means that the
quantities $A\otimes I$ and $I\otimes B$ are \emph{compatible}.  That
is, QM {\it assumes} that any quantity associated with one system is
compatible with any other quantity associated with a distant system
--- which is at least an interesting mathematical difference in
comparison to the quantities associated with a single system.

For a single system, there are quantities, such as position $Q$ and
momentum $P$, that have an uncertainty relation.  At least
statistically speaking, QM predicts that you'll won't find yourself in
a situation where sequences of position-then-momentum measurements
will stay within narrow bands.  The situation is different with
spatially separated systems.  QM predicts that for {\it any}
quantities $A$ and $B$ associated with spatially distant systems,
there are states in which both quantities have zero dispersion.

Intuitively, if there were an uncertainty relation between $A$ and
$B$, then Anne could signal Bente by measuring $A$, thereby forcing
(if we assume the collapse postulate) the system into an eigenstate
$\omega$ of $A$, so that the variance $V_\omega (B)$ would be nonzero.

%% now explain why expectation value on rhs is independent of lhs

So, the tensor product formalism ensures that there are no uncertainty
relations between Anne's quantities and Bente's quantities.  Another
feature of the tensor product formalism is that it's possible to ``mix
and match'' states.  To see what I mean, consider a contrasting
example:

\begin{quote} Suppose that there are twins, Arnie and Barnie.  Arnie
  and Barnie are perfectly in sync, so that if Arnie orders chicken
  for dinner, then so does Barnie --- even if they are on opposite
  sides of the world.  
\end{quote}

Spatially separated quantum systems are {\it not} like Arnie and
Barnie.  For any state $x$ of $H_A$ and for any state $y$ of a
spatially distant system $H_B$, it's possible for the composite system
to be in state $x\otimes y$.  For example, if $x$ is the state of
ordering chicken, and $y$ is the state of ordering beef, then there is
a state in which the first system orders chicken and the second system
orders beef.\footnote{Something more complicated happens when we take
  into account the permutation symmetries of QM.  In that case, there
  are two kinds of particles, those whose states are symmetric under
  permutation, and those whose states are antisymmetric under
  permutations.  However, those restrictions don't conflict with the
  property I'm describing here.}  That is, the quantum states of
spatially separated systems are freely combinable, a feature called
\emph{state preparation independence}.

Here's another way to put the issue.  Let $E$ be a nonzero projection
operator associated with Anne's system, and let $F$ be a nonzero
projection operator associated with Bente's system.  Then not only do
we have $[E,F]=0$ (the projections are compatible), but we also have
$E\wedge F\neq 0$.  In other words, Anne's system having property $E$
does not rule out Bente's system having property $F$.  If $F\neq I$,
then we also have $E\wedge \neg F\neq 0$, i.e.\ Anne's system having
property $E$ does not rule out Bente's system having property
$\neg F$.  In short, the properties of Anne's system are completely
independent of the properties of Bente's system.

For those who aren't convinced of the mathematics in the previous
passage: here $E$ is really a projection of the form $E\otimes I$, and
$F$ is really a projection of the form $I\otimes F$.  If $E$ has a
nonzero eigenvector $x$, and if $F$ has a nonzero eigenvector $y$,
then $(E\otimes I)(I\otimes F)=E\otimes F$ has a nonzero eigenvector
$x\otimes y$.  It follows that $E\otimes F\neq 0$.

As a summary, here are some of the ways in which QM is manifestly
local:
\begin{itemize}
\item microcausality: quantitites associated with distant regions are
  compatible with each other.
\item state preparation independence: if $D_1$ can be prepared and
  $D_2$ can be prepared, then $D_1\otimes D_2$ can be prepared.
\item quantum no signalling theorem
\item apparent parameter independence: statistics of measurements are
  independent of what is being measured at a distance.
\end{itemize}


\section{Bell's inequality for separable states}

It's not true, of course, that all quantum states violate Bell's
inequality.  Indeed, product states, such as $x\otimes y$, most
definitely do not violate Bell's inequality.

In this section, we define the notion of a separable state, and we
show that no separable states violate Bell's inequality.  The paradigm
case of separable states are those given by product vectors such as
$x\otimes y$.  In that case, for any product operator such as
$A\otimes B$, we have
\begin{eqnarray*} \langle x\otimes y,(A\otimes B)x\otimes y\rangle & =
  & \langle x\otimes y,Ax\otimes By\rangle \\
  & = & \langle x,Ax\rangle \cdot \langle y,By\rangle .\end{eqnarray*}
That is, expectation values factorize, which motivates the following definition.

\begin{defn} Let $\2H\otimes\2K$ be the Hilbert space for a composite
  quantum system, and let $\omega$ be a quantum state for this system.
  We say that $\omega$ is a \emph{product state} just in case
  $\omega (A\otimes B)=\omega (A\otimes I)\times \omega (I\otimes B)$,
  for all $A\in \bh$ and $B\in \bk$.  We say that $\omega$ is a
  \emph{separable state} just in case it is a convex combination of
  product states.  If a state is not separable, then we say that it is
  \emph{inseparable}.  Thus, the notion of an inseparable state is a
  generalization of the notion of an entangled state.  \end{defn}

\newcommand{\tr}[1]{\mathrm{Tr}(#1)}

We show first that product states satisfy Bell's inequality.  In fact,
the result is simple enough: if $\omega$ is a product state and
$R=A_1(B_1+B_2)+A_2(B_1-B_2)$ is a Bell observable, then
\[ \omega (R) \: = \: \omega (A_1)\omega (B_1)+\omega (A_1)\omega
  (B_2)+\omega (A_2)\omega (B_1)-\omega (A_2)\omega (B_2) .\]
Furthermore, since we've assumed that $\spec{(A_i)}\subseteq [-1,1]$
and $\spec{(B_i)}\subseteq [-1,1]$, it follows that
$-1\leq \omega (A_i)\leq 1$ and $-1\leq \omega (B_i)\leq 1$, and we
can just plug the numbers into the previous derivation of Bell's
inequality.  That is, $-2\leq \omega (R)\leq 2$.

Now suppose that $\rho$ is separable, that is
$\rho = \sum _{i=1}^n \lambda _i\omega _i$, where each $\omega _i$ is
a product state, and where $\sum _i\lambda _i=1$.  Since
$-2\leq \omega _i(R)\leq 2$ for $i=1,\dots ,n$, it follows that
$-2\leq \rho (R)\leq 2$.  Therefore, all separable states satisfy
Bell's inequality.

One might conjecture now that if a state is inseparable, then it
violates Bell's inequality.  It turns out, however, that this
conjecture is false.  There are certain quantum states, known as
\emph{Werner states}, that are inseparable but satisfy Bell's
inequality \cite[see][]{werner1989}.



\section{Summary}

Why are people arguing about the upshot of Bell's theorem?  My sense
is that the deeper source of the disagreement here are the different
attitudes toward the question: what should we do next in physics?  Or
to put the question in a more mundane way: which approach should I ---
as a student, or as a researcher --- invest my time and effort into?
In particular: should we proceed with the quantum theory we find in
today's textbooks, or should we work on an alternative such as Bohmian
mechanics?  Let me explain what Bell's theorem has to do with that
question.

Quantum mechanics definitely poses some conceptual challenges.  (I use
the phrase ``conceptual challenges'' as a euphemism where others say
that ``QM is incoherent'' or ``QM is not even a theory.'')  When one
learns about these challenges, and then learns that other approaches
--- such as Bohmian mechanics --- don't face these same challenges,
then one is faced with a dilemma.  Should one give up the current
version of QM?  One of the knee-jerk reactions among physicists has
been that Bohmian mechanics is not a good alternative to standard QM,
because Bohmian mechanics is nonlocal, and hence conflicts with
Einstein's relativity theory.

Thus, for defenders of Bohmian mechanics, it's sociologically and
psychologically important to establish that their theory does not have
a fatal flaw --- or at least, not a flaw that other approaches lack.
If Bohmians could demonstrate that {\it any} theory that saves the
phenomena is nonlocal, then they will have removed the primary
objection to their approach.  In other words, Bell's theorem could
provide Bohmians with the ultimate {\it tu quoque} response to
objections to their theory: ``You call my theory nonlocal?  Well, so
is yours.''

Bohmians are not the only guilty party here.  I suspect that
resistance to the claim that Bell's theorem establishes nonlocality is
primarily driven by the desire to maintain the status quo in physics.
When physicists are told that their favorite theory is ``incoherent''
or ``not even a theory,'' and when they are presented with an
alternative (such as Bohmian mechanics) which presumably does not have
these fatal flaws, then they might want to strike back with: ``well,
but Bohmian mechanics, unlike standard QM, contradicts relativity.''

On balance, I think the situation is simply the frustrating one that
Bell's theorem --- like any other mathematical theorem --- makes use
of many premises, some explicit, any many others implicit.  What that
means is that a violation of Bell's inequality only tells us that one
or another of those premises is false.  It doesn't tell us which one
of those premises is the guilty one, and for all we know, it might be
a framework assumption that we haven't yet made explicit.  Think here
about Einstein's path to the discovery of relativity theory.  When
Lorentz's ether theory made a false prediction, then the obvious thing
was to conclude that one of Lorentz's premises was false.  However,
nobody before Einstein thought that perhaps the false thing was the
implicit assumption that the notion of simultaneity is
observer-independent!  Is it possible then we have yet to identify
some implicit assumption in the derivation of Bell's inequality?

\section*{Discussion questions}

\begin{enumerate}
\item Bohmian mechanics says that all measurements
  ultimately reduce to position measurements.  How is it then that
  Bohmian mechanics violates Bell's inequality --- as it, must, since
  it reproduces all the predictions of QM? 
\item For those who have studied relativity theory: it's often said
  that relativity theory is the paradigm of a local physical theory.
  How is the locality assumption expressed in relativity
  theory?  Does the kind of locality required by relativity conflict
  with what we see in QM?
\item Is locality a discovery of science, or is it a presupposition
  for intelligible explanations?  Why is there a tendency to associate
  non-locality with something ``spooky'' or even
  ``mystical''?
\item If there is nonlocality in nature, then why can it apparently
  not be used to send information faster than the speed of light?  
\end{enumerate}



\section*{Appendix: Probabilities in the singlet state}

\begin{lemma} Let $\Omega$ be the singlet vector in $\2H\otimes\2H$,
  and let $\{ e_0,e_1\}$ be an orthonormal basis for $\2H$.  Then
  \[ \Omega \: = \: c(e_0\otimes e_1-e_1\otimes e_0 )
    ,\] for some complex number $c$. \label{schmidt} \end{lemma}

\begin{proof}[Sketch of proof] Write $\ket{0}=c_0e_0+c_1e_1$ and
  $\ket{1}=d_0e_0+d_1e_1$, and compute.  \end{proof}

\begin{lemma} If $\Omega$ is the singlet vector, then the reduced
  state on each subsystem is the the maximally mixed state. \end{lemma}

\begin{proof} Of course, this fact can be computed directly.  In
  addition, looking at Lemma \ref{schmidt}, if $P$ is any
  one-dimensional projection, then
  $\langle \psi ,(P\otimes I)\psi \rangle = \frac{1}{2}$.  \end{proof}


By default convention, the three basic spin operators on $\2H = \7C ^2$
are given by:
\[ S _x=\begin{pmatrix} 0 & 1 \\ 1 & 0 \end{pmatrix} \qquad
  S _z=\begin{pmatrix} 1 & 0 \\ 0 & -1 \end{pmatrix} \qquad
  S _y=\begin{pmatrix} 0 & -i \\ i & 0 \end{pmatrix} \]
It then follows that  \[ 
                   S _jS _k-S _kS_j =2i S_\ell , \qquad \text{and} \qquad S _jS _k+S _kS _j =
                               0 , \]
whenever $j,k,\ell$ are distinct. 
\begin{exercise} Show that the operators $\frac{1}{2}(1+S _z)$ and
  $\frac{1}{2}(1-S _z)$ are the spectral projections for $S
  _z$. \end{exercise} For our calculations, it will be convenient to
generate spin matrices from the following more basic matrices:
\[ V=\begin{pmatrix} 0 & 0 \\ 1 & 0 \end{pmatrix} \qquad \qquad V^*
  =\begin{pmatrix} 0 & 1 \\ 0 & 0 \end{pmatrix} \] Technically, we
shouldn't have written $V^*$ until we established that $V^*$ is the
adjoint of $V$, but it is a straightforward calculation.  Here $V$ is
defined by the fact that $V\ket{0}=\ket{1}$ and $V\ket{1}=0$, or in
Dirac notation, $V=|1\rangle\langle 0|$.  Similarly, $V^*$ is defined
by the fact that $V^*\ket{1}=\ket{0}$ and $V^*\ket{0}=0$, or in Dirac
notation, $V^*=|0\rangle\langle 1|$.  Hence,
\[ V^*V \: = \: |0\rangle\langle 1|1\rangle\langle 0| \: = \:
  |0\rangle\langle 0 | \] is the projection onto the $+1$ eigenspace
of $S_z$, and $VV^*=|1\rangle\langle 1|$ is the projection onto the
$-1$ eigenspace of $S_z$.  We also have $V^2=(V^*)^2=0$ and
$\tr{V}=\tr{V^*}=0$.

Now let
\[ A(\theta ) \: = \: \exp (i\theta )V+\exp (-i\theta )V^* ,\] where
$\theta$ is any angle in the $xy$ plane up from $x$.  In particular,
\[ A(0) \: = \: V+V^* \: = \: S_x , \] and
\[ A(\pi /2) \: = \: iV-iV^* \: = \: S_y ,\]
where we used the fact that $\exp (i\pi/2)=i$ and $\exp (-i\pi
/2)=-i$.  Note furthermore that
\begin{eqnarray*} A(\pi /4 )& = & 2^{-1/2}( V+iV+V^*-iV^*) \\
                            & = & 2^{-1/2}(V+V^*+iV-iV^*) \\
                            & = & 2^{-1/2}(S_x+S_y) ,\end{eqnarray*}
                                  and
                           \begin{eqnarray*} A(-\pi /4 ) & = & 2^{-1/2}(S_y-S_x) .\end{eqnarray*}
A straightforward calculation shows that
\[ A(\theta _1) A(\theta _2) \: = \: \exp (i(\theta _1-\theta _2))P +
  \exp (-i(\theta _1-\theta _2))(I-P) ,\] where $P=VV^*$.  Now let
$\omega$ be any state such that $\omega (P)=\omega (I-P)=1/2$.  [This
is true, for example, when $\omega$ is a tracial state, because then
$\omega (V^*V)=\omega (VV^*)$.]  Then it follows that
\begin{equation} \omega (A(\theta _1)A(\theta _2)) \: = \: \cos
  (\theta _1-\theta _2) \label{cosa} .\end{equation}

\begin{lemma} If $\Omega$ is the singlet vector, then
  \[ (A(\theta )\otimes I)\Omega \: = \: -(I\otimes A(\theta ))\Omega .\] \end{lemma}

\begin{proof} For notational simplicity, assume that all operators are
  multiplied by $\sqrt{2}$, so that we don't have to repeatedly
  renormalize vectors.  In that case, we have
  $(V\otimes I)\Omega =\ket{11}$ and
  $(V^*\otimes I)\Omega =-\ket{00}$, from which
  \[ (A(\theta )\otimes I)\Omega \: = \: \exp (i\theta )\ket{11}-\exp
    (-i\theta )\ket{00} .\] Similarly, $(I\otimes V)\Omega = -\ket{11}$ and
  $(I\otimes V^*)\Omega = \ket{00}$, from which 
\[ (I\otimes A(\theta ))\Omega \: = \: -\exp (i\theta )\ket{11}+\exp
  (-i\theta )\ket{00} . \]
\end{proof}

Putting together the previous results gives the following
expectation-value formula for the singlet state:
\begin{eqnarray*}
  \langle \Omega ,(A(\theta _1)\otimes -A(\theta _2))\Omega\rangle & = &
  \langle \Omega ,(A(\theta _1)\otimes I)(A(\theta _2)\otimes I)\Omega
                                                                         \rangle \\ 
& = & \langle \Omega , (A(\theta _1)A(\theta _2)\otimes I)\Omega
      \rangle \\  & = & \omega (A(\theta _1)A(\theta _2)) \\
  & = & \cos (\theta _1-\theta _2) . \end{eqnarray*} 
Here $\omega$ is the reduced state on the left-hand subsystem.  In particular, if we let $A_1=A(0)\otimes I$, $A_2=A(\pi /2 )\otimes I$,
$B_1=I\otimes -A(\pi /4 )$, and $B_2=I\otimes -A (-\pi /4 )$, then we
have the following expectation values:
\[ \begin{array}{l c l c c c l }
     \omega (A_1B_1) & = & \cos (-\pi /4) & = & \frac{1}{\sqrt{2}} \\ \\
     \omega (A_1B_2) & = & \cos (\pi /4) & = & \frac{1}{\sqrt{2}}   \\ \\
     \omega (A_2B_1) & = & \cos (\pi /4) & = & \frac{1}{\sqrt{2}}   \\ \\ 
     \omega (A_2B_2) & = & \cos (3\pi /4 ) & = & -\frac{1}{\sqrt{2}}  .\end{array} \]
                                             
The next thing to do is to compute the corresponding probabilities,
i.e.\ expectation values for spectral projections.  For fixed $\theta$,
we have $A(\theta )^2=I$, hence both $E_0=\frac{1}{2}(I+A(\theta ))$
and $E_1=\frac{1}{2}(I-A(\theta ))$ are projection operators such
that $E_0-E_1=A(\theta )$.  For two angles $\theta _1$ and
$\theta _2$, let $E=\frac{1}{2}(I+A(\theta _1))$ and let
 $F=\frac{1}{2}(I+A(\theta _2))$.  Then \begin{eqnarray*} EF & = &
   \tfrac{1}{4}(I+A(\theta _1)+A(\theta _2)+A(\theta _1)A(\theta _2))
   .\end{eqnarray*} If $\omega$ is a tracial state, then
 $\omega (A(\theta _i))=0$, and hence
 \[ \omega (EF) \: = \: \tfrac{1}{4}(1+\cos (\theta _1-\theta _2)) \:
   = \: \tfrac{1}{2}\cos ^2((\theta _1-\theta _2)/2) , \] where we
 used Equation \ref{cosa} and the trigonometric identity
 $\cos (2\theta )=2\cos ^2(\theta )-1$.  Similarly, if
 $E'=\frac{1}{2}(I-A(\theta _1))$ and $F'=(I-A( \theta _2))$, then
\[ 
  E'F' \: = \: \tfrac{1}{4}(I-A(\theta _1)-A(\theta _2)+A(\theta
  _1)A(\theta _2)) , \] from which it follows that $\omega (E'F')=\omega (EF)$.  Finally, 
\[ \begin{array}{c c c c l } \omega (E'F) &=& \omega
                                   ((I-E)F) & = & \omega (F)-\omega
                                  (EF)  \\
                               & = & \frac{1}{2}-\omega (EF) & =
                                     & \frac{1}{2}\sin ^2 ((\theta
                                     _1-\theta _2)/2)  .\end{array} \]
These calculations yield the following probabilities in the
singlet state:
\[ \begin{array}{r c l l l }
     P_{\theta _1\theta _2}(++) & = & \tfrac{1}{2}\sin ^2((\theta _1-\theta _2)/2 ) \\ \\
     P_{\theta _1\theta _2}(+-) & = & \tfrac{1}{2}\cos ^2((\theta _1-\theta _2)/2 ) \\ \\
     P_{\theta _1\theta _2}(-+) & = & \tfrac{1}{2}\cos ^2((\theta _1-\theta _2)/2 ) \\ \\ 
     P_{\theta _2\theta _2}(--) & = & \tfrac{1}{2}\sin ^2((\theta _1-\theta _2)/2
                 ) \end{array} \]
The following table gives probabilities for measurement settings
$a=\pi /3$ and $b=0$ on the left, and $b=0$ and $c=-\pi /3$ on the
right.  These numbers give a maximal violation of the probabilistic
version of Bell's inequality (Equation~\ref{pbell}).  
\[ \begin{array}{c | c || c | c | c | c}
                                    & & --     & -+     & +-    & ++  \\
     \hline\hline \pi / 3 & 0         &  1/8     &  3/8    & 3/8       & 1/8 \\
     \hline       \pi / 3 & -\pi /3   &  3/8    &  1/8       & 1/8   & 3/8  \\
     \hline       0       & 0         &  0      &  1/2   &   1/2   &  0     \\
     \hline       0       & -\pi /3   &  1/8    &  3/8   &  3/8  &  1/8 
   \end{array} \]

         
\section*{Appendix: Reduced states}

Any quantum state of a composite system $\omega$ gives rise to
restricted states of its subsystems.  In particular, any operator $A$
on $\2H$ corresponds to an operator $A\otimes I$ on $\2H\otimes \2K$,
so we may define
\[ \omega ^*(A) \: = \: \omega (A\otimes I ),\] for all $A\in \bh$.
It is straightforward to verify that $\omega ^*$ is an abstract state
on $\bh$.  And since every abstract state corresponds to a quantum
state (i.e.\ a density operator), it follows that for any density
operator $D$ on $\2H\otimes \2K$, there is a corresponding reduced
density operator $D^*$ on $\2H$.  We now see how to compute this
reduced density operator directly.

We first show that the trace is multiplicative across tensor products.
To this end, note that if we choose o.n. bases $\{ e_i\}$ for $\2H$
and $\{ f_i\}$ for $\2K$, then $\{ e_i\otimes f_j\}$ is an o.n. basis
for $\2H\otimes \2K$.  Hence,
\begin{eqnarray*} \tr{X\otimes Y} & = & \sum _i\sum _j \langle
                                        e_i\otimes f_j\mid Xe_i\otimes
                                        Yf_j\rangle ,\\
  & = & \sum _i\langle e_i\mid Xe_i\rangle \times \sum _j\langle
        f_j\mid Yf_j\rangle \\
                                  & = & \tr{X}\times \tr{Y}
                                        .\end{eqnarray*}
The multiplicativity of trace across tensor products also entails the Hilbert-Schmidt
inner product is multiplicative across inner products:
\begin{eqnarray*}
  \langle A_1\otimes B_1 , A_2\otimes B_2\rangle  & = &
                                                             \tr{(A_1\otimes
                                                             A_2)^*(B_1\otimes
                                                             B_2)} \\
                                                       & = & \tr{A_1^*B_1\otimes A_2^*B_2} \\
                                                       & = & \tr{A_1^*B_1}\times \tr{A_2^*B_2} \\
  & = & \langle A_1 , B_1\rangle \times \langle A_2 ,B_2\rangle  .\end{eqnarray*}

Let $\{ E_i\}$ be a family of one-dimensional projections on $\2H$
such that $\sum _iE_i=I$.  It follows that $\{ E_i\}$ is an o.n. basis
for $\bh$ relative to the Hilbert-Schmidt inner product, and hence,
for any operator $A\in \bh$,
\[ A \: = \: \sum _i\langle E_i,A\rangle E_i .\]
Thus,
\begin{eqnarray*} A\otimes I & = & \sum _i\langle E_i,A\rangle
                                   (E_i\otimes I) ,
\end{eqnarray*}
and
\begin{eqnarray*} \langle D,A\otimes I\rangle & = & \sum _i\langle
                                                   D,E_i\otimes
                                                   I\rangle \, \langle
                                                   E_i,A\rangle
                                                   .\end{eqnarray*} 
Now define
\[ D^* \: = \: \sum _i\lambda _iE_i ,\]
where \[ \lambda _i \: = \: \langle D,E_i\otimes I \rangle \: = \:
  \tr{D(E_i\otimes I)} .\]
It's clear then that $0\leq \lambda _i\leq 1$, and that $\sum
_i\lambda _i=1$.  Hence, $D^*$ is a density operator, and 
\begin{eqnarray*} \tr{D^*A} & = & \langle D^* , A\rangle \\
                            & = &   \sum _i\lambda _i\langle E_i , A \rangle \\
  & = & \sum _i\langle D,E_i\otimes I \rangle \, \langle E_i ,
        A\rangle \\
                            & = & \langle D , A\otimes I \rangle \\
                            & = & \tr{D(A\otimes I)} ,\end{eqnarray*}
                          for all $A\in \mathcal{B}(\mathcal{H})$.

\begin{example} Let $E$ be the projection onto the singlet state.
  Thus, $E$ is a density operator state for a composite system
  $\2H\otimes\2K$ consisting of two $2$-dimensional Hilbert spaces.
  Let $F_1,F_2$ be orthogonal, one-dimensional projection operators on $\2H$.  It can
  then be shown that $\tr{E(F_i\otimes I)}=\frac{1}{2}$.  Hence, the
  reduced density operator on $\2H$ is
  $(1/2)F_1+(1/2)F_2=(1/2)I$, i.e.\ the maximally
  mixed state.  This is a prime example of a case where the reduced
  state ``loses information.''  If we take the tensor product of the two density operators
  $(1/2)I$ and $(1/2)I$ on the subsystems, then we get a mixed state
  $(1/4)I$, and not the singlet state.     \end{example}



                                

\begin{lemma} Let $D_1$ be a density operator on $\2H$, and let $D_2$
  be a density operator on $\2K$.  Then the density operator
  $D_1\otimes D_2$ represents a product state on
  $\2H\otimes\2K$. \end{lemma}

\begin{proof} We compute:
  \begin{eqnarray*} \tr{(D_1\otimes
                                              D_2)(A\otimes B)} & = & \tr{D_1A\otimes D_1B} \\
                                        & = & \tr{D_1A}\times
                                              \tr{D_2B}  \\
    & = & \tr{(D_1\otimes D_2)(A\otimes I)}\times \tr{(D_1\otimes D_2)(I\otimes B)} .\end{eqnarray*}


\end{proof}







%% TO DO: in QM talk about how we do know that choice of measurement
%% doesn't affect statistics of the other system

%% act-outcome and outcome-outcome dependence

%% parameter dependence outcome dependence


%% Jarrett analysis




%% TO DO: conditionalization


\bibliographystyle{chicago}

\bibliography{/Users/hhalvors/teaching/phi327_s2020/qbib}



\end{document}

\section*{Appendix: Classical dynamical systems}

For finite state spaces, the mathematical representation of dynamical
evolution is not very interesting.  It's much more interesting for an
infinite space $X$ that might have further structure --- such as a
topology, or a metric, or a symplectic form.  In any case, whether $X$
is finite or infinite, one might assume that dynamical evolution is a
``flow'' on $X$, which we can represent by a parameterized family
$u_t:X\to X$, $t\in\7R$ of automorphisms of $X$.  Furthermore, it
would be natural to require that $u_{t+s}=u_tu_s$ and
$u_{-t}=u_t^{-1}$.  Notice, however, that this representation presumes
\emph{determinism}, i.e.\ that the state of a system at one time fixes
the state of the system of future times.  In other words, a family
such as $\{ u_t\mid t\in \mathbb{R} \}$ is a \emph{deterministic
  dynamical law}.

In contrast, a \emph{stochastic dynamical law} would specify a
probability distribution over future states.  For example, $p_t(x,-)$
could define a probability distribution on $X$.  We would then want to
specify some further properties of the map $t,x\mapsto p_t(x,-)$, but
we will not pursue that here.

There is another way that one can specify a stochastic dynamical law,
and that is as one-parameter family of morphisms on the space $M(X)$
of states (i.e.\ probability distributions) on $X$.  If a pure state
$p_y$ is mapped to a mixed state $q$, then that could naturally be
interpreted as a stochastic process, where the transition probability
from $p_y$ to $p_x$ is given by $q(x)$.





%%% Local Variables:
%%% mode: latex
%%% TeX-master: t
%%% End:
