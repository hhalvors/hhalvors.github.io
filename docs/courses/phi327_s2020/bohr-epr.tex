\documentclass[12pt,fleqn]{article}
\usepackage{amsfonts,amssymb,amsthm,amsmath}
\title{Bohr's Answer to EPR}
\author{Hans Halvorson}
\date{\today}
\usepackage{tikz}
\usepackage{url}
\usepackage{natbib}

\newcommand{\bh}{\mathbf{B}(\mathcal{H})}
\newcommand{\bk}{\mathbf{B}(\mathcal{K})}

\newcommand{\2}{\mathcal}      

\renewcommand{\emph}{\textbf}

\newcommand{\vp}{\varphi}

\newcommand{\7}[1]{\mathbb{#1}}

\swapnumbers

\newtheorem{prop}{Proposition}
\newtheorem{thm}[prop]{Theorem}
\newtheorem{lemma}[prop]{Lemma}
\newtheorem*{bellthm}{Bell's Theorem}
\theoremstyle{definition}
\newtheorem*{fact}{Fact}
\newtheorem*{defn}{Definition}
\newtheorem*{example}{Example}
\newtheorem*{question}{Question}
\newtheorem{exercise}{Exercise}
\newtheorem*{disc}{Discussion}

\newcommand{\ke}[1]{|#1\rangle}
 \newcommand{\ket}[1]{|#1\rangle}

\begin{document}

\maketitle

You'll recall that the EPR argument is supposed to be a reductio ad
absurdum of the idea that QM is complete.  The key step in their
argument is their ``proof'' that the reality condition implies that
two conjugate quantities (in particular, position and momentum) can
simultaneously have values.  Let's call this part of the argument the
\emph{EPR lemma}:
\bigskip 

\framebox[29em][c]{\begin{minipage}{27em} Given the reality criterion (and locality), it is
  possible for both {\it position} and {\it momentum} to have sharp
  values simultaneously.\end{minipage}}

\bigskip Already by 1927, Niels Bohr had come to the conclusion that
position and momentum cannot simultaneously have values.  He calls
this the \emph{complementarity principle}.  In this chapter, we'll
look more closely at {\it why} Bohr is convinced of complementarity.
As a preview, his reasons are {\it not} primarily mathematical.  In
particular, when asked to justify complementarity, Bohr doesn't
typically invoke von Neumann's NHV theorem.

So, Bohr does not accept the conclusion of the EPR lemma.  If the EPR
lemma is a valid argument, then either Bohr rejects one of its
premises, or (more radically) he rejects the framework in which the
argument is formulated.  In order to come to a fair assessment of what
Bohr's objection amounts to, let's first gather two sets of data.  The
first set of data consists of some general remarks about Bohr's
outlook.  The second set of data consists of specific citations from
the Bohr paper, to which I'll add some commentary.

\subsection*{Bohr's outlook}

\begin{itemize}
\item Bohr thinks that there is no fixed boundary between ``self'' and
  ``world'', or between ``subject'' and ``object.''  He illustrated
  that point with the example of a walking stick.  He says that the
  walking stick can be treated as an extension of a person's own body,
  or it can be treated as something {\it other}, i.e.\ an object of
  investigation.

  Other examples Bohr might have availed himself of: prostheses,
  eyeglasses, smartphones.

  Bohr thinks that this general point should be applied when talking
  about measuring devices.  i.e.\ a measuring device can be considered
  as an extension of one's own self.
  
\item Bohr thinks nothing can be the same after the discovery of the
  \emph{quantum of action}, i.e.\ the discovery that the physical
  quantity ``action'' has a basic unit (Planck's constant) that cannot
  be further subdivided.  In particular, Bohr claims that this fact
  implies a limit to the ``analysis'' of position and momentum
  relations between any two objects.  In any interaction between two
  physical objects, if the second object serves a reference frame,
  then the law of conservation momentum no longer applies to it.

  See: ``\ldots the finite and uncontrollable interaction between the
  objects and the measuring instruments in the field of quantum
  theory.'' (p 700) I think Bohr means to say that a person/subject is
  also a measuring instrument, but the quantum of action implies that
  she cannot possible keep track of her own state/condition when she
  performs a measurement.  e.g.\ if you perform a position
  measurement, then you can't keep track of how much your own momentum
  changed (``the transfer of momentum'') in the process of the
  measurement.

  See: ``\ldots the renunciation \dots of one or the other of two
  aspects of the description of physical phenomena \dots depends
  essentially on the impossibility of accurately controlling the
  reaction of the object on the measuring instruments.'' (p 699)

\item Bohr thinks that the complementarity between position and
  momentum is a relation between concepts, not between things out in
  the world.  He thinks that when position can be defined, momentum
  cannot be defined.  It's not that momentum is fuzzy (when position
  is sharp), it's that the momentum concept is inapplicable.

  Question: when is the momentum concept applicable?
\end{itemize}

\subsection*{Citations}

\begin{itemize}
\item The reality criterion ``contains an essential ambiguity when it
  is applied to quantum phenomena.''  (p 696, abstract)

  ``\ldots a criterion of reality like that proposed by the named
  authors contains \dots an essential ambiguity.'' (p 697)

  ``\dots essentially different experimental arrangements and
  procedures which are suited either for an unambiguous use of the
  idea of space location, or for a legitimate application of the
  conservation theorem of momentum.'' (p 699)

  ``\dots the impossibility of defining these quantities in an
  unambiguous way.'' (p 699)

  These sentences refer to one of Bohr's favorite concepts:
  ``ambiguity''.  He doesn't just mean that EPR have failed to express
  the reality criterion clearly.  Instead, he intends to point to a
  deeper problem that an intented description of reality can fail to
  make sense if conditions aren't right \dots or, more accurately, if
  one doesn't hold fixed certain presuppositions about one's own
  condition.

  
\item ``\dots lost our only basis for an unambiguous application of
  the idea of momentum \dots '' (p 700)

  Ambiguity again!  Here's how I'm reading Bohr: imagine that you see
  a ship off in the distance, and you want to tell another person
  where it is.  But imagine that you've become disoriented, so you
  don't know which direction is north, which is east, etc.  Then
  you've lost your basis for an unambiguous application of the idea of
  direction.  If you say that ``the ship is at two o'clock'' then your
  description is ambiguous.

  Similarly, imagine that you have no idea about how fast you're
  moving or in which direction.  Then reporting to another person that
  something is moving at $10$ mph is ambiguous, because they don't
  know which reference frame you are reporting from.

\item ``\ldots such measurements of momentum require only an
  unambiguous application of the classical law of conservation of
  momentum.'' (p 698)

  For example, to describe something as having momentum, one needs to
  be in a condition where one can apply the law of conservation of
  momentum unambiguously.  (I'm not totally sure what those conditions
  are, but I suspect it has something to do with not supposing one's
  own spacetime location to be fixed.)

\item ``From our point of view we now see that the wording of the
  above-mentioned criterion of reality proposed by Einstein, Podolsky
  and Rosen contains an ambiguity as regards the meaning of the
  expression `without in any way disturbing a system'.''  (p 700)

  Here I think Bohr is using ``ambiguity'' in a less technical sense.
  I think he just means that there are various things we can
  understand by ``disturbing a system.''  He himself distinguishes
  between ``mechanical disturbance'' (which he says doesn't happen)
  and ``an influence on the very conditions which define the possible
  types of predictions regarding the future behavior of the system.''

  This last sentence is a doozy.  To be honest, I think Bohr is
  struggling here with what to say about the issue.  By ``very
  conditions which define the possible types of predictions'' he might
  just be thinking of wavefunction collapse as an epistemic process,
  i.e.\ a sort of statistical updating (like Bayesian updating).  But
  if we put emphasis on ``types'', then it's not about changing one's
  predictions, it's about changing the the types of predictions.  That
  doesn't sound like simple updating.

  So, Bohr seems to accept that a measurement by Alice {\it can}
  disturb Bob's distant system in this second sense of ``disturb.''

\item Looking at that last sentence again: ``\ldots an influence on
  the very conditions \dots '' and the subsequent sentence ``Since
  these conditions constitute an inherent element of the description
  of any phenomenon to which the term `physical reality' can be
  properly attached.''  (p 700)

  What do you think Bohr means here by ``conditions''?

  If Bohr is equating ``conditions'' with ``physical state of
  affairs'', then it's not clear at all how his kind of disturbance
  would differ from mechanical disturbance.  So, I think he isn't
  equating these two.

  Why would ``conditions'' make up part of any description to which
  the term `physical reality' can be properly attached?

  There's also a clue here about what Bohr means by ``conditions,''
  because they make up a part of a ``description.''  So,
  ``conditions'' isn't a physical state of affairs, it's a kind of
  semantic thing.  Compare, for example, with the notion of a
  ``context'' that is in use among linguists and philosophers of
  language.

\item ``Of course there is in a case like that just considered no
  question of a \textbf{mechanical disturbance} of the system under
  investigation.'' (p 700)
  



\end{itemize}






\bibliographystyle{chicago}

\bibliography{/Users/hhalvors/teaching/phi327_s2020/qbib}



\end{document}

\section*{Appendix: Classical dynamical systems}

For finite state spaces, the mathematical representation of dynamical
evolution is not very interesting.  It's much more interesting for an
infinite space $X$ that might have further structure --- such as a
topology, or a metric, or a symplectic form.  In any case, whether $X$
is finite or infinite, one might assume that dynamical evolution is a
``flow'' on $X$, which we can represent by a parameterized family
$u_t:X\to X$, $t\in\7R$ of automorphisms of $X$.  Furthermore, it
would be natural to require that $u_{t+s}=u_tu_s$ and
$u_{-t}=u_t^{-1}$.  Notice, however, that this representation presumes
\emph{determinism}, i.e.\ that the state of a system at one time fixes
the state of the system of future times.  In other words, a family
such as $\{ u_t\mid t\in \mathbb{R} \}$ is a \emph{deterministic
  dynamical law}.

In contrast, a \emph{stochastic dynamical law} would specify a
probability distribution over future states.  For example, $p_t(x,-)$
could define a probability distribution on $X$.  We would then want to
specify some further properties of the map $t,x\mapsto p_t(x,-)$, but
we will not pursue that here.

There is another way that one can specify a stochastic dynamical law,
and that is as one-parameter family of morphisms on the space $M(X)$
of states (i.e.\ probability distributions) on $X$.  If a pure state
$p_y$ is mapped to a mixed state $q$, then that could naturally be
interpreted as a stochastic process, where the transition probability
from $p_y$ to $p_x$ is given by $q(x)$.





%%% Local Variables:
%%% mode: latex
%%% TeX-master: t
%%% End:
