\documentclass[12pt]{article}
\usepackage{fullpage}
\begin{document}

%% hypothetico deductivism about geometry

\section*{Poincar\'e, Non-Euclidean Geometries}

\begin{enumerate}

\item (p 103) What do most mathematicians think of Lobachevsky's
  geometry? What does Poincar{\'e} think about it?

\item (p 103) Write out Poincar{\'e}'s short argument for the
  conclusion that the geometrical axioms are not synthetic a priori
  judgments.

\item (p 104) What arguments does Poincar\'e give against the claim
  that the axioms of geometry are experimental truths?

\item (p 104) What kinds of considerations does Poincar\'e suggest are
  relevant for deciding which geometry to adopt? (Which kinds of
  considerations are \emph{not} relevant?) Which such considerations
  weigh in favor of Euclidean geometry?

\item (p 104) What evidence might one try to use to decide between
  Euclidean, spherical, and hyperbolic geometry? Why does Poincar{\'e}
  think that this evidence is not actually helpful?

\item (p 105) What does Poincar{\'e} think would happen if human
  beings were transported to a world whose native creatures used a
  non-euclidean geometry? (For further reflection: relate this to the
  discussion of the two notions of space on pp 119-120.)

\item (p 106, for deeper reflection, maybe a paper topic) Poincar{\'e}
  presents an example and concludes that ``these beings would adopt
  Lobachevskian geometry''. How does this claim sit with his earlier
  claim that we are \emph{free} to choose a geometry?


\end{enumerate}

\section*{Poincar\'e, On the Foundations of Geometry}

\begin{enumerate}

\item (p 117) Where does Poincar{\'e} think that our notion of space
  comes from? (Which philosophical movement does this align him with?)

\item (p 119, 120) What two different notions of space does
  Poincar{\'e} distinguish? Say a few words about the differences
  between these two notions of space.

\item (p 120) What would Poincar{\'e} say to Kant's claim that
  Euclidean geometry is a ``form of our sensibility''?

\item (p 124, for reflection) What, according to Poincar{\'e}, is the
  relation between ``nature'' and the laws we use to describe it? How
  does this claim sit with his earlier claim that we are absolutely
  free, within the bounds of consistency, to choose whichever geometry
  we want? (Compare also with p 145: ``Our choice is therefore not
  imposed by experience. It is simply guided by experience.'')

\item What role does ``convenience'' plan in Poincar{\'e}'s philosophy
  of geometry?

\item Does Poincar{\'e} think that there are \emph{objective}
  considerations in favor of choosing one geometry over another? (What
  does he say about simplicity?)



\end{enumerate}


\end{document}
