\documentclass[12pt]{article}
\usepackage{outlines}
\usepackage{amsfonts}
\usepackage{fullpage}
\begin{document}

\section*{Helmholtz: part 2}

\bigskip

\subsection*{Geometry: epistemology, metaphysics, semantics}

\begin{outline}[enumerate]

\1 Epistemology: what is the source of justification for our beliefs
about geometry?

\1 Metaphysics: does physical reality have intrinsic geometric
structure?

\1 Semantics: what do we mean when we make geometric claims? 


\end{outline}

\subsection*{Epistemology of geometry}

\begin{outline}[enumerate]

\1 Empiricism: best fit (could be wrong)

%% is there a sense in which a geometric system is an assumption of
%% measurement theory? Do I assume that the measuring device satisfies
%% Euclidean laws?

\2 Objection: how is the reference of terms fixed?

\2 Stronger objection(?): to check for fit, one must use geometry

Compare: justification of induction

Compare: justification of deduction

%% But could it just be that we stipulate definitions AND THEN it is a
%% matter of best fit? Or is the problem deeper in the sense that ...
%%
%% am I assuming that something satisfies Euclidean geometry?



\1 Holistic empiricism: a geometry $G$ is correct if it is a part of
the overall best theory $T$ 




\1 Kant: the phenomena is ``pre-processed'' by the rules of geometry
that our perceptual apparatus applies (couldn't be wrong)


\1 Strict conventionalism: geometrical axioms are analytic (true by
definition), and so empty of content (couldn't be wrong)

\end{outline}

\subsection*{Helmholtz's arguments}


\begin{outline}[enumerate]

\1 ``\ldots in testing rigid bodies for the invariance of their form,
the correctness of their planes and straight lines, in fact we must
use the very same geometrical propositions we sought to prove'' (p 47)

\1 ``\ldots the system of spatial measurement must presuppose the same
condition on which alone it is meaningful to assert congruence.'' (p
49)

\1 ``Postulates 2 and 4 must evidently be presupposed if congruence is
to be meaningful at all.'' (p 50)

\1 Geometric axioms make claims about physical reality only when
connected with mechanical principles 





\1 The lesson(s) of the convex mirror






\1 ``But if we can imagine such spaces of other sorts, it cannot be
maintained that the axioms of geometry are necessary consequences of
an a priori transcendental form of intuition, as Kant thought.'' (p
63)

\2 What ``other sorts'' of spaces is Helmholtz thinking of? Why does he
think we can imagine such spaces? 

\2 What is an a priori transcendental form of intuition?


\1 ``\ldots space [\emph{der Raum}] \dots does not at all correspond with
the most general conception of an aggregate of three dimensions.'' (p
61)

What does Helmholtz think is ``the most general conception of an
aggregate of three dimensions''?


\1 On the \underline{origin} and \underline{meaning} of geometrical axioms








\1 Against the transcendental (Kantian apriori) character of Euclid's
axioms




\1 Physical presuppositions for the applicability of geometry

``Thus all our geometric measurements depend on our instruments being
really, as we consider them, invariable in form.'' (p 63)

``Every comparative estimate of magnitudes or measurement of their
spatial relations proceeds therefore upon a supposition as to the
behaviour of certain physical things, either the human body or other
instruments employed.'' (pp 63--64)




\1 Geometry cannot be kept pure

``\ldots the axioms of geometry are not propositions pertaining only
to the pure doctrine of space.'' (p 67)

``We cannot however decide by pure geometry and without mechanical
considerations whether the coinciding bodies may not both have varied
in the same sense.'' (p 67)

``A soon as certain principles of mechanics are conjoined with the
axioms of geometry we obtain a system of propositions which has real
import, and which can be verified or overturned by empirical
observations.'' (p 68)




\end{outline}







\end{document}