\documentclass[12pt]{article}
\usepackage{outlines}
\usepackage{amsfonts}
\usepackage{fullpage}
\begin{document}

\section*{Hermann von Helmholtz: epistemology of geometry}

\bigskip 

\begin{outline}[enumerate]

\1 Hermann von Helmholtz (1821--1894)

\2 Physiology, perception, color vision

\2 Back to Kant?


\1 With regard to geometry, Helmholtz is \emph{not} a:

\begin{enumerate}
\item rationalist
\item Kantian
\item empiricist
\end{enumerate}

``\ldots in testing rigid bodies for the invariance of their form
\dots we must use the very same geometrical propositions we sought to
prove.'' (p 47)

``\ldots all original spatial measurement depends on asserting
congruence, and that, therefore, the system of spatial measurement
must presuppose the same conditions on which alone it is meaningful to
assert congruence.'' (p 49)


\1 Helmholtz's mathematical results and claims

\2 Defn: A \emph{manifold} $M$ consists of a set equipped with a
family of charts

\2 Defn: At each point $p\in M$ we can construct a vector space $T_p$
which we call the \emph{tangent space} at $p$

\2 Defn: A \emph{Riemannian metric} $g$ on $M$ is a smooth assignment of
inner products to the tangent spaces over $M$

\2 A Riemannian metric $g$ defines a tensor field called the
\emph{Riemannian curvature tensor}

\2 Fact: The curvature tensor can be ``contracted'' to give a
curvature function $\kappa :M\to\mathbb{R}$

\2 Defn: $(M,g)$ has \emph{constant curvature} if $\kappa$ is constant
on $M$

\2 Defn: $(M,g)$ is \emph{flat} if $\kappa (p)=0$ for all $p\in M$

\2 Fact: If figures in $(M,g)$ can be translated from one location to
another, then $(M,g)$ has constant curvature

``If figures of finite magnitude are moveable along all parts of such
a surface without changing their relative measurements on the surface
itself and if these figures can be rotated about any point whatever,
it must be the case that the surface has constant curvature.'' (p 48)

\2 Example: The surface of an egg does \emph{not} have constant
curvature

\1 Upshot: Space (\emph{Raum}) is a special case of a flat,
three-dimensional Riemannian manifold

Note: Helmholtz does not yet imagine (as Einstein later will) that
actual space might not be flat, and might not even be infinite

\1 Against the transcendental (Kantian apriori) character of Euclid's
axioms

``\ldots space [\emph{der Raum}] \dots does not at all correspond with
the most general conception of an aggregate of three dimensions.'' (p
61)

``But if we can imagine such spaces of other sorts, it cannot be
maintained that the axioms of geometry are necessary consequences of
an a priori transcendental form of intuition, as Kant thought.'' (p
63)


\1 Physical presuppositions for the applicability of geometry

``Thus all our geometric measurements depend on our instruments being
really, as we consider them, invariable in form.'' (p 63)

``Every comparative estimate of magnitudes or measurement of their
spatial relations proceeds therefore upon a supposition as to the
behaviour of certain physical things, either the human body or other
instruments employed.'' (pp 63--64)




\1 Geometry cannot be kept pure

``\ldots the axioms of geometry are not propositions pertaining only
to the pure doctrine of space.'' (p 67)

``We cannot however decide by pure geometry and without mechanical
considerations whether the coinciding bodies may not both have varied
in the same sense.'' (p 67)

``A soon as certain principles of mechanics are conjoined with the
axioms of geometry we obtain a system of propositions which has real
import, and which can be verified or overturned by empirical
observations.'' (p 68)




\end{outline}







\end{document}