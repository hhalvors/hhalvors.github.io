\documentclass[12pt]{article}
\usepackage{outlines}
\usepackage{amsfonts}
\usepackage{fullpage}
\begin{document}

\section*{Lecture 17}

\subsection*{Helmholtz}

\begin{outline}[enumerate]

\1 Geometry and the tabula rasa 

``\ldots in testing rigid bodies for the invariance of their form
\dots we must use the very same geometrical propositions we sought to
prove.'' (p 47)

``\ldots all original spatial measurement depends on asserting
congruence, and that, therefore, the system of spatial measurement
must presuppose the same conditions on which alone it is meaningful to
assert congruence.'' (p 49)


\1 Physical presuppositions for the applicability of geometry [or
``are there rigid bodies?'']

``Thus all our geometric measurements depend on our instruments [rods]
being really, as we consider them, invariable in form.'' (p 63)

``Every comparative estimate of magnitudes or measurement of their
spatial relations proceeds therefore upon a supposition as to the
behaviour of certain physical things, either the human body or other
instruments [rods] employed.'' (pp 63--64)


\1 Geometry doesn't mean anything by itself [it needs mechanics]

``We cannot however decide by pure geometry $(G)$ and without
mechanical considerations $(P)$ whether the coinciding bodies may not
both have varied in the same sense.'' (p 67)

``A soon as certain principles of mechanics $(P)$ are conjoined with
the axioms of geometry $(G)$ we obtain a system of propositions which
has real import, and which can be verified or overturned by empirical
observations.'' (p 68)

\end{outline}


\subsection*{Poincar\'e and 20th century analytic philosophy}


\begin{outline}[enumerate]

\1 Hume's fork (analytic-synthetic distinction): if something is
neither a relation of ideas nor a matter of fact, then it should be
cast in the flames

\1 Poincar\'e: the mathematics in empirical science is just relations
of ideas (analytic sentences)

\1 Logical positivism (Schlick, Reichenbach, Carnap): if something is
neither a relation of ideas nor verifiable, then it should be cast in
the flames

\2 Reichenbach: the metric in GTR is conventional

\1 W.v.O. Quine against logical positivism: the analytic-synthetic
distinction is incoherent

\1 Quine's children: all parts of a theory are on equal footing

\2 Example: Since GTR uses a manifold $(M,g)$, it assumes that
spacetime exists and has metric properties

\end{outline}

\subsection*{Poincar\'e's conventionalism}

\begin{outline}[enumerate]

\1 Geometry is either synthetic apriori \emph{or} experiential
\emph{or} conventional

\2 If $G$ were synthetic apriori, then there would be no non-euclidean
geometry

\2 Creatures who evolved in a different environment than ours might
experience the world in a non-euclidean way

\2 $G$ does not look like experiential truths

\3 We don't experiment with geometrical objects but only with material
objects

\3 If experiential, then $G$ would be subject to continual revision

\3 $G$ is based on a false presupposition of perfectly rigid bodies


\1 What does it mean to say that G is conventional?

\2 Choice of G is constrained only by consistency

\2 Choices of $G$ is guided by experimental facts

\2 Choice of $G$ is free

\2 $G$ consists of ``disguised definitions'' [analytic]

\2 Choice of $G$ is similar to choice of units (metric vs. imperial)
or choice of coordinates (Cartesian vs. polar)

\3 HH: there are disanalogies between these cases

\2 When choosing $G$, we aim for \underline{convenience}, not for
\underline{truth}


\1 Euclidean geometry is objectively more convenient

\2 EG is simplest in the same way that a linear equation is simpler
than a quadratic equation. (See the next essay)

\2 EG agrees with the properties of natural solids

\2 Everyone agrees that it would be more advantageous to give up the
laws of optics than the laws of EG


\end{outline}

\end{document}








\1 Against the transcendental (Kantian apriori) character of Euclid's
axioms

``\ldots space [\emph{der Raum}] \dots does not at all correspond with
the most general conception of an aggregate of three dimensions.'' (p
61)

``But if we can imagine such spaces of other sorts, it cannot be
maintained that the axioms of geometry are necessary consequences of
an a priori transcendental form of intuition, as Kant thought.'' (p
63)









\end{outline}







\end{document}