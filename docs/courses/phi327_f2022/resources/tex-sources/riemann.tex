\documentclass[12pt]{article}
\usepackage{amsfonts}
\title{Riemann's Revolution in Geometry}
\begin{document}

\maketitle

Today we are going to talk about the remarkable work of Bernhard
Riemann, which has come to serve as the backdrop for Einstein's
general theory of relativity. Before we turn to Riemann's article, let
me begin with a bit of mathematical context.

When people talk about ``non-euclidean geometry'', then normally mean
the sort of synthetic geometry that was developed by Bolyai and
Lobachevsky in the early part of the 19th century. Here the word
``synthetic'' means that the methods are similar to those of Euclid,
where the primary representational tool is natural language
supplemented by figures, and where the focus is on step by step
logical deduction. The contrasting approach is sometimes called
``analytic geometry'', and it is characterized by the use of equations
and calculations.

Now, Riemman's proposal in the 1854 paper is most certainly not
squarely in the tradition of synthetic geometry. And yet in a certain
sense, what he proposes is a generalization of all three synthetic
geometries, viz. Euclidean, hyperbolic, and spherical. I will explain
the sense in which it is a generalization in just a bit.

The second bit of mathematical context we need is a reminder about how
Euclidean geometry is related to the real numbers. The real number
line $\mathbb{R}$ is, of course, familiar to you. It consists of whole
numbers, fractions, and also all those infinite non-repeating decimals
that do not correspond to fractions. For our purposes, the real number
line is a good representation of how Newton understood time. In
particular, $\mathbb{R}$ comes with a natural notion of ordering: less
than or equal to. In addition, the real number line comes with a
natural notion of distance: the distance between two numbers $a,b$ in
$\mathbb{R}$ is just the absolute value $|b-a|$.

There is, however, one sense in which even Newton would find the real
numbers to have too much structure to represent time. In particular,
$\mathbb{R}$ has an ``origin'', viz.\ the number $0$. But Newton has
no notion of a preferred ``zero time''. So what he really needs is
something like the real numbers with a less than relationship, and
with distances, but without a zero. Actually what he needs is
$\mathbb{R}$ as an \emph{affine space}, a notion that I will not
define right now, but which will be useful later on.

So much for the mathematical representation of Newtonian absolute
time. Let's proceed now to Newtonian absolute space. First of all, the
Cartesian plane $\mathbb{R}^2$ consists of ordered pairs of real
numbers. For example, the origin can be represented by
$\langle 0,0\rangle$, while a point $\langle a,b\rangle$ is on the
unit circle just in case $a^2+b^2=1$. Now, the plane $\mathbb{R}^2$
does not have a natural ordering relation, but it does have a natural
distance function. In particular, for two real numbers
$\langle x_1,y_1\rangle$ and $\langle x_2,y_2\rangle$, we define the
distance to be
\[ \sqrt{(y_2-y_1)^2+(x_2-x_1)^2} .\] Of course, Newton thinks that
space is three dimensional, so he needs to add a third coordinate,
which we can think of as the $z$-axis. In this case, $\mathbb{R}^3$
consists of triples of real numbers, and the distance relationship is
given by the generalized Pythagorean formula:
\[ \sqrt{(y_2-y_1)^2+(x_2-x_1)^2+(z_2-z_1)} .\]

Newton says explicitly that space endures unchanged through time. We
might represent this mathematically by saying that this very same
$\mathbb{R}^3$ continues to exist at all times. That also means that
we can speak explicitly about whether or not a physical object is in
the \emph{same} place at two different times.

This result is actually not ideal for Newton, as it would commit him
not just to absolute acceleration, but also to absolute velocity. In
particular, we can say that an object is \emph{stationary} through a
time interval $I$ just in case that object is at the same point
$\mathbf{x}\in\mathbb{R}^3$ at all times $t$ in $I$. But none of
Newton's arguments assume that there really is a difference between
being stationary and moving. All that he needs is that there is a
difference between inertial (constant velocity) and accelerated
(changing velocity) motion.

Now, the idea of mixing space and time together did not really come up
until Einstein's work in the twentieth century. Nonetheless, we can
usefully talk about a notion of ``Newtonian spacetime''. In
particular, if we now add a fourth coordinate for time, then we can
say that Newtonian spacetime is the set
$\mathbb{R}\times \mathbb{R}^3$ where $\mathbb{R}$ represents time,
and where $\mathbb{R}^3$ represents space.

Bernhard Riemann was not content with the status of geometry as of
1854. In particular, he claims that geometers adopt a bunch of
postulates --- i.e.\ assumptions about the structure of space ---
without a clear understanding of why they accept them, and how they
are related to each other. Riemann's proposal will be to start from a
more primitive notion of a ``multiply extended quantity'', and then to
add additional structure to capture the specific features of the space
we actually live in. One way to see what he is doing here is to redraw
the line between what is known \emph{a priori} about space, and what
is only known \emph{a posteriori}. In this case, Riemann would be
saying that we know a priori that space is some or other multiply
extended quantity, and we know a posteriori that it has specific
metric features. I will now proceed to clarify what Riemann means by
``multiply extended quantity'' and by ``metric features''.

\section{Commentary Part I}

\begin{enumerate}

\item (\S 1) How much structure do we need to even talk about
  ``quantities''? What is a quantity after all? According to Riemann,
  the most fundamental idea here is of having a concept that admits of
  multiple instances. Of course we are familiar with such
  things. Think, for example, of ``height'' --- one concept with many
  different instances.

  Riemann then points out that for any finite collection of things,
  it's easy to make up a concept that they all fall under. But it's
  different for infinite collections of things. Some of these infinite
  collections naturally fall under a single concept, but others do
  not. Riemann says that color and the position of sensible objects
  are the most familiar cases of concepts whose instances form a
  continuous manifold.

  Now Riemann turns to the question of the quantitative comparison of
  different regions in a manifold. In the case of finite manifolds,
  the comparison is as simple as counting. But counting does not make
  sense for comparing two infinite collections. Consider, for example,
  the intervals $[0,1]$ and $[0,2]$. Both of these intervals contain
  infinitely many points, but the second is bigger in some sense.

  According to Riemann, the way to compare continuous quantities is by
  \emph{superposition}, i.e.\ by setting them side by side. But now we
  are on to an interesting notion: that there is some intrinsic
  connection between static facts about sizes and a concept of
  ``transport'' from one place to another.

\item (\S 2) In this section, Riemann explains the notion of a
  manifold by ``synthesizing'' it out of manifolds of lower
  dimension. He first asks us to consider a manifold where there are
  only two directions of motion: forward and backward. The idea, of
  course, is of an abstract line. He then asks us to consider the idea
  of moving such a line continuously, thereby generating a
  two-dimensional surface.

  This process can then continue, with the result that ``this
  construction can be characterized as a synthesis of a variability of
  $n+1$ dimensions from a variability of $n$ dimensions and
  variability of one dimension.''

\item (\S 3) In this section, Riemann describes how a point in a
  $n$-dimensional manifold can be specified by $n$ real numbers. He
  first asks us to consider a continuous function $x_1$ that assigns
  each element of the manifold a real number, and such that $x_1$ is
  never constant on a small neighborhood around any point. He then
  notes that for each fixed real number $r$, the set of points in the
  manifold that have value $r$ for $x_1$ will form a continuous
  manifold of fewer dimensions than the original one.

\end{enumerate}


\section{Commentary Part II}

\end{document}