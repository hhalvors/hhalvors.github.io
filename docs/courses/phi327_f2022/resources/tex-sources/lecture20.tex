\documentclass[12pt,fleqn]{article}
\usepackage{outlines}
\usepackage{amsfonts}
\usepackage{fullpage}
\usepackage{url}
\renewcommand{\emph}{\textbf}
\begin{document}

\section*{Lecture 20}

\subsection*{Conventionalism, realism, and the middle way}

\begin{outline}[enumerate]

\1 Why not just try to model the facts (structure of space) with
geometry?

\1 It seems like Einstein (following Helmholtz) is suggesting that we
divide the world into two parts

\2 Subject = practically rigid bodies (not asking whether or not they
are rigid)

\2 Object = the things whose behavior is being studied

\end{outline}

\subsection*{Mach, The Science of Mechanics}

\begin{outline}[enumerate]

  \1 Ernst Mach (1838--1916) [Poincar\'e (1854--1912), Bertrand
  Russell (1872--1970), Einstein (1879--1955), Bohr (1885--1962)]

  \url{https://plato.stanford.edu/entries/ernst-mach/}

  ``The frequent excursions which I have made into this province have
  all sprung from the profound conviction that the foundations of
  science as a whole, and of physics in particular, await their next
  greatest elucidations from the side of biology, and especially, from
  the analysis of the sensations.''

  \1 ``$K$ alters its direction and velocity solely through the
  influence of another body $K'$\,''

  \2 We cannot know how $K$ would act in the absence of $A,B,C,\dots $

  \2 ``Every means would be wanting of forming a judgment of the
  behavior of $K$ and of putting to the test what we had predicted ---
  which latter therefore would be bereft of all scientific
  significance.''

  \1 Open question: does this last statement of Mach's depend on the
  verification criterion of meaning?

  \1 What does Mach mean by saying that ``the universe is not
  \emph{twice} given''? (p 176)


\end{outline}

\subsection*{Some Minkowski geometry}

\begin{outline}[enumerate]

\1 There is a vector space $V$, and for any two points $p,q\in M$,
there is a vector $\overline{pq}\in V$.

\1 There is an inner product $\eta :V\times V\to\mathbb{R}$. The
vector space $V$ can be divided (in many ways!) into a
three-dimensional subspace on which $\eta$ is negative-definite and an
orthogonal subspace on which $\eta$ is positive-definite

\2 We say that $u\in V$ is \emph{timelike} if $\eta (u,u)>0$

\2 We say that $u\in V$ is \emph{spacelike} if $\eta (u,u)<0$

\2 We say that $u\in V$ is \emph{lightlike} or \emph{null} if
$\eta (u,u)=0$

\2 We say that $u\in V$ is \emph{causal} if $\eta (u,u)\geq 0$

\1 For any $p,q\in M$, we say that $p$ is timelike related to $q$ just
in case $\overline{pq}$ is timelike. We define spacelike and lightlike
similarly


\1 Given $p\in M$, the set of points that are causally related to $p$
consists of two subsets whose intersection is $p$. We label on of
these subsets as $F_p$ and the other as $B_p$. We call these sets the
\emph{lightcones} based at $p$.


\1 An \emph{inertial trajectory} is represented by a line $\alpha$ in
$M$ that always stays within the light cone (i.e.\ trajectories of
material bodies are never as fast as light)

\dots which is generated by a pair $(p,u)$ where $p\in M$ and $u$ is a
timelike vector in~$V$



\1 Simultaneity slices: Given $p\in M$ and $u$ a timelike vector in
$V$, the set
\[ (p,u)^\perp \: = \: \{ q\in M:\eta (u,\overline{pq})=0 \} \]
is a three-dimensional subspace of $M$


\1 Relativity of simultaneity: if $v$ is between $u$ and a null vector
$w$, then $(p,v)^\perp$ is between $(p,u)^\perp$ and $w$


\1 Coordinate systems

Let $p\in M$ and $u\in V$, and fix an orthonormal basis
$(e_1,e_2,e_3)$ of $u^\perp$. Then for any vector $v\in V$, there is a
unique quadruple $(t,a_1,a_2,a_3)$ of real numbers such that
\[ v = tu+a_1e_1+a_2e_2+a_3e_3 .\] We may then define a coordinate
chart $\phi :M\to\mathbb{R}^4$ by setting
$\phi (p+v)=(t,a_1,a_2,a_3)$, for each $v\in V$. Since for each
$q\in M$, there is a unique $v\in V$ such that $p+v=q$, it follows
that $\phi$ is defined on all of $M$.

Fact: The set $u^\perp$ has coordinates of the form
$(0,a_1,a_2,a_3)$. i.e.\ the events that are \emph{simultaneous} for
an observer at $p$ with velocity $u$.




  

\end{outline}

\section*{Putnam's argument for eternalism}

\begin{outline}[enumerate]

\1 Assumptions from common sense

\2 $A$-now is real

\2 Another person $B$ can move past $A$

\2 There are no privileged observers

If $A$ should take $B$ to be real, and $B$ should take $X$ to be real,
then $A$ should take $X$ to be real

\1 First version of argument

\2 There is an $X$ such that:

\3 $B$ should take $X$ to be real (since $X$ lies in the simultaneity
surface for $B$)

\3 $X$ lies in the future of the simultaneity surface for $A$

\2 So there is an $X$ that is in the future for $A$, but that $A$
should judge as real 

\end{outline}



\end{document}