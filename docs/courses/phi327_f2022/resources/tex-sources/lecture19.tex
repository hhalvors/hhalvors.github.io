\documentclass[12pt]{article}
\usepackage{outlines}
\usepackage{amsfonts}
\usepackage{fullpage}
\usepackage{url}
\begin{document}

\section*{Lecture 19}

\subsection*{Conventionalism, realism, and the middle way}

\begin{outline}[enumerate]

\1 Realism: space(time) has geometric structure, and our goal is to describe
it

\1 Conventionalism: our goal in adopting geometric axioms is to
establish a convenient framework for empirical research

\2 A ``non-truth-aimed'' choice. i.e. conceived of as largely
irrelevant to the question of fitting reality. Cf. choosing to write a
paper in Times New Roman font as opposed to Helvetica

\1 Middle way: perhaps Helmholtz, Mach, and Einstein are onto
something?

\2 ``On the facts underlying geometry''

\2 Hypothesis: there is a class of physical objects which are roughly
stable (rigid) relative to each other, i.e.\ form an approximately
Euclidean domain

\end{outline}



\subsection*{Einstein, ``Geometry and Experience''}

% Keywords: practically rigid body, body that is practically rigid in
% experience

\begin{outline}[enumerate]

\1 The puzzle (p 147): ``how is it possible that mathematics, being
  after all a product of human thought that is independent of
  experience, is so admirably appropriate to the objects of reality''?

\1 Einstein's solution: laws of mathematics (refer to reality
$\leftrightarrow$ not certain)

\2 Axiomatics has succeeded in separating the logical-formal from its
objective or intuitive content [de-interpretation, cf. Hilbert]

\1 Two interpretations of the axioms of geometry (e.g.\ ``through two
points in space there passes one and only one straight line'')

\2 Older interpretation: the axioms are self-evident under the
self-evident interpretation of the words contained in them

\2 Newer interpretation: the axioms are (a) to be taken in a purely
formal sense, (b) void of content of intuition or experience, (c) free
creations of the human mind, (d) first define the objects of which
geometry treats [implicit definition]

Aside on implicit definition: this would only work if the axioms also
have names for concrete (physical) things. This way of thinking is
still adopted under the name of ``functionalism''

``\ldots mathematics as such cannot predicate anything about objects
of our intuition or real objects'' (p 148)

``\ldots the system of concepts of axiomatic geometry alone cannot
make any assertions as to the behavior of real objects of this kind,
which we will call \underline{rigid bodies}'' (p 148)


\1 Re-interpretation

``\ldots geometry must be stripped of its merely logical-formal
character by \underline{assigning} to the empty conceptual schemata of
axiomatic geometry objects of reality that are capable of being
experienced'' (p 148)

``To accomplish this, we need only add the proposition: Solid bodies
are related, with respect to their possible relative positions, as are
bodies in Euclidean geometry of three dimensions. Then the
propositions of Euclid contain assertions as to the behavior of
\underline{practically rigid bodies}.'' (p 148)


\1 Einstein versus Poincar\'e

\2 Poincar\'e: rigid bodies are a fiction

\2 Einstein: Poincar\'e is right in principle, but the objection ``is
by no means so profound as might appear from a hasty examination'' (p
150)

``\ldots it is not a difficult task to determine the physical state of
a measuring body so accurately that its behavior in relation to the
relative positions of other measuring bodies will be sufficiently free
from ambiguity to allow it to be sustituted for the `rigid' body.'' (p
150)

\2 Einstein: if we reject the equation ``body of axiomatic Euclidean
geometry'' = ``practically rigid body of reality'', then
conventionalism follows

``If one rejects the relation between the practically rigid body and
geometry, indeed one will not easily free oneself from the convention
that Euclidean geometry is to be retained as the simplest.'' (p 149)

\2 Rotating frames of reference $\Rightarrow$ non-euclidean geometry
(p 149, top)


\2 Open question: is Einstein suggesting that we presuppose that our
measuring instruments (``rigid rods'') are Euclidean? Are we to use a
Euclidean framework to conclude that other parts of the world are best
described by non-euclidean geometry?




\1 What's fundamental?

``It is also clear that the solid body and the clock do not play the
part of irreducible elements in the conceptual edifice of physics, but
that of composite constructs. \dots But it is my conviction that in
the present stage of the development of theoretical physics these
concepts must still be invoked as independent concepts.'' (p 149)

  
\end{outline}

\subsection*{Mach, The Science of Mechanics}

\begin{outline}[enumerate]

  \1 Ernst Mach (1838--1916) [Poincar\'e (1854--1912), Bertrand
  Russell (1872--1970), Einstein (1879--1955), Bohr (1885--1962)]

  \url{https://plato.stanford.edu/entries/ernst-mach/}

  ``The frequent excursions which I have made into this province have
  all sprung from the profound conviction that the foundations of
  science as a whole, and of physics in particular, await their next
  greatest elucidations from the side of biology, and especially, from
  the analysis of the sensations.''

  \1 ``$K$ alters its direction and velocity solely through the
  influence of another body $K'$\,''

  \2 We cannot know how $K$ would act in the absence of $A,B,C,\dots $

  \2 ``Every means would be wanting of forming a judgment of the
  behavior of $K$ and of putting to the test what we had predicted ---
  which latter therefore would be bereft of all scientific
  significance.''

  \1 Open question: does this last statement of Mach's depend on the
  verification criterion of meaning?

  \1 What does Mach mean by saying that ``the universe is not
  \emph{twice} given''? (p 176)


\end{outline}



% \subsection*{Paper ideas}

% \begin{itemize}
% \item Extract from Einstein's articles in \emph{Beyond Geometry} his
%   argument for why we ought to adopt non-euclidean geometry. To do
%   this well, you will probably need to have a background in relativity
%   theory.


% \end{itemize}

\end{document}
