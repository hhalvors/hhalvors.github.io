\documentclass[11pt,fleqn]{article}
\usepackage{outlines}
\usepackage{amsfonts}
\usepackage{amsthm}
\theoremstyle{definition}
\newtheorem*{prop}{Proposition}
\newtheorem*{defn}{Definition}
\newtheorem*{ass}{Assumption}
\usepackage{fullpage}
\usepackage{url}
\usepackage{wrapfig}
% \renewcommand{\emph}{\textbf}
\setlength{\parskip}{0.5em}
\setlength{\parindent}{0em}

\usepackage{xcolor}
\usepackage{float}
\usepackage[outline]{contour} % glow around text

\usepackage{tikz}
\usetikzlibrary{arrows.meta}
\usetikzlibrary{decorations.markings}

\usepackage{tikz-3dplot}

\usepackage{pgfplots}

\usepackage{colortbl}

% A few colors
%--------------------------------------------------------------
\definecolor{plum}{rgb}{0.36078, 0.20784, 0.4}
\definecolor{chameleon}{rgb}{0.30588, 0.60392, 0.023529}
\definecolor{cornflower}{rgb}{0.12549, 0.29020, 0.52941}
\definecolor{scarlet}{rgb}{0.8, 0, 0}
\definecolor{brick}{rgb}{0.64314, 0, 0}
\definecolor{sunrise}{rgb}{0.80784, 0.36078, 0}
\definecolor{lightblue}{rgb}{0.15,0.35,0.75}

\begin{document}

\section*{Lecture 24: relativity wrap-up}

\subsection*{The overall lesson of relativity theory?}

\begin{outline}[enumerate]
\1 ``Special Relativity is, fundamentally, a postulate about the
structure of space-time.'' (p 83)
  \2 A forgotten lesson about geometry?
  \2 But special relativity is false! 
  \2 General relativity, and how theories work
\1 ``The key claim of Relativity is the \emph{nonexistence} of
  simultaneity as a real physical relation among events.'' (p 92)
\1 ``\ldots the `constancy of the speed of light' cannot be a
  fundamental physical principle.'' (p 96)
\end{outline}


\subsection*{Length contraction}

Maudlin's claim: ``The coordinate-based Lorentz-Fitzgerald contraction
is not, in any straightforward sense, the physical contraction of
anything. \ldots there is a coordinate-based Lorentz-Fitzgerald
contraction \dots and also a \emph{physical} Lorentz-Fitzgerald
contraction that does have its explanation `in terms of forces or the
like'.'' (p 99)

We begin at the popular science level:
\begin{quote} Claim: according to STR, if you are carrying a meter
  stick and you go very fast, then the meter stick will shrink in
  length. \end{quote}

Why should you already be suspicious of the way that the claim is put?

Open question (that we will not deal with right now): To what extent
has the claim been tested empirically? To what extent can it be tested
empirically?

\begin{quote} Rindler's analogy: rotation in space \end{quote}

\begin{quote} It's not so simple to determine the length of an object
  that is moving quickly past you. \end{quote}

However, there is something correct about the claim. We want to
understand (a) what exactly is correct, and (b) what it means
vis-a-vis the metaphysics and epistemology of spacetime

As to (b), there are two extremist points of view:
\begin{itemize}
\item (Minkowskian) There aren't really any three-dimensional
  objects. There are four-dimensional objects, and the phenomenon of
  length contraction is how those four-dimensional objects appear in
  different reference frames.
\item (Lorentzian) Length contraction is a physical effect of one
  physical substance (the aether) on another physical substance (the
  material that makes up the rod).
\end{itemize}

\subsection*{Clocks and coordinates}

\begin{outline}[enumerate]
  \1 The hyperboloid
  \1 Distances along worldlines
  \1 Boosting parallel worldlines 
\end{outline}



\subsection*{The geometry of length contraction}

I will describe length contraction in two different ways. (1) A single
measuring rod moving relative to some observer. (2) Two measuring rods
in motion relative to each other.

\begin{defn} The \emph{proper length} of a rod is its spatial length
  in its own reference frame. \end{defn}

Of course, the notion of proper length only makes sense for a rod that
is in inertial motion. For a rod that is accelerating, funny stuff can
happen.

%% TO DO: where is proper time 1 on the hyperboloid? 

\begin{itemize}
\item There is one sense in which there is \emph{no} length
  contraction: If we take a rod $\gamma$ and apply a Lorentz boost
  $L$, then the resulting object $L(\gamma )$ has the same spatial
  length as the original object (see Figure 1). We should not be
  surprised by this because Lorentz transformations preserve spacetime
  lengths, and they also preserve the relation ``is spacelike
  related'' and the predicate ``is a spacelike object''. So it is
  incorrect to say ``if $\gamma$ were moving faster, then it would be
  shorter.''
\item There is another sense in which there is length contraction. In
  particular, if we look at the two-dimensional sheet swept out by a
  rod $\gamma$ (moving relative to B), and if we measure the
  instantaneous length in B's frame of reference, then this
  instantaneous length will be shorter than $|\gamma |$. To be
  precise:
  \begin{prop} Let $\gamma$ be a line segment that is perpendicular to
    the timelike vector $u\in V$. Let $v\in V$ be a timelike vector,
    and let $P(\gamma )$ be the projection of $\gamma$ onto
    $v^\perp$. Then $|P(\gamma )|\leq |\gamma |$, with equality only
    if $v$ is proportional to $u$.
  \end{prop}
  This last proposition can be read as saying that if one observer A
  is carrying an extended object $\gamma$, then A's account of the
  length of $\gamma$ is expected to be longer than the estimate of an
  observer B who is in motion relative to A. (For reflection: how
  might we operationalize length measurement, and why would we expect
  that observers in relative motion would measure things as having
  different lengths?)
\end{itemize}



\end{document}

\colorlet{mydarkred}{red!55!black}
\colorlet{myfieldred}{mydarkred!5} % for S' background
\colorlet{myblue}{blue!80!black}
\colorlet{mypurple}{blue!40!red!80!black}
\colorlet{mylightblue}{blue!50!black!30}
\colorlet{mylightblue2}{myblue!10}
\colorlet{mybrown}{brown!20!orange!90!black}
\colorlet{mydarkbrown}{brown!20!orange!55!black}
\colorlet{myred}{red!85!black}
\colorlet{mylightred}{red!85!black!12}
\colorlet{mydarkgreen}{green!50!black}
\colorlet{mydarkblue}{blue!50!black}

\tikzstyle{world line}=[myblue!40,line width=0.3]
\tikzstyle{world line t}=[mypurple!50!myblue!40,line width=0.3]
\tikzstyle{world line'}=[mydarkred!40,line width=0.3]
\tikzstyle{rod}=[mydarkbrown,draw=mydarkbrown,double=mybrown,double distance=2pt,
                 line width=0.2,line cap=round,shorten >=1pt,shorten <=1pt]

% COMMON AXES
\pgfdeclarelayer{back} % to draw on background
\pgfsetlayers{back,main} % set order
\def\xmin{0.23}
\def\xmax{2}
\def\Nlines{6} % number of world lines (at constant x/t)
\def\DNxp{0}   % difference in number of world lines of x' axis
\def\DNyp{0}   % difference in number of world lines of ct' axis
\def\DNy{0}    % difference in number of world lines of ct axis
\def\ang{20}   % angle between x and x' axes
\def\xplabelang{180} % anchor angle of x' axis label
%\pgfmathsetmacro\ang{atan(0.44)} % angle between x and x' axes
\def\axes{
  \pgfmathsetmacro\d{\xmax/(\Nlines+0.4)} % grid size
  \pgfmathsetmacro\D{\d/cos(\ang)/sqrt(1-tan(\ang)^2)} % grid size, boosted
  \pgfmathsetmacro\ymax{\xmax+\DNy*\d} % maximum of y = ct axis
  \pgfmathsetmacro\xmaxp{(\xmax/\d+\DNxp)*\D} % maximum of x' axis
  \pgfmathsetmacro\ymaxp{(\xmax/\d+\DNyp)*\D} % maximum of y' = ct' axis
  \coordinate (O) at (0,0);
  \coordinate (X) at (\xmax+0.15,0);
  \coordinate (T) at (0,\ymax+0.15);
  \coordinate (X') at (\ang:\xmaxp+0.2);
  \coordinate (T') at (90-\ang:\ymaxp+0.2);
  
  % FILL
  \begin{pgfonlayer}{back} % draw on back
    \fill[myfieldred]
      (O) --++ (\ang:\xmaxp) --++ (90-\ang:\ymaxp) --++ (\ang:-\xmaxp) -- cycle;
  \end{pgfonlayer}
  
  % WORLD LINE GRID
  \message{  Making world lines...^^J}
  \pgfmathsetmacro\Nylines{\Nlines+\DNy} % number of world lines at constant ct'
  \foreach \i [evaluate={\x=\i*\d;}] in {1,...,\Nlines}{
    %\message{  Running i/N=\i/\Nlines, x=\x...^^J}
    \draw[world line]   (\x,0) -- (\x,\ymax);
  }
  \foreach \i [evaluate={\t=\i*\d;}] in {1,...,\Nylines}{
    %\message{  Running i/N=\i/\Nlines, t=\t...^^J}
    \draw[world line t] (0,\t) -- (\xmax,\t);
  }
  
  % BOOSTED WORLD LINE GRID
  \message{  Making world lines for boosted frame...^^J}
  \pgfmathsetmacro\Nxplines{\Nlines+\DNxp} % number of world lines at constant x'
  \pgfmathsetmacro\Nyplines{\Nlines+\DNyp} % number of world lines at constant ct'
  \foreach \i [evaluate={\x=\i*\D;}] in {1,...,\Nxplines}{
    %\message{  Running i/N=\i/\Nlines, x=\x...^^J}
    \draw[world line'] (\ang:\x) --++ (90-\ang:\ymaxp);
  }
  \foreach \i [evaluate={\t=\i*\D;}] in {1,...,\Nyplines}{
    %\message{  Running i/N=\i/\Nlines, t=\t...^^J}
    \draw[world line'] (90-\ang:\t) --++ (\ang:\xmaxp);
  }
  
  % AXES
  \draw[->,thick] (0,-\xmin) -- (T) node[left=-1] {$ct$};
  \draw[->,thick] (-\xmin,0) -- (X) node[below=0] {$x$};
  \draw[->,thick,mydarkred] (90-\ang:-\xmin) -- (T')
    node[right=5,above=-1] {$ct'$};
  \draw[->,thick,mydarkred] (\ang:-\xmin) -- (X')
    node[anchor=\xplabelang,inner sep=2] {$x'$};
}



% SPACETIME DIAGRAM - LENGTH CONTRACTION of rod at rest in S
% Inspiration: http://people.uncw.edu/hermanr/GR/Minkowski/Minkowski.pdf
\def\ang{23} % angle between x and x' axes
\begin{tikzpicture}[scale=1.8]
  \message{Length contraction (rod at rest in S)^^J}
  
  % AXES
  \def\Nlines{7} % number of world lines (at constant x/t)
  \axes
  
  % SETTINGS
  \pgfmathsetmacro\xA{2*\d} % triangle left corner x coordinate in S
  \pgfmathsetmacro\yA{4*\d} % triangle left corner y=ct coordinate in S
  \pgfmathsetmacro\Lz{4*\d} % proper/rest length L0 in S
  \pgfmathsetmacro\L{\Lz/cos(\ang)} % length L in S'
  \coordinate (L) at (\Lz,0); % rod end in S
  \coordinate (L') at (\ang:\L); % rod end in S'
  \coordinate (A) at (\xA,\yA); % point A in triangle
  \coordinate (B) at (\xA+\Lz,\yA); % point B in triangle
  \coordinate (B') at (\xA+\Lz,{\yA+\Lz*tan(\ang)}); % point B' in triangle
  
  % FILL
  \begin{pgfonlayer}{back} % draw on back
    \fill[mylightblue2] (\xA,-\xmin) rectangle (\xA+\Lz,\xmax);
  \end{pgfonlayer}
  \draw[->,thick,mydarkbrown] (\xA,-\xmin) --++ (0,\xmin+\xmax+0.2);
  \draw[->,thick,mydarkbrown] (\xA+\Lz,-\xmin) --++ (0,\xmin+\xmax+0.2);
  
  % ROD
  \draw[rod] (\xA,0) --++ (L)
    node[midway,below=-1] {$L_0$};
  \draw[rod] (\ang:{\xA/cos(\ang)}) --++ (L')
    node[pos=0.485,above=1] {\contour{mylightblue2}{$L$}};
  
  % TRIANGLE
  \draw[very thick,myred,rounded corners=0.1]
    (A) -- (B') node[midway,above=0] {\contour{mylightblue2}{$\Delta x'$}}
        -- (B) node[midway,right=-2] {\contour{myfieldred}{$c\Delta t$}}
        -- cycle node[pos=0.51,below=-1] {\contour{mylightblue2}{$\Delta x$}};
  %\fill[myfieldred] (A)++(185:0.089) circle(0.04);
  %\fill[mydarkred] (A) circle(0.03) node[below=1,left=-2.7] {A};
  \fill[myfieldred] (A)++(200:0.1) circle(0.04);
  \fill[mydarkred] (A) circle(0.03) node[below=1,left=-2.3] {\contour{myfieldred}{A}};
  \fill[mydarkred] (B) circle(0.03) node[below=1,right=0] {\contour{myfieldred}{B}};
  \fill[mydarkred] (B') circle(0.03) node[above=2,right=-1] {\contour{myfieldred}{B$'$}};
  
\end{tikzpicture}



% SPACETIME DIAGRAM - LENGTH CONTRACTION of moving rod (at rest in S')
\begin{tikzpicture}[scale=1.8]
  \message{Length contraction (rod at rest in S')^^J}
  
  % AXES
  \def\Nlines{6} % number of world lines (at constant x/t)
  \axes
  
  % SETTINGS
  \pgfmathsetmacro\Lz{4*\D} % proper/rest length L0 in S'
  \pgfmathsetmacro\L{cos(2*\ang)/cos(\ang)*\Lz} % contracted length L in S
  \coordinate (L) at (\L,0); % rod end in S
  \coordinate (L') at (\ang:\Lz); % rod end in S'
  \coordinate (A) at (90-\ang:{3*\d/cos(\ang)}); % point A in triangle
  \coordinate (B) at ($(A)+(L)$); % point B' in triangle
  \coordinate (B') at ($(A)+(L')$); % point B in triangle
  
  % FILL
  \begin{pgfonlayer}{back} % draw on back
    \fill[mylightred]
      (90-\ang:-\xmin) -- (90-\ang:\xmaxp) --++ (\ang:\Lz) -- (L) --++ (90-\ang:-\xmin) -- cycle;
  \end{pgfonlayer}
  \draw[->,thick,mydarkbrown] (L)++(90-\ang:-\xmin) -- (L) -- (L') --++ (90-\ang:\xmaxp+0.2);
  
  % ROD
  \draw[rod] (O) -- (L)
    node[midway,below=-1] {$L$};
  \draw[rod] (O) -- (L')
    node[pos=0.49,above=1] {\contour{mylightred}{$L_0$}};
  
  % TRIANGLE
  \draw[very thick,myred,rounded corners=0.1]
    (A) -- (B') node[midway,above=0] {\contour{mylightred}{$\Delta x'$}}
        -- (B) node[midway,right=-2] {\contour{myfieldred}{$c\Delta t'$}}
        -- cycle node[pos=0.525,below=-1] {\contour{mylightred}{$\Delta x$}};
  \fill[mydarkred] (A) circle(0.03) node[below=0,left=0] {\contour{white}{A}};
  \fill[mydarkred] (B) circle(0.03) node[below=1,right=0] {\contour{myfieldred}{B}};
  \fill[mydarkred] (B') circle(0.03) node[below=0.5,right=-1] {\contour{myfieldred}{B$'$}};
  
\end{tikzpicture}

\begin{tikzpicture}[scale=2]
  \message{Invariant hyperboloids with equations^^J}

  \pgfmathsetmacro\ang{atan(0.52)} % angle between x and x' axes
  \pgfmathsetmacro\d{0.64*\xmax/\Nlines} % grid size
  \pgfmathsetmacro\D{\d/cos(\ang)/sqrt(1-tan(\ang)^2)} % grid size, boosted
  \pgfmathsetmacro\Ax{3*\D*sin(\ang)} % x coordinate of event A
  \pgfmathsetmacro\Ay{3*\D*cos(\ang)} % y coordinate of event A
  \pgfmathsetmacro\Bx{4*\D*cos(\ang)} % x coordinate of event B
  \pgfmathsetmacro\By{4*\D*sin(\ang)} % y coordinate of event B
  \pgfmathsetmacro\st{3*\d} % spacetime interval
  \pgfmathsetmacro\sx{4*\d} % spacetime interval

  \def\tick#1#2{\draw[thick] (#1) ++ (#2:0.06) --++ (#2-180:0.12)}
\def\tickp#1#2{\draw[thick,mydarkred] (#1) ++ (#2:0.06) --++ (#2-180:0.12)}
  
  % AXES
  \axes
  \node[mydarkgreen,below=1] at (\Ax/2,{sqrt((\st)^2+(\Ax/2)^2)}) {$\phi$};
  \node[mydarkblue,left=1.5] at ({sqrt(\sx^2+(0.54*\By)^2)},0.54*\By) {\contour{white}{$\phi$}};
  
  % TICKS
  \tick{0,\Ay}{0} node[mydarkgreen,above=0,left=-2]
    {$ct_1\cosh\phi$};
  \tick{\Ax,0}{90} node[mydarkgreen,right=4,below=-4]
    {\contour{white}{$ct_1\sinh\phi$}};
  \tick{\Bx,0}{90} node[mydarkblue,right=8,below=-4]
    {\contour{white}{$x_1\cosh\phi$}};
  \tick{0,\By}{0} node[mydarkblue,below=0,left=-2]
    {$x_1\sinh\phi$};
  
  % EVENT LABELS
  \node[mydarkred,anchor=0,inner sep=3] at (C) {\contour{myfieldred}{R}};
  \node[mydarkred,below right] at (1.0*\xmax,3.57*\d) {$
    \begin{aligned}
      %\mathrm{R} &= (x_1,ct_1) \\
      %           &= (x_1\cosh\phi-ct_1\sinh\phi,\\[-0.3em]
      %           &\hspace{1.7em} ct_1\cosh\phi-x_1\sinh\phi)\\
      \mathrm{R}
      &=
      \left\{\begin{aligned}
        ct &= ct_1 \\
         x &=  x_1
      \end{aligned}\right.\\
      &=
      \left\{\begin{aligned}
        ct' &= ct_1\cosh\phi -  x_1\sinh\phi \\
         x' &=  x_1\cosh\phi - ct_1\sinh\phi
      \end{aligned}\right.
    \end{aligned}
  $};
  %\node[mydarkred,anchor=-173,inner sep=3] at (C') {\contour{myfieldred}{R$'$}};
  \node[mydarkred,anchor=167,inner sep=3] at (C') {$
    \begin{aligned}
      \mathrm{\contour{myfieldred}{R}}
      &=
      \left\{\begin{aligned}
        ct &= ct_1\cosh\phi +  x_1\sinh\phi \\
         x &=  x_1\cosh\phi + ct_1\sinh\phi
      \end{aligned}\right.\\
      &=
      \left\{\begin{aligned}
        ct' &= ct_1 \\
         x' &=  x_1
      \end{aligned}\right.
    \end{aligned}
  $};
  
\end{tikzpicture}



% SPACETIME DIAGRAM - INVARIANT HYPERBOLOIDS with equations 2
\begin{tikzpicture}[scale=2]
  \message{Invariant hyperboloids with equations 2^^J}
  
  % AXES
  \axes
  
  % TICKS
  \tick{0,\Ay}{0} node[mydarkgreen,above=0,left=-2]
    {$\gamma ct_1$};
  \tick{\Ax,0}{90} node[mydarkgreen,right=1,below=-1]
    {\contour{white}{$\gamma\beta ct_1$}};
  \tick{\Bx,0}{90} node[mydarkblue,right=0,below=-1]
    {\contour{white}{$\gamma x_1$}};
  \tick{0,\By}{0} node[mydarkblue,below=0,left=-2]
    {$\gamma\beta x_1$};
  
  % EVENT LABELS
  \node[mydarkred,anchor=0,inner sep=3] at (C) {\contour{myfieldred}{R}};
  \node[mydarkred,below right] at (1.0*\xmax,3.57*\d) {$
    \begin{aligned}
      \mathrm{R}
      &=
      \left\{\begin{aligned}
        ct &= ct_1 \\
         x &=  x_1
      \end{aligned}\right.\\
      &=
      \left\{\begin{aligned}
        ct' &= \gamma(ct_1 - \beta  x_1) \\
         x' &= \gamma( x_1 - \beta ct_1)
      \end{aligned}\right.
    \end{aligned}
  $};
  %\node[mydarkred,anchor=-173,inner sep=3] at (C') {\contour{myfieldred}{R$'$}};
  \node[mydarkred,anchor=167,inner sep=3] at (C') {$
    \begin{aligned}
      \mathrm{\contour{myfieldred}{R}}
      &=
      \left\{\begin{aligned}
        ct &= \gamma(ct_1 + \beta  x_1) \\
         x &= \gamma( x_1 + \beta ct_1)
      \end{aligned}\right.\\
      &=
      \left\{\begin{aligned}
        ct' &= ct_1 \\
         x' &=  x_1
      \end{aligned}\right.
    \end{aligned}
  $};
  
\end{tikzpicture}




\end{document}
%%% Local Variables:
%%% mode: latex
%%% TeX-master: t
%%% End:
