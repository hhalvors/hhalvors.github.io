\documentclass[12pt]{article}
\usepackage{fullpage}
\begin{document}

\section*{Warm up philosophical questions}

\begin{enumerate}

\item What does it really mean to say something like ``that is a
  circle on the chalkboard''? Under what conditions is such a claim
  true? How would one go about testing such a claim? (You might want
  also to think about other geometric objects such as lines,
  triangles, etc.) The point here is that lines, triangles, circles
  are \emph{abstract} objects, whereas things written on a chalkboard
  are \emph{concrete objects}. So what does it mean to say that a
  concrete object (e.g.\ ``that on the chalkboard'') is the same as an
  abstract object (e.g.\ ``a circle'').

  \begin{enumerate}
  \item Is the ``is'' in ``$x$ is a circle'' (where $x$ is a physical
    object) an assertion of identity or of predication? For example,
    when I say ``my shirt is red'' then I am not asserting an identity
    ``shirt = red'', but that ``my shirt'' falls under the predicate
    ``is red''. In contrast, when I say that Superman is Clark Kent, I
    am asserting an identity ``Superman = Clark
    Kent''. \end{enumerate}

\item Physicists today have no problem saying things like ``our
  universe is a curved four-dimensional manifold''. What do they mean
  by that? Is their claim justified? Why or why not?

\item What problems might there be with taking the truths of geometry
  to be purely analytic, i.e.\ true by definition?

\item What problems might there be with taking a neo-empiricist stance
  about the axioms of geometry? (A neo-empiricist stance might say
  that we accept a certain geometry because it is part of the best
  overall explanation of the things we experience.)

\item What problems might there be with saying that reality is
  actually made out of geometrical objects, and that our (or physics')
  goal is to say true things about them?

\item (Euthyphro meets physics) Does light follow a line because it is
  straight, or is a line straight because light follows it?  

\end{enumerate}


\section*{On the Factual Foundations of Geometry}

\begin{enumerate}

\item (p 47) What dilemma does Helmholtz start the article with? (For
  deeper reflection: How does this dilemma relate to the traditional
  dichotomy between analytic and synthetic propositions?)

\item (p 47) What does Helmholtz think is the problem with trying to
  ``test rigid bodies for the invariance of their form''?

\item (p 47) What advantage does Helmholtz think that analytic
  geometry has?

\item (p 50) What must be presupposed in order for us to be able to
  use geometry to make meaningful statements about the world?


\end{enumerate}


\section*{On the Origin and Meaning of Geometrical Axioms}

\begin{enumerate}

\item (p 54) What does Helmholtz think is the main difficulty for
  figuring out what justifies the geometric axioms?

\item (p 54) What does Helmholtz think is actually happening (in our
  minds) when we assume that certain geometric objects can be
  constructed? 

\item (p 59) What does Helmholtz think is an advantage of Riemann's
  approach to geometry?

\item (p 63, top) What does Helmholtz say about the claim that we know
  a priori that space is Euclidean? (What argument does he give?)

\item (p 63, bottom) What, according to Helmholtz, must be assumed in
  order to compare magnitudes of physical objects?

\item (p 64) What does Helmholtz think it would feel like if suddenly
  everything were doubled in size?   

\item (p 66) What does Helmholtz think about the idea that human
  beings are physiologically pre-wired to perceive space in a
  Euclidean fashion?

\item (p 67) Why can geometry alone not really tell us if two physical
  bodies are congruent with each other?

\item (p 67) What does Helmholtz mean by ``taking the notion of
  rigidity as a mere ideal''? Do you think he is endorsing this notion
  or not?

\item (p 67) What examples of ``propositions relating to the
  mechanical properties of natural bodies'' does Helmholtz give?

\item (p 68) Does Helmholtz think that we have apriori knowledge of
  the axioms of geometry, in the sense of Kant's ``transcendental form
  given before experience?''


\item (p 68) What else is needed in order for the axioms of geometry
  to represent relations of real (i.e.\ physical) things? (For further
  reflection: what would Helmholtz say about the idea that an axiom of
  geometry has been falsified by experience?)

\item (p 68) Harder question: Does Helmholtz think that the axioms of
  geometry can be confirmed or disconfirmed by empirical observations?

\item Explain in a few sentences what Helmholtz means by the title of
  this essay.

\end{enumerate}



\end{document}