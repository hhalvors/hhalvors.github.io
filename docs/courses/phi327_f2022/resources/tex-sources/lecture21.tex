\documentclass[12pt,fleqn]{article}
\usepackage{outlines}
\usepackage{amsfonts}
\usepackage{fullpage}
\usepackage{url}
\renewcommand{\emph}{\textbf}
\setlength{\parskip}{1em}
\setlength{\parindent}{0em}
\begin{document}

\section*{Lecture 21: Special relativity}

\subsection*{Outline of philosophical issues}

\begin{outline}[enumerate]

\1 presentism versus eternalism (Putnam's argument)

\1 the twin paradox

\1 Lorentz contraction and four-dimensional objects

\1 suprise issue 



% \1 Relativity of simultaneity: does it profoundly change our view of
% time?

% \2 Presentism versus Eternalism

% \2 Free Will versus Fatalism



% \1 Three philosophical interpretations of STR

% \2 Four dimensional universe (Minkowski, Einstein, Ted Sider, Yuri
% Balashov)

% \2 Neo-Lorentzian (W.L. Craig, Bohmians): there actually is a
% preferred frame of reference, even if it is impossible for us to
% detect it

% \2 Fragmentalism (Kit Fine)


% %% Rovelli neither presentism nor eternalism


% \1 Is the fundamental ontology of the universe $3$ or $4$ dimensional?


% \1 How are we supposed to think of the frame-relative descriptions
% versus the four-dimensional model? 


% \1 Does spacetime geometry explain physical states of affairs? (Should
% we think of spacetime as a thing in the causal order?)

% %% Lorentz contraction 

% %% Brown and Timpson, glorious non-entity

% %% Bell spaceship


% \1 Should physical theories proceed constructively, from existence
% claims to principles?

% %% Principle versus constructive theories

\end{outline}

\subsection*{Intuition pumps}

\begin{outline}[enumerate]

\1 Constancy of the speed of light in all reference frames

\2 One way speed (addition of velocities)  

\2 Round trip speed  

% \1 Lorentz contraction and dilation

% \1 Time dilation

\1 Relativity of simultaneity 

\end{outline}



\subsection*{The geometry of simultaneity}

light cones $\Leftrightarrow$ null tangent vector

inertial trajectory $\Leftrightarrow$ timelike tangent vector

Minkowski metric and orthogonality

spacelike hypersurface $\Leftrightarrow$ spacelike tangent vector

constancy of the speed of light $c=1$

? hyperbolic angles


\newpage 

\subsection*{Putnam's argument for eternalism}

\begin{outline}[enumerate]

\1 Conclusion: ``\dots contingent statements about future events
already have a truth value'' (p 247)

``\dots the notion of being `real' turns out toe be coextensive with
the \textit{tenseless} notion of existence''

\1 Assumptions from common sense

\2 I-now am real 

\2 Two people A and B can move past each other

\2 There are no privileged observers (democracy of reference frames)

If A takes B to be real, and B takes X to be real, then A should take
X to be real

If A takes B to be just as reliable as himself, and if B currently (in
A's frame of reference) believes that ``the event X has occurred'',
then A should believe that ``the event X has occurred''

\1 Putnam's construction

\2 Let A and B be located at the same place, but have velocity
vectors that are not parallel 

\2 There is an X such that:

\3 B should take X to be real (since X lies in the simultaneity
surface for B)

\3 X lies in the future of the simultaneity surface for A

\2 So there is an X that is in the future for A, but that A
should judge as real


\1 Ways out from Putnam's conclusion

Treat all spacelike separated events as vaguely ``now''



\1 Modified version of Putnam's construction

Consider two observers A and B such that B lies on A's
simultaneity surface, and such that the instantaneous velocity vectors
of A and B are not parallel

In this case, there is an event X that lies on B's simultaneity
surface, and such that X is in the \underline{causal future} of A


\1 Responses to Putnam

\2 Neo-lorentzian

There is a preferred frame of reference

\2 Howard Stein

\3 ``Now'' for me is my past lightcone

\3 ``Now'' for me is a point

\2 Momentum and context 

Statements are made in a context

Context includes state of motion

\end{outline}

\subsection*{Points for discussion}

\begin{itemize}
\item Are there (objective) facts about which events are simultaneous?
  \begin{itemize}
  \item Can it be objectively true that X and Y are simultaneous relative to A?
  \end{itemize}
\item ``\ldots the problem of the reality and the determinateness of
  future events is now solved. Moreover, it is solved by physics and
  not by philosophy. We have learned that we live in a
  four-dimensional and not a three-dimensional world'' (p 247)

\end{itemize}

\small

\textbf{More literature on this topic}
\begin{itemize}
\item C.W. Rietdijk, ``A rigorous proof of determinism derived from
  the special theory of relativity''
\item H. Stein, ``On Minkowski spacetime''
\item H. Stein, ``On relativity theory and the openness of the
  future''
\item T. Sider, \textit{Four Dimensionalism: An Ontology of
    Persistence and Time}
\item M. Hinchliff, ``A defense of presentism in a relativistic
  setting''
\item S. Saunders, ``How relativity contradicts presentism''
\item C. Rovelli, ``Neither presentism nor eternalism''
\item H. Halvorson, ``Momentum and context''
\item A. Ney, \textit{Metaphysics}
\end{itemize}

\end{document}





\begin{outline}[enumerate]

\1 Minkowski spacetime $M$ is an $n$-dimensional (with $n\geq 2$)
  \emph{affine space}, equipped with a Lorentzian metric

\2 There is an $n$-dimensional vector space $V$, and for any two points
  $p,q\in M$, there is a vector $\overline{pq}\in V$.

\2 There is an inner product $\eta :V\times V\to\mathbb{R}$. The
vector space $V$ can be divided (in many ways!) into a
three-dimensional subspace on which $\eta$ is negative-definite and an
orthogonal subspace on which $\eta$ is positive-definite

\2 We say that $u\in V$ is \emph{timelike} if $\eta (u,u)>0$

\2 We say that $u\in V$ is \emph{spacelike} if $\eta (u,u)<0$

\2 We say that $u\in V$ is \emph{lightlike} or \emph{null} if
$\eta (u,u)=0$

\2 We say that $u\in V$ is \emph{causal} if $\eta (u,u)\geq 0$

\1 For any $p,q\in M$, we say that $p$ is timelike related to $q$ just
in case $\overline{pq}$ is timelike. We define spacelike and lightlike
similarly


\1 Given $p\in M$, the set of points that are causally related to $p$
consists of two subsets whose intersection is $p$. We label on of
these subsets as $F_p$ and the other as $B_p$. We call these sets the
\emph{lightcones} based at $p$

The interior of $F_p$ is called the \emph{causal future} of $p$

The interior of $B_p$ is called the \emph{causal past} of $p$ 


\1 An \emph{inertial trajectory} is represented by a line $\alpha$ in
$M$ that always stays within the light cone (i.e.\ trajectories of
material bodies are never as fast as light)

\dots which is generated by a pair $(p,u)$ where $p\in M$ and $u$ is a
timelike vector in~$V$



\1 Simultaneity slices: Given $p\in M$ and $u$ a timelike vector in
$V$, the set
\[ (p,u)^\perp \: = \: \{ q\in M:\eta (u,\overline{pq})=0 \} \]
is a three-dimensional subspace of $M$


\1 Relativity of simultaneity: if $v$ is between $u$ and a null vector
$w$, then $(p,v)^\perp$ is between $(p,u)^\perp$ and $w$


\1 Coordinate systems

Let $p\in M$ and $u\in V$, and fix an orthonormal basis
$(e_1,e_2,e_3)$ of $u^\perp$. Then for any vector $v\in V$, there is a
unique quadruple $(t,a_1,a_2,a_3)$ of real numbers such that
\[ v = tu+a_1e_1+a_2e_2+a_3e_3 .\] We may then define a coordinate
chart $\phi :M\to\mathbb{R}^4$ by setting
$\phi (p+v)=(t,a_1,a_2,a_3)$, for each $v\in V$. Since for each
$q\in M$, there is a unique $v\in V$ such that $p+v=q$, it follows
that $\phi$ is defined on all of $M$.

Fact: The set $u^\perp$ has coordinates of the form
$(0,a_1,a_2,a_3)$. i.e.\ the events that are \emph{simultaneous} for
an observer at $p$ with velocity $u$.


\end{outline}

\subsection*{Putnam's argument for eternalism}

\begin{outline}[enumerate]

\1 Assumptions from common sense

\2 A-now is real

\2 Two people A and B can be at roughly the same place, but with
different states of motion

\2 There are no privileged observers

If A should take B to be real, and B should take X to be real,
then A should take X to be real

If A believes that B exists, and A believes that B correctly
asserts that ``the event X has occurred'', then A should believe
that X has occurred

\1 Putnam's construction

\2 Let A and B be located at the same place, but have velocity
vectors that are not parallel 

\2 There is an X such that:

\3 B should take X to be real (since X lies in the simultaneity
surface for B)

\3 X lies in the future of the simultaneity surface for A

\2 So there is an X that is in the future for A, but that A
should judge as real



\1 Modified version of Putnam's construction

Consider two observers A and B such that B lies on A's
simultaneity surface, and such that the instantaneous velocity vectors
of A and B are not parallel

In this case, there is an even X that lies on B's simultaneity
surface, and such that X is in the \emph{causal future} of A 


\end{outline}

\end{document}
%%% Local Variables:
%%% mode: latex
%%% TeX-master: t
%%% End:
