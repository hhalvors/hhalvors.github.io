\documentclass[12pt]{article}
\usepackage{fullpage}
\begin{document}

%% hypothetico deductivism about geometry

\section*{Einstein, Geometry and Experience}

\begin{enumerate}

\item (p 147) What are the two things that Einstein cites as reasons
  for the high repute of mathematics?

\item (p 147) What riddle about mathematics does Einstein present, and
  what is his solution to the riddle? (For deeper reflection: how is
  the thing he mentions supposed to solve the riddle?)

\item (p 147) What does Einstein think that ``axiomatics'' was good
  for?

\item (p 147) What, according to axiomatics, is the subject matter of
  mathematics?

\item (p 147--148) What are the two different ways of interpreting an
  axiom of geometry? (For further reflection: which does Einstein
  think is the better way?)

\item (p 149) Under what assumption about reference frames are we,
  according to Einstein, forced to give up Euclidean geometry?

\item (p 149) What specific ``relation'' between physical reality and
  geometry does Poincar\'e deny? For what reason is he inclined to
  deny it? How does rejecting this relation lead to conventionalism?
  (For deeper reflection: does it seem to you that Einstein has
  characterized Poincar\'e's position accurately?)

\item (p 149) What does Einstein mean by saying that ``\emph{sub
    specie aeterni} Poincar\'e is right in this interpretation''? If
  Einstein thinks this, then why does he adopt a different position
  than Poincar\'e?

\item (p 150) What does Einstein think must be assumed about clocks in
  order for Riemannian geometry to be applicable? What experimental
  support does Einstein claim for this assumption?

\item (p 150) What experimental evidence does Einstein think we have
  for Riemannian geometry (as opposed to even more general notions of
  space)?

\item Compare and contrast Poincar\'e's and Einstein's views about
  rigid bodies.

\item What is Einstein's view about measuring rods and clocks? Are
  measuring rods rigid bodies? Can we assume that a meter stick will
  stay the same length if we transport it from Princeton to Paris?



\end{enumerate}


\end{document}