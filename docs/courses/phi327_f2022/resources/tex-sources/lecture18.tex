\documentclass[12pt]{article}
\usepackage{outlines}
\usepackage{amsfonts}
\usepackage{fullpage}
\begin{document}

\section*{Lecture 18}

\subsection*{Poincar\'e, On the Foundations of Geometry}

\begin{outline}[enumerate]

\1 Advance warning: Poincar\'e says some things about human
  perception that may have been empirically falsified. How much of his
  argument can stand without these assumptions?

\1 There are two spaces (sensible and geometric) and they don't look
at all like each other

\2 Sensible space has very little (if any) intrinsic mathematical
structure [this is where Poincar\'e makes empirical assumptions that
are questionable].

\3 Sensations have no spatial character (p
117)

\3 No notion of distance between sensations (p 117)

\3 No notion of contiguity of sensations

\2 Geometric space is not a form of intuition, but a cognitive tool
for reasoning

\2 There is no bridge from sensible to geometric space


\1 Geometry would mean nothing to creatures that could not move ---
and more generally, that did not have a will

\2 We judge two sensations as being of the same thing if we can follow
an object with our eye (p 121)

\2 The distinction between \underline{displacement} and
\underline{alteration} depends on our ability to move

\3 We consider two displacements to be the same if both can be
``undone'' by the same internal change

\3 These (equivalence classes of) displacements form a group; if they
didn't, then there could be no geometry




\1 Geometry is not our attempt to picture ``what is out there''

\2 Recall the focus on the group of displacements

\2 The standard of simplicity [under convenience] is not some
pre-existing geometric ideal. The displacements of EG 
form a simpler group than the displacements of LG 



\1 A fiction of rigid bodies is created by our factorizing changes
into (a) displacement and (b) alteration


\end{outline}  

\subsection*{Einstein, Geometry and Experience}

\begin{outline}[enumerate]

\1 The puzzle (p 147): ``how is it possible that mathematics, being
  after all a product of human thought that is independent of
  experience, is so admirably appropriate to the objects of reality''?

\1 Einstein's solution: laws of mathematics (refer to reality
$\leftrightarrow$ not certain)

\2 Axiomatics has succeeded in separating the logical-formal from its
objective or intuitive content [de-interpretation, cf. Hilbert]

\1 Two interpretations of the axioms of geometry (e.g.\ ``through two
points in space there passes one and only one straight line'')

\2 Older interpretation: the axioms are self-evident under the
self-evident interpretation of the words contained in them

\2 Newer interpretation: the axioms are (a) to be taken in a purely
formal sense, (b) void of content of intuition or experience, (c) free
creations of the human mind, (d) first definte the objects of which
geometry treats [implicit definition]

``\ldots mathematics as such cannot predicate anything about objects
of our intuition or real objects'' (p 148)

``\ldots the system of concepts of axiomatic geometry alone cannot
make any assertions as to the behavior of real objects of this kind,
which we will call \underline{rigid bodies}'' (p 148)


\1 Re-interpretation

``\ldots geometry must be stripped of its merely logical-formal
character by \underline{assigning} to the empty conceptual schemata of
axiomatic geometry objects of reality that are capable of being
experienced'' (p 148)

``To accomplish this, we need only add the proposition: Solid bodies
are related, with respect to their possible relative positions, as are
bodies in Euclidean geometry of three dimensions'' (p 148)


\1 Einstein versus Poincar\'e

\2 Einstein: we get conventionalism if we reject the equation (body of
axiomatic Euclidean geometry $\approx$ practically rigid body of
reality)

HH: What!? Is E saying that we should assume that our measuring
devices satisfy Euclidean laws?

  
\end{outline}

\end{document}








\1 Against the transcendental (Kantian apriori) character of Euclid's
axioms

``\ldots space [\emph{der Raum}] \dots does not at all correspond with
the most general conception of an aggregate of three dimensions.'' (p
61)

``But if we can imagine such spaces of other sorts, it cannot be
maintained that the axioms of geometry are necessary consequences of
an a priori transcendental form of intuition, as Kant thought.'' (p
63)









\end{outline}







\end{document}