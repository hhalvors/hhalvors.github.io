\documentclass[fleqn]{beamer}

\usetheme[
  progressbar=frametitle,    % or try: 'none', 'foot'
  numbering=fraction,        % frame number / total
  sectionpage=none,
  subsectionpage=none
]{metropolis}

\title{Week 4 \\ Part C \\ The AI as Scientist}
\author{Hans Halvorson}
\date{May 19, 2025}

\usepackage[backend=biber,style=authoryear]{biblatex}
\addbibresource{ai.bib}


% Load math packages
\usepackage{amsmath,amssymb}

% Highlighting commands
\usepackage{soul}

% Optional: custom keyword formatting
\newcommand{\kw}[1]{\textbf{#1}}

% Only show section numbers in TOC
\setbeamertemplate{section in toc}[sections numbered]

% Section title slide at each new section
\AtBeginSection[]
{
  \begin{frame}
    \frametitle{Table of Contents}
    \tableofcontents[currentsection]
  \end{frame}
}

% Minimalist footer and navigation
\setbeamertemplate{footline}[frame number]
\setbeamertemplate{navigation symbols}{}

% Ensure numbered sections (useful with TOC)
\setcounter{secnumdepth}{1}

\begin{document}


\begin{frame}
  \titlepage
\end{frame}

\begin{frame}{Table of Contents}
\tableofcontents
\end{frame}

\section{Does ML change the scientific method?}

\begin{frame}{Foundations of scientific inference}

  \begin{itemize}
  \item In the early modern period (17th century and forward),
    philosophers reflected on the foundations of scientific inference
    --- i.e.\ how empirical data might justify our belief/acceptance
    of theoretical hypotheses
  \item This investigation was especially prominent among English,
    Scottish, and Irish philosophers --- the so-called \textbf{British
      empiricists}
  \item The earliest model was \textbf{straight induction}: a
    generalization $T$ is abstracted from many repeated instances
    $E_1,E_2,\dots $ of a phenomenon.
    \begin{itemize}
      \item Since the first 1,000 ravens were black, we are justified
        in believing that the next raven will be black.
      \end{itemize}
  \end{itemize}

\end{frame}

\begin{frame}{Hume's problem of induction}

  \begin{itemize}
  \item David Hume (1711--1776) asked: what is the justification for
    the inductive method? Why believe that this method is rational?
  \item The link between past instances $E_1,E_2,\dots $ and future
    projection $T$ seems to be mediated by an assumption:
    \begin{itemize}
      \item \textbf{Uniformity of Nature:} The future will resemble
        the past.
      \end{itemize}
    \item But what is the justification for UN?
  \end{itemize}

\end{frame}

\begin{frame}

  \begin{itemize}
  \item Many philosophers conclude that theorizing requires more
    creative input from the scientist.
    \begin{itemize}
    \item Karl Popper: hypothetico-deductive method
    \end{itemize}
  \item Many philosophers conclude that scientific inference can only
    proceed against a backdrop of assumptions that are \emph{not}
    justified by scientific inference.
   \begin{itemize}   
   \item Some (e.g.\ Bayesians) are hopeful that ``inductive
     learners'' would converge in the long run.
    \end{itemize}
  \end{itemize}

\end{frame}

%% Book covers: Howson and Urbach, Earman, Norton

\begin{frame}{Does Big Data change our understanding of science?} 

\includegraphics[scale=0.25]{theory}

\end{frame}

\begin{frame}

  \includegraphics[scale=0.25]{kitchin}


\end{frame}

\begin{frame}

  ``\ul{Generative modelling} offers a way to discern the most
  credible theory from various explanations for observational
  data. This is achieved solely through the data, without any
  predetermined understanding of the potential physical mechanisms
  operating within the studied system.'' (Rodrigues, ``Machine
  learning in physics'', referring to Schawinski, Turp, and Zhang,
  ``Exploring galaxy evolution with generative models'' 2018)

\end{frame}

\begin{frame}{A new empiricism?} 

  ``Our approach of using generative models like the Fader network to
  forward model physical processes and test hypotheses in a
  data-driven way has significant potential in astrophysics and other
  fields. Its central advantage is its data-driven nature which makes
  no assumptions on the underlying physics.'' (Schawinski et al. 2018)

\end{frame}

\begin{frame}{A new empiricism?}

  ``Deep learning leverages deep neural networks to automatically
  learn representations from the data.'' (Rodrigues, p 5)

\end{frame}


\section{Conceptual engineering}

\begin{frame}

  \begin{itemize}
  \item ``Surprising and creative ideas are the foundation of advances
    in science'' (p 764)
  \item What kind of thing is a ``new idea''?
    \begin{itemize}
    \item A new conjecture \newline Ex. ``Perhaps injecting some of
      this virus will prevent the person from getting a worse case?''
    \item An expansion of the conceptual framework
    \end{itemize}
  \end{itemize}

\end{frame}

\begin{frame}

  ``Ved forskellige lejligheder har jeg forsøgt at vise, at den
  belæring, som fysikkens nyere udvikling har givet os med hensyn til
  nødvendigheden af en stadig udvidelse af begrebsrammerne for
  indordningen af nye erfaringer, fører os til en almindelig
  erkendelsesteoretisk indstilling, der turde være egnet til at undgå
  tilsyneladende begrebsvanskeligheder, også på andre af videnskabens
  områder.'' (Niels Bohr, Causality and Complementarity 1936)

\end{frame}

\begin{frame}

  ``\dots at ingen erfaring er definerbar uden en logisk ramme, og at
  enhver mangel på harmoni synligt i et sådant forhold, kun kan
  fjernes ved en behørig udvidelse af den begrebslige ramme.'' (Niels
  Bohr, Fysisk videnskab og studiet af religioner 1953)

\end{frame}

\begin{frame}{Theoretical concepts}

  \begin{itemize}
  \item Modern science is characterized by the introduction of
    \textbf{novel concepts} that enable new levels of understanding to
    be achieved
   \item Examples:
  \begin{itemize}
  \item Genes
  \item Forces   
  \item Electromagnetic fields
  \item The quantum of action
  \item Spin (quantum two-valuedness)
  \item Pauli exclusion principle
  \item Lorentz transformation
  \item Geometric phases   
  \end{itemize}
\end{itemize}

\end{frame}

\begin{frame}

  \begin{itemize}
  \item ``It would be truly exciting to see an AI uncover hidden
    patterns or irregularities in scientific data previously
    overlooked by humans, which could lead to \ul{new ideas} and,
    ultimately, to \ul{new conceptual understanding}. As of now, we
    are not aware of such cases.'' (p 765)
  \item ``The concepts rediscovered in all of those works were not new
    and, thus, the most important challenge for the future is to learn
    how to extract \ul{previously unknown concepts}.'' (p 766)
  \end{itemize}


\end{frame}


\begin{frame}{Rational novelty?}
  
\begin{columns}[T] \begin{column}{0.55\textwidth}  
    \begin{itemize}
    \item 20th century philosophers of science: novel scientific
      concepts must have some rational connection to
      already-understood concepts
  \begin{itemize}
  \item Something is needed for these new concepts to be
    \textbf{intelligible}.
  \item If I simply introduced a phrase ``slithy toves'', you would
    have no idea what I meant
  \end{itemize}
\end{itemize}

\end{column}

\begin{column}{0.35\textwidth}

  \includegraphics[scale=0.15]{aufbau}

\end{column}
\end{columns}

\begin{itemize}
\item Rudolf Carnap (1891--1970) originally proposed that the new
  concepts should all be \kw{logical constructions} from better
  understood concepts
\end{itemize}

\end{frame}

\begin{frame}

  \includegraphics[scale=0.4]{begriff}

\end{frame}

\begin{frame}

  \begin{itemize}
  \item Even with a more advanced understanding of ``logical
    construction'', the criterion of logical constructability seems
    far too strict
  \item Even according to more advanced accounts of ``logical
    construction'', no genuine novelty arises (and that is the point
    of the relation being logical)
  \end{itemize}

\end{frame}

\begin{frame}

  \begin{itemize}
  \item A novel concept is similar to a \textbf{theoretical posit},
    i.e. something whose existence is conjectured to explain the
    phenomena.
  \begin{itemize}
  \item Le Verrier's (1846) prediction of the existence of Neptune
  \item Gell-Mann's prediction of ...
  \item Prediction of the Higgs boson 
  \end{itemize}
\end{itemize}

\end{frame}

\begin{frame}{Can AI Perform Scientific Inference?}

\small
\textbf{Bayesian conditionalizing:} Yes, AI systems can update beliefs using Bayes’ rule.  
\begin{itemize}
  \item Widely implemented in machine learning and probabilistic modeling.
  \item But: relies on well-specified prior probabilities and likelihoods.
\end{itemize}

\vspace{0.5em} \textbf{Inference to the Best Explanation (IBE):}
Partially
\begin{itemize}
  \item AI can rank hypotheses based on fit, simplicity, etc.
  \item But: IBE involves theory choice, explanatory power, and background knowledge — often context-sensitive and informal.
  \item Current AI lacks deep semantic understanding or explanatory intuition.
\end{itemize}

\end{frame}

\begin{frame}{Can LLMs Perform Scientific Inference?}

\small
\textbf{LLMs (e.g., GPT-4, Claude):} Predict text based on patterns in large datasets.

\vspace{0.5em} \textbf{Bayesian Updating:} Not natively
\begin{itemize}
  \item LLMs are not probabilistic reasoners in the Bayesian sense.
  \item They can talk about Bayes’ rule, but don’t maintain internal probabilistic belief states.
\end{itemize}

\vspace{0.5em} \textbf{Inference to the Best Explanation (IBE):}
Superficial imitation
\begin{itemize}
  \item Can generate plausible-sounding explanations.
  \item Can rank hypotheses based on heuristics (e.g., coherence, simplicity) — if prompted.
  \item But lacks grounding in actual theory choice, background understanding, or causal modeling.
\end{itemize}

\vspace{0.5em} \textbf{Bottom line:} LLMs simulate explanatory
language — but do not \emph{genuinely} infer, explain, or understand.

\end{frame}

\begin{frame}{Toward an Intelligent Scientific Agent}

\small
\textbf{Comparison of AI Systems for Scientific Inference}

\vspace{0.5em}
\begin{tabular}{@{}lcc@{}}
\textbf{Capability} & \textbf{LLMs} & \textbf{Probabilistic Models} \\
\hline
Natural language fluency       & Yes & No \\
Bayesian updating              & No  & Yes \\
Causal explanation             & Partial (imitated) & Yes (if built-in) \\
Theory generation              & Partial (pattern-based) & Limited (depends on priors) \\
Hypothesis revision            & No  & Yes \\
Scientific judgment            & No  & Partial (domain-specific) \\
\end{tabular}

\vspace{1em}
\textbf{Hybrid efforts:}  
Ongoing research aims to combine LLMs with Bayesian, causal, and symbolic reasoning models.

\end{frame}

\begin{frame}{Can AGI Do Science Like a Human?}

\small
\textbf{Current AI systems (e.g., LLMs):}
\begin{itemize}
  \item Simulate scientific discourse and generate plausible hypotheses.
  \item Summarize theories, analyze data, complete analogies.
  \item Lack genuine belief representation, explanatory goals, and epistemic norms.
\end{itemize}

\vspace{0.5em}
\textbf{What human-like scientific reasoning requires:}
\begin{itemize}
  \item Formulating and revising hypotheses in light of evidence.
  \item Understanding causal mechanisms, not just correlations.
  \item Distinguishing relevance and explanatory depth.
  \item Participating in social and normative dimensions of inquiry.
\end{itemize}

\vspace{0.5em} \textbf{Conclusion:} Simulating science $\neq$ doing
science. AGI would need more than language fluency --- it must
integrate causal reasoning, epistemic goals, and reflective judgment.

\end{frame}




\begin{frame}

  \begin{itemize}
  \item A simpler case: automated theorem proving
    \begin{itemize}
    \item Even in pure mathematics (i.e.\ deductive logic), humans
      rely on intuitions about which proof strategies to pursue
    \end{itemize}   
  \item What about the role of scientists in making value judgments?
    \begin{itemize}
    \item When should an experiment be re-run?
    \item Inductive risk
    \end{itemize}
  \end{itemize}
  
\end{frame}
  
\begin{frame}

  \begin{itemize}
  \item After Carnap realized that logical reduction was too strict,
    he tried various other more liberal relationships (between new
    concepts and old)
    \begin{itemize}
    \item Partial reduction
    \item Implicit definition
    \item Ramsey sentences
    \end{itemize}
  \end{itemize}

\end{frame}

\begin{frame}

  \begin{itemize}
  \item Both camps (realist and antirealist) of late 20th century
    philosophers of science gave up on the project of understanding
    conceptual novelty
    \begin{itemize}
    \item Realists: new concepts are good when they latch onto the
      joints of reality
    \item van Fraassen: the aim is simply to build models that are
      empirically adequate
    \item Kuhn: the concepts of the new paradigm are often
      \ul{incommensurable} with those of the old paradigm
    \end{itemize}
  \item As a result, we have few plausible models of what conceptual
    innovation/growth might amount to
  \end{itemize}

\end{frame}

\begin{frame}{Conceptual engineering and mathematization}

  \begin{itemize}
  \item Conceptual advances in modern physics have often followed the
    development of new mathematical frameworks
    \begin{itemize}
    \item Einstein was able to treat gravity as a field by using the
      resources of Riemannian geometry
    \item Heisenberg was (apparently) able to overcome the
      contradictions of the old quantum theory by employing
      non-commutative matrix algebra
    \item The classification of fundamental particles was enabled by
      the theory of representations of Lie groups
    \end{itemize}
  \end{itemize}



\end{frame}

\begin{frame}{Extending concepts to new domains}

  \begin{itemize}
  \item Einstein: the concept of two events being simultaneous breaks
    down when velocities are high compared to the speed of light

  \end{itemize}


\end{frame}

\begin{frame}

  

  \begin{itemize} 
  \item With genuinely novel concepts, the framework for inquiry is
    changed --- and it is difficult to imagine how AI could do that
  \item Can AI only do ``normal science''? Or could AI make a
    revolutionary advance? What would that look like?
    \begin{itemize}
    \item Compare to description of how AI might try to explain
      something that is ``above our heads'' \end{itemize}
  \item Krenn et al.\: AI is good at ``rediscovery tasks'', but not
    much evidence yet that it can drive discoveries. \end{itemize}

\end{frame}


\begin{frame}[allowframebreaks]{References}
\nocite{*}
\scriptsize
\printbibliography
\end{frame}


\end{document}


%%% Local Variables:
%%% mode: latex
%%% TeX-master: t
%%% End:
