\documentclass[fleqn]{beamer}

\usetheme[
  progressbar=frametitle,    % or try: 'none', 'foot'
  numbering=fraction,        % frame number / total
  sectionpage=none,
  subsectionpage=none
]{metropolis}

\title{Week 4 \\ Part A \\ Scientific Understanding}
\author{Hans Halvorson}
\date{May 19, 2025}

\usepackage[backend=biber, style=authortitle]{biblatex}
\addbibresource{refs.bib}
\setlength{\bibitemsep}{0.5em}
\AtBeginBibliography{\scriptsize}

% Load math packages
\usepackage{amsmath,amssymb}

% Highlighting commands
\usepackage{soul}

% Optional: custom keyword formatting
\newcommand{\kw}[1]{\textbf{#1}}

% Only show section numbers in TOC
\setbeamertemplate{section in toc}[sections numbered]

% Section title slide at each new section
\AtBeginSection[]
{
  \begin{frame}
    \frametitle{Table of Contents}
    \tableofcontents[currentsection]
  \end{frame}
}

% Minimalist footer and navigation
\setbeamertemplate{footline}[frame number]
\setbeamertemplate{navigation symbols}{}

% Ensure numbered sections (useful with TOC)
\setcounter{secnumdepth}{1}

\begin{document}

\begin{frame}
  \titlepage
\end{frame}

\begin{frame}{Table of Contents}
\tableofcontents
\end{frame}

\section{Introduction}

\begin{frame}{De Regt and Dieks on Quantum Non-Locality}

\begin{columns}
% Left column: text
\column{0.6\textwidth}
\small
\begin{itemize}
\item Henk de Regt (Nijmegen) and Dennis Dieks (Utrecht) are
  philosophers of science from the Netherlands. They have made central
  contributions to discussions about the foundations of quantum
  mechanics.
  \item De Regt and Dieks have wrestled with the apparent unintelligibility of quantum non-locality.
  \begin{itemize}
    \item Some interpretations of QM try to make non-locality intelligible by appeal to superluminal causation.
  \end{itemize}
\end{itemize}

% Right column: images
\column{0.4\textwidth}
\centering
\includegraphics[width=0.5\linewidth]{deregt.jpg} \\
{\scriptsize Henk de Regt}

\vspace{1em}

\includegraphics[width=0.5\linewidth]{dieks.jpg} \\
{\scriptsize Dennis Dieks}

\end{columns}

\end{frame}


\begin{frame}

  \begin{itemize}
  \item De Regt and Dieks offer us a ``theory'' of what scientific
    understanding is --- or, at least, what its characteristic signs
    are
  \item Their account of understanding is a successor to several
    competing accounts of scientific explanation that were offered
    between 1960 and 2000
  \item Their account is intended to show that understanding is
    \textbf{epistemically relevant}
  \end{itemize}

\end{frame}

\section{What drives science?}

\begin{frame}{What Drives Science?}

\begin{columns}
% Left column: text
\column{0.6\textwidth}
\small
\textbf{Empiricism:} The goal of science is to predict the results of experiments.

\vspace{0.5em}
\begin{quote}
  “According to empiricists such as Hempel and van Fraassen, the
  epistemic aim of science is (roughly stated) the production of
  factual knowledge about natural phenomena.” (p. 141)
\end{quote}

\vspace{0.5em}
\begin{itemize}
  \item 1870–1950: Ernst Mach, Logical Positivism
  \item 1960– : Carl Hempel, Bas van Fraassen, Brad Wray
\end{itemize}

% Right column: image
\column{0.4\textwidth}
\centering
\includegraphics[width=0.9\linewidth]{image.jpg} \\
\vspace{0.5em}
{\scriptsize Bas van Fraassen,\\ \textit{The Scientific Image} (1980)}

\end{columns}

\end{frame}


\begin{frame}{What drives science?}

  \begin{itemize}
  \item \textbf{Scientific Realism:} Science aims to \textbf{explain}
    phenomena.
    \begin{itemize}
    \item The realist reaction to logical positivism has been dominant
      among philosophers in the anglo-american tradition since the
      1960s
    \end{itemize}
  \end{itemize}

\end{frame}

\begin{frame}

  \begin{itemize}
  \item Empiricists see the aim of science as knowing \textbf{that}
    while realists see the aim of science as knowing \textbf{why}
  \item Is understanding needed?
  \end{itemize}

\end{frame}

\begin{frame}{What drives science?}

  \begin{itemize}
  \item De Regt and Dieks: ``We will argue that achieving
    understanding is among the general (macro-level) aims of science''
    (p 140)
  \item ``Understanding is an inextricable element of the aims of
    science'' (p 142)
  \end{itemize}

\end{frame}


\section{Philosophy of science background}

\begin{frame}{Origins of logical positivism}

  \begin{columns}
  \begin{column}{0.55\textwidth}
    Gottlob Frege (1848--1925) was a German mathematician who argued
    for a strict separation of the (objective) logical from the
    (subjective) psychological
    \begin{itemize}
    \item He was a key player in formalizing the logical foundations
      of mathematics
    \item The validity of a ``inference'' is an objective fact,
      independent of any person who is thinking about it
    \end{itemize}
  \end{column}

\begin{column}{0.45\textwidth}
  \includegraphics[scale=0.15]{frege}
\end{column}
\end{columns}  

\end{frame}

\begin{frame}{Origins of logical positivism}
 \begin{columns}
  \begin{column}{0.55\textwidth}
    \begin{itemize}
    \item Rudolf Carnap (1891--1970) was a student of Frege, also
      trained in physics and philosophy
    \item Carnap's idea: apply Fregean logical rigor to the empirical
      sciences --- \textbf{logic of science program}  
    \end{itemize}
    \end{column}
    \begin{column}{0.35\textwidth}
      \includegraphics[scale=0.3]{carnap.pdf}
    \end{column}
   \end{columns}

\end{frame}

\begin{frame}{A scientific theory of science?}

  \begin{itemize}
  \item Carnap was concerned with talking \emph{about} science in a
    scientifically rigorous fashion
  \item Mathematical rigor: definable in some well-understood formal
    system
  \item At first, Carnap thought that not even ``truth'' qualified as
    a scientifically legitimate concept
  \item He never recognized ``explains'' as a scientifically
    legitimate concept
  \end{itemize}

\end{frame}

\begin{frame}{The post-war realist turn}

  ``To explain the phenomena in the world of our experience, to answer
  the question `why?' rather than only the question `what?', is one of
  the foremost objectives of all rational inquiry; and especially,
  scientific research in its various branches strives to go beyond a
  mere description of its subject matter by providing an explanation
  of the phenomena it investigates.'' (Hempel and Oppenheim 1948)

\end{frame}

\begin{frame}{Carl Hempel}

  \begin{columns}
    \begin{column}{0.6\textwidth}
  \begin{itemize}
  \item Carl Hempel (1905--1997) was a member of the Berlin Circle,
    and immigrated to the US in 1937
  \item Hempel: ``explains'' is a worldly relation that holds between
    facts, quite independent of the person (or group) of people
    considering them
  \item The goal of science is to find the (objective) explanation for
    the phenomena
  \end{itemize}
\end{column}

      \begin{column}{0.35\textwidth}
        \includegraphics[scale=0.4]{hempel}
      \end{column}
\end{columns}
  
\end{frame}


\begin{frame}

  \begin{columns}
  \begin{column}{0.6\textwidth}
    ``Such expressions as `realm of understanding' and
    `comprehensible' do not belong to the vocabulary of logic, for
    they refer to the psychological and pragmatic aspects of
    explanation.'' (Carl Hempel)
  \end{column}

  \begin{column}{0.4\textwidth}
    \includegraphics[scale=2.5]{aspects}
  \end{column}
  \end{columns}

\end{frame}

\begin{frame} 

  ``Carl Hempel \dots argued that `understanding' is subjective and
  merely a psychological by-product of scientific activity and is,
  therefore, not relevant for the philosophy of science.''  (Krenn et
  al., p 762)

\end{frame}

\section{Theories of scientific explanation}

\begin{frame}{Major Accounts of Scientific Explanation (Historical Overview)}

\scriptsize
\begin{itemize}
  \item \textbf{Hempel (1940s–60s)}: Deductive-Nomological Model
    \begin{itemize}
      \item Explanation = logical deduction from laws + initial conditions
      \item Criticized for overgenerating (irrelevance) and symmetry
    \end{itemize}

  \item \textbf{Salmon (1970s–80s)}: Causal Models
    \begin{itemize}
      \item Statistical relevance $\Rightarrow$ causal relevance
      \item Explanation = tracing causal/mechanical processes
    \end{itemize}

  \item \textbf{Friedman \& Kitcher (1970s–80s)}: Unificationist Accounts
    \begin{itemize}
      \item Explanation = increased understanding via unification
      \item Fewer independent assumptions; general argument patterns
    \end{itemize}
\end{itemize}

\end{frame}




\begin{frame}{Deductive nomological account}

  Nomos = law

  \medskip Hempel proposed a general schema according to which a fact
  $E$ (the explanandum) is explained by being deduced logically from a
  covering law $L$ and an initial condition $C$ (the explanans).

  \vfill \begin{tabular}{ll}
  Initial condition \\
  Law \\ \hline
  Explanandum (to be explained)
  \end{tabular}

\end{frame}

\begin{frame}{DN both over- and undergenerates}

  \begin{itemize}
  \item DN undergenerates: There are legitimate scientific
    explanations that do not match the strict, DN format
  \item Hempel nuanced the DN account to include statistical
    explanations
  \item DN overgenerates: There are pseudo-explanations that match the
    strict DN format
    \begin{itemize}
    \item Asymmetry: Flagpole
    \item Relevance: Birth control pills
    \end{itemize}
  \end{itemize}

\end{frame}

\begin{frame}{Flagpole and Shadow: A Problem for Hempel}

\begin{columns}
\column{0.55\textwidth}

\scriptsize
\vspace{-0.5em}
\textbf{D-N Explanation (Hempel):}
\begin{itemize}
  \item Given: flagpole height $h$, sun angle $\theta$
  \item Law: $\tan(\theta) = \frac{h}{s}$
  \item Deduce: shadow length $s$
\end{itemize}

\vspace{0.5em}
\textbf{But also:}
\begin{itemize}
  \item Given $s$ and $\theta$, deduce $h$
\end{itemize}

\vspace{0.5em}
\textbf{What's wrong?}
\begin{itemize}
  \item Both directions are deductive...
  \item ...but only one feels explanatory.
\end{itemize}

\column{0.45\textwidth}

\begin{tikzpicture}[scale=1.2, every node/.style={font=\scriptsize}]
  % Ground
  \draw[thick] (0,0) -- (3.5,0);

  % Flagpole
  \draw[very thick] (0,0) -- (0,3);
  \node[left] at (0,1.5) {$h$};

  % Shadow
  \draw[thick] (0,0) -- (3,0);
  \node[below] at (1.5,0) {$s$};

  % Sun rays and triangle
  \draw[dashed] (0,3) -- (3,0);
  \node at (2.6,1.6) {$\theta$};

  % Right angle box
  \draw (0,0) rectangle +(0.2,0.2);
\end{tikzpicture}

\end{columns}

\end{frame}


\begin{frame}{Hempel's D-N Model Overgenerates}

\vspace{-0.5em}
\textbf{Premises (Laws + Initial Conditions):}
\begin{itemize}
  \item[(L)] All males who take birth control pills regularly fail to get pregnant.
  \item[(C)] John Jones is a male who has been taking birth control pills regularly.
\end{itemize}

\vspace{0.5em}
\textbf{Conclusion (E):}
\begin{itemize}
  \item[(E)] John Jones fails to get pregnant.
\end{itemize}

\vspace{0.5em}
\textbf{Problem:}
\begin{itemize}
  \item The derivation is logically valid under the D-N model.
  \item But it fails to identify a genuine \textit{explanatory} relation.
  \item Taking birth control pills is irrelevant to John's failure to become pregnant.
\end{itemize}

\vspace{0.5em} \textbf{Conclusion:} D-N explanation captures logical
derivability, but not explanatory relevance.

\end{frame}



\begin{frame}

  \begin{columns}
    \begin{column}{0.6\textwidth}
      \begin{itemize}
      \item For more than forty years, philosophers debated about the
        ``right'' account of scientific explanation.
      \item ``Newton-Smith (2000) observes that fifty years of
        discussion have not led to consensus, but, on the contrary, to
        many rival models of explanation.''
      \end{itemize}
    \end{column}
    \begin{column}{0.4\textwidth}
      \includegraphics[scale=0.15]{decades}
    \end{column}
  \end{columns}


\end{frame}

\begin{frame}{Causal-mechanical account}

  \begin{itemize}
  \item A natural addendum to the DN account of explanation would be
    to require that the explanans is \textbf{causally relevant} to the
    explanandum.
  \item Wesley Salmon developed the idea that a scientific explanation
    is a description of the causal mechanism that results in the
    production of the phenomenon.  \end{itemize}

\end{frame}

\begin{frame}{Causal-mechanical account}

  \begin{itemize}
  \item The causal-mechanical account taps into the old tradition of
    mechanistic explanation (and visualizability)
  \end{itemize}



\end{frame}

\begin{frame}{Unificationist account}

  ``Science advances our understanding of nature by showing us how to
  derive descriptions of many phenomena, using the same patterns of
  derivation again and again, and, in demonstrating this, it teaches
  us how to reduce the number of types of facts we have to accept as
  ultimate (or brute).'' (Philip Kitcher)

\end{frame}

\begin{frame}{Stalemate}

  \begin{itemize}
  \item Proved beyond difficult to isolate the core idea of
    \textbf{explanation} that holds throughout all the different
    sciences
  \item The methodology of ``whatever examples I can remember from
    when I was a physics student'' was recognized as unacceptable
  \end{itemize}

\end{frame}

\section{Resurrecting ``understanding''}

\begin{frame}{Against the old critique of understanding}

  \begin{itemize}
  \item ``The present paper argues, \emph{pace} Newton-Smith, that
    understanding can play the desired unifying role.'' (p 137)
  \item ``Should we rely on the view of practicing scientists...?'' (p
    138)
  \end{itemize}

\end{frame}
\begin{frame}

  \begin{itemize}    
  \item De Regt and Dieks argue that understanding is an
    \textbf{epistemically relevant} concept.  \newline ``We will argue
    that understanding ... is epistemically relevant'' (p 138)
  \item De Regt and Dieks argue that understanding transcends
    individual psychology \newline ``We will argue that understanding
    ...  transcends the domain of individual psychology.'' (p. 138)
  \end{itemize}

\end{frame}

\begin{frame}{Understanding is contextual}

  \begin{itemize}
  \item HH: A phenomenon can be \ul{contextual} and yet be
    second-order objective
  \item For example, it is second-order objective that ``Oslo is less
    than 500km from us''
  \item HH: de Regt and Dieks have not yet shown the sense in which
    understanding or intelligibility is second-order objective
  \end{itemize}

\end{frame}

\nocite{*}

\begin{frame}[allowframebreaks]{Selected References}
\printbibliography[heading=none, keyword=scientific-explanation]
\end{frame}


\end{document}


%%% Local Variables:
%%% mode: latex
%%% TeX-master: t
%%% End:
