\documentclass[fleqn]{beamer}
\usetheme{metropolis}
\usepackage[utf8]{inputenc}
\usepackage[T1]{fontenc}
\usepackage{amsmath, amssymb}
\usepackage{graphicx}
\usepackage{hyperref}
\usepackage{amsfonts}
\usepackage{tikz-cd}
\usepackage{fontawesome5} % for map marker icon



\usepackage{tikz}
\usetikzlibrary{decorations.pathreplacing, arrows.meta, positioning}

\usepackage{minted}
\usemintedstyle{tango} 

\title{Niels Bohr on Causal Explanation}
\subtitle{}
\author{Hans Halvorson}
\institute{Princeton University}
\date{August 12, 2025}

\usepackage{doi}
\usepackage{amssymb}
\usepackage[backend=biber,natbib=true,style=authoryear]{biblatex}
\addbibresource{~/drafts/master.bib}



\begin{document}

\begin{frame}
   \titlepage
 \end{frame}

\begin{frame}{Introduction}

  Bertrand Russell thought that causation was a folk concept that had
  no place in a fundamental description of the world.

  \medskip \begin{quote} The law of causality, I believe, like much
    that passes muster among philosophers, is a relic of a bygone age,
    surviving, like the monarchy, only because it is erroneously
    supposed to do no harm. \citep[p 12]{russell1913} \end{quote}
    
\end{frame}

\begin{frame}{Introduction}

  Niels Bohr thought that physics provides occasion to clarify and
  sharpen our most fundamental concepts.

  \medskip \begin{quote} The significance of physical science for
    philosophy does not merely lie in the steady increase of our
    experience of inanimate matter, but above all in the opportunity
    of testing the foundation and scope of some of our most elementary
    concepts. \citep[p 308]{qphil1958} \end{quote}


\end{frame}

\section{Philosophical background to Bohr's stance}

\begin{frame}{Rationalism versus Empiricism}

  \begin{itemize}
  \item The Law of Causality = The Principle of Sufficient Reason:
    Nothing happens without a cause 
  \item David Hume (1711--1776) argued that ``$C$ causes $E$'' cannot
    mean anything more than events of type $E$ have tended, in our
    experience, to be followed by events of type $E$.
  \item Immanuel Kant (1724--1804) argued that the PSR is valid in the
    realm of experience.
  \end{itemize}

\end{frame}

\begin{frame}{Høffding on causality}

  \begin{itemize}
   \item Hume is mistaken in thinking that ``thing'' is unproblematic
    while ``cause'' is problematic.
  \item It's impossible to prove or disprove the law of causality.
  \item Regulative ideal
  \end{itemize}

\end{frame}  

\begin{frame}{Peculiarities of Bohr's view}

  \begin{itemize}
  \item Continuity
  \item Conservation laws
  \item Causal concepts are indispensable for the human experience of
    the world. (Høffding had pointed out that our concepts evolved
    under pressure to act.)
  \end{itemize}

\end{frame}

\begin{frame}{Growing pains for causality}

  \begin{itemize}
  \item Rutherford's worry about Bohr's atomic model
  \item BKS
  \item Beta decay
  \end{itemize}

\end{frame}

\begin{frame}{Causality as a heuristic for discovery}



\end{frame}

\begin{frame}{Complementarity as a generalization of the principle of
    causality}

  \begin{itemize}
  \item Bohr: The discovery of QM settles (for now) the validity of
    the conservation principles.
  \item Several of Bohr's later articles focus on the concept of
    causality.
  \end{itemize}

\end{frame}

\begin{frame}

  \begin{quote} Space time co-ordination and dynamical conservation
    laws may be considered as \emph{two complementary aspects of
      ordinary causality} which in this field exclude one another to a
    certain extent, although neither of them has lost its intrinsic
    validity. \citep[p 376]{faraday1932}. \end{quote}


\end{frame}

\begin{frame}{Causality as an idealization}

  \begin{itemize}
  \item No system is truly closed (cf. Høffding); no particle is truly
    free from all external forces.
  \end{itemize}

    


\end{frame}

\begin{frame}[allowframebreaks]{References}

\printbibliography[heading=none]

\end{frame}


\end{document}

