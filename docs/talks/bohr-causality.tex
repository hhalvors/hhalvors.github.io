\documentclass[fleqn]{beamer}
\usetheme{metropolis}
\usepackage[utf8]{inputenc}
\usepackage[T1]{fontenc}
\usepackage{amsmath, amssymb}
\usepackage{graphicx}
\usepackage{hyperref}
\usepackage{amsfonts}

\title{Niels Bohr on Causal Explanation}
\subtitle{}
\author{Hans Halvorson}
\institute{Princeton University \& Niels Bohr Archive}
\date{August 12, 2025}

\usepackage{doi}
\usepackage{amssymb}
\usepackage[backend=biber,natbib=true,style=authoryear]{biblatex}
\addbibresource{~/drafts/master.bib}



\begin{document}

\begin{frame}
   \titlepage
 \end{frame}

\begin{frame}{Introduction}

  Bertrand Russell thought that causation is a folk [anthropomorphic]
  concept that has no place in a fundamental description of the world.

  \medskip \begin{quote} The law of causality, I believe, like much
    that passes muster among philosophers, is a relic of a bygone age,
    surviving, like the monarchy, only because it is erroneously
    supposed to do no harm. \citep[p 12]{russell1913} \end{quote}
    
\end{frame}

\begin{frame}{Introduction}

  Niels Bohr thought that physics provides occasion to clarify and
  sharpen our most fundamental concepts.

  \medskip \begin{quote} The significance of physical science for
    philosophy does not merely lie in the steady increase of our
    experience of inanimate matter, but above all in the opportunity
    of testing the foundation and scope of some of our most elementary
    concepts. \citep[p 308]{qphil1958} \end{quote}


\end{frame}

\section{Philosophical background to Bohr's stance}

\begin{frame}{Rationalism versus Empiricism}

  \begin{itemize}
  \item The Law of Causality = The Principle of Sufficient Reason:
    Nothing happens without a cause 
  \item Hume (1711--1776) argued that ``$C$ causes $E$'' cannot mean
    anything more than that events of type $C$ have tended, in our
    experience, to be followed by events of type $E$.
  \item Kant (1724--1804) argued that the PSR is valid in the realm of
    experience.
  \end{itemize}

\end{frame}

\begin{frame}{Harald Høffding on causality}

  \begin{itemize}
  \item Hume is mistaken to assume that ``thing'' is unproblematic,
    while ``cause'' is problematic.

     \medskip \begin{quote} The concept of a thing cannot be clear or
       valid, when the concept of cause is not. \citep[p
       238]{hoff} \end{quote}
   \item It's impossible to prove or disprove the law of causality.

     \medskip \begin{quote} Experience can never provide a full
       confirmation of the Law of Causality. \citep[p
       241]{hoff} \end{quote}
  \item Methodological principle

    \medskip \begin{quote} The Principle of the Conservation of Energy
      has the significance of a methodological principle which pushes
      us to look for equivalents for each quantum of matter or energy
      that seems to appear or disappear. \citep[p
      36]{hoff} \end{quote}
    
  \end{itemize}

  

\end{frame}

\begin{frame}{Harald Høffding on causality}

  \begin{itemize} \item Our minds attempt to find continuity among
    events

    \medskip \begin{quote} In every circumstance, we seek to conceive
      of that which happens as a continuous process, whose first and
      last elements we call cause and effect. The concept of causality
      is the expression for this striving. \citep[p
      240]{hoff} \end{quote} \end{itemize}



\end{frame}

\begin{frame}{Peculiarities of Bohr's views about causality}

  \begin{itemize}

  \item Conservation laws

    \medskip \begin{quote} A careful reading of Bohr’s discussions of
      the notion of complementarity reveals that in this context he
      takes the applicability of a causal description to mean the
      applicability of the laws of conservation of energy and
      momentum. \citep[p 313]{bensch} \end{quote}
  \item Indispensable for the human experience of the world.

    \medskip \begin{quote} All account of physical experience is, of
      course, ultimately based on common language, adapted to
      orientation in our surroundings and to tracing relationships
      between cause and effect. \citep[p 308]{qphil1958}
    \end{quote}

  \end{itemize}

\end{frame}

\section{The quantum challenge for causality}

\begin{frame}{Bohr's doubts about causality}

  \begin{quote} The unrestricted applicability of the causal mode of
    description to physical phenomena has hardly been seriously
    questioned until Planck's discovery of the quantum of
    action. \citep[p 11]{causal-problem} \end{quote}

  \bigskip \begin{quote} The interesting arguments brought forward
    more recently by Einstein \dots rather than supporting the theory
    of light quanta will seem to bring the legitimacy of a direct
    applcation of the theorems of conservation of energy and momentum
    to the radiation process in doubt. \citep[p
    413]{bohr1921} \end{quote}

\end{frame}

\begin{frame}{Bohr's doubts about causality}

  \begin{quote} As a result of the previous considerations, a general
    description of the phenomena, in which the laws of the
    conservation of energy and momentum retain in detail their
    validity in their classical formulation, cannot be carried
    through. \citep[p 40]{bohr1924} \end{quote}

  \begin{itemize}
  \item Bohr-Kramers-Slater theory
\item Beta decay
\end{itemize}



\end{frame}

\begin{frame}{Complementarity as a generalization of the principle of
    causality}

  \begin{itemize}
  \item Bohr: The discovery of QM settles (for now) the validity of
    the conservation principles.

    \medskip \begin{quote} The establishment of rational methods of
      quantum mechanics and electrodynamics [has] proved the
      compatibility of the existence of the quantum of action with the
      strict validity of the conservation laws in all such phenomena
      as electron diffraction and Compton effect. \citep[p
      25]{conservation1936} \end{quote}
  \item Several of Bohr's later articles focus on the concept of
    causality.
  \end{itemize}

\end{frame}

\begin{frame}

  \begin{quote} With the foregoing analysis we have described the new
    point of view brought forward by the quantum theory. Sometimes one
    has described it as leaving aside the idea of causality. I think
    we should rather say that in the quantum theory we try to express
    some laws of nature which lie so deep that they cannot be
    visualised, or, which cannot be accounted for by the usual
    description in terms of motion. \citep{continuity1931} \end{quote}


\end{frame}

\begin{frame}

  \begin{quote}
    Every application of conservation theorems, for instance to the
    energy balance in atomic reactions, involves an essential
    renunciation as regards the pursuance in space and time of the
    individual atomic particles.  In other words, the use of the idea
    of stationary states stands in a mutually exclusive relationship
    to the applicability of space-time pictures. \citep[p
    375]{faraday1932} \end{quote}


\end{frame}

\begin{frame}

  \begin{quote} Space time co-ordination and dynamical conservation
    laws may be considered as \emph{two complementary aspects of
      ordinary causality} which in this field exclude one another to a
    certain extent, although neither of them has lost its intrinsic
    validity. \citep[p 376]{faraday1932}. \end{quote}


\end{frame}

\begin{frame}{Causality as an idealization}

  No system is truly closed; no particle is truly free from all
  external forces.

  \medskip \begin{quote} We are not acquainted with any absolutely
    isolated and closed totalities; and only for such is the Law of
    Conservation of Energy valid in the strictest sense. \citep[p
    36]{hoff} \end{quote}

  \medskip \begin{quote} The very nature of quantum theory thus forces
    us to regard the space-time coordination and the claim of
    causality \dots as complementary but exclusive features of the
    description, symbolizing the idealization of observation and
    definition respectively. \citep[54]{bohr1927} \end{quote}

\end{frame}

\section{Summary}

\begin{frame}{Summary}

  \begin{quote} Far from containing any mysticism contrary to the
    spirit of science, the view-point of `complementarity' forms
    indeed a consistent generalization of the ideal of
    causality. \citep[p 269]{natural1939} \end{quote}


\end{frame}

\begin{frame}{Summary}

  \begin{quote} \dots far from involving any arbitrary renunciation of
    the ideal of causality, the wider frame of complementarity
    directly expresses our position as regards the account of
    fundamental properties of matter presupposed in classical physical
    description, but outside its scope. \citep[p
    314]{qphil1958} \end{quote}

\end{frame}


\begin{frame}[allowframebreaks]{References}

  \nocite{faye1979}

\printbibliography[heading=none]

\end{frame}


\end{document}


%%% Local Variables:
%%% mode: latex
%%% TeX-master: t
%%% End:
