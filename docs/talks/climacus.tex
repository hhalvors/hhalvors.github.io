\documentclass[12pt]{beamer}
\usetheme{metropolis}
\usepackage{amsmath}
\usepackage{hyperref}
\usepackage{graphicx}

\title{Climacus' Two Ways of Reflection}
\author{Hans Halvorson}
\date{May 16, 2025}

\begin{document}

\maketitle

\section{Motivation and Setup}

\begin{frame}{Overview}
\begin{itemize}
  \item What kind of work is \emph{Concluding Unscientific Postscript}?
  \item Climacus’ picture of “two ways of reflection”
  \item Relevance to scientific objectivity and rational inquiry
\end{itemize}
\end{frame}

\begin{frame}{The Postscript as Epistemological Intervention}
\begin{itemize}
  \item Not traditional metaphysics, epistemology, or ethics
  \item Aimed to dislodge the Hegelian mindset
  \item Epistemic self-understanding as central concern
\end{itemize}
\begin{block}{Climacus}
“The speculative thinker forgets that he exists, and proceeds to define existence.” (p. 173)
\end{block}
\end{frame}

\section{The Two Ways}

\begin{frame}{Climacus’ Distinction}
\begin{block}{Two Ways of Reflection}
“Two ways, in general, are open for an existing individual: either he can do his utmost to forget that he is an existing individual, or he can concentrate his entire energy upon the fact that he is an existing individual.” (p. 109)
\end{block}
\begin{itemize}
  \item Objective way: abstract, detached, impersonal
  \item Subjective way: passionate, situated, decision-bound
\end{itemize}
\end{frame}

\begin{frame}{Misreading Kierkegaard}
\begin{itemize}
  \item Robert Adams: defines objective reasoning as what most fair-minded people would accept
  \item But this misrepresents Kierkegaard’s concern
  \item K. critiques a false ideal of impersonal, godlike epistemic position
\end{itemize}
\begin{block}{Adams (1977)}
“Let us say that a piece of reasoning is ‘objective’ just in case...”
\end{block}
\end{frame}

\section{The Objective Way}

\begin{frame}{Features of the Objective Way}
\begin{itemize}
  \item Characterized by disinterested inquiry and abstraction
  \item Seeks timeless, impersonal truth
  \item Tends to erase the finite knower
\end{itemize}
\begin{block}{Climacus}
“From an objective standpoint, Christianity is a \emph{res in facto posita}, whose truth it is proposed to investigate in a purely objective manner.” (p. 29)
\end{block}
\end{frame}

\begin{frame}{The Comic Ideal of Objectivity}
\begin{block}{Climacus}
“At its maximum this [objective] way will lead to the contradiction that only the objective has come into being, while the subjective has gone out.” (p. 173)
\end{block}
\begin{itemize}
  \item The illusion of pure objectivity becomes comic
  \item Objective ideal ignores that the inquirer is changing while inquiring
\end{itemize}
\end{frame}

\begin{frame}{Objectivity as Detachment}
\begin{block}{Climacus}
“...they are still not infinitely, personally, impassionedly interested. On the contrary, they would even rather not be so. Their observations are to be objective, disinterested.” (Hannay, p. 20)
\end{block}
\begin{itemize}
  \item Climacus critiques disinterestedness as an ideal of scholarly detachment
  \item Real inquiry demands existential investment
\end{itemize}
\end{frame}

\section{The Subjective Way}

\begin{frame}{The Subjective Turn}
\begin{itemize}
  \item K. recommends the subjective way—but not as fideism
  \item Subjective way demands passionate commitment and decision
  \item Objective inquiry can inform, but not replace, decision
\end{itemize}
\begin{block}{Climacus}
“The existing individual who chooses the subjective way apprehends instantly the entire dialectical difficulty...” (p. 178)
\end{block}
\end{frame}

\begin{frame}{Subjectivity and Risk}
\begin{itemize}
  \item To exist is to be becoming—so subjectivity is inescapable
  \item Objective ideal fails to accommodate the risk of existence
\end{itemize}
\begin{block}{Climacus}
“Since man is a synthesis of the temporal and the eternal, the happiness that the speculative philosopher may enjoy will be an illusion...” (p. 54)
\end{block}
\end{frame}

\section{Complementarity and Conflict}

\begin{frame}{False Dilemma?}
\begin{itemize}
  \item Why not combine objectivity and subjectivity?
  \item Climacus says they are complementary, not additive
\end{itemize}
\begin{block}{Climacus}
“The more objective the contemplative inquirer, the less he bases an eternal happiness... since there can be no question of an eternal happiness except for the passionately and infinitely interested subject.” (p. 33)
\end{block}
\end{frame}

\begin{frame}{Complementarity as Epistemological Structure}
\begin{itemize}
  \item K.’s model: knower must be in a definite mode to know
  \item Objective indifference may block access to existential truths
\end{itemize}
\begin{block}{Climacus}
“In the case of a kind of observation where it is requisite that the observer should be in a specific condition, it naturally follows that if he is not in this condition, he will observe nothing.” (p. 51)
\end{block}
\end{frame}

\section{Implications}

\begin{frame}{Impact on Philosophy of Science}
\begin{itemize}
  \item Objectivity is a noble ideal, but must be humanized
  \item Inquiry is always agent-relative
  \item Bohr and van Fraassen take this lesson seriously
\end{itemize}
\begin{block}{Bohr}
“The epistemological lesson of quantum physics is that the observer must choose between complementary modes of description.”
\end{block}
\end{frame}

\begin{frame}{Beyond Hegel and Quine}
\begin{itemize}
  \item Hegel: the inner is the outer, subject-object collapse
  \item Quine: first-person discourse eliminable in favor of third-person
  \item Kierkegaard resists both: insists on irreducibility of subjective mode
\end{itemize}
\end{frame}

\section{Conclusion}

\begin{frame}{Final Reflections}
\begin{itemize}
  \item Climacus’ “two ways” model is existential, not metaphysical
  \item The critique is not anti-reason, but anti-hubris
  \item Epistemic self-understanding is the true task
\end{itemize}
\end{frame}


\end{document}

%%% Local Variables:
%%% mode: latex
%%% TeX-master: t
%%% End:
