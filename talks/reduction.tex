%% TO DO

%% Read Mark Johnston

%% Unity of science -- not willing to give up on reduction!

%% Nagelian reduction - Frigg Hartmann

%% relation of "more or less real" ... maybe some relation to recent
%% metaphysics, Shaffer sort of view that everything exists

%% is there a reduction relation for physical theories that is
%% different in kind than that for mathematical theories?

%% n. what is wrong with reducibility? 



\documentclass[fleqn]{beamer}
\usetheme{metropolis}
\usepackage[utf8]{inputenc}
\usepackage[T1]{fontenc}
\usepackage{amsmath, amssymb}
\usepackage{graphicx}
\usepackage{hyperref}
\title{Reduction Redux}
\subtitle{}
\author{Hans Halvorson}
\institute{Princeton University}
\date{April 4, 2025}

\usepackage[style=authoryear, backend=biber, natbib=true, doi=true]{biblatex} % Chicago style with Biber backend
\addbibresource{redux.bib}

\usepackage{tikz}
\usetikzlibrary{arrows.meta, positioning}

\begin{document}


\begin{frame}
  \titlepage
\end{frame}

\begin{frame}{Prescript}

  \small 

  This topic is too big to be dealt with in a one hour lecture,
  especially because I intend to cause trouble for the dominant
  picture, and to replace it with something quite different. The
  dominant picture is that reality (or our description of it) is like
  a tower, with microphysics at the bottom, and where ``reduction''
  breaks the symmetry of the spatial metaphor. (If $B$ reduces to $A$,
  it's as if $A$ bears weight of $B$.)

  The first stage of this project is to try to find an explication of
  ``$B$ reduces to $A$''. In this talk, I point out that ``$B$ is a
  Morita extension of $A$'' is a useful generalization of classical,
  Nagelian reduction.

  We could go on and do lots of other fancy formal work, e.g.\ with
  category theory. But I don't think it will help the dominant
  picture. Scientific theories form an eco-system, not a tower.
  
\end{frame}

% Add your content frames here
\begin{frame}{Outline}
  \tableofcontents
\end{frame}

% Example section



\section{Reducible or not: Who cares?}

\begin{frame}{Reduction as religious stance}

  \begin{enumerate}
  \item The church of reduction
    \begin{itemize}
    \item Lewis (Albert, Loewer, Butterfield)
    \item Dennett (Wallace)
    \end{itemize}
  \item The church of anti-reduction
    \begin{itemize}
    \item Chalmers
    \item Thomas Nagel  
    \item Ellis, Drossel, Koons, Simpson
    \end{itemize}
  \item Non-practicing reductionists
    \begin{itemize}
    \item Davidson 
    \end{itemize}
  \end{enumerate}

\end{frame}

%% Quantum -- more generally, state of play in physics

\begin{frame}{Where reduction pops up}

  \begin{itemize}
  \item within physics
    \begin{itemize}
    \item thermodyanmics to statistical mechanics
    \item classical world from quantum 
    \end{itemize}
  \item between (empirical) sciences
  \item manifest image to scientific image
    \begin{itemize}
    \item mind to brain
    \end{itemize}
  \item within mathematics
  \end{itemize}

\end{frame}  


\section{The forward march of physics?}

\begin{frame}{Success?}

  \begin{itemize}
  \item The mechanical worldview: Galileo, Descartes, etc.
    \begin{itemize}
    \item Primary and secondary qualities
    \end{itemize}
  \item Spacetime
  \item Reduction of thermodynamics to statistical mechanics?
  \end{itemize}

\end{frame}


\begin{frame}{Trouble on the horizon?}

  \begin{itemize}
  \item The reversibility problem in thermodynamics  
  \item It is \emph{not} as if the \emph{user} of quantum mechanics
    has any immediate reason to believe that he has run up against a
    non-material element of reality.
  \item So if QM poses a problem for physicalism, it is a different
    sort of problem then the one encountered in psychology or biology.
  \item It is the \emph{meta-theory} of QM that causes trouble: it
    doesn't seem possible to use QM to describe someone using QM to
    describe reality.
  \end{itemize}

\end{frame}

\begin{frame}

  This last problem is often called ``the measurement problem''. But I
  prefer to frame it as ``the description problem'', as it crosses
  over into the space of reasons.

  The problem only arises if one thinks of the first theorist as able
  to stand in ``aboutness'' relations to a state of affairs.

\end{frame}

\begin{frame}{Two kinds of solutions}

  \begin{itemize}
  \item Mono solutions
    \begin{itemize}
    \item Bohm
    \item Everett
    \end{itemize}  
  \item Duo solutions
  \begin{itemize}  
  \item Heisenberg cut --- old and much derided, e.g.\ by Bell
  \item Quantum hylomorphism --- new
  \end{itemize}
\end{itemize}

\end{frame}

\begin{frame}{The religion in reduction}

  \begin{itemize}
  \item This debate reminds me of philosophy of religion with its
    countless arguments for and against God's existence.
  \item After surveying a dozen or so of these (inconclusive and
    conflicting) arguments, the returns start to diminish (for me at
    least).
  \item Let's seek insight rather than an existential archimedean
    point. \end{itemize}

\end{frame}


\section{Explications of reduction}
%% Logic

\begin{frame}{From emotion to analysis}

  \begin{itemize}
  \item The previous one-hundred years in philosophy have been
    genuinely different --- because of its connection with symbolic
    logic and mathematics.
  \item Carnapian explication: where we take a vague concept or thesis
    and provide a (mathematically) precise counterpart
  \item Example: inference rules for classical predicate
    calculus
  \end{itemize}

\end{frame}

\begin{frame}{Preliminaries}

  \begin{itemize}
  \item What kinds of things stand in the ``reduction'' relation?
  \item I will take the relata to be \textbf{theories}, and I will
    take theories to be represented by pairs $L,T$, where $L$ is a
    language and $T$ is a set of $L$-sentences.
  \item I do \emph{not} assume that ``same theory'' means $L=L'$ and
    $T=T'$.
  \end{itemize}
\end{frame}

\begin{frame}{What reduction could not be?}

  \begin{itemize}
  \item Just as some have complained that equivalence (of theories) is
    not a purely formal matter, they might also complain that
    reduction is not a purely formal matter.
  \item Is reduction a ``worldly relation''?
  \end{itemize}  
\end{frame}

\begin{frame}{Criteria for success}

  \begin{itemize}
  \item What kind of understanding will such an account give?
  \item What kind of answers will such an account give?
  \item Is it a factual question whether one theory is reducible to
    another?
  \end{itemize}

\end{frame}

\begin{frame}{Classical reduction}

  \textbf{Definition:} $T$ explicitly defines $R$ in terms of
  $\Sigma _0$ just in case:
  \[ T\:\vdash \: \forall x(R(x)\leftrightarrow \theta (x)) \]

  \textbf{Proposition:} For every $\Sigma$-formula $\varphi (x)$,
  there is a $\Sigma _0$-formula $\varphi ^*(x)$ such that
  $T\vdash \forall x(\varphi (x)\leftrightarrow \varphi ^* (x))$.

\end{frame}

\begin{frame}{Identity theory}

  \begin{itemize}
  \item The \textbf{identity theory} is stronger than explicit
    definition.
  \item Hence, if explicit definition is too restrictive, so is the
    identity theory.
  \end{itemize}

\end{frame}

\begin{frame}{Challenges to explicit definability}

  \begin{itemize}
  \item Ramsey: ``[I]f we proceed by explicit definition we cannot add
    to our theory without changing the definitions, and so the meaning
    of the whole.''
  \item Carnap: ``It is, in general, not possible to give explicit
    definitions for theoretical terms on the basis of $\Sigma _0$.''
    \citeyearpar[p 42]{carnap}
  \item Putnam: Multiple realizability
  \item Fodor: The special sciences
  \item Jackson: Explanatory gap
  \end{itemize}

\end{frame}

\begin{frame}

  \begin{itemize}
  \item I'm not primarily interested in whether mind is reducible to
    brain.
  \item I want to know what kind of thing \emph{reduction} is, and
    whether \emph{classical reduction} admits of fruitful
    generalization.
  \end{itemize}

\end{frame}




\begin{frame}{Open the semantic floodgates!}

  %% Canberra plan

  \begin{itemize}
  \item 1970s: Buzzwords about relations binding the mental to the
    physical: determination, supervenience, functionalism, etc.
  \item ca.  1965--2020 common wisdom: you want some kind of root in
    the physicalist basis (determination, supervenience, emergence,
    \dots ), but \emph{not} reducibility.
  \item These philosophers \textbf{failed} to enable interdisciplinary
    discussion
  \item \textbf{Functional definition} is one of the more promising
    ideas. (See especially \cite{lewis1970,lewis1972})
    \begin{itemize}
    \item Popular in recent philosophy of science, e.g.\ spacetime
      functionalism \citep[see][]{butterfield2023}
    \end{itemize}
  \end{itemize}


\end{frame}

\begin{frame}

  ``Accounts of inter-theoretic reduction differ between the
  language-first and math-first views of theories in quite similar
  ways to their respective accounts of theoretical equivalence.''
  \citep[p 356]{lang}

  ``On the math-first view, reduction is something like instantiation:
  the realizing by some sub-structure of the low-level theory's models
  of the structure of the higher-level theory's models.'' \citep[p
  357]{lang}

\end{frame}  

\begin{frame}{Semantic strategies}

  \begin{itemize}
  \item After the invention of model theory, formal philosophy fell
    into confusion about what could be achieved by semantic methods.
  \item Many philosophers claimed that old problems were artifacts of
    the syntactic approach.
  \item The confusion was increased by an ambiguity about whether
    semantics is about mathematical models or about concrete
    existents.
  \end{itemize}

\end{frame}

\begin{frame}

  \begin{itemize}
  \item Supervenience: No change to the new objects without a change
    in the old objects.
  \item Any elementary embedding $h:M|_{\Sigma _0}\to N|_{\Sigma _0}$
    lifts uniquely to an elementary embedding $\overline{h}:M\to N$.
  \end{itemize}


\end{frame}

\begin{frame}{From semantics to syntax}

  \begin{itemize}
  \item \citet{hellman} point out that supervenience (determination)
    is just implicit definition, and it implies (via Beth's theorem)
    explicit definition. But they claim that the result is irrelevant
    since they aren't talking about models of a theory!
  \item \citet{bealer} argues that functional definition is just
    implicit definition and so implies explicit definition.
  \end{itemize}


\end{frame}

\begin{frame}{Formalizing functionalism}

  \[ t = \iota x \phi (x)  \]

  Lewis: ``This is what I have called \textbf{functional
    definition}. The $\Sigma$-terms have been defined as the occupants
  of the causal roles specified by the theory $T$; as \emph{the}
  entities, whatever those may be, that bear certain causal relations
  to one another and to the referents of the $\Sigma _0$-terms.''
  \citeyearpar[p 254]{lewis1972}
  
\end{frame}


\begin{frame}{Reduction of domains}

  \begin{itemize}
  \item Classical reduction only permits relations (on a fixed domain)
    to be reduced to relations (on that same domain).
  \item If reduction is inverse to definition, then what we need is a
    method of defining new domains out of old ones.
  \end{itemize}

\end{frame}

\begin{frame}{Sort Constructions in First-Order Logic}
\centering
\begin{tikzpicture}[scale=1, every node/.style={font=\small}]
  % Product (bottom-left)
  \node at (-2,-1) {\textbf{Product}};
  \node (prod) at (-2,-1.75) {$\sigma_0 \times \sigma_1$};
  \node (s0) at (-3,-3) {$\sigma_0$};
  \node (s1) at (-1,-3) {$\sigma_1$};
  \draw[->] (prod) -- (s0);
  \draw[->] (prod) -- (s1);

    % Coproduct (bottom-right)
  \node at (3,-1) {\textbf{Coproduct}};
  \node (s0cop) at (2,-3) {$\sigma_0$};
  \node (s1cop) at (4,-3) {$\sigma_1$};
  \node (cop) at (3,-1.75) {$\sigma_0 \coprod \sigma_1$};
  \draw[->] (s0cop) -- (cop);
  \draw[->] (s1cop) -- (cop);


  % Quotient (top-left)
  \node at (-2,2.5) {\textbf{Quotient}};
  \node (sigma) at (-2,2) {$\sigma$};
  \node (sigmaq) at (-2,1) {$\sigma'$};
  \draw[->] (sigma) -- node[right] {$e$} (sigmaq);

  % Subsort (top-right)
  \node at (3,2.5) {\textbf{Subsort}};
  \node (sigmap) at (3,1) {$\sigma'$};
  \node (sigma2) at (3,2) {$\sigma$};
  \draw[->] (sigmap) -- node[right] {$i$} (sigma2);

\end{tikzpicture}
\end{frame}


\begin{frame}

  \begin{itemize}
  \item \textbf{Definition:} $T^+$ is a \textbf{Morita extension} of
    $T$ just in case $T^+$ is at the end of a finite chain of
    extensions by the sort constructions and explicit definitions on
    $T$.
  \item \textbf{Proposition:} If $T^+$ is a Morita extension of $T$,
    then there is reduction map $G:T^+\to T$.
  \end{itemize}

\end{frame}  


\begin{frame}{Example: The reducibility of universalism to atomism}

  \begin{itemize}
  \item The nihilist theory $T$ defines a domain of mereological
    $n$-sums (product, quotient by permutation symmetry) and so a
    universalist theory $T'$.
  \item For each formula $\phi (x)$ in $T'$, there is a corresponding
    formula $\phi ^*(y_1,y_2)$ in $T$.
  \item $T'\vdash\phi$ iff $T\vdash \phi ^*$
  \end{itemize}


\end{frame}



\begin{frame}{Nota bene}

  The previous two theories are actually equivalent.

  A more natural example might be:

  \begin{itemize}
  \item $T$ is the theory of the integers.
  \item $T'$ is the theory of the rational numbers.
  \end{itemize}



\end{frame}




\section{Conclusion}




\begin{frame}{The fundamental dilemma for reduction}

  \begin{itemize}
  \item If the $N$ to $O$ relation is too close, then $N$ seems like
    another name for $O$.
  \item If the $N$ to $O$ relation is too far apart, then $N$ seems
    ``ontologically problematic''.
  \end{itemize}


\end{frame}


\begin{frame}{Manifesto}

  \begin{itemize}
  \item Carnap the existentialist: freedom to develop new theoretical
    concepts.
  \item Reduction should give ``inference tickets'' in both
    directions.
  \item Compare (HH view of theoretical equivalence): if two theories
    are equivalent, you can switch between them.
  \item Against absolutism
  \end{itemize}



\end{frame}

\section{References}


\begin{frame}[allowframebreaks]{References}

\printbibliography[heading=none]

\end{frame}

\begin{frame}{Postscript}

  \small I predict that classical Nagelian reduction is going to to
  turn out to be too restrictive, no matter how sophisticated we get
  with the formal machinery, e.g.\ generalized translations. But it's
  still interesting to get a handle on the possible formal
  relationships between formal theories. In the case of propositional
  theories, translations are naturally classified by:
  \textbf{conservative}, \textbf{essentially surjective}, and of
  course, as being one half of an equivalence. Nagelian reductions are
  inverse of definitional extensions, and that entails that they are
  one half of an equivalence. Thus, the paradigm example of a
  reduction of formal theories would be both conservative and
  essentially surjective. That doesn't seem like a great model for
  examples we know from physics. For example, it does not seem that
  thermodynamics can define all of the concepts of classical
  mechanics.



\end{frame}


% End of document
\end{document}




%%% Local Variables:
%%% mode: latex
%%% TeX-master: t
%%% End:
